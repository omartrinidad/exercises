\documentclass[12pt]{article}

\usepackage[margin=1in]{geometry} 
\usepackage{amsmath,amsthm,amssymb,amsfonts,calc}
\usepackage{caption}
\usepackage{algorithm}
\usepackage{algpseudocode}
\usepackage{titlesec}

% Avoid section numbers
\titleformat{\section}
{\normalfont\Large\bfseries}% The style of the section title
{}                          % a prefix
{0pt}                       % How much space exists between the prefix and the title
{}    % How the section is represented

\titleformat{\subsection}
{\normalfont\large\bfseries}% The style of the section title
{}                          % a prefix
{0pt}                       % How much space exists between the prefix and the title
{}    % How the section is represented

\begin{document}
 
\title{Sheet 5} 
\maketitle

\section{Assignment 24}

The figure \ref{fig:learning} represents the typical learning curve for a MLP
trained with Backpropagation of Error using the single step learning strategy.
As we can see there is a first stage where the error decreases a little bit,
and after that we can see that this change happen faster. After this, the
change will be slow. And finally, we will reach a plateu.

\begin{figure}[h]
    \centering
    % GNUPLOT: LaTeX picture
\setlength{\unitlength}{0.240900pt}
\ifx\plotpoint\undefined\newsavebox{\plotpoint}\fi
\sbox{\plotpoint}{\rule[-0.200pt]{0.400pt}{0.400pt}}%
\begin{picture}(1500,900)(0,0)
\sbox{\plotpoint}{\rule[-0.200pt]{0.400pt}{0.400pt}}%
\put(171.0,131.0){\rule[-0.200pt]{4.818pt}{0.400pt}}
\put(151,131){\makebox(0,0)[r]{ 0.6}}
\put(1419.0,131.0){\rule[-0.200pt]{4.818pt}{0.400pt}}
\put(171.0,277.0){\rule[-0.200pt]{4.818pt}{0.400pt}}
\put(151,277){\makebox(0,0)[r]{ 0.8}}
\put(1419.0,277.0){\rule[-0.200pt]{4.818pt}{0.400pt}}
\put(171.0,422.0){\rule[-0.200pt]{4.818pt}{0.400pt}}
\put(151,422){\makebox(0,0)[r]{ 1}}
\put(1419.0,422.0){\rule[-0.200pt]{4.818pt}{0.400pt}}
\put(171.0,568.0){\rule[-0.200pt]{4.818pt}{0.400pt}}
\put(151,568){\makebox(0,0)[r]{ 1.2}}
\put(1419.0,568.0){\rule[-0.200pt]{4.818pt}{0.400pt}}
\put(171.0,713.0){\rule[-0.200pt]{4.818pt}{0.400pt}}
\put(151,713){\makebox(0,0)[r]{ 1.4}}
\put(1419.0,713.0){\rule[-0.200pt]{4.818pt}{0.400pt}}
\put(171.0,859.0){\rule[-0.200pt]{4.818pt}{0.400pt}}
\put(151,859){\makebox(0,0)[r]{ 1.6}}
\put(1419.0,859.0){\rule[-0.200pt]{4.818pt}{0.400pt}}
\put(171.0,131.0){\rule[-0.200pt]{0.400pt}{4.818pt}}
\put(171,90){\makebox(0,0){ 0}}
\put(171.0,839.0){\rule[-0.200pt]{0.400pt}{4.818pt}}
\put(352.0,131.0){\rule[-0.200pt]{0.400pt}{4.818pt}}
\put(352,90){\makebox(0,0){ 500}}
\put(352.0,839.0){\rule[-0.200pt]{0.400pt}{4.818pt}}
\put(533.0,131.0){\rule[-0.200pt]{0.400pt}{4.818pt}}
\put(533,90){\makebox(0,0){ 1000}}
\put(533.0,839.0){\rule[-0.200pt]{0.400pt}{4.818pt}}
\put(714.0,131.0){\rule[-0.200pt]{0.400pt}{4.818pt}}
\put(714,90){\makebox(0,0){ 1500}}
\put(714.0,839.0){\rule[-0.200pt]{0.400pt}{4.818pt}}
\put(896.0,131.0){\rule[-0.200pt]{0.400pt}{4.818pt}}
\put(896,90){\makebox(0,0){ 2000}}
\put(896.0,839.0){\rule[-0.200pt]{0.400pt}{4.818pt}}
\put(1077.0,131.0){\rule[-0.200pt]{0.400pt}{4.818pt}}
\put(1077,90){\makebox(0,0){ 2500}}
\put(1077.0,839.0){\rule[-0.200pt]{0.400pt}{4.818pt}}
\put(1258.0,131.0){\rule[-0.200pt]{0.400pt}{4.818pt}}
\put(1258,90){\makebox(0,0){ 3000}}
\put(1258.0,839.0){\rule[-0.200pt]{0.400pt}{4.818pt}}
\put(1439.0,131.0){\rule[-0.200pt]{0.400pt}{4.818pt}}
\put(1439,90){\makebox(0,0){ 3500}}
\put(1439.0,839.0){\rule[-0.200pt]{0.400pt}{4.818pt}}
\put(171.0,131.0){\rule[-0.200pt]{0.400pt}{175.375pt}}
\put(171.0,131.0){\rule[-0.200pt]{305.461pt}{0.400pt}}
\put(1439.0,131.0){\rule[-0.200pt]{0.400pt}{175.375pt}}
\put(171.0,859.0){\rule[-0.200pt]{305.461pt}{0.400pt}}
\put(30,495){\makebox(0,0){E(t)}}
\put(805,29){\makebox(0,0){t : Number of patterns}}
\put(171,768){\usebox{\plotpoint}}
\put(171,753.67){\rule{0.241pt}{0.400pt}}
\multiput(171.00,753.17)(0.500,1.000){2}{\rule{0.120pt}{0.400pt}}
\put(171.0,754.0){\rule[-0.200pt]{0.400pt}{3.373pt}}
\put(171.67,766){\rule{0.400pt}{4.336pt}}
\multiput(171.17,775.00)(1.000,-9.000){2}{\rule{0.400pt}{2.168pt}}
\put(172.0,755.0){\rule[-0.200pt]{0.400pt}{6.986pt}}
\put(172.67,753){\rule{0.400pt}{1.686pt}}
\multiput(172.17,756.50)(1.000,-3.500){2}{\rule{0.400pt}{0.843pt}}
\put(173.0,760.0){\rule[-0.200pt]{0.400pt}{1.445pt}}
\put(174.0,753.0){\rule[-0.200pt]{0.400pt}{6.986pt}}
\put(173.67,749){\rule{0.400pt}{7.950pt}}
\multiput(173.17,749.00)(1.000,16.500){2}{\rule{0.400pt}{3.975pt}}
\put(174.0,749.0){\rule[-0.200pt]{0.400pt}{7.950pt}}
\put(174.67,748){\rule{0.400pt}{6.745pt}}
\multiput(174.17,748.00)(1.000,14.000){2}{\rule{0.400pt}{3.373pt}}
\put(175.0,748.0){\rule[-0.200pt]{0.400pt}{8.191pt}}
\put(176.0,751.0){\rule[-0.200pt]{0.400pt}{6.022pt}}
\put(175.67,754){\rule{0.400pt}{4.577pt}}
\multiput(175.17,763.50)(1.000,-9.500){2}{\rule{0.400pt}{2.289pt}}
\put(176.0,751.0){\rule[-0.200pt]{0.400pt}{5.300pt}}
\put(176.67,758){\rule{0.400pt}{0.964pt}}
\multiput(176.17,760.00)(1.000,-2.000){2}{\rule{0.400pt}{0.482pt}}
\put(177.0,754.0){\rule[-0.200pt]{0.400pt}{1.927pt}}
\put(178.0,758.0){\rule[-0.200pt]{0.400pt}{4.095pt}}
\put(177.67,760){\rule{0.400pt}{3.373pt}}
\multiput(177.17,767.00)(1.000,-7.000){2}{\rule{0.400pt}{1.686pt}}
\put(178.0,774.0){\usebox{\plotpoint}}
\put(179.0,760.0){\rule[-0.200pt]{0.400pt}{4.095pt}}
\put(178.67,753){\rule{0.400pt}{4.336pt}}
\multiput(178.17,762.00)(1.000,-9.000){2}{\rule{0.400pt}{2.168pt}}
\put(179.0,771.0){\rule[-0.200pt]{0.400pt}{1.445pt}}
\put(180.0,753.0){\rule[-0.200pt]{0.400pt}{6.263pt}}
\put(179.67,753){\rule{0.400pt}{3.854pt}}
\multiput(179.17,753.00)(1.000,8.000){2}{\rule{0.400pt}{1.927pt}}
\put(180.0,753.0){\rule[-0.200pt]{0.400pt}{6.263pt}}
\put(180.67,751){\rule{0.400pt}{2.650pt}}
\multiput(180.17,756.50)(1.000,-5.500){2}{\rule{0.400pt}{1.325pt}}
\put(181.0,762.0){\rule[-0.200pt]{0.400pt}{1.686pt}}
\put(182.0,751.0){\rule[-0.200pt]{0.400pt}{2.650pt}}
\put(181.67,743){\rule{0.400pt}{5.059pt}}
\multiput(181.17,743.00)(1.000,10.500){2}{\rule{0.400pt}{2.529pt}}
\put(182.0,743.0){\rule[-0.200pt]{0.400pt}{4.577pt}}
\put(183.0,756.0){\rule[-0.200pt]{0.400pt}{1.927pt}}
\put(182.67,759){\rule{0.400pt}{4.336pt}}
\multiput(182.17,768.00)(1.000,-9.000){2}{\rule{0.400pt}{2.168pt}}
\put(183.0,756.0){\rule[-0.200pt]{0.400pt}{5.059pt}}
\put(183.67,757){\rule{0.400pt}{3.614pt}}
\multiput(183.17,764.50)(1.000,-7.500){2}{\rule{0.400pt}{1.807pt}}
\put(184.0,759.0){\rule[-0.200pt]{0.400pt}{3.132pt}}
\put(184.67,744){\rule{0.400pt}{6.986pt}}
\multiput(184.17,744.00)(1.000,14.500){2}{\rule{0.400pt}{3.493pt}}
\put(185.0,744.0){\rule[-0.200pt]{0.400pt}{3.132pt}}
\put(185.67,747){\rule{0.400pt}{1.927pt}}
\multiput(185.17,747.00)(1.000,4.000){2}{\rule{0.400pt}{0.964pt}}
\put(186.0,747.0){\rule[-0.200pt]{0.400pt}{6.263pt}}
\put(187.0,747.0){\rule[-0.200pt]{0.400pt}{1.927pt}}
\put(187.0,747.0){\rule[-0.200pt]{0.400pt}{0.482pt}}
\put(187.0,749.0){\usebox{\plotpoint}}
\put(188.0,739.0){\rule[-0.200pt]{0.400pt}{2.409pt}}
\put(187.67,746){\rule{0.400pt}{1.204pt}}
\multiput(187.17,746.00)(1.000,2.500){2}{\rule{0.400pt}{0.602pt}}
\put(188.0,739.0){\rule[-0.200pt]{0.400pt}{1.686pt}}
\put(189.0,741.0){\rule[-0.200pt]{0.400pt}{2.409pt}}
\put(188.67,742){\rule{0.400pt}{2.650pt}}
\multiput(188.17,742.00)(1.000,5.500){2}{\rule{0.400pt}{1.325pt}}
\put(189.0,741.0){\usebox{\plotpoint}}
\put(189.67,766){\rule{0.400pt}{1.686pt}}
\multiput(189.17,766.00)(1.000,3.500){2}{\rule{0.400pt}{0.843pt}}
\put(190.0,753.0){\rule[-0.200pt]{0.400pt}{3.132pt}}
\put(191.0,755.0){\rule[-0.200pt]{0.400pt}{4.336pt}}
\put(190.67,752){\rule{0.400pt}{4.818pt}}
\multiput(190.17,762.00)(1.000,-10.000){2}{\rule{0.400pt}{2.409pt}}
\put(191.0,755.0){\rule[-0.200pt]{0.400pt}{4.095pt}}
\put(192.0,752.0){\rule[-0.200pt]{0.400pt}{3.373pt}}
\put(192,763.67){\rule{0.241pt}{0.400pt}}
\multiput(192.00,764.17)(0.500,-1.000){2}{\rule{0.120pt}{0.400pt}}
\put(192.0,765.0){\usebox{\plotpoint}}
\put(193.0,742.0){\rule[-0.200pt]{0.400pt}{5.300pt}}
\put(192.67,753){\rule{0.400pt}{0.723pt}}
\multiput(192.17,754.50)(1.000,-1.500){2}{\rule{0.400pt}{0.361pt}}
\put(193.0,742.0){\rule[-0.200pt]{0.400pt}{3.373pt}}
\put(193.67,740){\rule{0.400pt}{0.723pt}}
\multiput(193.17,740.00)(1.000,1.500){2}{\rule{0.400pt}{0.361pt}}
\put(194.0,740.0){\rule[-0.200pt]{0.400pt}{3.132pt}}
\put(194.67,763){\rule{0.400pt}{0.964pt}}
\multiput(194.17,763.00)(1.000,2.000){2}{\rule{0.400pt}{0.482pt}}
\put(195.0,743.0){\rule[-0.200pt]{0.400pt}{4.818pt}}
\put(196.0,741.0){\rule[-0.200pt]{0.400pt}{6.263pt}}
\put(195.67,747){\rule{0.400pt}{1.445pt}}
\multiput(195.17,747.00)(1.000,3.000){2}{\rule{0.400pt}{0.723pt}}
\put(196.0,741.0){\rule[-0.200pt]{0.400pt}{1.445pt}}
\put(197.0,739.0){\rule[-0.200pt]{0.400pt}{3.373pt}}
\put(196.67,747){\rule{0.400pt}{0.964pt}}
\multiput(196.17,749.00)(1.000,-2.000){2}{\rule{0.400pt}{0.482pt}}
\put(197.0,739.0){\rule[-0.200pt]{0.400pt}{2.891pt}}
\put(197.67,736){\rule{0.400pt}{0.482pt}}
\multiput(197.17,736.00)(1.000,1.000){2}{\rule{0.400pt}{0.241pt}}
\put(198.0,736.0){\rule[-0.200pt]{0.400pt}{2.650pt}}
\put(199.0,738.0){\rule[-0.200pt]{0.400pt}{2.650pt}}
\put(198.67,740){\rule{0.400pt}{4.818pt}}
\multiput(198.17,740.00)(1.000,10.000){2}{\rule{0.400pt}{2.409pt}}
\put(199.0,740.0){\rule[-0.200pt]{0.400pt}{2.168pt}}
\put(200.0,760.0){\rule[-0.200pt]{0.400pt}{1.445pt}}
\put(199.67,736){\rule{0.400pt}{4.095pt}}
\multiput(199.17,744.50)(1.000,-8.500){2}{\rule{0.400pt}{2.048pt}}
\put(200.0,753.0){\rule[-0.200pt]{0.400pt}{3.132pt}}
\put(201.0,736.0){\rule[-0.200pt]{0.400pt}{7.227pt}}
\put(200.67,733){\rule{0.400pt}{2.891pt}}
\multiput(200.17,739.00)(1.000,-6.000){2}{\rule{0.400pt}{1.445pt}}
\put(201.0,745.0){\rule[-0.200pt]{0.400pt}{5.059pt}}
\put(201.67,737){\rule{0.400pt}{1.927pt}}
\multiput(201.17,741.00)(1.000,-4.000){2}{\rule{0.400pt}{0.964pt}}
\put(202.0,733.0){\rule[-0.200pt]{0.400pt}{2.891pt}}
\put(202.67,740){\rule{0.400pt}{3.614pt}}
\multiput(202.17,740.00)(1.000,7.500){2}{\rule{0.400pt}{1.807pt}}
\put(203.0,737.0){\rule[-0.200pt]{0.400pt}{0.723pt}}
\put(204.0,739.0){\rule[-0.200pt]{0.400pt}{3.854pt}}
\put(203.67,756){\rule{0.400pt}{1.686pt}}
\multiput(203.17,756.00)(1.000,3.500){2}{\rule{0.400pt}{0.843pt}}
\put(204.0,739.0){\rule[-0.200pt]{0.400pt}{4.095pt}}
\put(204.67,744){\rule{0.400pt}{0.723pt}}
\multiput(204.17,744.00)(1.000,1.500){2}{\rule{0.400pt}{0.361pt}}
\put(205.0,744.0){\rule[-0.200pt]{0.400pt}{4.577pt}}
\put(205.67,753){\rule{0.400pt}{0.723pt}}
\multiput(205.17,754.50)(1.000,-1.500){2}{\rule{0.400pt}{0.361pt}}
\put(206.0,747.0){\rule[-0.200pt]{0.400pt}{2.168pt}}
\put(206.67,730){\rule{0.400pt}{0.482pt}}
\multiput(206.17,731.00)(1.000,-1.000){2}{\rule{0.400pt}{0.241pt}}
\put(207.0,732.0){\rule[-0.200pt]{0.400pt}{5.059pt}}
\put(207.67,748){\rule{0.400pt}{1.445pt}}
\multiput(207.17,748.00)(1.000,3.000){2}{\rule{0.400pt}{0.723pt}}
\put(208.0,730.0){\rule[-0.200pt]{0.400pt}{4.336pt}}
\put(209.0,752.0){\rule[-0.200pt]{0.400pt}{0.482pt}}
\put(208.67,750){\rule{0.400pt}{1.927pt}}
\multiput(208.17,754.00)(1.000,-4.000){2}{\rule{0.400pt}{0.964pt}}
\put(209.0,752.0){\rule[-0.200pt]{0.400pt}{1.445pt}}
\put(210.0,728.0){\rule[-0.200pt]{0.400pt}{5.300pt}}
\put(209.67,752){\rule{0.400pt}{1.445pt}}
\multiput(209.17,752.00)(1.000,3.000){2}{\rule{0.400pt}{0.723pt}}
\put(210.0,728.0){\rule[-0.200pt]{0.400pt}{5.782pt}}
\put(210.67,731){\rule{0.400pt}{4.818pt}}
\multiput(210.17,731.00)(1.000,10.000){2}{\rule{0.400pt}{2.409pt}}
\put(211.0,731.0){\rule[-0.200pt]{0.400pt}{6.504pt}}
\put(212.0,723.0){\rule[-0.200pt]{0.400pt}{6.745pt}}
\put(211.67,742){\rule{0.400pt}{2.168pt}}
\multiput(211.17,742.00)(1.000,4.500){2}{\rule{0.400pt}{1.084pt}}
\put(212.0,723.0){\rule[-0.200pt]{0.400pt}{4.577pt}}
\put(212.67,723){\rule{0.400pt}{8.191pt}}
\multiput(212.17,723.00)(1.000,17.000){2}{\rule{0.400pt}{4.095pt}}
\put(213.0,723.0){\rule[-0.200pt]{0.400pt}{6.745pt}}
\put(214.0,741.0){\rule[-0.200pt]{0.400pt}{3.854pt}}
\put(213.67,729){\rule{0.400pt}{5.059pt}}
\multiput(213.17,739.50)(1.000,-10.500){2}{\rule{0.400pt}{2.529pt}}
\put(214.0,741.0){\rule[-0.200pt]{0.400pt}{2.168pt}}
\put(214.67,722){\rule{0.400pt}{3.614pt}}
\multiput(214.17,729.50)(1.000,-7.500){2}{\rule{0.400pt}{1.807pt}}
\put(215.0,729.0){\rule[-0.200pt]{0.400pt}{1.927pt}}
\put(216.0,720.0){\rule[-0.200pt]{0.400pt}{0.482pt}}
\put(215.67,724){\rule{0.400pt}{7.468pt}}
\multiput(215.17,724.00)(1.000,15.500){2}{\rule{0.400pt}{3.734pt}}
\put(216.0,720.0){\rule[-0.200pt]{0.400pt}{0.964pt}}
\put(216.67,731){\rule{0.400pt}{3.373pt}}
\multiput(216.17,731.00)(1.000,7.000){2}{\rule{0.400pt}{1.686pt}}
\put(217.0,731.0){\rule[-0.200pt]{0.400pt}{5.782pt}}
\put(218.0,721.0){\rule[-0.200pt]{0.400pt}{5.782pt}}
\put(217.67,731){\rule{0.400pt}{1.445pt}}
\multiput(217.17,734.00)(1.000,-3.000){2}{\rule{0.400pt}{0.723pt}}
\put(218.0,721.0){\rule[-0.200pt]{0.400pt}{3.854pt}}
\put(218.67,720){\rule{0.400pt}{7.950pt}}
\multiput(218.17,736.50)(1.000,-16.500){2}{\rule{0.400pt}{3.975pt}}
\put(219.0,731.0){\rule[-0.200pt]{0.400pt}{5.300pt}}
\put(220.0,720.0){\rule[-0.200pt]{0.400pt}{7.950pt}}
\put(220,722.67){\rule{0.241pt}{0.400pt}}
\multiput(220.00,722.17)(0.500,1.000){2}{\rule{0.120pt}{0.400pt}}
\put(220.0,723.0){\rule[-0.200pt]{0.400pt}{7.227pt}}
\put(221.0,724.0){\rule[-0.200pt]{0.400pt}{1.686pt}}
\put(220.67,717){\rule{0.400pt}{4.336pt}}
\multiput(220.17,717.00)(1.000,9.000){2}{\rule{0.400pt}{2.168pt}}
\put(221.0,717.0){\rule[-0.200pt]{0.400pt}{3.373pt}}
\put(222.0,719.0){\rule[-0.200pt]{0.400pt}{3.854pt}}
\put(221.67,724){\rule{0.400pt}{4.577pt}}
\multiput(221.17,733.50)(1.000,-9.500){2}{\rule{0.400pt}{2.289pt}}
\put(222.0,719.0){\rule[-0.200pt]{0.400pt}{5.782pt}}
\put(222.67,716){\rule{0.400pt}{6.745pt}}
\multiput(222.17,730.00)(1.000,-14.000){2}{\rule{0.400pt}{3.373pt}}
\put(223.0,724.0){\rule[-0.200pt]{0.400pt}{4.818pt}}
\put(224.0,716.0){\rule[-0.200pt]{0.400pt}{7.950pt}}
\put(223.67,733){\rule{0.400pt}{0.482pt}}
\multiput(223.17,733.00)(1.000,1.000){2}{\rule{0.400pt}{0.241pt}}
\put(224.0,733.0){\rule[-0.200pt]{0.400pt}{3.854pt}}
\put(224.67,716){\rule{0.400pt}{0.964pt}}
\multiput(224.17,718.00)(1.000,-2.000){2}{\rule{0.400pt}{0.482pt}}
\put(225.0,720.0){\rule[-0.200pt]{0.400pt}{3.613pt}}
\put(226.0,716.0){\rule[-0.200pt]{0.400pt}{6.986pt}}
\put(225.67,730){\rule{0.400pt}{1.686pt}}
\multiput(225.17,730.00)(1.000,3.500){2}{\rule{0.400pt}{0.843pt}}
\put(226.0,730.0){\rule[-0.200pt]{0.400pt}{3.613pt}}
\put(226.67,715){\rule{0.400pt}{7.709pt}}
\multiput(226.17,731.00)(1.000,-16.000){2}{\rule{0.400pt}{3.854pt}}
\put(227.0,737.0){\rule[-0.200pt]{0.400pt}{2.409pt}}
\put(228.0,715.0){\rule[-0.200pt]{0.400pt}{6.986pt}}
\put(227.67,737){\rule{0.400pt}{0.482pt}}
\multiput(227.17,738.00)(1.000,-1.000){2}{\rule{0.400pt}{0.241pt}}
\put(228.0,739.0){\rule[-0.200pt]{0.400pt}{1.204pt}}
\put(229.0,714.0){\rule[-0.200pt]{0.400pt}{5.541pt}}
\put(229,723.67){\rule{0.241pt}{0.400pt}}
\multiput(229.00,724.17)(0.500,-1.000){2}{\rule{0.120pt}{0.400pt}}
\put(229.0,714.0){\rule[-0.200pt]{0.400pt}{2.650pt}}
\put(230.0,724.0){\rule[-0.200pt]{0.400pt}{3.854pt}}
\put(229.67,732){\rule{0.400pt}{3.132pt}}
\multiput(229.17,732.00)(1.000,6.500){2}{\rule{0.400pt}{1.566pt}}
\put(230.0,732.0){\rule[-0.200pt]{0.400pt}{1.927pt}}
\put(230.67,716){\rule{0.400pt}{1.927pt}}
\multiput(230.17,716.00)(1.000,4.000){2}{\rule{0.400pt}{0.964pt}}
\put(231.0,716.0){\rule[-0.200pt]{0.400pt}{6.986pt}}
\put(232.0,724.0){\rule[-0.200pt]{0.400pt}{1.686pt}}
\put(231.67,726){\rule{0.400pt}{0.964pt}}
\multiput(231.17,726.00)(1.000,2.000){2}{\rule{0.400pt}{0.482pt}}
\put(232.0,726.0){\rule[-0.200pt]{0.400pt}{1.204pt}}
\put(233.0,719.0){\rule[-0.200pt]{0.400pt}{2.650pt}}
\put(232.67,713){\rule{0.400pt}{3.373pt}}
\multiput(232.17,720.00)(1.000,-7.000){2}{\rule{0.400pt}{1.686pt}}
\put(233.0,719.0){\rule[-0.200pt]{0.400pt}{1.927pt}}
\put(234.0,713.0){\rule[-0.200pt]{0.400pt}{7.468pt}}
\put(233.67,717){\rule{0.400pt}{5.541pt}}
\multiput(233.17,728.50)(1.000,-11.500){2}{\rule{0.400pt}{2.770pt}}
\put(234.0,740.0){\rule[-0.200pt]{0.400pt}{0.964pt}}
\put(235.0,717.0){\rule[-0.200pt]{0.400pt}{5.059pt}}
\put(234.67,714){\rule{0.400pt}{1.445pt}}
\multiput(234.17,717.00)(1.000,-3.000){2}{\rule{0.400pt}{0.723pt}}
\put(235.0,720.0){\rule[-0.200pt]{0.400pt}{4.336pt}}
\put(235.67,733){\rule{0.400pt}{0.723pt}}
\multiput(235.17,734.50)(1.000,-1.500){2}{\rule{0.400pt}{0.361pt}}
\put(236.0,714.0){\rule[-0.200pt]{0.400pt}{5.300pt}}
\put(237.0,733.0){\rule[-0.200pt]{0.400pt}{1.927pt}}
\put(236.67,709){\rule{0.400pt}{5.059pt}}
\multiput(236.17,709.00)(1.000,10.500){2}{\rule{0.400pt}{2.529pt}}
\put(237.0,709.0){\rule[-0.200pt]{0.400pt}{7.709pt}}
\put(238.0,719.0){\rule[-0.200pt]{0.400pt}{2.650pt}}
\put(237.67,733){\rule{0.400pt}{0.482pt}}
\multiput(237.17,733.00)(1.000,1.000){2}{\rule{0.400pt}{0.241pt}}
\put(238.0,719.0){\rule[-0.200pt]{0.400pt}{3.373pt}}
\put(238.67,708){\rule{0.400pt}{2.409pt}}
\multiput(238.17,708.00)(1.000,5.000){2}{\rule{0.400pt}{1.204pt}}
\put(239.0,708.0){\rule[-0.200pt]{0.400pt}{6.504pt}}
\put(239.67,733){\rule{0.400pt}{1.204pt}}
\multiput(239.17,735.50)(1.000,-2.500){2}{\rule{0.400pt}{0.602pt}}
\put(240.0,718.0){\rule[-0.200pt]{0.400pt}{4.818pt}}
\put(241.0,733.0){\rule[-0.200pt]{0.400pt}{0.723pt}}
\put(240.67,704){\rule{0.400pt}{1.927pt}}
\multiput(240.17,704.00)(1.000,4.000){2}{\rule{0.400pt}{0.964pt}}
\put(241.0,704.0){\rule[-0.200pt]{0.400pt}{7.709pt}}
\put(242.0,712.0){\rule[-0.200pt]{0.400pt}{4.095pt}}
\put(241.67,716){\rule{0.400pt}{2.891pt}}
\multiput(241.17,716.00)(1.000,6.000){2}{\rule{0.400pt}{1.445pt}}
\put(242.0,716.0){\rule[-0.200pt]{0.400pt}{3.132pt}}
\put(243.0,705.0){\rule[-0.200pt]{0.400pt}{5.541pt}}
\put(243,722.67){\rule{0.241pt}{0.400pt}}
\multiput(243.00,722.17)(0.500,1.000){2}{\rule{0.120pt}{0.400pt}}
\put(243.0,705.0){\rule[-0.200pt]{0.400pt}{4.336pt}}
\put(243.67,725){\rule{0.400pt}{2.650pt}}
\multiput(243.17,725.00)(1.000,5.500){2}{\rule{0.400pt}{1.325pt}}
\put(244.0,724.0){\usebox{\plotpoint}}
\put(244.67,708){\rule{0.400pt}{0.964pt}}
\multiput(244.17,710.00)(1.000,-2.000){2}{\rule{0.400pt}{0.482pt}}
\put(245.0,712.0){\rule[-0.200pt]{0.400pt}{5.782pt}}
\put(246.0,708.0){\rule[-0.200pt]{0.400pt}{3.132pt}}
\put(245.67,716){\rule{0.400pt}{0.482pt}}
\multiput(245.17,716.00)(1.000,1.000){2}{\rule{0.400pt}{0.241pt}}
\put(246.0,716.0){\rule[-0.200pt]{0.400pt}{1.204pt}}
\put(247.0,700.0){\rule[-0.200pt]{0.400pt}{4.336pt}}
\put(247,703.67){\rule{0.241pt}{0.400pt}}
\multiput(247.00,703.17)(0.500,1.000){2}{\rule{0.120pt}{0.400pt}}
\put(247.0,700.0){\rule[-0.200pt]{0.400pt}{0.964pt}}
\put(248.0,705.0){\rule[-0.200pt]{0.400pt}{1.927pt}}
\put(248.0,713.0){\usebox{\plotpoint}}
\put(249.0,713.0){\rule[-0.200pt]{0.400pt}{3.373pt}}
\put(248.67,703){\rule{0.400pt}{1.445pt}}
\multiput(248.17,706.00)(1.000,-3.000){2}{\rule{0.400pt}{0.723pt}}
\put(249.0,709.0){\rule[-0.200pt]{0.400pt}{4.336pt}}
\put(250.0,699.0){\rule[-0.200pt]{0.400pt}{0.964pt}}
\put(249.67,729){\rule{0.400pt}{0.482pt}}
\multiput(249.17,730.00)(1.000,-1.000){2}{\rule{0.400pt}{0.241pt}}
\put(250.0,699.0){\rule[-0.200pt]{0.400pt}{7.709pt}}
\put(250.67,709){\rule{0.400pt}{4.577pt}}
\multiput(250.17,709.00)(1.000,9.500){2}{\rule{0.400pt}{2.289pt}}
\put(251.0,709.0){\rule[-0.200pt]{0.400pt}{4.818pt}}
\put(252,720.67){\rule{0.241pt}{0.400pt}}
\multiput(252.00,721.17)(0.500,-1.000){2}{\rule{0.120pt}{0.400pt}}
\put(252.0,722.0){\rule[-0.200pt]{0.400pt}{1.445pt}}
\put(252.67,698){\rule{0.400pt}{6.504pt}}
\multiput(252.17,698.00)(1.000,13.500){2}{\rule{0.400pt}{3.252pt}}
\put(253.0,698.0){\rule[-0.200pt]{0.400pt}{5.541pt}}
\put(253.67,696){\rule{0.400pt}{1.445pt}}
\multiput(253.17,696.00)(1.000,3.000){2}{\rule{0.400pt}{0.723pt}}
\put(254.0,696.0){\rule[-0.200pt]{0.400pt}{6.986pt}}
\put(254.67,723){\rule{0.400pt}{0.723pt}}
\multiput(254.17,724.50)(1.000,-1.500){2}{\rule{0.400pt}{0.361pt}}
\put(255.0,702.0){\rule[-0.200pt]{0.400pt}{5.782pt}}
\put(256.0,696.0){\rule[-0.200pt]{0.400pt}{6.504pt}}
\put(255.67,701){\rule{0.400pt}{3.132pt}}
\multiput(255.17,701.00)(1.000,6.500){2}{\rule{0.400pt}{1.566pt}}
\put(256.0,696.0){\rule[-0.200pt]{0.400pt}{1.204pt}}
\put(256.67,707){\rule{0.400pt}{1.204pt}}
\multiput(256.17,707.00)(1.000,2.500){2}{\rule{0.400pt}{0.602pt}}
\put(257.0,707.0){\rule[-0.200pt]{0.400pt}{1.686pt}}
\put(258.0,712.0){\rule[-0.200pt]{0.400pt}{1.686pt}}
\put(257.67,692){\rule{0.400pt}{6.504pt}}
\multiput(257.17,692.00)(1.000,13.500){2}{\rule{0.400pt}{3.252pt}}
\put(258.0,692.0){\rule[-0.200pt]{0.400pt}{6.504pt}}
\put(258.67,701){\rule{0.400pt}{5.541pt}}
\multiput(258.17,701.00)(1.000,11.500){2}{\rule{0.400pt}{2.770pt}}
\put(259.0,701.0){\rule[-0.200pt]{0.400pt}{4.336pt}}
\put(260.0,696.0){\rule[-0.200pt]{0.400pt}{6.745pt}}
\put(259.67,701){\rule{0.400pt}{4.818pt}}
\multiput(259.17,711.00)(1.000,-10.000){2}{\rule{0.400pt}{2.409pt}}
\put(260.0,696.0){\rule[-0.200pt]{0.400pt}{6.022pt}}
\put(260.67,706){\rule{0.400pt}{2.650pt}}
\multiput(260.17,706.00)(1.000,5.500){2}{\rule{0.400pt}{1.325pt}}
\put(261.0,701.0){\rule[-0.200pt]{0.400pt}{1.204pt}}
\put(262.0,692.0){\rule[-0.200pt]{0.400pt}{6.022pt}}
\put(261.67,690){\rule{0.400pt}{2.409pt}}
\multiput(261.17,695.00)(1.000,-5.000){2}{\rule{0.400pt}{1.204pt}}
\put(262.0,692.0){\rule[-0.200pt]{0.400pt}{1.927pt}}
\put(263.0,690.0){\rule[-0.200pt]{0.400pt}{7.709pt}}
\put(262.67,696){\rule{0.400pt}{6.023pt}}
\multiput(262.17,696.00)(1.000,12.500){2}{\rule{0.400pt}{3.011pt}}
\put(263.0,696.0){\rule[-0.200pt]{0.400pt}{6.263pt}}
\put(263.67,710){\rule{0.400pt}{1.927pt}}
\multiput(263.17,710.00)(1.000,4.000){2}{\rule{0.400pt}{0.964pt}}
\put(264.0,710.0){\rule[-0.200pt]{0.400pt}{2.650pt}}
\put(265.0,690.0){\rule[-0.200pt]{0.400pt}{6.745pt}}
\put(265.0,690.0){\usebox{\plotpoint}}
\put(266.0,690.0){\rule[-0.200pt]{0.400pt}{6.263pt}}
\put(265.67,696){\rule{0.400pt}{2.168pt}}
\multiput(265.17,696.00)(1.000,4.500){2}{\rule{0.400pt}{1.084pt}}
\put(266.0,696.0){\rule[-0.200pt]{0.400pt}{4.818pt}}
\put(267.0,693.0){\rule[-0.200pt]{0.400pt}{2.891pt}}
\put(266.67,690){\rule{0.400pt}{1.204pt}}
\multiput(266.17,692.50)(1.000,-2.500){2}{\rule{0.400pt}{0.602pt}}
\put(267.0,693.0){\rule[-0.200pt]{0.400pt}{0.482pt}}
\put(268.0,690.0){\rule[-0.200pt]{0.400pt}{7.227pt}}
\put(267.67,709){\rule{0.400pt}{1.445pt}}
\multiput(267.17,712.00)(1.000,-3.000){2}{\rule{0.400pt}{0.723pt}}
\put(268.0,715.0){\rule[-0.200pt]{0.400pt}{1.204pt}}
\put(268.67,695){\rule{0.400pt}{1.204pt}}
\multiput(268.17,697.50)(1.000,-2.500){2}{\rule{0.400pt}{0.602pt}}
\put(269.0,700.0){\rule[-0.200pt]{0.400pt}{2.168pt}}
\put(270.0,695.0){\rule[-0.200pt]{0.400pt}{5.059pt}}
\put(269.67,686){\rule{0.400pt}{1.927pt}}
\multiput(269.17,686.00)(1.000,4.000){2}{\rule{0.400pt}{0.964pt}}
\put(270.0,686.0){\rule[-0.200pt]{0.400pt}{7.227pt}}
\put(270.67,708){\rule{0.400pt}{2.168pt}}
\multiput(270.17,708.00)(1.000,4.500){2}{\rule{0.400pt}{1.084pt}}
\put(271.0,694.0){\rule[-0.200pt]{0.400pt}{3.373pt}}
\put(272.0,684.0){\rule[-0.200pt]{0.400pt}{7.950pt}}
\put(271.67,683){\rule{0.400pt}{7.468pt}}
\multiput(271.17,698.50)(1.000,-15.500){2}{\rule{0.400pt}{3.734pt}}
\put(272.0,684.0){\rule[-0.200pt]{0.400pt}{7.227pt}}
\put(272.67,698){\rule{0.400pt}{1.204pt}}
\multiput(272.17,698.00)(1.000,2.500){2}{\rule{0.400pt}{0.602pt}}
\put(273.0,683.0){\rule[-0.200pt]{0.400pt}{3.613pt}}
\put(274.0,700.0){\rule[-0.200pt]{0.400pt}{0.723pt}}
\put(274,702.67){\rule{0.241pt}{0.400pt}}
\multiput(274.00,703.17)(0.500,-1.000){2}{\rule{0.120pt}{0.400pt}}
\put(274.0,700.0){\rule[-0.200pt]{0.400pt}{0.964pt}}
\put(275.0,703.0){\rule[-0.200pt]{0.400pt}{1.445pt}}
\put(274.67,687){\rule{0.400pt}{6.745pt}}
\multiput(274.17,687.00)(1.000,14.000){2}{\rule{0.400pt}{3.373pt}}
\put(275.0,687.0){\rule[-0.200pt]{0.400pt}{5.300pt}}
\put(276.0,692.0){\rule[-0.200pt]{0.400pt}{5.541pt}}
\put(275.67,681){\rule{0.400pt}{8.191pt}}
\multiput(275.17,698.00)(1.000,-17.000){2}{\rule{0.400pt}{4.095pt}}
\put(276.0,692.0){\rule[-0.200pt]{0.400pt}{5.541pt}}
\put(276.67,695){\rule{0.400pt}{0.964pt}}
\multiput(276.17,695.00)(1.000,2.000){2}{\rule{0.400pt}{0.482pt}}
\put(277.0,681.0){\rule[-0.200pt]{0.400pt}{3.373pt}}
\put(277.67,682){\rule{0.400pt}{5.300pt}}
\multiput(277.17,682.00)(1.000,11.000){2}{\rule{0.400pt}{2.650pt}}
\put(278.0,682.0){\rule[-0.200pt]{0.400pt}{4.095pt}}
\put(279.0,704.0){\rule[-0.200pt]{0.400pt}{0.482pt}}
\put(278.67,695){\rule{0.400pt}{2.168pt}}
\multiput(278.17,699.50)(1.000,-4.500){2}{\rule{0.400pt}{1.084pt}}
\put(279.0,704.0){\rule[-0.200pt]{0.400pt}{0.482pt}}
\put(280.0,694.0){\usebox{\plotpoint}}
\put(279.67,692){\rule{0.400pt}{3.614pt}}
\multiput(279.17,699.50)(1.000,-7.500){2}{\rule{0.400pt}{1.807pt}}
\put(280.0,694.0){\rule[-0.200pt]{0.400pt}{3.132pt}}
\put(281,692){\usebox{\plotpoint}}
\put(280.67,687){\rule{0.400pt}{1.204pt}}
\multiput(280.17,687.00)(1.000,2.500){2}{\rule{0.400pt}{0.602pt}}
\put(281.0,687.0){\rule[-0.200pt]{0.400pt}{1.204pt}}
\put(281.67,688){\rule{0.400pt}{2.891pt}}
\multiput(281.17,688.00)(1.000,6.000){2}{\rule{0.400pt}{1.445pt}}
\put(282.0,688.0){\rule[-0.200pt]{0.400pt}{0.964pt}}
\put(283.0,667.0){\rule[-0.200pt]{0.400pt}{7.950pt}}
\put(282.67,681){\rule{0.400pt}{1.204pt}}
\multiput(282.17,681.00)(1.000,2.500){2}{\rule{0.400pt}{0.602pt}}
\put(283.0,667.0){\rule[-0.200pt]{0.400pt}{3.373pt}}
\put(284.0,665.0){\rule[-0.200pt]{0.400pt}{5.059pt}}
\put(283.67,670){\rule{0.400pt}{2.168pt}}
\multiput(283.17,670.00)(1.000,4.500){2}{\rule{0.400pt}{1.084pt}}
\put(284.0,665.0){\rule[-0.200pt]{0.400pt}{1.204pt}}
\put(285.0,672.0){\rule[-0.200pt]{0.400pt}{1.686pt}}
\put(284.67,679){\rule{0.400pt}{1.204pt}}
\multiput(284.17,681.50)(1.000,-2.500){2}{\rule{0.400pt}{0.602pt}}
\put(285.0,672.0){\rule[-0.200pt]{0.400pt}{2.891pt}}
\put(286,679){\usebox{\plotpoint}}
\put(285.67,663){\rule{0.400pt}{3.854pt}}
\multiput(285.17,671.00)(1.000,-8.000){2}{\rule{0.400pt}{1.927pt}}
\put(287.0,663.0){\rule[-0.200pt]{0.400pt}{2.409pt}}
\put(286.67,654){\rule{0.400pt}{1.686pt}}
\multiput(286.17,657.50)(1.000,-3.500){2}{\rule{0.400pt}{0.843pt}}
\put(287.0,661.0){\rule[-0.200pt]{0.400pt}{2.891pt}}
\put(288.0,654.0){\rule[-0.200pt]{0.400pt}{3.132pt}}
\put(288.0,651.0){\rule[-0.200pt]{0.400pt}{3.854pt}}
\put(288.0,651.0){\usebox{\plotpoint}}
\put(289.0,651.0){\rule[-0.200pt]{0.400pt}{4.095pt}}
\put(288.67,656){\rule{0.400pt}{0.482pt}}
\multiput(288.17,656.00)(1.000,1.000){2}{\rule{0.400pt}{0.241pt}}
\put(289.0,656.0){\rule[-0.200pt]{0.400pt}{2.891pt}}
\put(289.67,646){\rule{0.400pt}{4.336pt}}
\multiput(289.17,655.00)(1.000,-9.000){2}{\rule{0.400pt}{2.168pt}}
\put(290.0,658.0){\rule[-0.200pt]{0.400pt}{1.445pt}}
\put(291.0,641.0){\rule[-0.200pt]{0.400pt}{1.204pt}}
\put(290.67,640){\rule{0.400pt}{4.336pt}}
\multiput(290.17,649.00)(1.000,-9.000){2}{\rule{0.400pt}{2.168pt}}
\put(291.0,641.0){\rule[-0.200pt]{0.400pt}{4.095pt}}
\put(292.0,640.0){\rule[-0.200pt]{0.400pt}{2.891pt}}
\put(291.67,624){\rule{0.400pt}{4.336pt}}
\multiput(291.17,624.00)(1.000,9.000){2}{\rule{0.400pt}{2.168pt}}
\put(292.0,624.0){\rule[-0.200pt]{0.400pt}{6.745pt}}
\put(293.0,624.0){\rule[-0.200pt]{0.400pt}{4.336pt}}
\put(292.67,641){\rule{0.400pt}{1.927pt}}
\multiput(292.17,641.00)(1.000,4.000){2}{\rule{0.400pt}{0.964pt}}
\put(293.0,624.0){\rule[-0.200pt]{0.400pt}{4.095pt}}
\put(293.67,611){\rule{0.400pt}{3.373pt}}
\multiput(293.17,618.00)(1.000,-7.000){2}{\rule{0.400pt}{1.686pt}}
\put(294.0,625.0){\rule[-0.200pt]{0.400pt}{5.782pt}}
\put(295.0,611.0){\rule[-0.200pt]{0.400pt}{7.468pt}}
\put(294.67,631){\rule{0.400pt}{0.723pt}}
\multiput(294.17,632.50)(1.000,-1.500){2}{\rule{0.400pt}{0.361pt}}
\put(295.0,634.0){\rule[-0.200pt]{0.400pt}{1.927pt}}
\put(295.67,606){\rule{0.400pt}{0.723pt}}
\multiput(295.17,606.00)(1.000,1.500){2}{\rule{0.400pt}{0.361pt}}
\put(296.0,606.0){\rule[-0.200pt]{0.400pt}{6.022pt}}
\put(297.0,608.0){\usebox{\plotpoint}}
\put(296.67,616){\rule{0.400pt}{1.686pt}}
\multiput(296.17,619.50)(1.000,-3.500){2}{\rule{0.400pt}{0.843pt}}
\put(297.0,608.0){\rule[-0.200pt]{0.400pt}{3.613pt}}
\put(297.67,595){\rule{0.400pt}{3.614pt}}
\multiput(297.17,602.50)(1.000,-7.500){2}{\rule{0.400pt}{1.807pt}}
\put(298.0,610.0){\rule[-0.200pt]{0.400pt}{1.445pt}}
\put(299.0,595.0){\rule[-0.200pt]{0.400pt}{1.445pt}}
\put(298.67,593){\rule{0.400pt}{2.409pt}}
\multiput(298.17,593.00)(1.000,5.000){2}{\rule{0.400pt}{1.204pt}}
\put(299.0,593.0){\rule[-0.200pt]{0.400pt}{1.927pt}}
\put(299.67,607){\rule{0.400pt}{1.204pt}}
\multiput(299.17,607.00)(1.000,2.500){2}{\rule{0.400pt}{0.602pt}}
\put(300.0,603.0){\rule[-0.200pt]{0.400pt}{0.964pt}}
\put(301.0,583.0){\rule[-0.200pt]{0.400pt}{6.986pt}}
\put(300.67,578){\rule{0.400pt}{5.541pt}}
\multiput(300.17,589.50)(1.000,-11.500){2}{\rule{0.400pt}{2.770pt}}
\put(301.0,583.0){\rule[-0.200pt]{0.400pt}{4.336pt}}
\put(301.67,582){\rule{0.400pt}{3.854pt}}
\multiput(301.17,582.00)(1.000,8.000){2}{\rule{0.400pt}{1.927pt}}
\put(302.0,578.0){\rule[-0.200pt]{0.400pt}{0.964pt}}
\put(302.67,585){\rule{0.400pt}{4.095pt}}
\multiput(302.17,585.00)(1.000,8.500){2}{\rule{0.400pt}{2.048pt}}
\put(303.0,585.0){\rule[-0.200pt]{0.400pt}{3.132pt}}
\put(304.0,572.0){\rule[-0.200pt]{0.400pt}{7.227pt}}
\put(303.67,583){\rule{0.400pt}{3.132pt}}
\multiput(303.17,583.00)(1.000,6.500){2}{\rule{0.400pt}{1.566pt}}
\put(304.0,572.0){\rule[-0.200pt]{0.400pt}{2.650pt}}
\put(305.0,570.0){\rule[-0.200pt]{0.400pt}{6.263pt}}
\put(304.67,573){\rule{0.400pt}{1.445pt}}
\multiput(304.17,576.00)(1.000,-3.000){2}{\rule{0.400pt}{0.723pt}}
\put(305.0,570.0){\rule[-0.200pt]{0.400pt}{2.168pt}}
\put(306.0,569.0){\rule[-0.200pt]{0.400pt}{0.964pt}}
\put(305.67,564){\rule{0.400pt}{2.409pt}}
\multiput(305.17,569.00)(1.000,-5.000){2}{\rule{0.400pt}{1.204pt}}
\put(306.0,569.0){\rule[-0.200pt]{0.400pt}{1.204pt}}
\put(306.67,563){\rule{0.400pt}{3.614pt}}
\multiput(306.17,563.00)(1.000,7.500){2}{\rule{0.400pt}{1.807pt}}
\put(307.0,563.0){\usebox{\plotpoint}}
\put(308.0,575.0){\rule[-0.200pt]{0.400pt}{0.723pt}}
\put(307.67,566){\rule{0.400pt}{3.614pt}}
\multiput(307.17,573.50)(1.000,-7.500){2}{\rule{0.400pt}{1.807pt}}
\put(308.0,575.0){\rule[-0.200pt]{0.400pt}{1.445pt}}
\put(308.67,559){\rule{0.400pt}{0.482pt}}
\multiput(308.17,560.00)(1.000,-1.000){2}{\rule{0.400pt}{0.241pt}}
\put(309.0,561.0){\rule[-0.200pt]{0.400pt}{1.204pt}}
\put(310.0,557.0){\rule[-0.200pt]{0.400pt}{0.482pt}}
\put(309.67,544){\rule{0.400pt}{6.745pt}}
\multiput(309.17,558.00)(1.000,-14.000){2}{\rule{0.400pt}{3.373pt}}
\put(310.0,557.0){\rule[-0.200pt]{0.400pt}{3.613pt}}
\put(310.67,538){\rule{0.400pt}{7.468pt}}
\multiput(310.17,553.50)(1.000,-15.500){2}{\rule{0.400pt}{3.734pt}}
\put(311.0,544.0){\rule[-0.200pt]{0.400pt}{6.022pt}}
\put(312.0,538.0){\rule[-0.200pt]{0.400pt}{3.854pt}}
\put(311.67,543){\rule{0.400pt}{1.445pt}}
\multiput(311.17,546.00)(1.000,-3.000){2}{\rule{0.400pt}{0.723pt}}
\put(312.0,549.0){\rule[-0.200pt]{0.400pt}{1.204pt}}
\put(313.0,530.0){\rule[-0.200pt]{0.400pt}{3.132pt}}
\put(313,550.67){\rule{0.241pt}{0.400pt}}
\multiput(313.00,550.17)(0.500,1.000){2}{\rule{0.120pt}{0.400pt}}
\put(313.0,530.0){\rule[-0.200pt]{0.400pt}{5.059pt}}
\put(314.0,527.0){\rule[-0.200pt]{0.400pt}{6.022pt}}
\put(313.67,546){\rule{0.400pt}{1.204pt}}
\multiput(313.17,548.50)(1.000,-2.500){2}{\rule{0.400pt}{0.602pt}}
\put(314.0,527.0){\rule[-0.200pt]{0.400pt}{5.782pt}}
\put(314.67,526){\rule{0.400pt}{0.482pt}}
\multiput(314.17,527.00)(1.000,-1.000){2}{\rule{0.400pt}{0.241pt}}
\put(315.0,528.0){\rule[-0.200pt]{0.400pt}{4.336pt}}
\put(316.0,523.0){\rule[-0.200pt]{0.400pt}{0.723pt}}
\put(315.67,514){\rule{0.400pt}{3.854pt}}
\multiput(315.17,522.00)(1.000,-8.000){2}{\rule{0.400pt}{1.927pt}}
\put(316.0,523.0){\rule[-0.200pt]{0.400pt}{1.686pt}}
\put(316.67,535){\rule{0.400pt}{0.723pt}}
\multiput(316.17,535.00)(1.000,1.500){2}{\rule{0.400pt}{0.361pt}}
\put(317.0,514.0){\rule[-0.200pt]{0.400pt}{5.059pt}}
\put(317.67,507){\rule{0.400pt}{4.095pt}}
\multiput(317.17,507.00)(1.000,8.500){2}{\rule{0.400pt}{2.048pt}}
\put(318.0,507.0){\rule[-0.200pt]{0.400pt}{7.468pt}}
\put(318.67,507){\rule{0.400pt}{0.723pt}}
\multiput(318.17,507.00)(1.000,1.500){2}{\rule{0.400pt}{0.361pt}}
\put(319.0,507.0){\rule[-0.200pt]{0.400pt}{4.095pt}}
\put(319.67,523){\rule{0.400pt}{0.723pt}}
\multiput(319.17,524.50)(1.000,-1.500){2}{\rule{0.400pt}{0.361pt}}
\put(320.0,510.0){\rule[-0.200pt]{0.400pt}{3.854pt}}
\put(321.0,493.0){\rule[-0.200pt]{0.400pt}{7.227pt}}
\put(320.67,500){\rule{0.400pt}{3.373pt}}
\multiput(320.17,507.00)(1.000,-7.000){2}{\rule{0.400pt}{1.686pt}}
\put(321.0,493.0){\rule[-0.200pt]{0.400pt}{5.059pt}}
\put(322,485.67){\rule{0.241pt}{0.400pt}}
\multiput(322.00,486.17)(0.500,-1.000){2}{\rule{0.120pt}{0.400pt}}
\put(322.0,487.0){\rule[-0.200pt]{0.400pt}{3.132pt}}
\put(323.0,486.0){\rule[-0.200pt]{0.400pt}{2.650pt}}
\put(323.0,497.0){\usebox{\plotpoint}}
\put(324.0,483.0){\rule[-0.200pt]{0.400pt}{3.373pt}}
\put(323.67,479){\rule{0.400pt}{2.409pt}}
\multiput(323.17,484.00)(1.000,-5.000){2}{\rule{0.400pt}{1.204pt}}
\put(324.0,483.0){\rule[-0.200pt]{0.400pt}{1.445pt}}
\put(325.0,479.0){\rule[-0.200pt]{0.400pt}{2.409pt}}
\put(324.67,476){\rule{0.400pt}{0.482pt}}
\multiput(324.17,477.00)(1.000,-1.000){2}{\rule{0.400pt}{0.241pt}}
\put(325.0,478.0){\rule[-0.200pt]{0.400pt}{2.650pt}}
\put(326.0,476.0){\rule[-0.200pt]{0.400pt}{2.168pt}}
\put(325.67,473){\rule{0.400pt}{1.445pt}}
\multiput(325.17,476.00)(1.000,-3.000){2}{\rule{0.400pt}{0.723pt}}
\put(326.0,479.0){\rule[-0.200pt]{0.400pt}{1.445pt}}
\put(326.67,464){\rule{0.400pt}{3.373pt}}
\multiput(326.17,471.00)(1.000,-7.000){2}{\rule{0.400pt}{1.686pt}}
\put(327.0,473.0){\rule[-0.200pt]{0.400pt}{1.204pt}}
\put(327.67,474){\rule{0.400pt}{3.614pt}}
\multiput(327.17,474.00)(1.000,7.500){2}{\rule{0.400pt}{1.807pt}}
\put(328.0,464.0){\rule[-0.200pt]{0.400pt}{2.409pt}}
\put(329.0,471.0){\rule[-0.200pt]{0.400pt}{4.336pt}}
\put(328.67,457){\rule{0.400pt}{6.023pt}}
\multiput(328.17,469.50)(1.000,-12.500){2}{\rule{0.400pt}{3.011pt}}
\put(329.0,471.0){\rule[-0.200pt]{0.400pt}{2.650pt}}
\put(330.0,457.0){\rule[-0.200pt]{0.400pt}{1.927pt}}
\put(329.67,449){\rule{0.400pt}{4.336pt}}
\multiput(329.17,449.00)(1.000,9.000){2}{\rule{0.400pt}{2.168pt}}
\put(330.0,449.0){\rule[-0.200pt]{0.400pt}{3.854pt}}
\put(331.0,454.0){\rule[-0.200pt]{0.400pt}{3.132pt}}
\put(330.67,474){\rule{0.400pt}{0.723pt}}
\multiput(330.17,475.50)(1.000,-1.500){2}{\rule{0.400pt}{0.361pt}}
\put(331.0,454.0){\rule[-0.200pt]{0.400pt}{5.541pt}}
\put(332,451.67){\rule{0.241pt}{0.400pt}}
\multiput(332.00,451.17)(0.500,1.000){2}{\rule{0.120pt}{0.400pt}}
\put(332.0,452.0){\rule[-0.200pt]{0.400pt}{5.300pt}}
\put(333.0,448.0){\rule[-0.200pt]{0.400pt}{1.204pt}}
\put(332.67,449){\rule{0.400pt}{3.614pt}}
\multiput(332.17,449.00)(1.000,7.500){2}{\rule{0.400pt}{1.807pt}}
\put(333.0,448.0){\usebox{\plotpoint}}
\put(334.0,464.0){\usebox{\plotpoint}}
\put(333.67,429){\rule{0.400pt}{8.432pt}}
\multiput(333.17,446.50)(1.000,-17.500){2}{\rule{0.400pt}{4.216pt}}
\put(334.0,464.0){\usebox{\plotpoint}}
\put(335.0,429.0){\rule[-0.200pt]{0.400pt}{5.059pt}}
\put(334.67,424){\rule{0.400pt}{4.336pt}}
\multiput(334.17,433.00)(1.000,-9.000){2}{\rule{0.400pt}{2.168pt}}
\put(335.0,442.0){\rule[-0.200pt]{0.400pt}{1.927pt}}
\put(335.67,437){\rule{0.400pt}{2.891pt}}
\multiput(335.17,443.00)(1.000,-6.000){2}{\rule{0.400pt}{1.445pt}}
\put(336.0,424.0){\rule[-0.200pt]{0.400pt}{6.022pt}}
\put(336.67,424){\rule{0.400pt}{6.263pt}}
\multiput(336.17,424.00)(1.000,13.000){2}{\rule{0.400pt}{3.132pt}}
\put(337.0,424.0){\rule[-0.200pt]{0.400pt}{3.132pt}}
\put(337.67,421){\rule{0.400pt}{4.818pt}}
\multiput(337.17,421.00)(1.000,10.000){2}{\rule{0.400pt}{2.409pt}}
\put(338.0,421.0){\rule[-0.200pt]{0.400pt}{6.986pt}}
\put(338.67,414){\rule{0.400pt}{0.964pt}}
\multiput(338.17,414.00)(1.000,2.000){2}{\rule{0.400pt}{0.482pt}}
\put(339.0,414.0){\rule[-0.200pt]{0.400pt}{6.504pt}}
\put(340,418){\usebox{\plotpoint}}
\put(339.67,405){\rule{0.400pt}{3.132pt}}
\multiput(339.17,411.50)(1.000,-6.500){2}{\rule{0.400pt}{1.566pt}}
\put(340.67,400){\rule{0.400pt}{6.023pt}}
\multiput(340.17,400.00)(1.000,12.500){2}{\rule{0.400pt}{3.011pt}}
\put(341.0,400.0){\rule[-0.200pt]{0.400pt}{1.204pt}}
\put(341.67,395){\rule{0.400pt}{7.709pt}}
\multiput(341.17,395.00)(1.000,16.000){2}{\rule{0.400pt}{3.854pt}}
\put(342.0,395.0){\rule[-0.200pt]{0.400pt}{7.227pt}}
\put(342.67,398){\rule{0.400pt}{5.059pt}}
\multiput(342.17,398.00)(1.000,10.500){2}{\rule{0.400pt}{2.529pt}}
\put(343.0,398.0){\rule[-0.200pt]{0.400pt}{6.986pt}}
\put(343.67,407){\rule{0.400pt}{1.204pt}}
\multiput(343.17,407.00)(1.000,2.500){2}{\rule{0.400pt}{0.602pt}}
\put(344.0,407.0){\rule[-0.200pt]{0.400pt}{2.891pt}}
\put(345.0,404.0){\rule[-0.200pt]{0.400pt}{1.927pt}}
\put(344.67,399){\rule{0.400pt}{1.445pt}}
\multiput(344.17,402.00)(1.000,-3.000){2}{\rule{0.400pt}{0.723pt}}
\put(345.0,404.0){\usebox{\plotpoint}}
\put(346.0,399.0){\rule[-0.200pt]{0.400pt}{2.891pt}}
\put(345.67,383){\rule{0.400pt}{3.132pt}}
\multiput(345.17,389.50)(1.000,-6.500){2}{\rule{0.400pt}{1.566pt}}
\put(346.0,396.0){\rule[-0.200pt]{0.400pt}{3.613pt}}
\put(347.0,376.0){\rule[-0.200pt]{0.400pt}{1.686pt}}
\put(346.67,378){\rule{0.400pt}{3.132pt}}
\multiput(346.17,384.50)(1.000,-6.500){2}{\rule{0.400pt}{1.566pt}}
\put(347.0,376.0){\rule[-0.200pt]{0.400pt}{3.613pt}}
\put(347.67,370){\rule{0.400pt}{7.468pt}}
\multiput(347.17,370.00)(1.000,15.500){2}{\rule{0.400pt}{3.734pt}}
\put(348.0,370.0){\rule[-0.200pt]{0.400pt}{1.927pt}}
\put(348.67,366){\rule{0.400pt}{7.227pt}}
\multiput(348.17,366.00)(1.000,15.000){2}{\rule{0.400pt}{3.613pt}}
\put(349.0,366.0){\rule[-0.200pt]{0.400pt}{8.431pt}}
\put(349.67,366){\rule{0.400pt}{1.927pt}}
\multiput(349.17,366.00)(1.000,4.000){2}{\rule{0.400pt}{0.964pt}}
\put(350.0,366.0){\rule[-0.200pt]{0.400pt}{7.227pt}}
\put(351.0,366.0){\rule[-0.200pt]{0.400pt}{1.927pt}}
\put(350.67,353){\rule{0.400pt}{3.614pt}}
\multiput(350.17,360.50)(1.000,-7.500){2}{\rule{0.400pt}{1.807pt}}
\put(351.0,366.0){\rule[-0.200pt]{0.400pt}{0.482pt}}
\put(352,353){\usebox{\plotpoint}}
\put(351.67,353){\rule{0.400pt}{5.059pt}}
\multiput(351.17,353.00)(1.000,10.500){2}{\rule{0.400pt}{2.529pt}}
\put(353.0,374.0){\rule[-0.200pt]{0.400pt}{2.409pt}}
\put(352.67,356){\rule{0.400pt}{6.504pt}}
\multiput(352.17,356.00)(1.000,13.500){2}{\rule{0.400pt}{3.252pt}}
\put(353.0,356.0){\rule[-0.200pt]{0.400pt}{6.745pt}}
\put(353.67,356){\rule{0.400pt}{3.132pt}}
\multiput(353.17,356.00)(1.000,6.500){2}{\rule{0.400pt}{1.566pt}}
\put(354.0,356.0){\rule[-0.200pt]{0.400pt}{6.504pt}}
\put(355.0,356.0){\rule[-0.200pt]{0.400pt}{3.132pt}}
\put(354.67,365){\rule{0.400pt}{3.614pt}}
\multiput(354.17,365.00)(1.000,7.500){2}{\rule{0.400pt}{1.807pt}}
\put(355.0,356.0){\rule[-0.200pt]{0.400pt}{2.168pt}}
\put(356.0,351.0){\rule[-0.200pt]{0.400pt}{6.986pt}}
\put(355.67,354){\rule{0.400pt}{2.168pt}}
\multiput(355.17,354.00)(1.000,4.500){2}{\rule{0.400pt}{1.084pt}}
\put(356.0,351.0){\rule[-0.200pt]{0.400pt}{0.723pt}}
\put(356.67,377){\rule{0.400pt}{1.686pt}}
\multiput(356.17,380.50)(1.000,-3.500){2}{\rule{0.400pt}{0.843pt}}
\put(357.0,363.0){\rule[-0.200pt]{0.400pt}{5.059pt}}
\put(358.0,369.0){\rule[-0.200pt]{0.400pt}{1.927pt}}
\put(357.67,357){\rule{0.400pt}{3.132pt}}
\multiput(357.17,363.50)(1.000,-6.500){2}{\rule{0.400pt}{1.566pt}}
\put(358.0,369.0){\usebox{\plotpoint}}
\put(359.0,357.0){\rule[-0.200pt]{0.400pt}{1.686pt}}
\put(358.67,350){\rule{0.400pt}{4.577pt}}
\multiput(358.17,350.00)(1.000,9.500){2}{\rule{0.400pt}{2.289pt}}
\put(359.0,350.0){\rule[-0.200pt]{0.400pt}{3.373pt}}
\put(360.0,354.0){\rule[-0.200pt]{0.400pt}{3.613pt}}
\put(359.67,365){\rule{0.400pt}{0.482pt}}
\multiput(359.17,366.00)(1.000,-1.000){2}{\rule{0.400pt}{0.241pt}}
\put(360.0,354.0){\rule[-0.200pt]{0.400pt}{3.132pt}}
\put(360.67,368){\rule{0.400pt}{1.927pt}}
\multiput(360.17,368.00)(1.000,4.000){2}{\rule{0.400pt}{0.964pt}}
\put(361.0,365.0){\rule[-0.200pt]{0.400pt}{0.723pt}}
\put(362.0,376.0){\rule[-0.200pt]{0.400pt}{0.964pt}}
\put(361.67,353){\rule{0.400pt}{4.818pt}}
\multiput(361.17,363.00)(1.000,-10.000){2}{\rule{0.400pt}{2.409pt}}
\put(362.0,373.0){\rule[-0.200pt]{0.400pt}{1.686pt}}
\put(363.0,353.0){\rule[-0.200pt]{0.400pt}{7.227pt}}
\put(362.67,367){\rule{0.400pt}{3.132pt}}
\multiput(362.17,373.50)(1.000,-6.500){2}{\rule{0.400pt}{1.566pt}}
\put(363.0,380.0){\rule[-0.200pt]{0.400pt}{0.723pt}}
\put(364.0,367.0){\rule[-0.200pt]{0.400pt}{1.445pt}}
\put(363.67,351){\rule{0.400pt}{1.927pt}}
\multiput(363.17,351.00)(1.000,4.000){2}{\rule{0.400pt}{0.964pt}}
\put(364.0,351.0){\rule[-0.200pt]{0.400pt}{5.300pt}}
\put(364.67,353){\rule{0.400pt}{6.745pt}}
\multiput(364.17,367.00)(1.000,-14.000){2}{\rule{0.400pt}{3.373pt}}
\put(365.0,359.0){\rule[-0.200pt]{0.400pt}{5.300pt}}
\put(366,363.67){\rule{0.241pt}{0.400pt}}
\multiput(366.00,363.17)(0.500,1.000){2}{\rule{0.120pt}{0.400pt}}
\put(366.0,353.0){\rule[-0.200pt]{0.400pt}{2.650pt}}
\put(367.0,365.0){\rule[-0.200pt]{0.400pt}{3.854pt}}
\put(366.67,356){\rule{0.400pt}{0.723pt}}
\multiput(366.17,356.00)(1.000,1.500){2}{\rule{0.400pt}{0.361pt}}
\put(367.0,356.0){\rule[-0.200pt]{0.400pt}{6.022pt}}
\put(368.0,359.0){\rule[-0.200pt]{0.400pt}{5.782pt}}
\put(367.67,365){\rule{0.400pt}{1.927pt}}
\multiput(367.17,369.00)(1.000,-4.000){2}{\rule{0.400pt}{0.964pt}}
\put(368.0,373.0){\rule[-0.200pt]{0.400pt}{2.409pt}}
\put(368.67,362){\rule{0.400pt}{3.132pt}}
\multiput(368.17,362.00)(1.000,6.500){2}{\rule{0.400pt}{1.566pt}}
\put(369.0,362.0){\rule[-0.200pt]{0.400pt}{0.723pt}}
\put(370.0,366.0){\rule[-0.200pt]{0.400pt}{2.168pt}}
\put(370,373.67){\rule{0.241pt}{0.400pt}}
\multiput(370.00,374.17)(0.500,-1.000){2}{\rule{0.120pt}{0.400pt}}
\put(370.0,366.0){\rule[-0.200pt]{0.400pt}{2.168pt}}
\put(371,374){\usebox{\plotpoint}}
\put(370.67,367){\rule{0.400pt}{0.964pt}}
\multiput(370.17,369.00)(1.000,-2.000){2}{\rule{0.400pt}{0.482pt}}
\put(371.0,371.0){\rule[-0.200pt]{0.400pt}{0.723pt}}
\put(372.0,361.0){\rule[-0.200pt]{0.400pt}{1.445pt}}
\put(371.67,376){\rule{0.400pt}{0.723pt}}
\multiput(371.17,377.50)(1.000,-1.500){2}{\rule{0.400pt}{0.361pt}}
\put(372.0,361.0){\rule[-0.200pt]{0.400pt}{4.336pt}}
\put(372.67,353){\rule{0.400pt}{3.854pt}}
\multiput(372.17,353.00)(1.000,8.000){2}{\rule{0.400pt}{1.927pt}}
\put(373.0,353.0){\rule[-0.200pt]{0.400pt}{5.541pt}}
\put(374.0,367.0){\rule[-0.200pt]{0.400pt}{0.482pt}}
\put(373.67,368){\rule{0.400pt}{1.445pt}}
\multiput(373.17,371.00)(1.000,-3.000){2}{\rule{0.400pt}{0.723pt}}
\put(374.0,367.0){\rule[-0.200pt]{0.400pt}{1.686pt}}
\put(375.0,368.0){\rule[-0.200pt]{0.400pt}{0.482pt}}
\put(374.67,369){\rule{0.400pt}{0.964pt}}
\multiput(374.17,369.00)(1.000,2.000){2}{\rule{0.400pt}{0.482pt}}
\put(375.0,369.0){\usebox{\plotpoint}}
\put(376.0,356.0){\rule[-0.200pt]{0.400pt}{4.095pt}}
\put(375.67,359){\rule{0.400pt}{1.204pt}}
\multiput(375.17,359.00)(1.000,2.500){2}{\rule{0.400pt}{0.602pt}}
\put(376.0,356.0){\rule[-0.200pt]{0.400pt}{0.723pt}}
\put(376.67,374){\rule{0.400pt}{1.686pt}}
\multiput(376.17,377.50)(1.000,-3.500){2}{\rule{0.400pt}{0.843pt}}
\put(377.0,364.0){\rule[-0.200pt]{0.400pt}{4.095pt}}
\put(377.67,350){\rule{0.400pt}{1.686pt}}
\multiput(377.17,350.00)(1.000,3.500){2}{\rule{0.400pt}{0.843pt}}
\put(378.0,350.0){\rule[-0.200pt]{0.400pt}{5.782pt}}
\put(378.67,363){\rule{0.400pt}{2.891pt}}
\multiput(378.17,369.00)(1.000,-6.000){2}{\rule{0.400pt}{1.445pt}}
\put(379.0,357.0){\rule[-0.200pt]{0.400pt}{4.336pt}}
\put(380.0,360.0){\rule[-0.200pt]{0.400pt}{0.723pt}}
\put(379.67,363){\rule{0.400pt}{3.614pt}}
\multiput(379.17,363.00)(1.000,7.500){2}{\rule{0.400pt}{1.807pt}}
\put(380.0,360.0){\rule[-0.200pt]{0.400pt}{0.723pt}}
\put(380.67,364){\rule{0.400pt}{1.204pt}}
\multiput(380.17,366.50)(1.000,-2.500){2}{\rule{0.400pt}{0.602pt}}
\put(381.0,369.0){\rule[-0.200pt]{0.400pt}{2.168pt}}
\put(381.67,370){\rule{0.400pt}{1.927pt}}
\multiput(381.17,370.00)(1.000,4.000){2}{\rule{0.400pt}{0.964pt}}
\put(382.0,364.0){\rule[-0.200pt]{0.400pt}{1.445pt}}
\put(383.0,361.0){\rule[-0.200pt]{0.400pt}{4.095pt}}
\put(382.67,374){\rule{0.400pt}{1.204pt}}
\multiput(382.17,374.00)(1.000,2.500){2}{\rule{0.400pt}{0.602pt}}
\put(383.0,361.0){\rule[-0.200pt]{0.400pt}{3.132pt}}
\put(384.0,349.0){\rule[-0.200pt]{0.400pt}{7.227pt}}
\put(383.67,349){\rule{0.400pt}{7.468pt}}
\multiput(383.17,364.50)(1.000,-15.500){2}{\rule{0.400pt}{3.734pt}}
\put(384.0,349.0){\rule[-0.200pt]{0.400pt}{7.468pt}}
\put(385.0,349.0){\rule[-0.200pt]{0.400pt}{5.782pt}}
\put(384.67,354){\rule{0.400pt}{5.059pt}}
\multiput(384.17,354.00)(1.000,10.500){2}{\rule{0.400pt}{2.529pt}}
\put(385.0,354.0){\rule[-0.200pt]{0.400pt}{4.577pt}}
\put(385.67,358){\rule{0.400pt}{4.577pt}}
\multiput(385.17,358.00)(1.000,9.500){2}{\rule{0.400pt}{2.289pt}}
\put(386.0,358.0){\rule[-0.200pt]{0.400pt}{4.095pt}}
\put(386.67,357){\rule{0.400pt}{1.445pt}}
\multiput(386.17,360.00)(1.000,-3.000){2}{\rule{0.400pt}{0.723pt}}
\put(387.0,363.0){\rule[-0.200pt]{0.400pt}{3.373pt}}
\put(387.67,362){\rule{0.400pt}{0.964pt}}
\multiput(387.17,362.00)(1.000,2.000){2}{\rule{0.400pt}{0.482pt}}
\put(388.0,357.0){\rule[-0.200pt]{0.400pt}{1.204pt}}
\put(389.0,358.0){\rule[-0.200pt]{0.400pt}{1.927pt}}
\put(388.67,376){\rule{0.400pt}{0.482pt}}
\multiput(388.17,376.00)(1.000,1.000){2}{\rule{0.400pt}{0.241pt}}
\put(389.0,358.0){\rule[-0.200pt]{0.400pt}{4.336pt}}
\put(389.67,365){\rule{0.400pt}{3.373pt}}
\multiput(389.17,365.00)(1.000,7.000){2}{\rule{0.400pt}{1.686pt}}
\put(390.0,365.0){\rule[-0.200pt]{0.400pt}{3.132pt}}
\put(391.0,351.0){\rule[-0.200pt]{0.400pt}{6.745pt}}
\put(390.67,365){\rule{0.400pt}{1.204pt}}
\multiput(390.17,365.00)(1.000,2.500){2}{\rule{0.400pt}{0.602pt}}
\put(391.0,351.0){\rule[-0.200pt]{0.400pt}{3.373pt}}
\put(391.67,350){\rule{0.400pt}{5.541pt}}
\multiput(391.17,361.50)(1.000,-11.500){2}{\rule{0.400pt}{2.770pt}}
\put(392.0,370.0){\rule[-0.200pt]{0.400pt}{0.723pt}}
\put(393.0,350.0){\rule[-0.200pt]{0.400pt}{4.818pt}}
\put(392.67,353){\rule{0.400pt}{2.891pt}}
\multiput(392.17,353.00)(1.000,6.000){2}{\rule{0.400pt}{1.445pt}}
\put(393.0,353.0){\rule[-0.200pt]{0.400pt}{4.095pt}}
\put(393.67,368){\rule{0.400pt}{1.686pt}}
\multiput(393.17,368.00)(1.000,3.500){2}{\rule{0.400pt}{0.843pt}}
\put(394.0,365.0){\rule[-0.200pt]{0.400pt}{0.723pt}}
\put(395.0,375.0){\rule[-0.200pt]{0.400pt}{0.723pt}}
\put(394.67,362){\rule{0.400pt}{1.686pt}}
\multiput(394.17,365.50)(1.000,-3.500){2}{\rule{0.400pt}{0.843pt}}
\put(395.0,369.0){\rule[-0.200pt]{0.400pt}{2.168pt}}
\put(396.0,356.0){\rule[-0.200pt]{0.400pt}{1.445pt}}
\put(396,370.67){\rule{0.241pt}{0.400pt}}
\multiput(396.00,371.17)(0.500,-1.000){2}{\rule{0.120pt}{0.400pt}}
\put(396.0,356.0){\rule[-0.200pt]{0.400pt}{3.854pt}}
\put(397.0,349.0){\rule[-0.200pt]{0.400pt}{5.300pt}}
\put(396.67,374){\rule{0.400pt}{0.723pt}}
\multiput(396.17,374.00)(1.000,1.500){2}{\rule{0.400pt}{0.361pt}}
\put(397.0,349.0){\rule[-0.200pt]{0.400pt}{6.022pt}}
\put(397.67,350){\rule{0.400pt}{5.300pt}}
\multiput(397.17,361.00)(1.000,-11.000){2}{\rule{0.400pt}{2.650pt}}
\put(398.0,372.0){\rule[-0.200pt]{0.400pt}{1.204pt}}
\put(398.67,348){\rule{0.400pt}{5.059pt}}
\multiput(398.17,358.50)(1.000,-10.500){2}{\rule{0.400pt}{2.529pt}}
\put(399.0,350.0){\rule[-0.200pt]{0.400pt}{4.577pt}}
\put(399.67,359){\rule{0.400pt}{3.132pt}}
\multiput(399.17,359.00)(1.000,6.500){2}{\rule{0.400pt}{1.566pt}}
\put(400.0,348.0){\rule[-0.200pt]{0.400pt}{2.650pt}}
\put(401.0,346.0){\rule[-0.200pt]{0.400pt}{6.263pt}}
\put(401.0,346.0){\rule[-0.200pt]{0.400pt}{7.468pt}}
\put(401.0,377.0){\usebox{\plotpoint}}
\put(402.0,346.0){\rule[-0.200pt]{0.400pt}{7.468pt}}
\put(402.0,346.0){\usebox{\plotpoint}}
\put(402.67,353){\rule{0.400pt}{2.650pt}}
\multiput(402.17,353.00)(1.000,5.500){2}{\rule{0.400pt}{1.325pt}}
\put(403.0,346.0){\rule[-0.200pt]{0.400pt}{1.686pt}}
\put(404,350.67){\rule{0.241pt}{0.400pt}}
\multiput(404.00,351.17)(0.500,-1.000){2}{\rule{0.120pt}{0.400pt}}
\put(404.0,352.0){\rule[-0.200pt]{0.400pt}{2.891pt}}
\put(404.67,362){\rule{0.400pt}{1.927pt}}
\multiput(404.17,366.00)(1.000,-4.000){2}{\rule{0.400pt}{0.964pt}}
\put(405.0,351.0){\rule[-0.200pt]{0.400pt}{4.577pt}}
\put(406.0,360.0){\rule[-0.200pt]{0.400pt}{0.482pt}}
\put(405.67,361){\rule{0.400pt}{2.650pt}}
\multiput(405.17,366.50)(1.000,-5.500){2}{\rule{0.400pt}{1.325pt}}
\put(406.0,360.0){\rule[-0.200pt]{0.400pt}{2.891pt}}
\put(406.67,366){\rule{0.400pt}{1.204pt}}
\multiput(406.17,366.00)(1.000,2.500){2}{\rule{0.400pt}{0.602pt}}
\put(407.0,361.0){\rule[-0.200pt]{0.400pt}{1.204pt}}
\put(408.0,371.0){\rule[-0.200pt]{0.400pt}{1.445pt}}
\put(407.67,350){\rule{0.400pt}{2.891pt}}
\multiput(407.17,350.00)(1.000,6.000){2}{\rule{0.400pt}{1.445pt}}
\put(408.0,350.0){\rule[-0.200pt]{0.400pt}{6.504pt}}
\put(409,362){\usebox{\plotpoint}}
\put(408.67,365){\rule{0.400pt}{2.168pt}}
\multiput(408.17,369.50)(1.000,-4.500){2}{\rule{0.400pt}{1.084pt}}
\put(409.0,362.0){\rule[-0.200pt]{0.400pt}{2.891pt}}
\put(410.0,348.0){\rule[-0.200pt]{0.400pt}{4.095pt}}
\put(409.67,356){\rule{0.400pt}{1.927pt}}
\multiput(409.17,356.00)(1.000,4.000){2}{\rule{0.400pt}{0.964pt}}
\put(410.0,348.0){\rule[-0.200pt]{0.400pt}{1.927pt}}
\put(410.67,372){\rule{0.400pt}{1.204pt}}
\multiput(410.17,374.50)(1.000,-2.500){2}{\rule{0.400pt}{0.602pt}}
\put(411.0,364.0){\rule[-0.200pt]{0.400pt}{3.132pt}}
\put(412,349.67){\rule{0.241pt}{0.400pt}}
\multiput(412.00,349.17)(0.500,1.000){2}{\rule{0.120pt}{0.400pt}}
\put(412.0,350.0){\rule[-0.200pt]{0.400pt}{5.300pt}}
\put(413.0,345.0){\rule[-0.200pt]{0.400pt}{1.445pt}}
\put(412.67,352){\rule{0.400pt}{5.059pt}}
\multiput(412.17,352.00)(1.000,10.500){2}{\rule{0.400pt}{2.529pt}}
\put(413.0,345.0){\rule[-0.200pt]{0.400pt}{1.686pt}}
\put(413.67,347){\rule{0.400pt}{5.300pt}}
\multiput(413.17,347.00)(1.000,11.000){2}{\rule{0.400pt}{2.650pt}}
\put(414.0,347.0){\rule[-0.200pt]{0.400pt}{6.263pt}}
\put(414.67,371){\rule{0.400pt}{0.482pt}}
\multiput(414.17,372.00)(1.000,-1.000){2}{\rule{0.400pt}{0.241pt}}
\put(415.0,369.0){\rule[-0.200pt]{0.400pt}{0.964pt}}
\put(416.0,353.0){\rule[-0.200pt]{0.400pt}{4.336pt}}
\put(415.67,360){\rule{0.400pt}{0.723pt}}
\multiput(415.17,361.50)(1.000,-1.500){2}{\rule{0.400pt}{0.361pt}}
\put(416.0,353.0){\rule[-0.200pt]{0.400pt}{2.409pt}}
\put(416.67,346){\rule{0.400pt}{3.373pt}}
\multiput(416.17,346.00)(1.000,7.000){2}{\rule{0.400pt}{1.686pt}}
\put(417.0,346.0){\rule[-0.200pt]{0.400pt}{3.373pt}}
\put(418.0,348.0){\rule[-0.200pt]{0.400pt}{2.891pt}}
\put(418,369.67){\rule{0.241pt}{0.400pt}}
\multiput(418.00,370.17)(0.500,-1.000){2}{\rule{0.120pt}{0.400pt}}
\put(418.0,348.0){\rule[-0.200pt]{0.400pt}{5.541pt}}
\put(418.67,351){\rule{0.400pt}{1.927pt}}
\multiput(418.17,355.00)(1.000,-4.000){2}{\rule{0.400pt}{0.964pt}}
\put(419.0,359.0){\rule[-0.200pt]{0.400pt}{2.650pt}}
\put(419.67,348){\rule{0.400pt}{6.504pt}}
\multiput(419.17,361.50)(1.000,-13.500){2}{\rule{0.400pt}{3.252pt}}
\put(420.0,351.0){\rule[-0.200pt]{0.400pt}{5.782pt}}
\put(421.0,348.0){\rule[-0.200pt]{0.400pt}{6.745pt}}
\put(420.67,354){\rule{0.400pt}{1.445pt}}
\multiput(420.17,354.00)(1.000,3.000){2}{\rule{0.400pt}{0.723pt}}
\put(421.0,354.0){\rule[-0.200pt]{0.400pt}{5.300pt}}
\put(422.0,360.0){\rule[-0.200pt]{0.400pt}{0.482pt}}
\put(422,345.67){\rule{0.241pt}{0.400pt}}
\multiput(422.00,345.17)(0.500,1.000){2}{\rule{0.120pt}{0.400pt}}
\put(422.0,346.0){\rule[-0.200pt]{0.400pt}{3.854pt}}
\put(422.67,353){\rule{0.400pt}{1.927pt}}
\multiput(422.17,353.00)(1.000,4.000){2}{\rule{0.400pt}{0.964pt}}
\put(423.0,347.0){\rule[-0.200pt]{0.400pt}{1.445pt}}
\put(423.67,345){\rule{0.400pt}{5.782pt}}
\multiput(423.17,357.00)(1.000,-12.000){2}{\rule{0.400pt}{2.891pt}}
\put(424.0,361.0){\rule[-0.200pt]{0.400pt}{1.927pt}}
\put(425.0,345.0){\rule[-0.200pt]{0.400pt}{5.782pt}}
\put(425.0,369.0){\usebox{\plotpoint}}
\put(426.0,358.0){\rule[-0.200pt]{0.400pt}{2.650pt}}
\put(425.67,360){\rule{0.400pt}{3.132pt}}
\multiput(425.17,366.50)(1.000,-6.500){2}{\rule{0.400pt}{1.566pt}}
\put(426.0,358.0){\rule[-0.200pt]{0.400pt}{3.613pt}}
\put(427.0,357.0){\rule[-0.200pt]{0.400pt}{0.723pt}}
\put(426.67,350){\rule{0.400pt}{5.541pt}}
\multiput(426.17,361.50)(1.000,-11.500){2}{\rule{0.400pt}{2.770pt}}
\put(427.0,357.0){\rule[-0.200pt]{0.400pt}{3.854pt}}
\put(427.67,355){\rule{0.400pt}{3.132pt}}
\multiput(427.17,355.00)(1.000,6.500){2}{\rule{0.400pt}{1.566pt}}
\put(428.0,350.0){\rule[-0.200pt]{0.400pt}{1.204pt}}
\put(429.0,368.0){\rule[-0.200pt]{0.400pt}{1.686pt}}
\put(428.67,348){\rule{0.400pt}{1.686pt}}
\multiput(428.17,348.00)(1.000,3.500){2}{\rule{0.400pt}{0.843pt}}
\put(429.0,348.0){\rule[-0.200pt]{0.400pt}{6.504pt}}
\put(430.0,353.0){\rule[-0.200pt]{0.400pt}{0.482pt}}
\put(430,367.67){\rule{0.241pt}{0.400pt}}
\multiput(430.00,368.17)(0.500,-1.000){2}{\rule{0.120pt}{0.400pt}}
\put(430.0,353.0){\rule[-0.200pt]{0.400pt}{3.854pt}}
\put(431.0,340.0){\rule[-0.200pt]{0.400pt}{6.745pt}}
\put(430.67,360){\rule{0.400pt}{0.964pt}}
\multiput(430.17,362.00)(1.000,-2.000){2}{\rule{0.400pt}{0.482pt}}
\put(431.0,340.0){\rule[-0.200pt]{0.400pt}{5.782pt}}
\put(431.67,343){\rule{0.400pt}{3.373pt}}
\multiput(431.17,350.00)(1.000,-7.000){2}{\rule{0.400pt}{1.686pt}}
\put(432.0,357.0){\rule[-0.200pt]{0.400pt}{0.723pt}}
\put(432.67,351){\rule{0.400pt}{3.132pt}}
\multiput(432.17,357.50)(1.000,-6.500){2}{\rule{0.400pt}{1.566pt}}
\put(433.0,343.0){\rule[-0.200pt]{0.400pt}{5.059pt}}
\put(434.0,351.0){\rule[-0.200pt]{0.400pt}{1.927pt}}
\put(433.67,344){\rule{0.400pt}{1.445pt}}
\multiput(433.17,344.00)(1.000,3.000){2}{\rule{0.400pt}{0.723pt}}
\put(434.0,344.0){\rule[-0.200pt]{0.400pt}{3.613pt}}
\put(435.0,350.0){\rule[-0.200pt]{0.400pt}{0.723pt}}
\put(434.67,340){\rule{0.400pt}{4.336pt}}
\multiput(434.17,340.00)(1.000,9.000){2}{\rule{0.400pt}{2.168pt}}
\put(435.0,340.0){\rule[-0.200pt]{0.400pt}{3.132pt}}
\put(435.67,339){\rule{0.400pt}{4.577pt}}
\multiput(435.17,339.00)(1.000,9.500){2}{\rule{0.400pt}{2.289pt}}
\put(436.0,339.0){\rule[-0.200pt]{0.400pt}{4.577pt}}
\put(437.0,348.0){\rule[-0.200pt]{0.400pt}{2.409pt}}
\put(436.67,352){\rule{0.400pt}{0.482pt}}
\multiput(436.17,353.00)(1.000,-1.000){2}{\rule{0.400pt}{0.241pt}}
\put(437.0,348.0){\rule[-0.200pt]{0.400pt}{1.445pt}}
\put(438.0,341.0){\rule[-0.200pt]{0.400pt}{2.650pt}}
\put(437.67,370){\rule{0.400pt}{0.482pt}}
\multiput(437.17,371.00)(1.000,-1.000){2}{\rule{0.400pt}{0.241pt}}
\put(438.0,341.0){\rule[-0.200pt]{0.400pt}{7.468pt}}
\put(438.67,342){\rule{0.400pt}{5.541pt}}
\multiput(438.17,342.00)(1.000,11.500){2}{\rule{0.400pt}{2.770pt}}
\put(439.0,342.0){\rule[-0.200pt]{0.400pt}{6.745pt}}
\put(439.67,340){\rule{0.400pt}{2.409pt}}
\multiput(439.17,345.00)(1.000,-5.000){2}{\rule{0.400pt}{1.204pt}}
\put(440.0,350.0){\rule[-0.200pt]{0.400pt}{3.613pt}}
\put(440.67,359){\rule{0.400pt}{1.204pt}}
\multiput(440.17,361.50)(1.000,-2.500){2}{\rule{0.400pt}{0.602pt}}
\put(441.0,340.0){\rule[-0.200pt]{0.400pt}{5.782pt}}
\put(442.0,359.0){\rule[-0.200pt]{0.400pt}{2.650pt}}
\put(441.67,361){\rule{0.400pt}{0.482pt}}
\multiput(441.17,362.00)(1.000,-1.000){2}{\rule{0.400pt}{0.241pt}}
\put(442.0,363.0){\rule[-0.200pt]{0.400pt}{1.686pt}}
\put(443.0,349.0){\rule[-0.200pt]{0.400pt}{2.891pt}}
\put(443.0,349.0){\rule[-0.200pt]{0.400pt}{0.482pt}}
\put(443.0,351.0){\usebox{\plotpoint}}
\put(443.67,350){\rule{0.400pt}{3.373pt}}
\multiput(443.17,357.00)(1.000,-7.000){2}{\rule{0.400pt}{1.686pt}}
\put(444.0,351.0){\rule[-0.200pt]{0.400pt}{3.132pt}}
\put(445.0,350.0){\rule[-0.200pt]{0.400pt}{1.686pt}}
\put(444.67,337){\rule{0.400pt}{6.263pt}}
\multiput(444.17,337.00)(1.000,13.000){2}{\rule{0.400pt}{3.132pt}}
\put(445.0,337.0){\rule[-0.200pt]{0.400pt}{4.818pt}}
\put(446.0,340.0){\rule[-0.200pt]{0.400pt}{5.541pt}}
\put(445.67,343){\rule{0.400pt}{0.964pt}}
\multiput(445.17,343.00)(1.000,2.000){2}{\rule{0.400pt}{0.482pt}}
\put(446.0,340.0){\rule[-0.200pt]{0.400pt}{0.723pt}}
\put(447.0,347.0){\rule[-0.200pt]{0.400pt}{2.891pt}}
\put(446.67,346){\rule{0.400pt}{4.336pt}}
\multiput(446.17,346.00)(1.000,9.000){2}{\rule{0.400pt}{2.168pt}}
\put(447.0,346.0){\rule[-0.200pt]{0.400pt}{3.132pt}}
\put(447.67,340){\rule{0.400pt}{5.541pt}}
\multiput(447.17,351.50)(1.000,-11.500){2}{\rule{0.400pt}{2.770pt}}
\put(448.0,363.0){\usebox{\plotpoint}}
\put(448.67,353){\rule{0.400pt}{4.095pt}}
\multiput(448.17,361.50)(1.000,-8.500){2}{\rule{0.400pt}{2.048pt}}
\put(449.0,340.0){\rule[-0.200pt]{0.400pt}{7.227pt}}
\put(450.0,353.0){\rule[-0.200pt]{0.400pt}{2.168pt}}
\put(449.67,352){\rule{0.400pt}{0.964pt}}
\multiput(449.17,352.00)(1.000,2.000){2}{\rule{0.400pt}{0.482pt}}
\put(450.0,352.0){\rule[-0.200pt]{0.400pt}{2.409pt}}
\put(451.0,356.0){\rule[-0.200pt]{0.400pt}{1.927pt}}
\put(450.67,341){\rule{0.400pt}{2.409pt}}
\multiput(450.17,346.00)(1.000,-5.000){2}{\rule{0.400pt}{1.204pt}}
\put(451.0,351.0){\rule[-0.200pt]{0.400pt}{3.132pt}}
\put(451.67,358){\rule{0.400pt}{0.482pt}}
\multiput(451.17,358.00)(1.000,1.000){2}{\rule{0.400pt}{0.241pt}}
\put(452.0,341.0){\rule[-0.200pt]{0.400pt}{4.095pt}}
\put(452.67,346){\rule{0.400pt}{2.168pt}}
\multiput(452.17,346.00)(1.000,4.500){2}{\rule{0.400pt}{1.084pt}}
\put(453.0,346.0){\rule[-0.200pt]{0.400pt}{3.373pt}}
\put(454.0,340.0){\rule[-0.200pt]{0.400pt}{3.613pt}}
\put(454,354.67){\rule{0.241pt}{0.400pt}}
\multiput(454.00,354.17)(0.500,1.000){2}{\rule{0.120pt}{0.400pt}}
\put(454.0,340.0){\rule[-0.200pt]{0.400pt}{3.613pt}}
\put(455.0,356.0){\rule[-0.200pt]{0.400pt}{1.204pt}}
\put(454.67,352){\rule{0.400pt}{1.204pt}}
\multiput(454.17,354.50)(1.000,-2.500){2}{\rule{0.400pt}{0.602pt}}
\put(455.0,357.0){\rule[-0.200pt]{0.400pt}{0.964pt}}
\put(456.0,350.0){\rule[-0.200pt]{0.400pt}{0.482pt}}
\put(455.67,347){\rule{0.400pt}{1.686pt}}
\multiput(455.17,350.50)(1.000,-3.500){2}{\rule{0.400pt}{0.843pt}}
\put(456.0,350.0){\rule[-0.200pt]{0.400pt}{0.964pt}}
\put(456.67,357){\rule{0.400pt}{1.927pt}}
\multiput(456.17,361.00)(1.000,-4.000){2}{\rule{0.400pt}{0.964pt}}
\put(457.0,347.0){\rule[-0.200pt]{0.400pt}{4.336pt}}
\put(458.0,357.0){\rule[-0.200pt]{0.400pt}{2.891pt}}
\put(457.67,357){\rule{0.400pt}{0.482pt}}
\multiput(457.17,358.00)(1.000,-1.000){2}{\rule{0.400pt}{0.241pt}}
\put(458.0,359.0){\rule[-0.200pt]{0.400pt}{2.409pt}}
\put(459.0,357.0){\rule[-0.200pt]{0.400pt}{3.132pt}}
\put(458.67,353){\rule{0.400pt}{3.132pt}}
\multiput(458.17,353.00)(1.000,6.500){2}{\rule{0.400pt}{1.566pt}}
\put(459.0,353.0){\rule[-0.200pt]{0.400pt}{4.095pt}}
\put(459.67,352){\rule{0.400pt}{0.964pt}}
\multiput(459.17,354.00)(1.000,-2.000){2}{\rule{0.400pt}{0.482pt}}
\put(460.0,356.0){\rule[-0.200pt]{0.400pt}{2.409pt}}
\put(460.67,365){\rule{0.400pt}{0.482pt}}
\multiput(460.17,365.00)(1.000,1.000){2}{\rule{0.400pt}{0.241pt}}
\put(461.0,352.0){\rule[-0.200pt]{0.400pt}{3.132pt}}
\put(462.0,348.0){\rule[-0.200pt]{0.400pt}{4.577pt}}
\put(461.67,352){\rule{0.400pt}{1.445pt}}
\multiput(461.17,352.00)(1.000,3.000){2}{\rule{0.400pt}{0.723pt}}
\put(462.0,348.0){\rule[-0.200pt]{0.400pt}{0.964pt}}
\put(463.0,358.0){\rule[-0.200pt]{0.400pt}{2.409pt}}
\put(462.67,353){\rule{0.400pt}{3.373pt}}
\multiput(462.17,360.00)(1.000,-7.000){2}{\rule{0.400pt}{1.686pt}}
\put(463.0,367.0){\usebox{\plotpoint}}
\put(463.67,339){\rule{0.400pt}{1.204pt}}
\multiput(463.17,341.50)(1.000,-2.500){2}{\rule{0.400pt}{0.602pt}}
\put(464.0,344.0){\rule[-0.200pt]{0.400pt}{2.168pt}}
\put(464.67,350){\rule{0.400pt}{4.577pt}}
\multiput(464.17,350.00)(1.000,9.500){2}{\rule{0.400pt}{2.289pt}}
\put(465.0,339.0){\rule[-0.200pt]{0.400pt}{2.650pt}}
\put(466.0,346.0){\rule[-0.200pt]{0.400pt}{5.541pt}}
\put(465.67,361){\rule{0.400pt}{0.964pt}}
\multiput(465.17,363.00)(1.000,-2.000){2}{\rule{0.400pt}{0.482pt}}
\put(466.0,346.0){\rule[-0.200pt]{0.400pt}{4.577pt}}
\put(466.67,343){\rule{0.400pt}{3.132pt}}
\multiput(466.17,349.50)(1.000,-6.500){2}{\rule{0.400pt}{1.566pt}}
\put(467.0,356.0){\rule[-0.200pt]{0.400pt}{1.204pt}}
\put(468.0,343.0){\rule[-0.200pt]{0.400pt}{3.373pt}}
\put(467.67,345){\rule{0.400pt}{5.300pt}}
\multiput(467.17,345.00)(1.000,11.000){2}{\rule{0.400pt}{2.650pt}}
\put(468.0,345.0){\rule[-0.200pt]{0.400pt}{2.891pt}}
\put(468.67,335){\rule{0.400pt}{2.891pt}}
\multiput(468.17,335.00)(1.000,6.000){2}{\rule{0.400pt}{1.445pt}}
\put(469.0,335.0){\rule[-0.200pt]{0.400pt}{7.709pt}}
\put(470.0,347.0){\rule[-0.200pt]{0.400pt}{3.854pt}}
\put(469.67,341){\rule{0.400pt}{1.204pt}}
\multiput(469.17,343.50)(1.000,-2.500){2}{\rule{0.400pt}{0.602pt}}
\put(470.0,346.0){\rule[-0.200pt]{0.400pt}{4.095pt}}
\put(471.0,341.0){\rule[-0.200pt]{0.400pt}{6.745pt}}
\put(471,352.67){\rule{0.241pt}{0.400pt}}
\multiput(471.00,352.17)(0.500,1.000){2}{\rule{0.120pt}{0.400pt}}
\put(471.0,353.0){\rule[-0.200pt]{0.400pt}{3.854pt}}
\put(472.0,335.0){\rule[-0.200pt]{0.400pt}{4.577pt}}
\put(471.67,346){\rule{0.400pt}{2.409pt}}
\multiput(471.17,346.00)(1.000,5.000){2}{\rule{0.400pt}{1.204pt}}
\put(472.0,335.0){\rule[-0.200pt]{0.400pt}{2.650pt}}
\put(472.67,336){\rule{0.400pt}{3.373pt}}
\multiput(472.17,336.00)(1.000,7.000){2}{\rule{0.400pt}{1.686pt}}
\put(473.0,336.0){\rule[-0.200pt]{0.400pt}{4.818pt}}
\put(474.0,350.0){\rule[-0.200pt]{0.400pt}{1.927pt}}
\put(473.67,336){\rule{0.400pt}{2.409pt}}
\multiput(473.17,336.00)(1.000,5.000){2}{\rule{0.400pt}{1.204pt}}
\put(474.0,336.0){\rule[-0.200pt]{0.400pt}{5.300pt}}
\put(475.0,339.0){\rule[-0.200pt]{0.400pt}{1.686pt}}
\put(474.67,343){\rule{0.400pt}{4.577pt}}
\multiput(474.17,352.50)(1.000,-9.500){2}{\rule{0.400pt}{2.289pt}}
\put(475.0,339.0){\rule[-0.200pt]{0.400pt}{5.541pt}}
\put(475.67,348){\rule{0.400pt}{2.891pt}}
\multiput(475.17,354.00)(1.000,-6.000){2}{\rule{0.400pt}{1.445pt}}
\put(476.0,343.0){\rule[-0.200pt]{0.400pt}{4.095pt}}
\put(476.67,333){\rule{0.400pt}{8.432pt}}
\multiput(476.17,333.00)(1.000,17.500){2}{\rule{0.400pt}{4.216pt}}
\put(477.0,333.0){\rule[-0.200pt]{0.400pt}{3.613pt}}
\put(477.67,346){\rule{0.400pt}{4.577pt}}
\multiput(477.17,346.00)(1.000,9.500){2}{\rule{0.400pt}{2.289pt}}
\put(478.0,346.0){\rule[-0.200pt]{0.400pt}{5.300pt}}
\put(479.0,365.0){\rule[-0.200pt]{0.400pt}{0.482pt}}
\put(478.67,346){\rule{0.400pt}{0.482pt}}
\multiput(478.17,347.00)(1.000,-1.000){2}{\rule{0.400pt}{0.241pt}}
\put(479.0,348.0){\rule[-0.200pt]{0.400pt}{4.577pt}}
\put(480.0,341.0){\rule[-0.200pt]{0.400pt}{1.204pt}}
\put(479.67,345){\rule{0.400pt}{3.614pt}}
\multiput(479.17,345.00)(1.000,7.500){2}{\rule{0.400pt}{1.807pt}}
\put(480.0,341.0){\rule[-0.200pt]{0.400pt}{0.964pt}}
\put(481.0,343.0){\rule[-0.200pt]{0.400pt}{4.095pt}}
\put(480.67,362){\rule{0.400pt}{0.964pt}}
\multiput(480.17,364.00)(1.000,-2.000){2}{\rule{0.400pt}{0.482pt}}
\put(481.0,343.0){\rule[-0.200pt]{0.400pt}{5.541pt}}
\put(481.67,350){\rule{0.400pt}{3.373pt}}
\multiput(481.17,357.00)(1.000,-7.000){2}{\rule{0.400pt}{1.686pt}}
\put(482.0,362.0){\rule[-0.200pt]{0.400pt}{0.482pt}}
\put(483.0,347.0){\rule[-0.200pt]{0.400pt}{0.723pt}}
\put(482.67,346){\rule{0.400pt}{2.409pt}}
\multiput(482.17,351.00)(1.000,-5.000){2}{\rule{0.400pt}{1.204pt}}
\put(483.0,347.0){\rule[-0.200pt]{0.400pt}{2.168pt}}
\put(484.0,333.0){\rule[-0.200pt]{0.400pt}{3.132pt}}
\put(483.67,343){\rule{0.400pt}{0.964pt}}
\multiput(483.17,343.00)(1.000,2.000){2}{\rule{0.400pt}{0.482pt}}
\put(484.0,333.0){\rule[-0.200pt]{0.400pt}{2.409pt}}
\put(484.67,335){\rule{0.400pt}{5.782pt}}
\multiput(484.17,335.00)(1.000,12.000){2}{\rule{0.400pt}{2.891pt}}
\put(485.0,335.0){\rule[-0.200pt]{0.400pt}{2.891pt}}
\put(485.67,354){\rule{0.400pt}{3.132pt}}
\multiput(485.17,360.50)(1.000,-6.500){2}{\rule{0.400pt}{1.566pt}}
\put(486.0,359.0){\rule[-0.200pt]{0.400pt}{1.927pt}}
\put(486.67,344){\rule{0.400pt}{5.300pt}}
\multiput(486.17,355.00)(1.000,-11.000){2}{\rule{0.400pt}{2.650pt}}
\put(487.0,354.0){\rule[-0.200pt]{0.400pt}{2.891pt}}
\put(488.0,344.0){\rule[-0.200pt]{0.400pt}{1.445pt}}
\put(487.67,349){\rule{0.400pt}{0.723pt}}
\multiput(487.17,349.00)(1.000,1.500){2}{\rule{0.400pt}{0.361pt}}
\put(488.0,349.0){\usebox{\plotpoint}}
\put(489.0,352.0){\rule[-0.200pt]{0.400pt}{2.891pt}}
\put(488.67,334){\rule{0.400pt}{4.577pt}}
\multiput(488.17,343.50)(1.000,-9.500){2}{\rule{0.400pt}{2.289pt}}
\put(489.0,353.0){\rule[-0.200pt]{0.400pt}{2.650pt}}
\put(489.67,350){\rule{0.400pt}{0.964pt}}
\multiput(489.17,350.00)(1.000,2.000){2}{\rule{0.400pt}{0.482pt}}
\put(490.0,334.0){\rule[-0.200pt]{0.400pt}{3.854pt}}
\put(491.0,341.0){\rule[-0.200pt]{0.400pt}{3.132pt}}
\put(490.67,357){\rule{0.400pt}{1.927pt}}
\multiput(490.17,357.00)(1.000,4.000){2}{\rule{0.400pt}{0.964pt}}
\put(491.0,341.0){\rule[-0.200pt]{0.400pt}{3.854pt}}
\put(491.67,336){\rule{0.400pt}{4.336pt}}
\multiput(491.17,336.00)(1.000,9.000){2}{\rule{0.400pt}{2.168pt}}
\put(492.0,336.0){\rule[-0.200pt]{0.400pt}{6.986pt}}
\put(493.0,351.0){\rule[-0.200pt]{0.400pt}{0.723pt}}
\put(492.67,350){\rule{0.400pt}{3.854pt}}
\multiput(492.17,358.00)(1.000,-8.000){2}{\rule{0.400pt}{1.927pt}}
\put(493.0,351.0){\rule[-0.200pt]{0.400pt}{3.613pt}}
\put(493.67,332){\rule{0.400pt}{7.468pt}}
\multiput(493.17,332.00)(1.000,15.500){2}{\rule{0.400pt}{3.734pt}}
\put(494.0,332.0){\rule[-0.200pt]{0.400pt}{4.336pt}}
\put(494.67,346){\rule{0.400pt}{2.891pt}}
\multiput(494.17,346.00)(1.000,6.000){2}{\rule{0.400pt}{1.445pt}}
\put(495.0,346.0){\rule[-0.200pt]{0.400pt}{4.095pt}}
\put(496.0,343.0){\rule[-0.200pt]{0.400pt}{3.613pt}}
\put(495.67,356){\rule{0.400pt}{2.168pt}}
\multiput(495.17,356.00)(1.000,4.500){2}{\rule{0.400pt}{1.084pt}}
\put(496.0,343.0){\rule[-0.200pt]{0.400pt}{3.132pt}}
\put(497.0,356.0){\rule[-0.200pt]{0.400pt}{2.168pt}}
\put(496.67,350){\rule{0.400pt}{2.891pt}}
\multiput(496.17,356.00)(1.000,-6.000){2}{\rule{0.400pt}{1.445pt}}
\put(497.0,356.0){\rule[-0.200pt]{0.400pt}{1.445pt}}
\put(497.67,331){\rule{0.400pt}{3.854pt}}
\multiput(497.17,339.00)(1.000,-8.000){2}{\rule{0.400pt}{1.927pt}}
\put(498.0,347.0){\rule[-0.200pt]{0.400pt}{0.723pt}}
\put(499.0,331.0){\rule[-0.200pt]{0.400pt}{6.745pt}}
\put(498.67,344){\rule{0.400pt}{2.650pt}}
\multiput(498.17,344.00)(1.000,5.500){2}{\rule{0.400pt}{1.325pt}}
\put(499.0,344.0){\rule[-0.200pt]{0.400pt}{3.613pt}}
\put(499.67,332){\rule{0.400pt}{7.950pt}}
\multiput(499.17,332.00)(1.000,16.500){2}{\rule{0.400pt}{3.975pt}}
\put(500.0,332.0){\rule[-0.200pt]{0.400pt}{5.541pt}}
\put(500.67,335){\rule{0.400pt}{5.782pt}}
\multiput(500.17,335.00)(1.000,12.000){2}{\rule{0.400pt}{2.891pt}}
\put(501.0,335.0){\rule[-0.200pt]{0.400pt}{7.227pt}}
\put(502.0,331.0){\rule[-0.200pt]{0.400pt}{6.745pt}}
\put(501.67,337){\rule{0.400pt}{6.263pt}}
\multiput(501.17,350.00)(1.000,-13.000){2}{\rule{0.400pt}{3.132pt}}
\put(502.0,331.0){\rule[-0.200pt]{0.400pt}{7.709pt}}
\put(502.67,346){\rule{0.400pt}{2.891pt}}
\multiput(502.17,346.00)(1.000,6.000){2}{\rule{0.400pt}{1.445pt}}
\put(503.0,337.0){\rule[-0.200pt]{0.400pt}{2.168pt}}
\put(504.0,345.0){\rule[-0.200pt]{0.400pt}{3.132pt}}
\put(503.67,339){\rule{0.400pt}{1.686pt}}
\multiput(503.17,342.50)(1.000,-3.500){2}{\rule{0.400pt}{0.843pt}}
\put(504.0,345.0){\usebox{\plotpoint}}
\put(504.67,332){\rule{0.400pt}{5.541pt}}
\multiput(504.17,343.50)(1.000,-11.500){2}{\rule{0.400pt}{2.770pt}}
\put(505.0,339.0){\rule[-0.200pt]{0.400pt}{3.854pt}}
\put(506.0,330.0){\rule[-0.200pt]{0.400pt}{0.482pt}}
\put(505.67,330){\rule{0.400pt}{7.468pt}}
\multiput(505.17,345.50)(1.000,-15.500){2}{\rule{0.400pt}{3.734pt}}
\put(506.0,330.0){\rule[-0.200pt]{0.400pt}{7.468pt}}
\put(507.0,330.0){\rule[-0.200pt]{0.400pt}{0.482pt}}
\put(507.0,332.0){\usebox{\plotpoint}}
\put(508.0,332.0){\rule[-0.200pt]{0.400pt}{6.745pt}}
\put(507.67,340){\rule{0.400pt}{5.541pt}}
\multiput(507.17,340.00)(1.000,11.500){2}{\rule{0.400pt}{2.770pt}}
\put(508.0,340.0){\rule[-0.200pt]{0.400pt}{4.818pt}}
\put(508.67,351){\rule{0.400pt}{2.409pt}}
\multiput(508.17,351.00)(1.000,5.000){2}{\rule{0.400pt}{1.204pt}}
\put(509.0,351.0){\rule[-0.200pt]{0.400pt}{2.891pt}}
\put(510.0,351.0){\rule[-0.200pt]{0.400pt}{2.409pt}}
\put(509.67,359){\rule{0.400pt}{0.964pt}}
\multiput(509.17,361.00)(1.000,-2.000){2}{\rule{0.400pt}{0.482pt}}
\put(510.0,351.0){\rule[-0.200pt]{0.400pt}{2.891pt}}
\put(510.67,336){\rule{0.400pt}{3.373pt}}
\multiput(510.17,343.00)(1.000,-7.000){2}{\rule{0.400pt}{1.686pt}}
\put(511.0,350.0){\rule[-0.200pt]{0.400pt}{2.168pt}}
\put(512.0,336.0){\rule[-0.200pt]{0.400pt}{2.168pt}}
\put(511.67,333){\rule{0.400pt}{0.964pt}}
\multiput(511.17,335.00)(1.000,-2.000){2}{\rule{0.400pt}{0.482pt}}
\put(512.0,337.0){\rule[-0.200pt]{0.400pt}{1.927pt}}
\put(513.0,331.0){\rule[-0.200pt]{0.400pt}{0.482pt}}
\put(512.67,341){\rule{0.400pt}{2.891pt}}
\multiput(512.17,347.00)(1.000,-6.000){2}{\rule{0.400pt}{1.445pt}}
\put(513.0,331.0){\rule[-0.200pt]{0.400pt}{5.300pt}}
\put(514.0,341.0){\rule[-0.200pt]{0.400pt}{0.482pt}}
\put(513.67,334){\rule{0.400pt}{0.723pt}}
\multiput(513.17,334.00)(1.000,1.500){2}{\rule{0.400pt}{0.361pt}}
\put(514.0,334.0){\rule[-0.200pt]{0.400pt}{2.168pt}}
\put(514.67,353){\rule{0.400pt}{1.204pt}}
\multiput(514.17,353.00)(1.000,2.500){2}{\rule{0.400pt}{0.602pt}}
\put(515.0,337.0){\rule[-0.200pt]{0.400pt}{3.854pt}}
\put(516.0,342.0){\rule[-0.200pt]{0.400pt}{3.854pt}}
\put(515.67,349){\rule{0.400pt}{1.204pt}}
\multiput(515.17,349.00)(1.000,2.500){2}{\rule{0.400pt}{0.602pt}}
\put(516.0,342.0){\rule[-0.200pt]{0.400pt}{1.686pt}}
\put(517.0,354.0){\rule[-0.200pt]{0.400pt}{1.686pt}}
\put(516.67,335){\rule{0.400pt}{5.541pt}}
\multiput(516.17,346.50)(1.000,-11.500){2}{\rule{0.400pt}{2.770pt}}
\put(517.0,358.0){\rule[-0.200pt]{0.400pt}{0.723pt}}
\put(518.0,335.0){\rule[-0.200pt]{0.400pt}{4.095pt}}
\put(517.67,336){\rule{0.400pt}{4.336pt}}
\multiput(517.17,336.00)(1.000,9.000){2}{\rule{0.400pt}{2.168pt}}
\put(518.0,336.0){\rule[-0.200pt]{0.400pt}{3.854pt}}
\put(518.67,333){\rule{0.400pt}{6.263pt}}
\multiput(518.17,346.00)(1.000,-13.000){2}{\rule{0.400pt}{3.132pt}}
\put(519.0,354.0){\rule[-0.200pt]{0.400pt}{1.204pt}}
\put(519.67,341){\rule{0.400pt}{1.927pt}}
\multiput(519.17,341.00)(1.000,4.000){2}{\rule{0.400pt}{0.964pt}}
\put(520.0,333.0){\rule[-0.200pt]{0.400pt}{1.927pt}}
\put(521.0,349.0){\rule[-0.200pt]{0.400pt}{0.482pt}}
\put(520.67,332){\rule{0.400pt}{0.723pt}}
\multiput(520.17,332.00)(1.000,1.500){2}{\rule{0.400pt}{0.361pt}}
\put(521.0,332.0){\rule[-0.200pt]{0.400pt}{4.577pt}}
\put(521.67,353){\rule{0.400pt}{1.204pt}}
\multiput(521.17,355.50)(1.000,-2.500){2}{\rule{0.400pt}{0.602pt}}
\put(522.0,335.0){\rule[-0.200pt]{0.400pt}{5.541pt}}
\put(522.67,338){\rule{0.400pt}{4.095pt}}
\multiput(522.17,338.00)(1.000,8.500){2}{\rule{0.400pt}{2.048pt}}
\put(523.0,338.0){\rule[-0.200pt]{0.400pt}{3.613pt}}
\put(524.0,340.0){\rule[-0.200pt]{0.400pt}{3.613pt}}
\put(523.67,341){\rule{0.400pt}{5.059pt}}
\multiput(523.17,341.00)(1.000,10.500){2}{\rule{0.400pt}{2.529pt}}
\put(524.0,340.0){\usebox{\plotpoint}}
\put(525.0,347.0){\rule[-0.200pt]{0.400pt}{3.613pt}}
\put(524.67,350){\rule{0.400pt}{2.891pt}}
\multiput(524.17,356.00)(1.000,-6.000){2}{\rule{0.400pt}{1.445pt}}
\put(525.0,347.0){\rule[-0.200pt]{0.400pt}{3.613pt}}
\put(526.0,336.0){\rule[-0.200pt]{0.400pt}{3.373pt}}
\put(525.67,354){\rule{0.400pt}{0.482pt}}
\multiput(525.17,355.00)(1.000,-1.000){2}{\rule{0.400pt}{0.241pt}}
\put(526.0,336.0){\rule[-0.200pt]{0.400pt}{4.818pt}}
\put(526.67,333){\rule{0.400pt}{6.263pt}}
\multiput(526.17,333.00)(1.000,13.000){2}{\rule{0.400pt}{3.132pt}}
\put(527.0,333.0){\rule[-0.200pt]{0.400pt}{5.059pt}}
\put(527.67,328){\rule{0.400pt}{7.468pt}}
\multiput(527.17,328.00)(1.000,15.500){2}{\rule{0.400pt}{3.734pt}}
\put(528.0,328.0){\rule[-0.200pt]{0.400pt}{7.468pt}}
\put(529.0,348.0){\rule[-0.200pt]{0.400pt}{2.650pt}}
\put(528.67,342){\rule{0.400pt}{3.614pt}}
\multiput(528.17,349.50)(1.000,-7.500){2}{\rule{0.400pt}{1.807pt}}
\put(529.0,348.0){\rule[-0.200pt]{0.400pt}{2.168pt}}
\put(530.0,327.0){\rule[-0.200pt]{0.400pt}{3.613pt}}
\put(529.67,343){\rule{0.400pt}{0.964pt}}
\multiput(529.17,343.00)(1.000,2.000){2}{\rule{0.400pt}{0.482pt}}
\put(530.0,327.0){\rule[-0.200pt]{0.400pt}{3.854pt}}
\put(531.0,347.0){\rule[-0.200pt]{0.400pt}{2.168pt}}
\put(530.67,326){\rule{0.400pt}{8.191pt}}
\multiput(530.17,326.00)(1.000,17.000){2}{\rule{0.400pt}{4.095pt}}
\put(531.0,326.0){\rule[-0.200pt]{0.400pt}{7.227pt}}
\put(531.67,334){\rule{0.400pt}{5.059pt}}
\multiput(531.17,334.00)(1.000,10.500){2}{\rule{0.400pt}{2.529pt}}
\put(532.0,334.0){\rule[-0.200pt]{0.400pt}{6.263pt}}
\put(533.0,326.0){\rule[-0.200pt]{0.400pt}{6.986pt}}
\put(532.67,333){\rule{0.400pt}{5.782pt}}
\multiput(532.17,345.00)(1.000,-12.000){2}{\rule{0.400pt}{2.891pt}}
\put(533.0,326.0){\rule[-0.200pt]{0.400pt}{7.468pt}}
\put(534.0,333.0){\rule[-0.200pt]{0.400pt}{6.745pt}}
\put(534,337.67){\rule{0.241pt}{0.400pt}}
\multiput(534.00,337.17)(0.500,1.000){2}{\rule{0.120pt}{0.400pt}}
\put(534.0,338.0){\rule[-0.200pt]{0.400pt}{5.541pt}}
\put(535.0,339.0){\rule[-0.200pt]{0.400pt}{3.132pt}}
\put(534.67,328){\rule{0.400pt}{3.373pt}}
\multiput(534.17,335.00)(1.000,-7.000){2}{\rule{0.400pt}{1.686pt}}
\put(535.0,342.0){\rule[-0.200pt]{0.400pt}{2.409pt}}
\put(535.67,328){\rule{0.400pt}{6.745pt}}
\multiput(535.17,342.00)(1.000,-14.000){2}{\rule{0.400pt}{3.373pt}}
\put(536.0,328.0){\rule[-0.200pt]{0.400pt}{6.745pt}}
\put(536.67,339){\rule{0.400pt}{2.168pt}}
\multiput(536.17,343.50)(1.000,-4.500){2}{\rule{0.400pt}{1.084pt}}
\put(537.0,328.0){\rule[-0.200pt]{0.400pt}{4.818pt}}
\put(538.0,339.0){\rule[-0.200pt]{0.400pt}{4.336pt}}
\put(537.67,332){\rule{0.400pt}{3.854pt}}
\multiput(537.17,340.00)(1.000,-8.000){2}{\rule{0.400pt}{1.927pt}}
\put(538.0,348.0){\rule[-0.200pt]{0.400pt}{2.168pt}}
\put(538.67,352){\rule{0.400pt}{1.204pt}}
\multiput(538.17,354.50)(1.000,-2.500){2}{\rule{0.400pt}{0.602pt}}
\put(539.0,332.0){\rule[-0.200pt]{0.400pt}{6.022pt}}
\put(539.67,340){\rule{0.400pt}{4.577pt}}
\multiput(539.17,340.00)(1.000,9.500){2}{\rule{0.400pt}{2.289pt}}
\put(540.0,340.0){\rule[-0.200pt]{0.400pt}{2.891pt}}
\put(541.0,349.0){\rule[-0.200pt]{0.400pt}{2.409pt}}
\put(540.67,333){\rule{0.400pt}{4.095pt}}
\multiput(540.17,341.50)(1.000,-8.500){2}{\rule{0.400pt}{2.048pt}}
\put(541.0,349.0){\usebox{\plotpoint}}
\put(542.0,329.0){\rule[-0.200pt]{0.400pt}{0.964pt}}
\put(541.67,345){\rule{0.400pt}{1.686pt}}
\multiput(541.17,348.50)(1.000,-3.500){2}{\rule{0.400pt}{0.843pt}}
\put(542.0,329.0){\rule[-0.200pt]{0.400pt}{5.541pt}}
\put(542.67,328){\rule{0.400pt}{4.818pt}}
\multiput(542.17,338.00)(1.000,-10.000){2}{\rule{0.400pt}{2.409pt}}
\put(543.0,345.0){\rule[-0.200pt]{0.400pt}{0.723pt}}
\put(543.67,348){\rule{0.400pt}{2.409pt}}
\multiput(543.17,348.00)(1.000,5.000){2}{\rule{0.400pt}{1.204pt}}
\put(544.0,328.0){\rule[-0.200pt]{0.400pt}{4.818pt}}
\put(544.67,334){\rule{0.400pt}{2.891pt}}
\multiput(544.17,340.00)(1.000,-6.000){2}{\rule{0.400pt}{1.445pt}}
\put(545.0,346.0){\rule[-0.200pt]{0.400pt}{2.891pt}}
\put(546.0,334.0){\rule[-0.200pt]{0.400pt}{3.373pt}}
\put(545.67,330){\rule{0.400pt}{2.891pt}}
\multiput(545.17,336.00)(1.000,-6.000){2}{\rule{0.400pt}{1.445pt}}
\put(546.0,342.0){\rule[-0.200pt]{0.400pt}{1.445pt}}
\put(547.0,330.0){\rule[-0.200pt]{0.400pt}{4.818pt}}
\put(546.67,329){\rule{0.400pt}{5.300pt}}
\multiput(546.17,329.00)(1.000,11.000){2}{\rule{0.400pt}{2.650pt}}
\put(547.0,329.0){\rule[-0.200pt]{0.400pt}{5.059pt}}
\put(547.67,327){\rule{0.400pt}{3.614pt}}
\multiput(547.17,327.00)(1.000,7.500){2}{\rule{0.400pt}{1.807pt}}
\put(548.0,327.0){\rule[-0.200pt]{0.400pt}{5.782pt}}
\put(549.0,342.0){\rule[-0.200pt]{0.400pt}{3.373pt}}
\put(548.67,346){\rule{0.400pt}{2.168pt}}
\multiput(548.17,346.00)(1.000,4.500){2}{\rule{0.400pt}{1.084pt}}
\put(549.0,346.0){\rule[-0.200pt]{0.400pt}{2.409pt}}
\put(550.0,329.0){\rule[-0.200pt]{0.400pt}{6.263pt}}
\put(549.67,345){\rule{0.400pt}{0.482pt}}
\multiput(549.17,345.00)(1.000,1.000){2}{\rule{0.400pt}{0.241pt}}
\put(550.0,329.0){\rule[-0.200pt]{0.400pt}{3.854pt}}
\put(551.0,347.0){\rule[-0.200pt]{0.400pt}{1.445pt}}
\put(550.67,345){\rule{0.400pt}{2.168pt}}
\multiput(550.17,345.00)(1.000,4.500){2}{\rule{0.400pt}{1.084pt}}
\put(551.0,345.0){\rule[-0.200pt]{0.400pt}{1.927pt}}
\put(551.67,330){\rule{0.400pt}{0.964pt}}
\multiput(551.17,330.00)(1.000,2.000){2}{\rule{0.400pt}{0.482pt}}
\put(552.0,330.0){\rule[-0.200pt]{0.400pt}{5.782pt}}
\put(552.67,323){\rule{0.400pt}{8.432pt}}
\multiput(552.17,340.50)(1.000,-17.500){2}{\rule{0.400pt}{4.216pt}}
\put(553.0,334.0){\rule[-0.200pt]{0.400pt}{5.782pt}}
\put(554.0,323.0){\rule[-0.200pt]{0.400pt}{8.431pt}}
\put(553.67,341){\rule{0.400pt}{3.854pt}}
\multiput(553.17,349.00)(1.000,-8.000){2}{\rule{0.400pt}{1.927pt}}
\put(554.0,357.0){\usebox{\plotpoint}}
\put(555.0,341.0){\usebox{\plotpoint}}
\put(554.67,331){\rule{0.400pt}{4.577pt}}
\multiput(554.17,331.00)(1.000,9.500){2}{\rule{0.400pt}{2.289pt}}
\put(555.0,331.0){\rule[-0.200pt]{0.400pt}{2.650pt}}
\put(556.0,331.0){\rule[-0.200pt]{0.400pt}{4.577pt}}
\put(555.67,346){\rule{0.400pt}{0.723pt}}
\multiput(555.17,346.00)(1.000,1.500){2}{\rule{0.400pt}{0.361pt}}
\put(556.0,331.0){\rule[-0.200pt]{0.400pt}{3.613pt}}
\put(557,329.67){\rule{0.241pt}{0.400pt}}
\multiput(557.00,330.17)(0.500,-1.000){2}{\rule{0.120pt}{0.400pt}}
\put(557.0,331.0){\rule[-0.200pt]{0.400pt}{4.336pt}}
\put(558.0,324.0){\rule[-0.200pt]{0.400pt}{1.445pt}}
\put(558.0,324.0){\rule[-0.200pt]{0.400pt}{4.095pt}}
\put(558.0,341.0){\usebox{\plotpoint}}
\put(559.0,336.0){\rule[-0.200pt]{0.400pt}{1.204pt}}
\put(558.67,342){\rule{0.400pt}{2.650pt}}
\multiput(558.17,347.50)(1.000,-5.500){2}{\rule{0.400pt}{1.325pt}}
\put(559.0,336.0){\rule[-0.200pt]{0.400pt}{4.095pt}}
\put(560.0,322.0){\rule[-0.200pt]{0.400pt}{4.818pt}}
\put(559.67,346){\rule{0.400pt}{1.927pt}}
\multiput(559.17,350.00)(1.000,-4.000){2}{\rule{0.400pt}{0.964pt}}
\put(560.0,322.0){\rule[-0.200pt]{0.400pt}{7.709pt}}
\put(560.67,335){\rule{0.400pt}{0.482pt}}
\multiput(560.17,336.00)(1.000,-1.000){2}{\rule{0.400pt}{0.241pt}}
\put(561.0,337.0){\rule[-0.200pt]{0.400pt}{2.168pt}}
\put(561.67,335){\rule{0.400pt}{4.336pt}}
\multiput(561.17,344.00)(1.000,-9.000){2}{\rule{0.400pt}{2.168pt}}
\put(562.0,335.0){\rule[-0.200pt]{0.400pt}{4.336pt}}
\put(562.67,336){\rule{0.400pt}{2.168pt}}
\multiput(562.17,340.50)(1.000,-4.500){2}{\rule{0.400pt}{1.084pt}}
\put(563.0,335.0){\rule[-0.200pt]{0.400pt}{2.409pt}}
\put(564.0,329.0){\rule[-0.200pt]{0.400pt}{1.686pt}}
\put(563.67,327){\rule{0.400pt}{0.723pt}}
\multiput(563.17,328.50)(1.000,-1.500){2}{\rule{0.400pt}{0.361pt}}
\put(564.0,329.0){\usebox{\plotpoint}}
\put(564.67,323){\rule{0.400pt}{4.577pt}}
\multiput(564.17,323.00)(1.000,9.500){2}{\rule{0.400pt}{2.289pt}}
\put(565.0,323.0){\rule[-0.200pt]{0.400pt}{0.964pt}}
\put(566.0,326.0){\rule[-0.200pt]{0.400pt}{3.854pt}}
\put(565.67,345){\rule{0.400pt}{1.445pt}}
\multiput(565.17,348.00)(1.000,-3.000){2}{\rule{0.400pt}{0.723pt}}
\put(566.0,326.0){\rule[-0.200pt]{0.400pt}{6.022pt}}
\put(567.0,345.0){\rule[-0.200pt]{0.400pt}{2.650pt}}
\put(566.67,338){\rule{0.400pt}{0.723pt}}
\multiput(566.17,338.00)(1.000,1.500){2}{\rule{0.400pt}{0.361pt}}
\put(567.0,338.0){\rule[-0.200pt]{0.400pt}{4.336pt}}
\put(568.0,322.0){\rule[-0.200pt]{0.400pt}{4.577pt}}
\put(567.67,323){\rule{0.400pt}{7.468pt}}
\multiput(567.17,338.50)(1.000,-15.500){2}{\rule{0.400pt}{3.734pt}}
\put(568.0,322.0){\rule[-0.200pt]{0.400pt}{7.709pt}}
\put(568.67,321){\rule{0.400pt}{6.023pt}}
\multiput(568.17,333.50)(1.000,-12.500){2}{\rule{0.400pt}{3.011pt}}
\put(569.0,323.0){\rule[-0.200pt]{0.400pt}{5.541pt}}
\put(569.67,332){\rule{0.400pt}{3.854pt}}
\multiput(569.17,332.00)(1.000,8.000){2}{\rule{0.400pt}{1.927pt}}
\put(570.0,321.0){\rule[-0.200pt]{0.400pt}{2.650pt}}
\put(571.0,325.0){\rule[-0.200pt]{0.400pt}{5.541pt}}
\put(570.67,321){\rule{0.400pt}{1.686pt}}
\multiput(570.17,324.50)(1.000,-3.500){2}{\rule{0.400pt}{0.843pt}}
\put(571.0,325.0){\rule[-0.200pt]{0.400pt}{0.723pt}}
\put(572.0,321.0){\rule[-0.200pt]{0.400pt}{7.227pt}}
\put(572.0,351.0){\usebox{\plotpoint}}
\put(572.67,325){\rule{0.400pt}{5.541pt}}
\multiput(572.17,325.00)(1.000,11.500){2}{\rule{0.400pt}{2.770pt}}
\put(573.0,325.0){\rule[-0.200pt]{0.400pt}{6.263pt}}
\put(573.67,332){\rule{0.400pt}{1.927pt}}
\multiput(573.17,336.00)(1.000,-4.000){2}{\rule{0.400pt}{0.964pt}}
\put(574.0,340.0){\rule[-0.200pt]{0.400pt}{1.927pt}}
\put(575.0,332.0){\rule[-0.200pt]{0.400pt}{5.541pt}}
\put(574.67,326){\rule{0.400pt}{6.504pt}}
\multiput(574.17,326.00)(1.000,13.500){2}{\rule{0.400pt}{3.252pt}}
\put(575.0,326.0){\rule[-0.200pt]{0.400pt}{6.986pt}}
\put(576.0,353.0){\rule[-0.200pt]{0.400pt}{0.482pt}}
\put(575.67,326){\rule{0.400pt}{4.095pt}}
\multiput(575.17,334.50)(1.000,-8.500){2}{\rule{0.400pt}{2.048pt}}
\put(576.0,343.0){\rule[-0.200pt]{0.400pt}{2.891pt}}
\put(577.0,326.0){\rule[-0.200pt]{0.400pt}{6.263pt}}
\put(576.67,341){\rule{0.400pt}{3.132pt}}
\multiput(576.17,341.00)(1.000,6.500){2}{\rule{0.400pt}{1.566pt}}
\put(577.0,341.0){\rule[-0.200pt]{0.400pt}{2.650pt}}
\put(577.67,332){\rule{0.400pt}{0.482pt}}
\multiput(577.17,332.00)(1.000,1.000){2}{\rule{0.400pt}{0.241pt}}
\put(578.0,332.0){\rule[-0.200pt]{0.400pt}{5.300pt}}
\put(578.67,326){\rule{0.400pt}{5.782pt}}
\multiput(578.17,326.00)(1.000,12.000){2}{\rule{0.400pt}{2.891pt}}
\put(579.0,326.0){\rule[-0.200pt]{0.400pt}{1.927pt}}
\put(580.0,341.0){\rule[-0.200pt]{0.400pt}{2.168pt}}
\put(579.67,335){\rule{0.400pt}{3.614pt}}
\multiput(579.17,342.50)(1.000,-7.500){2}{\rule{0.400pt}{1.807pt}}
\put(580.0,341.0){\rule[-0.200pt]{0.400pt}{2.168pt}}
\put(580.67,345){\rule{0.400pt}{0.482pt}}
\multiput(580.17,345.00)(1.000,1.000){2}{\rule{0.400pt}{0.241pt}}
\put(581.0,335.0){\rule[-0.200pt]{0.400pt}{2.409pt}}
\put(581.67,325){\rule{0.400pt}{5.059pt}}
\multiput(581.17,335.50)(1.000,-10.500){2}{\rule{0.400pt}{2.529pt}}
\put(582.0,346.0){\usebox{\plotpoint}}
\put(583.0,324.0){\usebox{\plotpoint}}
\put(582.67,337){\rule{0.400pt}{3.132pt}}
\multiput(582.17,337.00)(1.000,6.500){2}{\rule{0.400pt}{1.566pt}}
\put(583.0,324.0){\rule[-0.200pt]{0.400pt}{3.132pt}}
\put(584.0,350.0){\rule[-0.200pt]{0.400pt}{0.964pt}}
\put(583.67,330){\rule{0.400pt}{5.541pt}}
\multiput(583.17,341.50)(1.000,-11.500){2}{\rule{0.400pt}{2.770pt}}
\put(584.0,353.0){\usebox{\plotpoint}}
\put(585.0,330.0){\rule[-0.200pt]{0.400pt}{0.964pt}}
\put(584.67,330){\rule{0.400pt}{5.300pt}}
\multiput(584.17,330.00)(1.000,11.000){2}{\rule{0.400pt}{2.650pt}}
\put(585.0,330.0){\rule[-0.200pt]{0.400pt}{0.964pt}}
\put(585.67,329){\rule{0.400pt}{5.782pt}}
\multiput(585.17,329.00)(1.000,12.000){2}{\rule{0.400pt}{2.891pt}}
\put(586.0,329.0){\rule[-0.200pt]{0.400pt}{5.541pt}}
\put(587.0,327.0){\rule[-0.200pt]{0.400pt}{6.263pt}}
\put(586.67,340){\rule{0.400pt}{1.204pt}}
\multiput(586.17,342.50)(1.000,-2.500){2}{\rule{0.400pt}{0.602pt}}
\put(587.0,327.0){\rule[-0.200pt]{0.400pt}{4.336pt}}
\put(587.67,336){\rule{0.400pt}{4.095pt}}
\multiput(587.17,344.50)(1.000,-8.500){2}{\rule{0.400pt}{2.048pt}}
\put(588.0,340.0){\rule[-0.200pt]{0.400pt}{3.132pt}}
\put(588.67,322){\rule{0.400pt}{2.891pt}}
\multiput(588.17,322.00)(1.000,6.000){2}{\rule{0.400pt}{1.445pt}}
\put(589.0,322.0){\rule[-0.200pt]{0.400pt}{3.373pt}}
\put(589.67,325){\rule{0.400pt}{5.541pt}}
\multiput(589.17,325.00)(1.000,11.500){2}{\rule{0.400pt}{2.770pt}}
\put(590.0,325.0){\rule[-0.200pt]{0.400pt}{2.168pt}}
\put(591.0,343.0){\rule[-0.200pt]{0.400pt}{1.204pt}}
\put(590.67,340){\rule{0.400pt}{2.891pt}}
\multiput(590.17,346.00)(1.000,-6.000){2}{\rule{0.400pt}{1.445pt}}
\put(591.0,343.0){\rule[-0.200pt]{0.400pt}{2.168pt}}
\put(592.0,324.0){\rule[-0.200pt]{0.400pt}{3.854pt}}
\put(591.67,322){\rule{0.400pt}{1.445pt}}
\multiput(591.17,325.00)(1.000,-3.000){2}{\rule{0.400pt}{0.723pt}}
\put(592.0,324.0){\rule[-0.200pt]{0.400pt}{0.964pt}}
\put(593.0,322.0){\rule[-0.200pt]{0.400pt}{4.577pt}}
\put(592.67,333){\rule{0.400pt}{1.686pt}}
\multiput(592.17,336.50)(1.000,-3.500){2}{\rule{0.400pt}{0.843pt}}
\put(593.0,340.0){\usebox{\plotpoint}}
\put(593.67,318){\rule{0.400pt}{6.023pt}}
\multiput(593.17,318.00)(1.000,12.500){2}{\rule{0.400pt}{3.011pt}}
\put(594.0,318.0){\rule[-0.200pt]{0.400pt}{3.613pt}}
\put(594.67,320){\rule{0.400pt}{3.373pt}}
\multiput(594.17,320.00)(1.000,7.000){2}{\rule{0.400pt}{1.686pt}}
\put(595.0,320.0){\rule[-0.200pt]{0.400pt}{5.541pt}}
\put(596.0,334.0){\rule[-0.200pt]{0.400pt}{2.409pt}}
\put(596,322.67){\rule{0.241pt}{0.400pt}}
\multiput(596.00,322.17)(0.500,1.000){2}{\rule{0.120pt}{0.400pt}}
\put(596.0,323.0){\rule[-0.200pt]{0.400pt}{5.059pt}}
\put(597.0,324.0){\rule[-0.200pt]{0.400pt}{4.095pt}}
\put(596.67,337){\rule{0.400pt}{3.614pt}}
\multiput(596.17,337.00)(1.000,7.500){2}{\rule{0.400pt}{1.807pt}}
\put(597.0,337.0){\rule[-0.200pt]{0.400pt}{0.964pt}}
\put(598.0,342.0){\rule[-0.200pt]{0.400pt}{2.409pt}}
\put(597.67,344){\rule{0.400pt}{1.204pt}}
\multiput(597.17,344.00)(1.000,2.500){2}{\rule{0.400pt}{0.602pt}}
\put(598.0,342.0){\rule[-0.200pt]{0.400pt}{0.482pt}}
\put(599,349){\usebox{\plotpoint}}
\put(598.67,344){\rule{0.400pt}{1.204pt}}
\multiput(598.17,346.50)(1.000,-2.500){2}{\rule{0.400pt}{0.602pt}}
\put(599.67,317){\rule{0.400pt}{4.577pt}}
\multiput(599.17,317.00)(1.000,9.500){2}{\rule{0.400pt}{2.289pt}}
\put(600.0,317.0){\rule[-0.200pt]{0.400pt}{6.504pt}}
\put(600.67,323){\rule{0.400pt}{6.504pt}}
\multiput(600.17,336.50)(1.000,-13.500){2}{\rule{0.400pt}{3.252pt}}
\put(601.0,336.0){\rule[-0.200pt]{0.400pt}{3.373pt}}
\put(602.0,323.0){\rule[-0.200pt]{0.400pt}{1.204pt}}
\put(602.0,326.0){\rule[-0.200pt]{0.400pt}{0.482pt}}
\put(602.0,326.0){\usebox{\plotpoint}}
\put(602.67,320){\rule{0.400pt}{4.336pt}}
\multiput(602.17,320.00)(1.000,9.000){2}{\rule{0.400pt}{2.168pt}}
\put(603.0,320.0){\rule[-0.200pt]{0.400pt}{1.445pt}}
\put(604.0,338.0){\rule[-0.200pt]{0.400pt}{0.482pt}}
\put(603.67,317){\rule{0.400pt}{4.577pt}}
\multiput(603.17,317.00)(1.000,9.500){2}{\rule{0.400pt}{2.289pt}}
\put(604.0,317.0){\rule[-0.200pt]{0.400pt}{5.541pt}}
\put(605.0,320.0){\rule[-0.200pt]{0.400pt}{3.854pt}}
\put(605.0,320.0){\usebox{\plotpoint}}
\put(605.67,318){\rule{0.400pt}{7.227pt}}
\multiput(605.17,333.00)(1.000,-15.000){2}{\rule{0.400pt}{3.613pt}}
\put(606.0,320.0){\rule[-0.200pt]{0.400pt}{6.745pt}}
\put(606.67,330){\rule{0.400pt}{3.132pt}}
\multiput(606.17,336.50)(1.000,-6.500){2}{\rule{0.400pt}{1.566pt}}
\put(607.0,318.0){\rule[-0.200pt]{0.400pt}{6.022pt}}
\put(607.67,327){\rule{0.400pt}{4.095pt}}
\multiput(607.17,327.00)(1.000,8.500){2}{\rule{0.400pt}{2.048pt}}
\put(608.0,327.0){\rule[-0.200pt]{0.400pt}{0.723pt}}
\put(609.0,320.0){\rule[-0.200pt]{0.400pt}{5.782pt}}
\put(608.67,339){\rule{0.400pt}{1.686pt}}
\multiput(608.17,342.50)(1.000,-3.500){2}{\rule{0.400pt}{0.843pt}}
\put(609.0,320.0){\rule[-0.200pt]{0.400pt}{6.263pt}}
\put(610.0,325.0){\rule[-0.200pt]{0.400pt}{3.373pt}}
\put(609.67,328){\rule{0.400pt}{0.482pt}}
\multiput(609.17,329.00)(1.000,-1.000){2}{\rule{0.400pt}{0.241pt}}
\put(610.0,325.0){\rule[-0.200pt]{0.400pt}{1.204pt}}
\put(610.67,329){\rule{0.400pt}{2.650pt}}
\multiput(610.17,329.00)(1.000,5.500){2}{\rule{0.400pt}{1.325pt}}
\put(611.0,328.0){\usebox{\plotpoint}}
\put(612.0,338.0){\rule[-0.200pt]{0.400pt}{0.482pt}}
\put(611.67,325){\rule{0.400pt}{4.577pt}}
\multiput(611.17,334.50)(1.000,-9.500){2}{\rule{0.400pt}{2.289pt}}
\put(612.0,338.0){\rule[-0.200pt]{0.400pt}{1.445pt}}
\put(613.0,325.0){\rule[-0.200pt]{0.400pt}{5.541pt}}
\put(613,343.67){\rule{0.241pt}{0.400pt}}
\multiput(613.00,344.17)(0.500,-1.000){2}{\rule{0.120pt}{0.400pt}}
\put(613.0,345.0){\rule[-0.200pt]{0.400pt}{0.723pt}}
\put(613.67,317){\rule{0.400pt}{2.891pt}}
\multiput(613.17,317.00)(1.000,6.000){2}{\rule{0.400pt}{1.445pt}}
\put(614.0,317.0){\rule[-0.200pt]{0.400pt}{6.504pt}}
\put(614.67,327){\rule{0.400pt}{2.891pt}}
\multiput(614.17,327.00)(1.000,6.000){2}{\rule{0.400pt}{1.445pt}}
\put(615.0,327.0){\rule[-0.200pt]{0.400pt}{0.482pt}}
\put(616.0,329.0){\rule[-0.200pt]{0.400pt}{2.409pt}}
\put(615.67,316){\rule{0.400pt}{4.577pt}}
\multiput(615.17,325.50)(1.000,-9.500){2}{\rule{0.400pt}{2.289pt}}
\put(616.0,329.0){\rule[-0.200pt]{0.400pt}{1.445pt}}
\put(617.0,316.0){\rule[-0.200pt]{0.400pt}{4.336pt}}
\put(616.67,322){\rule{0.400pt}{2.168pt}}
\multiput(616.17,326.50)(1.000,-4.500){2}{\rule{0.400pt}{1.084pt}}
\put(617.0,331.0){\rule[-0.200pt]{0.400pt}{0.723pt}}
\put(618.0,322.0){\rule[-0.200pt]{0.400pt}{5.782pt}}
\put(617.67,319){\rule{0.400pt}{3.614pt}}
\multiput(617.17,319.00)(1.000,7.500){2}{\rule{0.400pt}{1.807pt}}
\put(618.0,319.0){\rule[-0.200pt]{0.400pt}{6.504pt}}
\put(619,337.67){\rule{0.241pt}{0.400pt}}
\multiput(619.00,338.17)(0.500,-1.000){2}{\rule{0.120pt}{0.400pt}}
\put(619.0,334.0){\rule[-0.200pt]{0.400pt}{1.204pt}}
\put(620.0,338.0){\rule[-0.200pt]{0.400pt}{2.409pt}}
\put(619.67,318){\rule{0.400pt}{4.336pt}}
\multiput(619.17,318.00)(1.000,9.000){2}{\rule{0.400pt}{2.168pt}}
\put(620.0,318.0){\rule[-0.200pt]{0.400pt}{7.227pt}}
\put(621,336){\usebox{\plotpoint}}
\put(620.67,324){\rule{0.400pt}{1.445pt}}
\multiput(620.17,327.00)(1.000,-3.000){2}{\rule{0.400pt}{0.723pt}}
\put(621.0,330.0){\rule[-0.200pt]{0.400pt}{1.445pt}}
\put(622.0,324.0){\rule[-0.200pt]{0.400pt}{5.059pt}}
\put(621.67,319){\rule{0.400pt}{6.263pt}}
\multiput(621.17,319.00)(1.000,13.000){2}{\rule{0.400pt}{3.132pt}}
\put(622.0,319.0){\rule[-0.200pt]{0.400pt}{6.263pt}}
\put(622.67,314){\rule{0.400pt}{2.168pt}}
\multiput(622.17,314.00)(1.000,4.500){2}{\rule{0.400pt}{1.084pt}}
\put(623.0,314.0){\rule[-0.200pt]{0.400pt}{7.468pt}}
\put(623.67,324){\rule{0.400pt}{1.445pt}}
\multiput(623.17,327.00)(1.000,-3.000){2}{\rule{0.400pt}{0.723pt}}
\put(624.0,323.0){\rule[-0.200pt]{0.400pt}{1.686pt}}
\put(625.0,324.0){\rule[-0.200pt]{0.400pt}{1.927pt}}
\put(624.67,316){\rule{0.400pt}{4.336pt}}
\multiput(624.17,316.00)(1.000,9.000){2}{\rule{0.400pt}{2.168pt}}
\put(625.0,316.0){\rule[-0.200pt]{0.400pt}{3.854pt}}
\put(626.0,334.0){\rule[-0.200pt]{0.400pt}{3.132pt}}
\put(625.67,323){\rule{0.400pt}{1.204pt}}
\multiput(625.17,325.50)(1.000,-2.500){2}{\rule{0.400pt}{0.602pt}}
\put(626.0,328.0){\rule[-0.200pt]{0.400pt}{4.577pt}}
\put(626.67,336){\rule{0.400pt}{2.650pt}}
\multiput(626.17,341.50)(1.000,-5.500){2}{\rule{0.400pt}{1.325pt}}
\put(627.0,323.0){\rule[-0.200pt]{0.400pt}{5.782pt}}
\put(627.67,318){\rule{0.400pt}{4.577pt}}
\multiput(627.17,318.00)(1.000,9.500){2}{\rule{0.400pt}{2.289pt}}
\put(628.0,318.0){\rule[-0.200pt]{0.400pt}{4.336pt}}
\put(629.0,322.0){\rule[-0.200pt]{0.400pt}{3.613pt}}
\put(628.67,320){\rule{0.400pt}{6.504pt}}
\multiput(628.17,333.50)(1.000,-13.500){2}{\rule{0.400pt}{3.252pt}}
\put(629.0,322.0){\rule[-0.200pt]{0.400pt}{6.022pt}}
\put(630.0,320.0){\rule[-0.200pt]{0.400pt}{3.132pt}}
\put(629.67,329){\rule{0.400pt}{1.204pt}}
\multiput(629.17,329.00)(1.000,2.500){2}{\rule{0.400pt}{0.602pt}}
\put(630.0,329.0){\rule[-0.200pt]{0.400pt}{0.964pt}}
\put(631.0,320.0){\rule[-0.200pt]{0.400pt}{3.373pt}}
\put(630.67,323){\rule{0.400pt}{2.650pt}}
\multiput(630.17,323.00)(1.000,5.500){2}{\rule{0.400pt}{1.325pt}}
\put(631.0,320.0){\rule[-0.200pt]{0.400pt}{0.723pt}}
\put(631.67,338){\rule{0.400pt}{0.723pt}}
\multiput(631.17,338.00)(1.000,1.500){2}{\rule{0.400pt}{0.361pt}}
\put(632.0,334.0){\rule[-0.200pt]{0.400pt}{0.964pt}}
\put(632.67,314){\rule{0.400pt}{5.059pt}}
\multiput(632.17,314.00)(1.000,10.500){2}{\rule{0.400pt}{2.529pt}}
\put(633.0,314.0){\rule[-0.200pt]{0.400pt}{6.504pt}}
\put(634.0,335.0){\rule[-0.200pt]{0.400pt}{1.686pt}}
\put(633.67,326){\rule{0.400pt}{1.686pt}}
\multiput(633.17,326.00)(1.000,3.500){2}{\rule{0.400pt}{0.843pt}}
\put(634.0,326.0){\rule[-0.200pt]{0.400pt}{3.854pt}}
\put(635.0,333.0){\rule[-0.200pt]{0.400pt}{0.482pt}}
\put(634.67,326){\rule{0.400pt}{3.854pt}}
\multiput(634.17,326.00)(1.000,8.000){2}{\rule{0.400pt}{1.927pt}}
\put(635.0,326.0){\rule[-0.200pt]{0.400pt}{2.168pt}}
\put(635.67,324){\rule{0.400pt}{3.614pt}}
\multiput(635.17,324.00)(1.000,7.500){2}{\rule{0.400pt}{1.807pt}}
\put(636.0,324.0){\rule[-0.200pt]{0.400pt}{4.336pt}}
\put(636.67,333){\rule{0.400pt}{3.373pt}}
\multiput(636.17,340.00)(1.000,-7.000){2}{\rule{0.400pt}{1.686pt}}
\put(637.0,339.0){\rule[-0.200pt]{0.400pt}{1.927pt}}
\put(638.0,333.0){\rule[-0.200pt]{0.400pt}{3.132pt}}
\put(637.67,319){\rule{0.400pt}{2.650pt}}
\multiput(637.17,324.50)(1.000,-5.500){2}{\rule{0.400pt}{1.325pt}}
\put(638.0,330.0){\rule[-0.200pt]{0.400pt}{3.854pt}}
\put(639.0,319.0){\rule[-0.200pt]{0.400pt}{4.818pt}}
\put(638.67,321){\rule{0.400pt}{2.409pt}}
\multiput(638.17,321.00)(1.000,5.000){2}{\rule{0.400pt}{1.204pt}}
\put(639.0,321.0){\rule[-0.200pt]{0.400pt}{4.336pt}}
\put(639.67,317){\rule{0.400pt}{1.686pt}}
\multiput(639.17,320.50)(1.000,-3.500){2}{\rule{0.400pt}{0.843pt}}
\put(640.0,324.0){\rule[-0.200pt]{0.400pt}{1.686pt}}
\put(641.0,317.0){\rule[-0.200pt]{0.400pt}{4.577pt}}
\put(640.67,315){\rule{0.400pt}{2.168pt}}
\multiput(640.17,319.50)(1.000,-4.500){2}{\rule{0.400pt}{1.084pt}}
\put(641.0,324.0){\rule[-0.200pt]{0.400pt}{2.891pt}}
\put(642.0,315.0){\rule[-0.200pt]{0.400pt}{5.541pt}}
\put(641.67,328){\rule{0.400pt}{1.445pt}}
\multiput(641.17,331.00)(1.000,-3.000){2}{\rule{0.400pt}{0.723pt}}
\put(642.0,334.0){\rule[-0.200pt]{0.400pt}{0.964pt}}
\put(643.0,319.0){\rule[-0.200pt]{0.400pt}{2.168pt}}
\put(642.67,332){\rule{0.400pt}{2.650pt}}
\multiput(642.17,337.50)(1.000,-5.500){2}{\rule{0.400pt}{1.325pt}}
\put(643.0,319.0){\rule[-0.200pt]{0.400pt}{5.782pt}}
\put(643.67,339){\rule{0.400pt}{0.964pt}}
\multiput(643.17,341.00)(1.000,-2.000){2}{\rule{0.400pt}{0.482pt}}
\put(644.0,332.0){\rule[-0.200pt]{0.400pt}{2.650pt}}
\put(645.0,339.0){\rule[-0.200pt]{0.400pt}{0.964pt}}
\put(644.67,317){\rule{0.400pt}{2.650pt}}
\multiput(644.17,322.50)(1.000,-5.500){2}{\rule{0.400pt}{1.325pt}}
\put(645.0,328.0){\rule[-0.200pt]{0.400pt}{3.613pt}}
\put(645.67,315){\rule{0.400pt}{6.986pt}}
\multiput(645.17,329.50)(1.000,-14.500){2}{\rule{0.400pt}{3.493pt}}
\put(646.0,317.0){\rule[-0.200pt]{0.400pt}{6.504pt}}
\put(647.0,315.0){\rule[-0.200pt]{0.400pt}{6.986pt}}
\put(646.67,313){\rule{0.400pt}{6.986pt}}
\multiput(646.17,313.00)(1.000,14.500){2}{\rule{0.400pt}{3.493pt}}
\put(647.0,313.0){\rule[-0.200pt]{0.400pt}{7.468pt}}
\put(647.67,313){\rule{0.400pt}{7.709pt}}
\multiput(647.17,313.00)(1.000,16.000){2}{\rule{0.400pt}{3.854pt}}
\put(648.0,313.0){\rule[-0.200pt]{0.400pt}{6.986pt}}
\put(649,345){\usebox{\plotpoint}}
\put(648.67,312){\rule{0.400pt}{7.950pt}}
\multiput(648.17,328.50)(1.000,-16.500){2}{\rule{0.400pt}{3.975pt}}
\put(649.67,318){\rule{0.400pt}{6.263pt}}
\multiput(649.17,318.00)(1.000,13.000){2}{\rule{0.400pt}{3.132pt}}
\put(650.0,312.0){\rule[-0.200pt]{0.400pt}{1.445pt}}
\put(650.67,318){\rule{0.400pt}{1.686pt}}
\multiput(650.17,318.00)(1.000,3.500){2}{\rule{0.400pt}{0.843pt}}
\put(651.0,318.0){\rule[-0.200pt]{0.400pt}{6.263pt}}
\put(652.0,317.0){\rule[-0.200pt]{0.400pt}{1.927pt}}
\put(651.67,322){\rule{0.400pt}{2.650pt}}
\multiput(651.17,322.00)(1.000,5.500){2}{\rule{0.400pt}{1.325pt}}
\put(652.0,317.0){\rule[-0.200pt]{0.400pt}{1.204pt}}
\put(652.67,324){\rule{0.400pt}{4.336pt}}
\multiput(652.17,324.00)(1.000,9.000){2}{\rule{0.400pt}{2.168pt}}
\put(653.0,324.0){\rule[-0.200pt]{0.400pt}{2.168pt}}
\put(654.0,337.0){\rule[-0.200pt]{0.400pt}{1.204pt}}
\put(653.67,324){\rule{0.400pt}{3.373pt}}
\multiput(653.17,331.00)(1.000,-7.000){2}{\rule{0.400pt}{1.686pt}}
\put(654.0,337.0){\usebox{\plotpoint}}
\put(655.0,321.0){\rule[-0.200pt]{0.400pt}{0.723pt}}
\put(654.67,336){\rule{0.400pt}{0.482pt}}
\multiput(654.17,337.00)(1.000,-1.000){2}{\rule{0.400pt}{0.241pt}}
\put(655.0,321.0){\rule[-0.200pt]{0.400pt}{4.095pt}}
\put(656.0,311.0){\rule[-0.200pt]{0.400pt}{6.022pt}}
\put(655.67,324){\rule{0.400pt}{1.686pt}}
\multiput(655.17,327.50)(1.000,-3.500){2}{\rule{0.400pt}{0.843pt}}
\put(656.0,311.0){\rule[-0.200pt]{0.400pt}{4.818pt}}
\put(656.67,325){\rule{0.400pt}{0.964pt}}
\multiput(656.17,327.00)(1.000,-2.000){2}{\rule{0.400pt}{0.482pt}}
\put(657.0,324.0){\rule[-0.200pt]{0.400pt}{1.204pt}}
\put(658.0,325.0){\rule[-0.200pt]{0.400pt}{3.373pt}}
\put(657.67,320){\rule{0.400pt}{3.854pt}}
\multiput(657.17,320.00)(1.000,8.000){2}{\rule{0.400pt}{1.927pt}}
\put(658.0,320.0){\rule[-0.200pt]{0.400pt}{4.577pt}}
\put(659.0,329.0){\rule[-0.200pt]{0.400pt}{1.686pt}}
\put(658.67,332){\rule{0.400pt}{0.482pt}}
\multiput(658.17,333.00)(1.000,-1.000){2}{\rule{0.400pt}{0.241pt}}
\put(659.0,329.0){\rule[-0.200pt]{0.400pt}{1.204pt}}
\put(659.67,336){\rule{0.400pt}{1.686pt}}
\multiput(659.17,339.50)(1.000,-3.500){2}{\rule{0.400pt}{0.843pt}}
\put(660.0,332.0){\rule[-0.200pt]{0.400pt}{2.650pt}}
\put(661,324.67){\rule{0.241pt}{0.400pt}}
\multiput(661.00,325.17)(0.500,-1.000){2}{\rule{0.120pt}{0.400pt}}
\put(661.0,326.0){\rule[-0.200pt]{0.400pt}{2.409pt}}
\put(662.0,314.0){\rule[-0.200pt]{0.400pt}{2.650pt}}
\put(661.67,324){\rule{0.400pt}{4.336pt}}
\multiput(661.17,324.00)(1.000,9.000){2}{\rule{0.400pt}{2.168pt}}
\put(662.0,314.0){\rule[-0.200pt]{0.400pt}{2.409pt}}
\put(662.67,312){\rule{0.400pt}{2.409pt}}
\multiput(662.17,312.00)(1.000,5.000){2}{\rule{0.400pt}{1.204pt}}
\put(663.0,312.0){\rule[-0.200pt]{0.400pt}{7.227pt}}
\put(664.0,322.0){\rule[-0.200pt]{0.400pt}{2.650pt}}
\put(663.67,320){\rule{0.400pt}{4.818pt}}
\multiput(663.17,320.00)(1.000,10.000){2}{\rule{0.400pt}{2.409pt}}
\put(664.0,320.0){\rule[-0.200pt]{0.400pt}{3.132pt}}
\put(664.67,320){\rule{0.400pt}{4.577pt}}
\multiput(664.17,320.00)(1.000,9.500){2}{\rule{0.400pt}{2.289pt}}
\put(665.0,320.0){\rule[-0.200pt]{0.400pt}{4.818pt}}
\put(665.67,317){\rule{0.400pt}{6.023pt}}
\multiput(665.17,317.00)(1.000,12.500){2}{\rule{0.400pt}{3.011pt}}
\put(666.0,317.0){\rule[-0.200pt]{0.400pt}{5.300pt}}
\put(667.0,321.0){\rule[-0.200pt]{0.400pt}{5.059pt}}
\put(666.67,332){\rule{0.400pt}{0.482pt}}
\multiput(666.17,332.00)(1.000,1.000){2}{\rule{0.400pt}{0.241pt}}
\put(667.0,321.0){\rule[-0.200pt]{0.400pt}{2.650pt}}
\put(668.0,320.0){\rule[-0.200pt]{0.400pt}{3.373pt}}
\put(667.67,335){\rule{0.400pt}{0.723pt}}
\multiput(667.17,335.00)(1.000,1.500){2}{\rule{0.400pt}{0.361pt}}
\put(668.0,320.0){\rule[-0.200pt]{0.400pt}{3.613pt}}
\put(668.67,325){\rule{0.400pt}{0.964pt}}
\multiput(668.17,327.00)(1.000,-2.000){2}{\rule{0.400pt}{0.482pt}}
\put(669.0,329.0){\rule[-0.200pt]{0.400pt}{2.168pt}}
\put(670.0,322.0){\rule[-0.200pt]{0.400pt}{0.723pt}}
\put(669.67,335){\rule{0.400pt}{0.964pt}}
\multiput(669.17,335.00)(1.000,2.000){2}{\rule{0.400pt}{0.482pt}}
\put(670.0,322.0){\rule[-0.200pt]{0.400pt}{3.132pt}}
\put(671.0,311.0){\rule[-0.200pt]{0.400pt}{6.745pt}}
\put(670.67,326){\rule{0.400pt}{3.614pt}}
\multiput(670.17,326.00)(1.000,7.500){2}{\rule{0.400pt}{1.807pt}}
\put(671.0,311.0){\rule[-0.200pt]{0.400pt}{3.613pt}}
\put(672.0,309.0){\rule[-0.200pt]{0.400pt}{7.709pt}}
\put(671.67,327){\rule{0.400pt}{1.927pt}}
\multiput(671.17,331.00)(1.000,-4.000){2}{\rule{0.400pt}{0.964pt}}
\put(672.0,309.0){\rule[-0.200pt]{0.400pt}{6.263pt}}
\put(673.0,311.0){\rule[-0.200pt]{0.400pt}{3.854pt}}
\put(672.67,314){\rule{0.400pt}{6.023pt}}
\multiput(672.17,314.00)(1.000,12.500){2}{\rule{0.400pt}{3.011pt}}
\put(673.0,311.0){\rule[-0.200pt]{0.400pt}{0.723pt}}
\put(673.67,313){\rule{0.400pt}{4.818pt}}
\multiput(673.17,323.00)(1.000,-10.000){2}{\rule{0.400pt}{2.409pt}}
\put(674.0,333.0){\rule[-0.200pt]{0.400pt}{1.445pt}}
\put(675.0,312.0){\usebox{\plotpoint}}
\put(674.67,313){\rule{0.400pt}{2.650pt}}
\multiput(674.17,318.50)(1.000,-5.500){2}{\rule{0.400pt}{1.325pt}}
\put(675.0,312.0){\rule[-0.200pt]{0.400pt}{2.891pt}}
\put(676.0,313.0){\rule[-0.200pt]{0.400pt}{3.854pt}}
\put(675.67,309){\rule{0.400pt}{1.204pt}}
\multiput(675.17,309.00)(1.000,2.500){2}{\rule{0.400pt}{0.602pt}}
\put(676.0,309.0){\rule[-0.200pt]{0.400pt}{4.818pt}}
\put(677.0,314.0){\rule[-0.200pt]{0.400pt}{4.095pt}}
\put(677.0,314.0){\rule[-0.200pt]{0.400pt}{4.095pt}}
\put(677.0,314.0){\usebox{\plotpoint}}
\put(678,332.67){\rule{0.241pt}{0.400pt}}
\multiput(678.00,332.17)(0.500,1.000){2}{\rule{0.120pt}{0.400pt}}
\put(678.0,314.0){\rule[-0.200pt]{0.400pt}{4.577pt}}
\put(679.0,330.0){\rule[-0.200pt]{0.400pt}{0.964pt}}
\put(678.67,321){\rule{0.400pt}{4.818pt}}
\multiput(678.17,331.00)(1.000,-10.000){2}{\rule{0.400pt}{2.409pt}}
\put(679.0,330.0){\rule[-0.200pt]{0.400pt}{2.650pt}}
\put(680.0,318.0){\rule[-0.200pt]{0.400pt}{0.723pt}}
\put(679.67,325){\rule{0.400pt}{0.723pt}}
\multiput(679.17,326.50)(1.000,-1.500){2}{\rule{0.400pt}{0.361pt}}
\put(680.0,318.0){\rule[-0.200pt]{0.400pt}{2.409pt}}
\put(681.0,325.0){\rule[-0.200pt]{0.400pt}{3.613pt}}
\put(680.67,311){\rule{0.400pt}{3.854pt}}
\multiput(680.17,319.00)(1.000,-8.000){2}{\rule{0.400pt}{1.927pt}}
\put(681.0,327.0){\rule[-0.200pt]{0.400pt}{3.132pt}}
\put(681.67,327){\rule{0.400pt}{1.927pt}}
\multiput(681.17,331.00)(1.000,-4.000){2}{\rule{0.400pt}{0.964pt}}
\put(682.0,311.0){\rule[-0.200pt]{0.400pt}{5.782pt}}
\put(683.0,327.0){\rule[-0.200pt]{0.400pt}{2.168pt}}
\put(682.67,312){\rule{0.400pt}{4.095pt}}
\multiput(682.17,312.00)(1.000,8.500){2}{\rule{0.400pt}{2.048pt}}
\put(683.0,312.0){\rule[-0.200pt]{0.400pt}{5.782pt}}
\put(684.0,317.0){\rule[-0.200pt]{0.400pt}{2.891pt}}
\put(683.67,318){\rule{0.400pt}{4.818pt}}
\multiput(683.17,318.00)(1.000,10.000){2}{\rule{0.400pt}{2.409pt}}
\put(684.0,317.0){\usebox{\plotpoint}}
\put(684.67,315){\rule{0.400pt}{1.927pt}}
\multiput(684.17,319.00)(1.000,-4.000){2}{\rule{0.400pt}{0.964pt}}
\put(685.0,323.0){\rule[-0.200pt]{0.400pt}{3.613pt}}
\put(685.67,318){\rule{0.400pt}{2.168pt}}
\multiput(685.17,318.00)(1.000,4.500){2}{\rule{0.400pt}{1.084pt}}
\put(686.0,315.0){\rule[-0.200pt]{0.400pt}{0.723pt}}
\put(687.0,324.0){\rule[-0.200pt]{0.400pt}{0.723pt}}
\put(686.67,324){\rule{0.400pt}{0.964pt}}
\multiput(686.17,326.00)(1.000,-2.000){2}{\rule{0.400pt}{0.482pt}}
\put(687.0,324.0){\rule[-0.200pt]{0.400pt}{0.964pt}}
\put(688.0,320.0){\rule[-0.200pt]{0.400pt}{0.964pt}}
\put(687.67,332){\rule{0.400pt}{0.482pt}}
\multiput(687.17,333.00)(1.000,-1.000){2}{\rule{0.400pt}{0.241pt}}
\put(688.0,320.0){\rule[-0.200pt]{0.400pt}{3.373pt}}
\put(689.0,318.0){\rule[-0.200pt]{0.400pt}{3.373pt}}
\put(688.67,312){\rule{0.400pt}{5.300pt}}
\multiput(688.17,323.00)(1.000,-11.000){2}{\rule{0.400pt}{2.650pt}}
\put(689.0,318.0){\rule[-0.200pt]{0.400pt}{3.854pt}}
\put(689.67,305){\rule{0.400pt}{4.336pt}}
\multiput(689.17,314.00)(1.000,-9.000){2}{\rule{0.400pt}{2.168pt}}
\put(690.0,312.0){\rule[-0.200pt]{0.400pt}{2.650pt}}
\put(690.67,315){\rule{0.400pt}{3.854pt}}
\multiput(690.17,323.00)(1.000,-8.000){2}{\rule{0.400pt}{1.927pt}}
\put(691.0,305.0){\rule[-0.200pt]{0.400pt}{6.263pt}}
\put(692.0,305.0){\rule[-0.200pt]{0.400pt}{2.409pt}}
\put(691.67,319){\rule{0.400pt}{1.204pt}}
\multiput(691.17,319.00)(1.000,2.500){2}{\rule{0.400pt}{0.602pt}}
\put(692.0,305.0){\rule[-0.200pt]{0.400pt}{3.373pt}}
\put(693.0,324.0){\rule[-0.200pt]{0.400pt}{0.482pt}}
\put(692.67,321){\rule{0.400pt}{3.373pt}}
\multiput(692.17,321.00)(1.000,7.000){2}{\rule{0.400pt}{1.686pt}}
\put(693.0,321.0){\rule[-0.200pt]{0.400pt}{1.204pt}}
\put(693.67,317){\rule{0.400pt}{0.482pt}}
\multiput(693.17,318.00)(1.000,-1.000){2}{\rule{0.400pt}{0.241pt}}
\put(694.0,319.0){\rule[-0.200pt]{0.400pt}{3.854pt}}
\put(694.67,330){\rule{0.400pt}{2.168pt}}
\multiput(694.17,334.50)(1.000,-4.500){2}{\rule{0.400pt}{1.084pt}}
\put(695.0,317.0){\rule[-0.200pt]{0.400pt}{5.300pt}}
\put(695.67,309){\rule{0.400pt}{5.059pt}}
\multiput(695.17,309.00)(1.000,10.500){2}{\rule{0.400pt}{2.529pt}}
\put(696.0,309.0){\rule[-0.200pt]{0.400pt}{5.059pt}}
\put(697.0,313.0){\rule[-0.200pt]{0.400pt}{4.095pt}}
\put(696.67,330){\rule{0.400pt}{0.482pt}}
\multiput(696.17,330.00)(1.000,1.000){2}{\rule{0.400pt}{0.241pt}}
\put(697.0,313.0){\rule[-0.200pt]{0.400pt}{4.095pt}}
\put(698.0,332.0){\rule[-0.200pt]{0.400pt}{1.204pt}}
\put(697.67,326){\rule{0.400pt}{0.964pt}}
\multiput(697.17,328.00)(1.000,-2.000){2}{\rule{0.400pt}{0.482pt}}
\put(698.0,330.0){\rule[-0.200pt]{0.400pt}{1.686pt}}
\put(698.67,334){\rule{0.400pt}{0.964pt}}
\multiput(698.17,336.00)(1.000,-2.000){2}{\rule{0.400pt}{0.482pt}}
\put(699.0,326.0){\rule[-0.200pt]{0.400pt}{2.891pt}}
\put(700.0,334.0){\rule[-0.200pt]{0.400pt}{0.723pt}}
\put(699.67,305){\rule{0.400pt}{4.095pt}}
\multiput(699.17,305.00)(1.000,8.500){2}{\rule{0.400pt}{2.048pt}}
\put(700.0,305.0){\rule[-0.200pt]{0.400pt}{7.709pt}}
\put(701.0,313.0){\rule[-0.200pt]{0.400pt}{2.168pt}}
\put(700.67,321){\rule{0.400pt}{1.686pt}}
\multiput(700.17,321.00)(1.000,3.500){2}{\rule{0.400pt}{0.843pt}}
\put(701.0,313.0){\rule[-0.200pt]{0.400pt}{1.927pt}}
\put(702.0,319.0){\rule[-0.200pt]{0.400pt}{2.168pt}}
\put(701.67,305){\rule{0.400pt}{5.059pt}}
\multiput(701.17,315.50)(1.000,-10.500){2}{\rule{0.400pt}{2.529pt}}
\put(702.0,319.0){\rule[-0.200pt]{0.400pt}{1.686pt}}
\put(702.67,308){\rule{0.400pt}{4.336pt}}
\multiput(702.17,317.00)(1.000,-9.000){2}{\rule{0.400pt}{2.168pt}}
\put(703.0,305.0){\rule[-0.200pt]{0.400pt}{5.059pt}}
\put(703.67,314){\rule{0.400pt}{5.059pt}}
\multiput(703.17,324.50)(1.000,-10.500){2}{\rule{0.400pt}{2.529pt}}
\put(704.0,308.0){\rule[-0.200pt]{0.400pt}{6.504pt}}
\put(704.67,311){\rule{0.400pt}{3.854pt}}
\multiput(704.17,311.00)(1.000,8.000){2}{\rule{0.400pt}{1.927pt}}
\put(705.0,311.0){\rule[-0.200pt]{0.400pt}{0.723pt}}
\put(705.67,305){\rule{0.400pt}{6.504pt}}
\multiput(705.17,318.50)(1.000,-13.500){2}{\rule{0.400pt}{3.252pt}}
\put(706.0,327.0){\rule[-0.200pt]{0.400pt}{1.204pt}}
\put(706.67,326){\rule{0.400pt}{0.723pt}}
\multiput(706.17,327.50)(1.000,-1.500){2}{\rule{0.400pt}{0.361pt}}
\put(707.0,305.0){\rule[-0.200pt]{0.400pt}{5.782pt}}
\put(708.0,326.0){\rule[-0.200pt]{0.400pt}{1.445pt}}
\put(707.67,318){\rule{0.400pt}{0.964pt}}
\multiput(707.17,318.00)(1.000,2.000){2}{\rule{0.400pt}{0.482pt}}
\put(708.0,318.0){\rule[-0.200pt]{0.400pt}{3.373pt}}
\put(709.0,309.0){\rule[-0.200pt]{0.400pt}{3.132pt}}
\put(708.67,318){\rule{0.400pt}{4.577pt}}
\multiput(708.17,318.00)(1.000,9.500){2}{\rule{0.400pt}{2.289pt}}
\put(709.0,309.0){\rule[-0.200pt]{0.400pt}{2.168pt}}
\put(709.67,330){\rule{0.400pt}{0.482pt}}
\multiput(709.17,330.00)(1.000,1.000){2}{\rule{0.400pt}{0.241pt}}
\put(710.0,330.0){\rule[-0.200pt]{0.400pt}{1.686pt}}
\put(710.67,317){\rule{0.400pt}{3.132pt}}
\multiput(710.17,323.50)(1.000,-6.500){2}{\rule{0.400pt}{1.566pt}}
\put(711.0,330.0){\rule[-0.200pt]{0.400pt}{0.482pt}}
\put(712.0,314.0){\rule[-0.200pt]{0.400pt}{0.723pt}}
\put(711.67,321){\rule{0.400pt}{3.373pt}}
\multiput(711.17,321.00)(1.000,7.000){2}{\rule{0.400pt}{1.686pt}}
\put(712.0,314.0){\rule[-0.200pt]{0.400pt}{1.686pt}}
\put(713.0,306.0){\rule[-0.200pt]{0.400pt}{6.986pt}}
\put(712.67,316){\rule{0.400pt}{0.482pt}}
\multiput(712.17,316.00)(1.000,1.000){2}{\rule{0.400pt}{0.241pt}}
\put(713.0,306.0){\rule[-0.200pt]{0.400pt}{2.409pt}}
\put(713.67,308){\rule{0.400pt}{6.023pt}}
\multiput(713.17,320.50)(1.000,-12.500){2}{\rule{0.400pt}{3.011pt}}
\put(714.0,318.0){\rule[-0.200pt]{0.400pt}{3.613pt}}
\put(714.67,320){\rule{0.400pt}{1.927pt}}
\multiput(714.17,324.00)(1.000,-4.000){2}{\rule{0.400pt}{0.964pt}}
\put(715.0,308.0){\rule[-0.200pt]{0.400pt}{4.818pt}}
\put(716.0,320.0){\rule[-0.200pt]{0.400pt}{3.613pt}}
\put(716,317.67){\rule{0.241pt}{0.400pt}}
\multiput(716.00,317.17)(0.500,1.000){2}{\rule{0.120pt}{0.400pt}}
\put(716.0,318.0){\rule[-0.200pt]{0.400pt}{4.095pt}}
\put(716.67,318){\rule{0.400pt}{3.614pt}}
\multiput(716.17,325.50)(1.000,-7.500){2}{\rule{0.400pt}{1.807pt}}
\put(717.0,319.0){\rule[-0.200pt]{0.400pt}{3.373pt}}
\put(718.0,318.0){\rule[-0.200pt]{0.400pt}{1.445pt}}
\put(717.67,321){\rule{0.400pt}{0.482pt}}
\multiput(717.17,321.00)(1.000,1.000){2}{\rule{0.400pt}{0.241pt}}
\put(718.0,321.0){\rule[-0.200pt]{0.400pt}{0.723pt}}
\put(718.67,307){\rule{0.400pt}{3.132pt}}
\multiput(718.17,313.50)(1.000,-6.500){2}{\rule{0.400pt}{1.566pt}}
\put(719.0,320.0){\rule[-0.200pt]{0.400pt}{0.723pt}}
\put(720.0,307.0){\rule[-0.200pt]{0.400pt}{3.613pt}}
\put(719.67,312){\rule{0.400pt}{1.445pt}}
\multiput(719.17,312.00)(1.000,3.000){2}{\rule{0.400pt}{0.723pt}}
\put(720.0,312.0){\rule[-0.200pt]{0.400pt}{2.409pt}}
\put(721.0,318.0){\rule[-0.200pt]{0.400pt}{1.927pt}}
\put(720.67,305){\rule{0.400pt}{3.373pt}}
\multiput(720.17,312.00)(1.000,-7.000){2}{\rule{0.400pt}{1.686pt}}
\put(721.0,319.0){\rule[-0.200pt]{0.400pt}{1.686pt}}
\put(721.67,309){\rule{0.400pt}{5.782pt}}
\multiput(721.17,309.00)(1.000,12.000){2}{\rule{0.400pt}{2.891pt}}
\put(722.0,305.0){\rule[-0.200pt]{0.400pt}{0.964pt}}
\put(723.0,310.0){\rule[-0.200pt]{0.400pt}{5.541pt}}
\put(722.67,313){\rule{0.400pt}{3.854pt}}
\multiput(722.17,321.00)(1.000,-8.000){2}{\rule{0.400pt}{1.927pt}}
\put(723.0,310.0){\rule[-0.200pt]{0.400pt}{4.577pt}}
\put(723.67,301){\rule{0.400pt}{3.373pt}}
\multiput(723.17,308.00)(1.000,-7.000){2}{\rule{0.400pt}{1.686pt}}
\put(724.0,313.0){\rule[-0.200pt]{0.400pt}{0.482pt}}
\put(725.0,301.0){\rule[-0.200pt]{0.400pt}{1.445pt}}
\put(724.67,301){\rule{0.400pt}{4.095pt}}
\multiput(724.17,301.00)(1.000,8.500){2}{\rule{0.400pt}{2.048pt}}
\put(725.0,301.0){\rule[-0.200pt]{0.400pt}{1.445pt}}
\put(726.0,318.0){\rule[-0.200pt]{0.400pt}{0.482pt}}
\put(725.67,305){\rule{0.400pt}{6.023pt}}
\multiput(725.17,305.00)(1.000,12.500){2}{\rule{0.400pt}{3.011pt}}
\put(726.0,305.0){\rule[-0.200pt]{0.400pt}{3.613pt}}
\put(727.0,306.0){\rule[-0.200pt]{0.400pt}{5.782pt}}
\put(726.67,319){\rule{0.400pt}{2.168pt}}
\multiput(726.17,319.00)(1.000,4.500){2}{\rule{0.400pt}{1.084pt}}
\put(727.0,306.0){\rule[-0.200pt]{0.400pt}{3.132pt}}
\put(728,328){\usebox{\plotpoint}}
\put(727.67,307){\rule{0.400pt}{5.059pt}}
\multiput(727.17,317.50)(1.000,-10.500){2}{\rule{0.400pt}{2.529pt}}
\put(729.0,307.0){\rule[-0.200pt]{0.400pt}{3.854pt}}
\put(728.67,300){\rule{0.400pt}{1.686pt}}
\multiput(728.17,300.00)(1.000,3.500){2}{\rule{0.400pt}{0.843pt}}
\put(729.0,300.0){\rule[-0.200pt]{0.400pt}{5.541pt}}
\put(729.67,314){\rule{0.400pt}{0.964pt}}
\multiput(729.17,314.00)(1.000,2.000){2}{\rule{0.400pt}{0.482pt}}
\put(730.0,307.0){\rule[-0.200pt]{0.400pt}{1.686pt}}
\put(730.67,300){\rule{0.400pt}{3.854pt}}
\multiput(730.17,300.00)(1.000,8.000){2}{\rule{0.400pt}{1.927pt}}
\put(731.0,300.0){\rule[-0.200pt]{0.400pt}{4.336pt}}
\put(731.67,319){\rule{0.400pt}{2.891pt}}
\multiput(731.17,319.00)(1.000,6.000){2}{\rule{0.400pt}{1.445pt}}
\put(732.0,316.0){\rule[-0.200pt]{0.400pt}{0.723pt}}
\put(733.0,310.0){\rule[-0.200pt]{0.400pt}{5.059pt}}
\put(732.67,311){\rule{0.400pt}{1.204pt}}
\multiput(732.17,313.50)(1.000,-2.500){2}{\rule{0.400pt}{0.602pt}}
\put(733.0,310.0){\rule[-0.200pt]{0.400pt}{1.445pt}}
\put(734.0,311.0){\rule[-0.200pt]{0.400pt}{5.541pt}}
\put(734,326.67){\rule{0.241pt}{0.400pt}}
\multiput(734.00,327.17)(0.500,-1.000){2}{\rule{0.120pt}{0.400pt}}
\put(734.0,328.0){\rule[-0.200pt]{0.400pt}{1.445pt}}
\put(734.67,301){\rule{0.400pt}{6.745pt}}
\multiput(734.17,301.00)(1.000,14.000){2}{\rule{0.400pt}{3.373pt}}
\put(735.0,301.0){\rule[-0.200pt]{0.400pt}{6.263pt}}
\put(735.67,308){\rule{0.400pt}{3.854pt}}
\multiput(735.17,316.00)(1.000,-8.000){2}{\rule{0.400pt}{1.927pt}}
\put(736.0,324.0){\rule[-0.200pt]{0.400pt}{1.204pt}}
\put(736.67,323){\rule{0.400pt}{0.723pt}}
\multiput(736.17,324.50)(1.000,-1.500){2}{\rule{0.400pt}{0.361pt}}
\put(737.0,308.0){\rule[-0.200pt]{0.400pt}{4.336pt}}
\put(738.0,304.0){\rule[-0.200pt]{0.400pt}{4.577pt}}
\put(737.67,319){\rule{0.400pt}{3.132pt}}
\multiput(737.17,319.00)(1.000,6.500){2}{\rule{0.400pt}{1.566pt}}
\put(738.0,304.0){\rule[-0.200pt]{0.400pt}{3.613pt}}
\put(738.67,303){\rule{0.400pt}{1.686pt}}
\multiput(738.17,306.50)(1.000,-3.500){2}{\rule{0.400pt}{0.843pt}}
\put(739.0,310.0){\rule[-0.200pt]{0.400pt}{5.300pt}}
\put(739.67,301){\rule{0.400pt}{6.504pt}}
\multiput(739.17,314.50)(1.000,-13.500){2}{\rule{0.400pt}{3.252pt}}
\put(740.0,303.0){\rule[-0.200pt]{0.400pt}{6.022pt}}
\put(741.0,301.0){\rule[-0.200pt]{0.400pt}{5.541pt}}
\put(740.67,304){\rule{0.400pt}{6.023pt}}
\multiput(740.17,304.00)(1.000,12.500){2}{\rule{0.400pt}{3.011pt}}
\put(741.0,304.0){\rule[-0.200pt]{0.400pt}{4.818pt}}
\put(741.67,300){\rule{0.400pt}{4.095pt}}
\multiput(741.17,300.00)(1.000,8.500){2}{\rule{0.400pt}{2.048pt}}
\put(742.0,300.0){\rule[-0.200pt]{0.400pt}{6.986pt}}
\put(743.0,305.0){\rule[-0.200pt]{0.400pt}{2.891pt}}
\put(742.67,322){\rule{0.400pt}{0.482pt}}
\multiput(742.17,322.00)(1.000,1.000){2}{\rule{0.400pt}{0.241pt}}
\put(743.0,305.0){\rule[-0.200pt]{0.400pt}{4.095pt}}
\put(744.0,307.0){\rule[-0.200pt]{0.400pt}{4.095pt}}
\put(743.67,299){\rule{0.400pt}{6.745pt}}
\multiput(743.17,313.00)(1.000,-14.000){2}{\rule{0.400pt}{3.373pt}}
\put(744.0,307.0){\rule[-0.200pt]{0.400pt}{4.818pt}}
\put(744.67,308){\rule{0.400pt}{3.373pt}}
\multiput(744.17,308.00)(1.000,7.000){2}{\rule{0.400pt}{1.686pt}}
\put(745.0,299.0){\rule[-0.200pt]{0.400pt}{2.168pt}}
\put(746.0,322.0){\rule[-0.200pt]{0.400pt}{2.409pt}}
\put(745.67,300){\rule{0.400pt}{5.300pt}}
\multiput(745.17,311.00)(1.000,-11.000){2}{\rule{0.400pt}{2.650pt}}
\put(746.0,322.0){\rule[-0.200pt]{0.400pt}{2.409pt}}
\put(746.67,312){\rule{0.400pt}{3.132pt}}
\multiput(746.17,318.50)(1.000,-6.500){2}{\rule{0.400pt}{1.566pt}}
\put(747.0,300.0){\rule[-0.200pt]{0.400pt}{6.022pt}}
\put(748.0,301.0){\rule[-0.200pt]{0.400pt}{2.650pt}}
\put(747.67,304){\rule{0.400pt}{6.263pt}}
\multiput(747.17,317.00)(1.000,-13.000){2}{\rule{0.400pt}{3.132pt}}
\put(748.0,301.0){\rule[-0.200pt]{0.400pt}{6.986pt}}
\put(748.67,300){\rule{0.400pt}{0.723pt}}
\multiput(748.17,301.50)(1.000,-1.500){2}{\rule{0.400pt}{0.361pt}}
\put(749.0,303.0){\usebox{\plotpoint}}
\put(750.0,297.0){\rule[-0.200pt]{0.400pt}{0.723pt}}
\put(749.67,299){\rule{0.400pt}{0.723pt}}
\multiput(749.17,300.50)(1.000,-1.500){2}{\rule{0.400pt}{0.361pt}}
\put(750.0,297.0){\rule[-0.200pt]{0.400pt}{1.204pt}}
\put(751,299){\usebox{\plotpoint}}
\put(750.67,300){\rule{0.400pt}{4.095pt}}
\multiput(750.17,300.00)(1.000,8.500){2}{\rule{0.400pt}{2.048pt}}
\put(751.0,299.0){\usebox{\plotpoint}}
\put(751.67,304){\rule{0.400pt}{2.891pt}}
\multiput(751.17,304.00)(1.000,6.000){2}{\rule{0.400pt}{1.445pt}}
\put(752.0,304.0){\rule[-0.200pt]{0.400pt}{3.132pt}}
\put(752.67,304){\rule{0.400pt}{3.854pt}}
\multiput(752.17,312.00)(1.000,-8.000){2}{\rule{0.400pt}{1.927pt}}
\put(753.0,316.0){\rule[-0.200pt]{0.400pt}{0.964pt}}
\put(753.67,298){\rule{0.400pt}{4.577pt}}
\multiput(753.17,307.50)(1.000,-9.500){2}{\rule{0.400pt}{2.289pt}}
\put(754.0,304.0){\rule[-0.200pt]{0.400pt}{3.132pt}}
\put(755.0,298.0){\rule[-0.200pt]{0.400pt}{5.782pt}}
\put(754.67,315){\rule{0.400pt}{0.723pt}}
\multiput(754.17,316.50)(1.000,-1.500){2}{\rule{0.400pt}{0.361pt}}
\put(755.0,318.0){\rule[-0.200pt]{0.400pt}{0.964pt}}
\put(756.0,315.0){\rule[-0.200pt]{0.400pt}{1.445pt}}
\put(755.67,310){\rule{0.400pt}{2.891pt}}
\multiput(755.17,310.00)(1.000,6.000){2}{\rule{0.400pt}{1.445pt}}
\put(756.0,310.0){\rule[-0.200pt]{0.400pt}{2.650pt}}
\put(756.67,298){\rule{0.400pt}{1.686pt}}
\multiput(756.17,298.00)(1.000,3.500){2}{\rule{0.400pt}{0.843pt}}
\put(757.0,298.0){\rule[-0.200pt]{0.400pt}{5.782pt}}
\put(757.67,313){\rule{0.400pt}{0.964pt}}
\multiput(757.17,315.00)(1.000,-2.000){2}{\rule{0.400pt}{0.482pt}}
\put(758.0,305.0){\rule[-0.200pt]{0.400pt}{2.891pt}}
\put(759.0,313.0){\usebox{\plotpoint}}
\put(758.67,299){\rule{0.400pt}{4.818pt}}
\multiput(758.17,299.00)(1.000,10.000){2}{\rule{0.400pt}{2.409pt}}
\put(759.0,299.0){\rule[-0.200pt]{0.400pt}{3.613pt}}
\put(760.0,298.0){\rule[-0.200pt]{0.400pt}{5.059pt}}
\put(759.67,300){\rule{0.400pt}{3.614pt}}
\multiput(759.17,300.00)(1.000,7.500){2}{\rule{0.400pt}{1.807pt}}
\put(760.0,298.0){\rule[-0.200pt]{0.400pt}{0.482pt}}
\put(761,326.67){\rule{0.241pt}{0.400pt}}
\multiput(761.00,327.17)(0.500,-1.000){2}{\rule{0.120pt}{0.400pt}}
\put(761.0,315.0){\rule[-0.200pt]{0.400pt}{3.132pt}}
\put(761.67,297){\rule{0.400pt}{3.373pt}}
\multiput(761.17,297.00)(1.000,7.000){2}{\rule{0.400pt}{1.686pt}}
\put(762.0,297.0){\rule[-0.200pt]{0.400pt}{7.227pt}}
\put(763.0,304.0){\rule[-0.200pt]{0.400pt}{1.686pt}}
\put(762.67,309){\rule{0.400pt}{2.891pt}}
\multiput(762.17,309.00)(1.000,6.000){2}{\rule{0.400pt}{1.445pt}}
\put(763.0,304.0){\rule[-0.200pt]{0.400pt}{1.204pt}}
\put(764.0,321.0){\rule[-0.200pt]{0.400pt}{0.964pt}}
\put(763.67,302){\rule{0.400pt}{0.964pt}}
\multiput(763.17,302.00)(1.000,2.000){2}{\rule{0.400pt}{0.482pt}}
\put(764.0,302.0){\rule[-0.200pt]{0.400pt}{5.541pt}}
\put(764.67,312){\rule{0.400pt}{4.095pt}}
\multiput(764.17,320.50)(1.000,-8.500){2}{\rule{0.400pt}{2.048pt}}
\put(765.0,306.0){\rule[-0.200pt]{0.400pt}{5.541pt}}
\put(766.0,307.0){\rule[-0.200pt]{0.400pt}{1.204pt}}
\put(766,307.67){\rule{0.241pt}{0.400pt}}
\multiput(766.00,307.17)(0.500,1.000){2}{\rule{0.120pt}{0.400pt}}
\put(766.0,307.0){\usebox{\plotpoint}}
\put(767.0,300.0){\rule[-0.200pt]{0.400pt}{2.168pt}}
\put(767.0,300.0){\rule[-0.200pt]{0.400pt}{3.132pt}}
\put(767.0,313.0){\usebox{\plotpoint}}
\put(768.0,294.0){\rule[-0.200pt]{0.400pt}{4.577pt}}
\put(767.67,296){\rule{0.400pt}{4.095pt}}
\multiput(767.17,296.00)(1.000,8.500){2}{\rule{0.400pt}{2.048pt}}
\put(768.0,294.0){\rule[-0.200pt]{0.400pt}{0.482pt}}
\put(769.0,296.0){\rule[-0.200pt]{0.400pt}{4.095pt}}
\put(768.67,313){\rule{0.400pt}{2.891pt}}
\multiput(768.17,319.00)(1.000,-6.000){2}{\rule{0.400pt}{1.445pt}}
\put(769.0,296.0){\rule[-0.200pt]{0.400pt}{6.986pt}}
\put(769.67,304){\rule{0.400pt}{4.577pt}}
\multiput(769.17,313.50)(1.000,-9.500){2}{\rule{0.400pt}{2.289pt}}
\put(770.0,313.0){\rule[-0.200pt]{0.400pt}{2.409pt}}
\put(771.0,303.0){\usebox{\plotpoint}}
\put(770.67,304){\rule{0.400pt}{3.614pt}}
\multiput(770.17,311.50)(1.000,-7.500){2}{\rule{0.400pt}{1.807pt}}
\put(771.0,303.0){\rule[-0.200pt]{0.400pt}{3.854pt}}
\put(772.0,304.0){\rule[-0.200pt]{0.400pt}{4.818pt}}
\put(771.67,295){\rule{0.400pt}{3.373pt}}
\multiput(771.17,295.00)(1.000,7.000){2}{\rule{0.400pt}{1.686pt}}
\put(772.0,295.0){\rule[-0.200pt]{0.400pt}{6.986pt}}
\put(773.0,309.0){\rule[-0.200pt]{0.400pt}{3.373pt}}
\put(772.67,309){\rule{0.400pt}{3.854pt}}
\multiput(772.17,309.00)(1.000,8.000){2}{\rule{0.400pt}{1.927pt}}
\put(773.0,309.0){\rule[-0.200pt]{0.400pt}{3.373pt}}
\put(773.67,319){\rule{0.400pt}{2.168pt}}
\multiput(773.17,323.50)(1.000,-4.500){2}{\rule{0.400pt}{1.084pt}}
\put(774.0,325.0){\rule[-0.200pt]{0.400pt}{0.723pt}}
\put(775.0,297.0){\rule[-0.200pt]{0.400pt}{5.300pt}}
\put(774.67,307){\rule{0.400pt}{2.891pt}}
\multiput(774.17,313.00)(1.000,-6.000){2}{\rule{0.400pt}{1.445pt}}
\put(775.0,297.0){\rule[-0.200pt]{0.400pt}{5.300pt}}
\put(776.0,307.0){\rule[-0.200pt]{0.400pt}{1.927pt}}
\put(775.67,297){\rule{0.400pt}{3.854pt}}
\multiput(775.17,305.00)(1.000,-8.000){2}{\rule{0.400pt}{1.927pt}}
\put(776.0,313.0){\rule[-0.200pt]{0.400pt}{0.482pt}}
\put(776.67,319){\rule{0.400pt}{0.964pt}}
\multiput(776.17,321.00)(1.000,-2.000){2}{\rule{0.400pt}{0.482pt}}
\put(777.0,297.0){\rule[-0.200pt]{0.400pt}{6.263pt}}
\put(777.67,318){\rule{0.400pt}{1.204pt}}
\multiput(777.17,320.50)(1.000,-2.500){2}{\rule{0.400pt}{0.602pt}}
\put(778.0,319.0){\rule[-0.200pt]{0.400pt}{0.964pt}}
\put(779,318){\usebox{\plotpoint}}
\put(778.67,313){\rule{0.400pt}{2.168pt}}
\multiput(778.17,317.50)(1.000,-4.500){2}{\rule{0.400pt}{1.084pt}}
\put(779.0,318.0){\rule[-0.200pt]{0.400pt}{0.964pt}}
\put(780.0,313.0){\rule[-0.200pt]{0.400pt}{2.168pt}}
\put(779.67,320){\rule{0.400pt}{0.482pt}}
\multiput(779.17,320.00)(1.000,1.000){2}{\rule{0.400pt}{0.241pt}}
\put(780.0,320.0){\rule[-0.200pt]{0.400pt}{0.482pt}}
\put(780.67,296){\rule{0.400pt}{2.168pt}}
\multiput(780.17,300.50)(1.000,-4.500){2}{\rule{0.400pt}{1.084pt}}
\put(781.0,305.0){\rule[-0.200pt]{0.400pt}{4.095pt}}
\put(781.67,307){\rule{0.400pt}{2.409pt}}
\multiput(781.17,307.00)(1.000,5.000){2}{\rule{0.400pt}{1.204pt}}
\put(782.0,296.0){\rule[-0.200pt]{0.400pt}{2.650pt}}
\put(783.0,317.0){\rule[-0.200pt]{0.400pt}{0.482pt}}
\put(782.67,312){\rule{0.400pt}{0.964pt}}
\multiput(782.17,312.00)(1.000,2.000){2}{\rule{0.400pt}{0.482pt}}
\put(783.0,312.0){\rule[-0.200pt]{0.400pt}{1.686pt}}
\put(784.0,295.0){\rule[-0.200pt]{0.400pt}{5.059pt}}
\put(783.67,317){\rule{0.400pt}{0.482pt}}
\multiput(783.17,318.00)(1.000,-1.000){2}{\rule{0.400pt}{0.241pt}}
\put(784.0,295.0){\rule[-0.200pt]{0.400pt}{5.782pt}}
\put(785.0,304.0){\rule[-0.200pt]{0.400pt}{3.132pt}}
\put(784.67,293){\rule{0.400pt}{5.059pt}}
\multiput(784.17,303.50)(1.000,-10.500){2}{\rule{0.400pt}{2.529pt}}
\put(785.0,304.0){\rule[-0.200pt]{0.400pt}{2.409pt}}
\put(785.67,321){\rule{0.400pt}{0.964pt}}
\multiput(785.17,323.00)(1.000,-2.000){2}{\rule{0.400pt}{0.482pt}}
\put(786.0,293.0){\rule[-0.200pt]{0.400pt}{7.709pt}}
\put(787.0,317.0){\rule[-0.200pt]{0.400pt}{0.964pt}}
\put(786.67,292){\rule{0.400pt}{6.745pt}}
\multiput(786.17,306.00)(1.000,-14.000){2}{\rule{0.400pt}{3.373pt}}
\put(787.0,317.0){\rule[-0.200pt]{0.400pt}{0.723pt}}
\put(788.0,292.0){\rule[-0.200pt]{0.400pt}{7.709pt}}
\put(787.67,303){\rule{0.400pt}{2.650pt}}
\multiput(787.17,303.00)(1.000,5.500){2}{\rule{0.400pt}{1.325pt}}
\put(788.0,303.0){\rule[-0.200pt]{0.400pt}{5.059pt}}
\put(788.67,297){\rule{0.400pt}{2.891pt}}
\multiput(788.17,297.00)(1.000,6.000){2}{\rule{0.400pt}{1.445pt}}
\put(789.0,297.0){\rule[-0.200pt]{0.400pt}{4.095pt}}
\put(789.67,296){\rule{0.400pt}{0.723pt}}
\multiput(789.17,296.00)(1.000,1.500){2}{\rule{0.400pt}{0.361pt}}
\put(790.0,296.0){\rule[-0.200pt]{0.400pt}{3.132pt}}
\put(791.0,299.0){\rule[-0.200pt]{0.400pt}{2.409pt}}
\put(790.67,297){\rule{0.400pt}{5.541pt}}
\multiput(790.17,297.00)(1.000,11.500){2}{\rule{0.400pt}{2.770pt}}
\put(791.0,297.0){\rule[-0.200pt]{0.400pt}{2.891pt}}
\put(792.0,313.0){\rule[-0.200pt]{0.400pt}{1.686pt}}
\put(791.67,294){\rule{0.400pt}{5.059pt}}
\multiput(791.17,304.50)(1.000,-10.500){2}{\rule{0.400pt}{2.529pt}}
\put(792.0,313.0){\rule[-0.200pt]{0.400pt}{0.482pt}}
\put(792.67,314){\rule{0.400pt}{0.482pt}}
\multiput(792.17,315.00)(1.000,-1.000){2}{\rule{0.400pt}{0.241pt}}
\put(793.0,294.0){\rule[-0.200pt]{0.400pt}{5.300pt}}
\put(794.0,297.0){\rule[-0.200pt]{0.400pt}{4.095pt}}
\put(793.67,314){\rule{0.400pt}{2.891pt}}
\multiput(793.17,314.00)(1.000,6.000){2}{\rule{0.400pt}{1.445pt}}
\put(794.0,297.0){\rule[-0.200pt]{0.400pt}{4.095pt}}
\put(794.67,296){\rule{0.400pt}{1.686pt}}
\multiput(794.17,296.00)(1.000,3.500){2}{\rule{0.400pt}{0.843pt}}
\put(795.0,296.0){\rule[-0.200pt]{0.400pt}{7.227pt}}
\put(796.0,303.0){\rule[-0.200pt]{0.400pt}{1.445pt}}
\put(795.67,301){\rule{0.400pt}{1.927pt}}
\multiput(795.17,301.00)(1.000,4.000){2}{\rule{0.400pt}{0.964pt}}
\put(796.0,301.0){\rule[-0.200pt]{0.400pt}{1.927pt}}
\put(796.67,291){\rule{0.400pt}{6.986pt}}
\multiput(796.17,291.00)(1.000,14.500){2}{\rule{0.400pt}{3.493pt}}
\put(797.0,291.0){\rule[-0.200pt]{0.400pt}{4.336pt}}
\put(797.67,298){\rule{0.400pt}{3.614pt}}
\multiput(797.17,298.00)(1.000,7.500){2}{\rule{0.400pt}{1.807pt}}
\put(798.0,298.0){\rule[-0.200pt]{0.400pt}{5.300pt}}
\put(798.67,294){\rule{0.400pt}{2.168pt}}
\multiput(798.17,298.50)(1.000,-4.500){2}{\rule{0.400pt}{1.084pt}}
\put(799.0,303.0){\rule[-0.200pt]{0.400pt}{2.409pt}}
\put(800.0,294.0){\rule[-0.200pt]{0.400pt}{5.300pt}}
\put(799.67,293){\rule{0.400pt}{3.132pt}}
\multiput(799.17,293.00)(1.000,6.500){2}{\rule{0.400pt}{1.566pt}}
\put(800.0,293.0){\rule[-0.200pt]{0.400pt}{5.541pt}}
\put(801.0,294.0){\rule[-0.200pt]{0.400pt}{2.891pt}}
\put(800.67,305){\rule{0.400pt}{1.927pt}}
\multiput(800.17,309.00)(1.000,-4.000){2}{\rule{0.400pt}{0.964pt}}
\put(801.0,294.0){\rule[-0.200pt]{0.400pt}{4.577pt}}
\put(801.67,294){\rule{0.400pt}{0.482pt}}
\multiput(801.17,295.00)(1.000,-1.000){2}{\rule{0.400pt}{0.241pt}}
\put(802.0,296.0){\rule[-0.200pt]{0.400pt}{2.168pt}}
\put(802.67,302){\rule{0.400pt}{3.132pt}}
\multiput(802.17,308.50)(1.000,-6.500){2}{\rule{0.400pt}{1.566pt}}
\put(803.0,294.0){\rule[-0.200pt]{0.400pt}{5.059pt}}
\put(803.67,301){\rule{0.400pt}{4.095pt}}
\multiput(803.17,309.50)(1.000,-8.500){2}{\rule{0.400pt}{2.048pt}}
\put(804.0,302.0){\rule[-0.200pt]{0.400pt}{3.854pt}}
\put(804.67,291){\rule{0.400pt}{5.300pt}}
\multiput(804.17,291.00)(1.000,11.000){2}{\rule{0.400pt}{2.650pt}}
\put(805.0,291.0){\rule[-0.200pt]{0.400pt}{2.409pt}}
\put(805.67,295){\rule{0.400pt}{4.818pt}}
\multiput(805.17,295.00)(1.000,10.000){2}{\rule{0.400pt}{2.409pt}}
\put(806.0,295.0){\rule[-0.200pt]{0.400pt}{4.336pt}}
\put(806.67,314){\rule{0.400pt}{0.482pt}}
\multiput(806.17,315.00)(1.000,-1.000){2}{\rule{0.400pt}{0.241pt}}
\put(807.0,315.0){\usebox{\plotpoint}}
\put(807.67,296){\rule{0.400pt}{1.204pt}}
\multiput(807.17,298.50)(1.000,-2.500){2}{\rule{0.400pt}{0.602pt}}
\put(808.0,301.0){\rule[-0.200pt]{0.400pt}{3.132pt}}
\put(809.0,294.0){\rule[-0.200pt]{0.400pt}{0.482pt}}
\put(808.67,309){\rule{0.400pt}{2.891pt}}
\multiput(808.17,315.00)(1.000,-6.000){2}{\rule{0.400pt}{1.445pt}}
\put(809.0,294.0){\rule[-0.200pt]{0.400pt}{6.504pt}}
\put(809.67,296){\rule{0.400pt}{5.059pt}}
\multiput(809.17,296.00)(1.000,10.500){2}{\rule{0.400pt}{2.529pt}}
\put(810.0,296.0){\rule[-0.200pt]{0.400pt}{3.132pt}}
\put(810.67,289){\rule{0.400pt}{6.504pt}}
\multiput(810.17,302.50)(1.000,-13.500){2}{\rule{0.400pt}{3.252pt}}
\put(811.0,316.0){\usebox{\plotpoint}}
\put(812.0,289.0){\rule[-0.200pt]{0.400pt}{7.468pt}}
\put(811.67,289){\rule{0.400pt}{4.818pt}}
\multiput(811.17,299.00)(1.000,-10.000){2}{\rule{0.400pt}{2.409pt}}
\put(812.0,309.0){\rule[-0.200pt]{0.400pt}{2.650pt}}
\put(813.0,289.0){\rule[-0.200pt]{0.400pt}{7.468pt}}
\put(812.67,298){\rule{0.400pt}{2.168pt}}
\multiput(812.17,302.50)(1.000,-4.500){2}{\rule{0.400pt}{1.084pt}}
\put(813.0,307.0){\rule[-0.200pt]{0.400pt}{3.132pt}}
\put(814.0,298.0){\rule[-0.200pt]{0.400pt}{3.854pt}}
\put(813.67,312){\rule{0.400pt}{1.927pt}}
\multiput(813.17,312.00)(1.000,4.000){2}{\rule{0.400pt}{0.964pt}}
\put(814.0,312.0){\rule[-0.200pt]{0.400pt}{0.482pt}}
\put(814.67,315){\rule{0.400pt}{1.927pt}}
\multiput(814.17,315.00)(1.000,4.000){2}{\rule{0.400pt}{0.964pt}}
\put(815.0,315.0){\rule[-0.200pt]{0.400pt}{1.204pt}}
\put(815.67,291){\rule{0.400pt}{0.964pt}}
\multiput(815.17,293.00)(1.000,-2.000){2}{\rule{0.400pt}{0.482pt}}
\put(816.0,295.0){\rule[-0.200pt]{0.400pt}{6.745pt}}
\put(816.67,314){\rule{0.400pt}{1.445pt}}
\multiput(816.17,314.00)(1.000,3.000){2}{\rule{0.400pt}{0.723pt}}
\put(817.0,291.0){\rule[-0.200pt]{0.400pt}{5.541pt}}
\put(818.0,289.0){\rule[-0.200pt]{0.400pt}{7.468pt}}
\put(817.67,314){\rule{0.400pt}{1.686pt}}
\multiput(817.17,314.00)(1.000,3.500){2}{\rule{0.400pt}{0.843pt}}
\put(818.0,289.0){\rule[-0.200pt]{0.400pt}{6.022pt}}
\put(818.67,289){\rule{0.400pt}{0.964pt}}
\multiput(818.17,289.00)(1.000,2.000){2}{\rule{0.400pt}{0.482pt}}
\put(819.0,289.0){\rule[-0.200pt]{0.400pt}{7.709pt}}
\put(820,291.67){\rule{0.241pt}{0.400pt}}
\multiput(820.00,291.17)(0.500,1.000){2}{\rule{0.120pt}{0.400pt}}
\put(820.0,292.0){\usebox{\plotpoint}}
\put(821.0,293.0){\rule[-0.200pt]{0.400pt}{1.686pt}}
\put(820.67,294){\rule{0.400pt}{0.723pt}}
\multiput(820.17,295.50)(1.000,-1.500){2}{\rule{0.400pt}{0.361pt}}
\put(821.0,297.0){\rule[-0.200pt]{0.400pt}{0.723pt}}
\put(822.0,294.0){\rule[-0.200pt]{0.400pt}{0.964pt}}
\put(821.67,287){\rule{0.400pt}{0.964pt}}
\multiput(821.17,287.00)(1.000,2.000){2}{\rule{0.400pt}{0.482pt}}
\put(822.0,287.0){\rule[-0.200pt]{0.400pt}{2.650pt}}
\put(823.0,291.0){\rule[-0.200pt]{0.400pt}{1.927pt}}
\put(822.67,290){\rule{0.400pt}{2.409pt}}
\multiput(822.17,290.00)(1.000,5.000){2}{\rule{0.400pt}{1.204pt}}
\put(823.0,290.0){\rule[-0.200pt]{0.400pt}{2.168pt}}
\put(823.67,291){\rule{0.400pt}{5.541pt}}
\multiput(823.17,302.50)(1.000,-11.500){2}{\rule{0.400pt}{2.770pt}}
\put(824.0,300.0){\rule[-0.200pt]{0.400pt}{3.373pt}}
\put(825.0,291.0){\rule[-0.200pt]{0.400pt}{5.541pt}}
\put(824.67,313){\rule{0.400pt}{0.964pt}}
\multiput(824.17,313.00)(1.000,2.000){2}{\rule{0.400pt}{0.482pt}}
\put(825.0,313.0){\usebox{\plotpoint}}
\put(826.0,297.0){\rule[-0.200pt]{0.400pt}{4.818pt}}
\put(826.0,297.0){\usebox{\plotpoint}}
\put(827.0,297.0){\rule[-0.200pt]{0.400pt}{5.300pt}}
\put(827.0,319.0){\usebox{\plotpoint}}
\put(827.67,290){\rule{0.400pt}{3.614pt}}
\multiput(827.17,297.50)(1.000,-7.500){2}{\rule{0.400pt}{1.807pt}}
\put(828.0,305.0){\rule[-0.200pt]{0.400pt}{3.373pt}}
\put(829.0,290.0){\rule[-0.200pt]{0.400pt}{0.964pt}}
\put(829.0,294.0){\usebox{\plotpoint}}
\put(830.0,294.0){\rule[-0.200pt]{0.400pt}{3.373pt}}
\put(829.67,293){\rule{0.400pt}{1.927pt}}
\multiput(829.17,293.00)(1.000,4.000){2}{\rule{0.400pt}{0.964pt}}
\put(830.0,293.0){\rule[-0.200pt]{0.400pt}{3.613pt}}
\put(831.0,288.0){\rule[-0.200pt]{0.400pt}{3.132pt}}
\put(830.67,289){\rule{0.400pt}{4.818pt}}
\multiput(830.17,289.00)(1.000,10.000){2}{\rule{0.400pt}{2.409pt}}
\put(831.0,288.0){\usebox{\plotpoint}}
\put(831.67,287){\rule{0.400pt}{3.132pt}}
\multiput(831.17,287.00)(1.000,6.500){2}{\rule{0.400pt}{1.566pt}}
\put(832.0,287.0){\rule[-0.200pt]{0.400pt}{5.300pt}}
\put(833.0,290.0){\rule[-0.200pt]{0.400pt}{2.409pt}}
\put(832.67,296){\rule{0.400pt}{5.300pt}}
\multiput(832.17,307.00)(1.000,-11.000){2}{\rule{0.400pt}{2.650pt}}
\put(833.0,290.0){\rule[-0.200pt]{0.400pt}{6.745pt}}
\put(833.67,308){\rule{0.400pt}{0.723pt}}
\multiput(833.17,309.50)(1.000,-1.500){2}{\rule{0.400pt}{0.361pt}}
\put(834.0,296.0){\rule[-0.200pt]{0.400pt}{3.613pt}}
\put(835.0,308.0){\rule[-0.200pt]{0.400pt}{0.723pt}}
\put(834.67,300){\rule{0.400pt}{0.723pt}}
\multiput(834.17,301.50)(1.000,-1.500){2}{\rule{0.400pt}{0.361pt}}
\put(835.0,303.0){\rule[-0.200pt]{0.400pt}{1.927pt}}
\put(835.67,293){\rule{0.400pt}{2.168pt}}
\multiput(835.17,297.50)(1.000,-4.500){2}{\rule{0.400pt}{1.084pt}}
\put(836.0,300.0){\rule[-0.200pt]{0.400pt}{0.482pt}}
\put(837.0,293.0){\rule[-0.200pt]{0.400pt}{5.300pt}}
\put(836.67,299){\rule{0.400pt}{4.336pt}}
\multiput(836.17,299.00)(1.000,9.000){2}{\rule{0.400pt}{2.168pt}}
\put(837.0,299.0){\rule[-0.200pt]{0.400pt}{3.854pt}}
\put(838.0,308.0){\rule[-0.200pt]{0.400pt}{2.168pt}}
\put(837.67,293){\rule{0.400pt}{3.854pt}}
\multiput(837.17,301.00)(1.000,-8.000){2}{\rule{0.400pt}{1.927pt}}
\put(838.0,308.0){\usebox{\plotpoint}}
\put(839.0,293.0){\rule[-0.200pt]{0.400pt}{1.927pt}}
\put(839.0,301.0){\usebox{\plotpoint}}
\put(839.67,302){\rule{0.400pt}{3.854pt}}
\multiput(839.17,302.00)(1.000,8.000){2}{\rule{0.400pt}{1.927pt}}
\put(840.0,301.0){\usebox{\plotpoint}}
\put(841.0,288.0){\rule[-0.200pt]{0.400pt}{7.227pt}}
\put(840.67,302){\rule{0.400pt}{3.614pt}}
\multiput(840.17,309.50)(1.000,-7.500){2}{\rule{0.400pt}{1.807pt}}
\put(841.0,288.0){\rule[-0.200pt]{0.400pt}{6.986pt}}
\put(842.0,302.0){\rule[-0.200pt]{0.400pt}{1.927pt}}
\put(842.0,287.0){\rule[-0.200pt]{0.400pt}{5.541pt}}
\put(842.0,287.0){\usebox{\plotpoint}}
\put(842.67,289){\rule{0.400pt}{7.227pt}}
\multiput(842.17,304.00)(1.000,-15.000){2}{\rule{0.400pt}{3.613pt}}
\put(843.0,287.0){\rule[-0.200pt]{0.400pt}{7.709pt}}
\put(843.67,296){\rule{0.400pt}{1.204pt}}
\multiput(843.17,298.50)(1.000,-2.500){2}{\rule{0.400pt}{0.602pt}}
\put(844.0,289.0){\rule[-0.200pt]{0.400pt}{2.891pt}}
\put(845.0,286.0){\rule[-0.200pt]{0.400pt}{2.409pt}}
\put(845.0,286.0){\usebox{\plotpoint}}
\put(846.0,286.0){\rule[-0.200pt]{0.400pt}{5.300pt}}
\put(845.67,296){\rule{0.400pt}{1.204pt}}
\multiput(845.17,298.50)(1.000,-2.500){2}{\rule{0.400pt}{0.602pt}}
\put(846.0,301.0){\rule[-0.200pt]{0.400pt}{1.686pt}}
\put(847.0,290.0){\rule[-0.200pt]{0.400pt}{1.445pt}}
\put(846.67,286){\rule{0.400pt}{3.854pt}}
\multiput(846.17,294.00)(1.000,-8.000){2}{\rule{0.400pt}{1.927pt}}
\put(847.0,290.0){\rule[-0.200pt]{0.400pt}{2.891pt}}
\put(848.0,286.0){\rule[-0.200pt]{0.400pt}{2.409pt}}
\put(848,289.67){\rule{0.241pt}{0.400pt}}
\multiput(848.00,289.17)(0.500,1.000){2}{\rule{0.120pt}{0.400pt}}
\put(848.0,290.0){\rule[-0.200pt]{0.400pt}{1.445pt}}
\put(848.67,288){\rule{0.400pt}{3.132pt}}
\multiput(848.17,288.00)(1.000,6.500){2}{\rule{0.400pt}{1.566pt}}
\put(849.0,288.0){\rule[-0.200pt]{0.400pt}{0.723pt}}
\put(850.0,290.0){\rule[-0.200pt]{0.400pt}{2.650pt}}
\put(849.67,297){\rule{0.400pt}{2.409pt}}
\multiput(849.17,302.00)(1.000,-5.000){2}{\rule{0.400pt}{1.204pt}}
\put(850.0,290.0){\rule[-0.200pt]{0.400pt}{4.095pt}}
\put(851.0,297.0){\rule[-0.200pt]{0.400pt}{5.059pt}}
\put(850.67,310){\rule{0.400pt}{1.445pt}}
\multiput(850.17,313.00)(1.000,-3.000){2}{\rule{0.400pt}{0.723pt}}
\put(851.0,316.0){\rule[-0.200pt]{0.400pt}{0.482pt}}
\put(852,310){\usebox{\plotpoint}}
\put(852,310){\usebox{\plotpoint}}
\put(851.67,289){\rule{0.400pt}{5.059pt}}
\multiput(851.17,299.50)(1.000,-10.500){2}{\rule{0.400pt}{2.529pt}}
\put(852.67,298){\rule{0.400pt}{0.482pt}}
\multiput(852.17,298.00)(1.000,1.000){2}{\rule{0.400pt}{0.241pt}}
\put(853.0,289.0){\rule[-0.200pt]{0.400pt}{2.168pt}}
\put(854.0,291.0){\rule[-0.200pt]{0.400pt}{2.168pt}}
\put(853.67,293){\rule{0.400pt}{2.891pt}}
\multiput(853.17,299.00)(1.000,-6.000){2}{\rule{0.400pt}{1.445pt}}
\put(854.0,291.0){\rule[-0.200pt]{0.400pt}{3.373pt}}
\put(855.0,292.0){\usebox{\plotpoint}}
\put(855.0,292.0){\rule[-0.200pt]{0.400pt}{4.336pt}}
\put(855.0,310.0){\usebox{\plotpoint}}
\put(855.67,296){\rule{0.400pt}{3.373pt}}
\multiput(855.17,296.00)(1.000,7.000){2}{\rule{0.400pt}{1.686pt}}
\put(856.0,296.0){\rule[-0.200pt]{0.400pt}{3.373pt}}
\put(856.67,292){\rule{0.400pt}{5.300pt}}
\multiput(856.17,303.00)(1.000,-11.000){2}{\rule{0.400pt}{2.650pt}}
\put(857.0,310.0){\rule[-0.200pt]{0.400pt}{0.964pt}}
\put(858.0,292.0){\rule[-0.200pt]{0.400pt}{3.373pt}}
\put(857.67,298){\rule{0.400pt}{4.336pt}}
\multiput(857.17,298.00)(1.000,9.000){2}{\rule{0.400pt}{2.168pt}}
\put(858.0,298.0){\rule[-0.200pt]{0.400pt}{1.927pt}}
\put(859.0,316.0){\usebox{\plotpoint}}
\put(858.67,298){\rule{0.400pt}{3.614pt}}
\multiput(858.17,305.50)(1.000,-7.500){2}{\rule{0.400pt}{1.807pt}}
\put(859.0,313.0){\rule[-0.200pt]{0.400pt}{0.964pt}}
\put(860.0,298.0){\rule[-0.200pt]{0.400pt}{2.650pt}}
\put(859.67,302){\rule{0.400pt}{3.373pt}}
\multiput(859.17,302.00)(1.000,7.000){2}{\rule{0.400pt}{1.686pt}}
\put(860.0,302.0){\rule[-0.200pt]{0.400pt}{1.686pt}}
\put(860.67,286){\rule{0.400pt}{0.964pt}}
\multiput(860.17,288.00)(1.000,-2.000){2}{\rule{0.400pt}{0.482pt}}
\put(861.0,290.0){\rule[-0.200pt]{0.400pt}{6.263pt}}
\put(862.0,286.0){\rule[-0.200pt]{0.400pt}{4.818pt}}
\put(861.67,293){\rule{0.400pt}{1.445pt}}
\multiput(861.17,296.00)(1.000,-3.000){2}{\rule{0.400pt}{0.723pt}}
\put(862.0,299.0){\rule[-0.200pt]{0.400pt}{1.686pt}}
\put(863.0,289.0){\rule[-0.200pt]{0.400pt}{0.964pt}}
\put(862.67,296){\rule{0.400pt}{0.482pt}}
\multiput(862.17,296.00)(1.000,1.000){2}{\rule{0.400pt}{0.241pt}}
\put(863.0,289.0){\rule[-0.200pt]{0.400pt}{1.686pt}}
\put(864.0,294.0){\rule[-0.200pt]{0.400pt}{0.964pt}}
\put(863.67,299){\rule{0.400pt}{1.686pt}}
\multiput(863.17,302.50)(1.000,-3.500){2}{\rule{0.400pt}{0.843pt}}
\put(864.0,294.0){\rule[-0.200pt]{0.400pt}{2.891pt}}
\put(864.67,292){\rule{0.400pt}{4.818pt}}
\multiput(864.17,302.00)(1.000,-10.000){2}{\rule{0.400pt}{2.409pt}}
\put(865.0,299.0){\rule[-0.200pt]{0.400pt}{3.132pt}}
\put(865.67,298){\rule{0.400pt}{4.095pt}}
\multiput(865.17,298.00)(1.000,8.500){2}{\rule{0.400pt}{2.048pt}}
\put(866.0,292.0){\rule[-0.200pt]{0.400pt}{1.445pt}}
\put(867,315){\usebox{\plotpoint}}
\put(866.67,295){\rule{0.400pt}{0.482pt}}
\multiput(866.17,296.00)(1.000,-1.000){2}{\rule{0.400pt}{0.241pt}}
\put(867.0,297.0){\rule[-0.200pt]{0.400pt}{4.336pt}}
\put(868.0,295.0){\rule[-0.200pt]{0.400pt}{5.059pt}}
\put(867.67,283){\rule{0.400pt}{3.614pt}}
\multiput(867.17,283.00)(1.000,7.500){2}{\rule{0.400pt}{1.807pt}}
\put(868.0,283.0){\rule[-0.200pt]{0.400pt}{7.950pt}}
\put(869.0,295.0){\rule[-0.200pt]{0.400pt}{0.723pt}}
\put(868.67,296){\rule{0.400pt}{4.095pt}}
\multiput(868.17,304.50)(1.000,-8.500){2}{\rule{0.400pt}{2.048pt}}
\put(869.0,295.0){\rule[-0.200pt]{0.400pt}{4.336pt}}
\put(869.67,286){\rule{0.400pt}{0.964pt}}
\multiput(869.17,286.00)(1.000,2.000){2}{\rule{0.400pt}{0.482pt}}
\put(870.0,286.0){\rule[-0.200pt]{0.400pt}{2.409pt}}
\put(871.0,288.0){\rule[-0.200pt]{0.400pt}{0.482pt}}
\put(870.67,286){\rule{0.400pt}{4.818pt}}
\multiput(870.17,296.00)(1.000,-10.000){2}{\rule{0.400pt}{2.409pt}}
\put(871.0,288.0){\rule[-0.200pt]{0.400pt}{4.336pt}}
\put(871.67,289){\rule{0.400pt}{4.095pt}}
\multiput(871.17,297.50)(1.000,-8.500){2}{\rule{0.400pt}{2.048pt}}
\put(872.0,286.0){\rule[-0.200pt]{0.400pt}{4.818pt}}
\put(872.67,308){\rule{0.400pt}{1.204pt}}
\multiput(872.17,310.50)(1.000,-2.500){2}{\rule{0.400pt}{0.602pt}}
\put(873.0,289.0){\rule[-0.200pt]{0.400pt}{5.782pt}}
\put(873.67,306){\rule{0.400pt}{2.168pt}}
\multiput(873.17,306.00)(1.000,4.500){2}{\rule{0.400pt}{1.084pt}}
\put(874.0,306.0){\rule[-0.200pt]{0.400pt}{0.482pt}}
\put(875.0,296.0){\rule[-0.200pt]{0.400pt}{4.577pt}}
\put(874.67,298){\rule{0.400pt}{3.614pt}}
\multiput(874.17,298.00)(1.000,7.500){2}{\rule{0.400pt}{1.807pt}}
\put(875.0,296.0){\rule[-0.200pt]{0.400pt}{0.482pt}}
\put(876.0,285.0){\rule[-0.200pt]{0.400pt}{6.745pt}}
\put(876.0,285.0){\usebox{\plotpoint}}
\put(877.0,281.0){\rule[-0.200pt]{0.400pt}{0.964pt}}
\put(876.67,284){\rule{0.400pt}{3.854pt}}
\multiput(876.17,292.00)(1.000,-8.000){2}{\rule{0.400pt}{1.927pt}}
\put(877.0,281.0){\rule[-0.200pt]{0.400pt}{4.577pt}}
\put(877.67,283){\rule{0.400pt}{3.373pt}}
\multiput(877.17,283.00)(1.000,7.000){2}{\rule{0.400pt}{1.686pt}}
\put(878.0,283.0){\usebox{\plotpoint}}
\put(878.67,308){\rule{0.400pt}{1.204pt}}
\multiput(878.17,310.50)(1.000,-2.500){2}{\rule{0.400pt}{0.602pt}}
\put(879.0,297.0){\rule[-0.200pt]{0.400pt}{3.854pt}}
\put(879.67,285){\rule{0.400pt}{3.132pt}}
\multiput(879.17,285.00)(1.000,6.500){2}{\rule{0.400pt}{1.566pt}}
\put(880.0,285.0){\rule[-0.200pt]{0.400pt}{5.541pt}}
\put(881.0,291.0){\rule[-0.200pt]{0.400pt}{1.686pt}}
\put(880.67,299){\rule{0.400pt}{3.614pt}}
\multiput(880.17,299.00)(1.000,7.500){2}{\rule{0.400pt}{1.807pt}}
\put(881.0,291.0){\rule[-0.200pt]{0.400pt}{1.927pt}}
\put(881.67,284){\rule{0.400pt}{2.891pt}}
\multiput(881.17,290.00)(1.000,-6.000){2}{\rule{0.400pt}{1.445pt}}
\put(882.0,296.0){\rule[-0.200pt]{0.400pt}{4.336pt}}
\put(882.67,285){\rule{0.400pt}{3.854pt}}
\multiput(882.17,293.00)(1.000,-8.000){2}{\rule{0.400pt}{1.927pt}}
\put(883.0,284.0){\rule[-0.200pt]{0.400pt}{4.095pt}}
\put(884.0,281.0){\rule[-0.200pt]{0.400pt}{0.964pt}}
\put(883.67,286){\rule{0.400pt}{4.577pt}}
\multiput(883.17,295.50)(1.000,-9.500){2}{\rule{0.400pt}{2.289pt}}
\put(884.0,281.0){\rule[-0.200pt]{0.400pt}{5.782pt}}
\put(885.0,279.0){\rule[-0.200pt]{0.400pt}{1.686pt}}
\put(884.67,286){\rule{0.400pt}{5.782pt}}
\multiput(884.17,298.00)(1.000,-12.000){2}{\rule{0.400pt}{2.891pt}}
\put(885.0,279.0){\rule[-0.200pt]{0.400pt}{7.468pt}}
\put(885.67,282){\rule{0.400pt}{7.468pt}}
\multiput(885.17,297.50)(1.000,-15.500){2}{\rule{0.400pt}{3.734pt}}
\put(886.0,286.0){\rule[-0.200pt]{0.400pt}{6.504pt}}
\put(886.67,293){\rule{0.400pt}{3.132pt}}
\multiput(886.17,293.00)(1.000,6.500){2}{\rule{0.400pt}{1.566pt}}
\put(887.0,282.0){\rule[-0.200pt]{0.400pt}{2.650pt}}
\put(888.0,283.0){\rule[-0.200pt]{0.400pt}{5.541pt}}
\put(887.67,289){\rule{0.400pt}{3.132pt}}
\multiput(887.17,295.50)(1.000,-6.500){2}{\rule{0.400pt}{1.566pt}}
\put(888.0,283.0){\rule[-0.200pt]{0.400pt}{4.577pt}}
\put(889,280.67){\rule{0.241pt}{0.400pt}}
\multiput(889.00,280.17)(0.500,1.000){2}{\rule{0.120pt}{0.400pt}}
\put(889.0,281.0){\rule[-0.200pt]{0.400pt}{1.927pt}}
\put(889.67,299){\rule{0.400pt}{3.132pt}}
\multiput(889.17,305.50)(1.000,-6.500){2}{\rule{0.400pt}{1.566pt}}
\put(890.0,282.0){\rule[-0.200pt]{0.400pt}{7.227pt}}
\put(890.67,280){\rule{0.400pt}{0.964pt}}
\multiput(890.17,280.00)(1.000,2.000){2}{\rule{0.400pt}{0.482pt}}
\put(891.0,280.0){\rule[-0.200pt]{0.400pt}{4.577pt}}
\put(891.67,289){\rule{0.400pt}{1.445pt}}
\multiput(891.17,292.00)(1.000,-3.000){2}{\rule{0.400pt}{0.723pt}}
\put(892.0,284.0){\rule[-0.200pt]{0.400pt}{2.650pt}}
\put(893,289){\usebox{\plotpoint}}
\put(893,297.67){\rule{0.241pt}{0.400pt}}
\multiput(893.00,298.17)(0.500,-1.000){2}{\rule{0.120pt}{0.400pt}}
\put(893.0,289.0){\rule[-0.200pt]{0.400pt}{2.409pt}}
\put(894,298){\usebox{\plotpoint}}
\put(893.67,284){\rule{0.400pt}{5.059pt}}
\multiput(893.17,284.00)(1.000,10.500){2}{\rule{0.400pt}{2.529pt}}
\put(894.0,284.0){\rule[-0.200pt]{0.400pt}{3.373pt}}
\put(895.0,280.0){\rule[-0.200pt]{0.400pt}{6.022pt}}
\put(895.0,280.0){\usebox{\plotpoint}}
\put(895.67,294){\rule{0.400pt}{4.095pt}}
\multiput(895.17,294.00)(1.000,8.500){2}{\rule{0.400pt}{2.048pt}}
\put(896.0,280.0){\rule[-0.200pt]{0.400pt}{3.373pt}}
\put(896.67,289){\rule{0.400pt}{1.686pt}}
\multiput(896.17,292.50)(1.000,-3.500){2}{\rule{0.400pt}{0.843pt}}
\put(897.0,296.0){\rule[-0.200pt]{0.400pt}{3.613pt}}
\put(898.0,289.0){\rule[-0.200pt]{0.400pt}{0.723pt}}
\put(897.67,287){\rule{0.400pt}{2.650pt}}
\multiput(897.17,287.00)(1.000,5.500){2}{\rule{0.400pt}{1.325pt}}
\put(898.0,287.0){\rule[-0.200pt]{0.400pt}{1.204pt}}
\put(898.67,283){\rule{0.400pt}{2.650pt}}
\multiput(898.17,283.00)(1.000,5.500){2}{\rule{0.400pt}{1.325pt}}
\put(899.0,283.0){\rule[-0.200pt]{0.400pt}{3.613pt}}
\put(900.0,293.0){\usebox{\plotpoint}}
\put(899.67,282){\rule{0.400pt}{4.336pt}}
\multiput(899.17,291.00)(1.000,-9.000){2}{\rule{0.400pt}{2.168pt}}
\put(900.0,293.0){\rule[-0.200pt]{0.400pt}{1.686pt}}
\put(901.0,277.0){\rule[-0.200pt]{0.400pt}{1.204pt}}
\put(900.67,298){\rule{0.400pt}{2.650pt}}
\multiput(900.17,303.50)(1.000,-5.500){2}{\rule{0.400pt}{1.325pt}}
\put(901.0,277.0){\rule[-0.200pt]{0.400pt}{7.709pt}}
\put(901.67,281){\rule{0.400pt}{0.482pt}}
\multiput(901.17,281.00)(1.000,1.000){2}{\rule{0.400pt}{0.241pt}}
\put(902.0,281.0){\rule[-0.200pt]{0.400pt}{4.095pt}}
\put(902.67,291){\rule{0.400pt}{3.373pt}}
\multiput(902.17,298.00)(1.000,-7.000){2}{\rule{0.400pt}{1.686pt}}
\put(903.0,283.0){\rule[-0.200pt]{0.400pt}{5.300pt}}
\put(904.0,291.0){\rule[-0.200pt]{0.400pt}{2.891pt}}
\put(903.67,282){\rule{0.400pt}{0.723pt}}
\multiput(903.17,283.50)(1.000,-1.500){2}{\rule{0.400pt}{0.361pt}}
\put(904.0,285.0){\rule[-0.200pt]{0.400pt}{4.336pt}}
\put(905.0,282.0){\rule[-0.200pt]{0.400pt}{6.022pt}}
\put(904.67,281){\rule{0.400pt}{4.577pt}}
\multiput(904.17,281.00)(1.000,9.500){2}{\rule{0.400pt}{2.289pt}}
\put(905.0,281.0){\rule[-0.200pt]{0.400pt}{6.263pt}}
\put(906.0,291.0){\rule[-0.200pt]{0.400pt}{2.168pt}}
\put(905.67,277){\rule{0.400pt}{7.227pt}}
\multiput(905.17,292.00)(1.000,-15.000){2}{\rule{0.400pt}{3.613pt}}
\put(906.0,291.0){\rule[-0.200pt]{0.400pt}{3.854pt}}
\put(906.67,293){\rule{0.400pt}{0.482pt}}
\multiput(906.17,293.00)(1.000,1.000){2}{\rule{0.400pt}{0.241pt}}
\put(907.0,277.0){\rule[-0.200pt]{0.400pt}{3.854pt}}
\put(908.0,287.0){\rule[-0.200pt]{0.400pt}{1.927pt}}
\put(907.67,279){\rule{0.400pt}{7.709pt}}
\multiput(907.17,295.00)(1.000,-16.000){2}{\rule{0.400pt}{3.854pt}}
\put(908.0,287.0){\rule[-0.200pt]{0.400pt}{5.782pt}}
\put(908.67,284){\rule{0.400pt}{6.263pt}}
\multiput(908.17,297.00)(1.000,-13.000){2}{\rule{0.400pt}{3.132pt}}
\put(909.0,279.0){\rule[-0.200pt]{0.400pt}{7.468pt}}
\put(910.0,284.0){\rule[-0.200pt]{0.400pt}{2.409pt}}
\put(909.67,287){\rule{0.400pt}{0.482pt}}
\multiput(909.17,288.00)(1.000,-1.000){2}{\rule{0.400pt}{0.241pt}}
\put(910.0,289.0){\rule[-0.200pt]{0.400pt}{1.204pt}}
\put(910.67,297){\rule{0.400pt}{3.854pt}}
\multiput(910.17,297.00)(1.000,8.000){2}{\rule{0.400pt}{1.927pt}}
\put(911.0,287.0){\rule[-0.200pt]{0.400pt}{2.409pt}}
\put(911.67,287){\rule{0.400pt}{1.204pt}}
\multiput(911.17,289.50)(1.000,-2.500){2}{\rule{0.400pt}{0.602pt}}
\put(912.0,292.0){\rule[-0.200pt]{0.400pt}{5.059pt}}
\put(913.0,277.0){\rule[-0.200pt]{0.400pt}{2.409pt}}
\put(912.67,290){\rule{0.400pt}{3.132pt}}
\multiput(912.17,290.00)(1.000,6.500){2}{\rule{0.400pt}{1.566pt}}
\put(913.0,277.0){\rule[-0.200pt]{0.400pt}{3.132pt}}
\put(913.67,284){\rule{0.400pt}{5.059pt}}
\multiput(913.17,284.00)(1.000,10.500){2}{\rule{0.400pt}{2.529pt}}
\put(914.0,284.0){\rule[-0.200pt]{0.400pt}{4.577pt}}
\put(915.0,281.0){\rule[-0.200pt]{0.400pt}{5.782pt}}
\put(914.67,298){\rule{0.400pt}{2.409pt}}
\multiput(914.17,298.00)(1.000,5.000){2}{\rule{0.400pt}{1.204pt}}
\put(915.0,281.0){\rule[-0.200pt]{0.400pt}{4.095pt}}
\put(915.67,287){\rule{0.400pt}{6.023pt}}
\multiput(915.17,299.50)(1.000,-12.500){2}{\rule{0.400pt}{3.011pt}}
\put(916.0,308.0){\rule[-0.200pt]{0.400pt}{0.964pt}}
\put(917.0,286.0){\usebox{\plotpoint}}
\put(916.67,307){\rule{0.400pt}{0.964pt}}
\multiput(916.17,309.00)(1.000,-2.000){2}{\rule{0.400pt}{0.482pt}}
\put(917.0,286.0){\rule[-0.200pt]{0.400pt}{6.022pt}}
\put(918.0,278.0){\rule[-0.200pt]{0.400pt}{6.986pt}}
\put(917.67,280){\rule{0.400pt}{3.614pt}}
\multiput(917.17,287.50)(1.000,-7.500){2}{\rule{0.400pt}{1.807pt}}
\put(918.0,278.0){\rule[-0.200pt]{0.400pt}{4.095pt}}
\put(919,280){\usebox{\plotpoint}}
\put(918.67,282){\rule{0.400pt}{2.168pt}}
\multiput(918.17,286.50)(1.000,-4.500){2}{\rule{0.400pt}{1.084pt}}
\put(919.0,280.0){\rule[-0.200pt]{0.400pt}{2.650pt}}
\put(920,282){\usebox{\plotpoint}}
\put(919.67,278){\rule{0.400pt}{0.964pt}}
\multiput(919.17,280.00)(1.000,-2.000){2}{\rule{0.400pt}{0.482pt}}
\put(921.0,278.0){\rule[-0.200pt]{0.400pt}{4.336pt}}
\put(920.67,278){\rule{0.400pt}{0.723pt}}
\multiput(920.17,278.00)(1.000,1.500){2}{\rule{0.400pt}{0.361pt}}
\put(921.0,278.0){\rule[-0.200pt]{0.400pt}{4.336pt}}
\put(922.0,281.0){\rule[-0.200pt]{0.400pt}{5.059pt}}
\put(921.67,291){\rule{0.400pt}{2.168pt}}
\multiput(921.17,291.00)(1.000,4.500){2}{\rule{0.400pt}{1.084pt}}
\put(922.0,291.0){\rule[-0.200pt]{0.400pt}{2.650pt}}
\put(922.67,303){\rule{0.400pt}{1.686pt}}
\multiput(922.17,303.00)(1.000,3.500){2}{\rule{0.400pt}{0.843pt}}
\put(923.0,300.0){\rule[-0.200pt]{0.400pt}{0.723pt}}
\put(923.67,292){\rule{0.400pt}{3.373pt}}
\multiput(923.17,292.00)(1.000,7.000){2}{\rule{0.400pt}{1.686pt}}
\put(924.0,292.0){\rule[-0.200pt]{0.400pt}{4.336pt}}
\put(925.0,280.0){\rule[-0.200pt]{0.400pt}{6.263pt}}
\put(924.67,291){\rule{0.400pt}{1.204pt}}
\multiput(924.17,291.00)(1.000,2.500){2}{\rule{0.400pt}{0.602pt}}
\put(925.0,280.0){\rule[-0.200pt]{0.400pt}{2.650pt}}
\put(926.0,296.0){\rule[-0.200pt]{0.400pt}{2.650pt}}
\put(925.67,293){\rule{0.400pt}{1.204pt}}
\multiput(925.17,295.50)(1.000,-2.500){2}{\rule{0.400pt}{0.602pt}}
\put(926.0,298.0){\rule[-0.200pt]{0.400pt}{2.168pt}}
\put(927.0,293.0){\rule[-0.200pt]{0.400pt}{3.613pt}}
\put(926.67,290){\rule{0.400pt}{0.482pt}}
\multiput(926.17,291.00)(1.000,-1.000){2}{\rule{0.400pt}{0.241pt}}
\put(927.0,292.0){\rule[-0.200pt]{0.400pt}{3.854pt}}
\put(927.67,295){\rule{0.400pt}{1.445pt}}
\multiput(927.17,295.00)(1.000,3.000){2}{\rule{0.400pt}{0.723pt}}
\put(928.0,290.0){\rule[-0.200pt]{0.400pt}{1.204pt}}
\put(929.0,298.0){\rule[-0.200pt]{0.400pt}{0.723pt}}
\put(928.67,282){\rule{0.400pt}{6.504pt}}
\multiput(928.17,295.50)(1.000,-13.500){2}{\rule{0.400pt}{3.252pt}}
\put(929.0,298.0){\rule[-0.200pt]{0.400pt}{2.650pt}}
\put(929.67,279){\rule{0.400pt}{7.709pt}}
\multiput(929.17,295.00)(1.000,-16.000){2}{\rule{0.400pt}{3.854pt}}
\put(930.0,282.0){\rule[-0.200pt]{0.400pt}{6.986pt}}
\put(930.67,302){\rule{0.400pt}{2.409pt}}
\multiput(930.17,307.00)(1.000,-5.000){2}{\rule{0.400pt}{1.204pt}}
\put(931.0,279.0){\rule[-0.200pt]{0.400pt}{7.950pt}}
\put(931.67,286){\rule{0.400pt}{4.577pt}}
\multiput(931.17,295.50)(1.000,-9.500){2}{\rule{0.400pt}{2.289pt}}
\put(932.0,302.0){\rule[-0.200pt]{0.400pt}{0.723pt}}
\put(932.67,287){\rule{0.400pt}{0.964pt}}
\multiput(932.17,289.00)(1.000,-2.000){2}{\rule{0.400pt}{0.482pt}}
\put(933.0,286.0){\rule[-0.200pt]{0.400pt}{1.204pt}}
\put(934.0,278.0){\rule[-0.200pt]{0.400pt}{2.168pt}}
\put(933.67,278){\rule{0.400pt}{4.095pt}}
\multiput(933.17,286.50)(1.000,-8.500){2}{\rule{0.400pt}{2.048pt}}
\put(934.0,278.0){\rule[-0.200pt]{0.400pt}{4.095pt}}
\put(935,278){\usebox{\plotpoint}}
\put(934.67,306){\rule{0.400pt}{1.445pt}}
\multiput(934.17,309.00)(1.000,-3.000){2}{\rule{0.400pt}{0.723pt}}
\put(935.0,278.0){\rule[-0.200pt]{0.400pt}{8.191pt}}
\put(935.67,294){\rule{0.400pt}{4.336pt}}
\multiput(935.17,294.00)(1.000,9.000){2}{\rule{0.400pt}{2.168pt}}
\put(936.0,294.0){\rule[-0.200pt]{0.400pt}{2.891pt}}
\put(937.0,307.0){\rule[-0.200pt]{0.400pt}{1.204pt}}
\put(936.67,296){\rule{0.400pt}{3.614pt}}
\multiput(936.17,303.50)(1.000,-7.500){2}{\rule{0.400pt}{1.807pt}}
\put(937.0,307.0){\rule[-0.200pt]{0.400pt}{0.964pt}}
\put(937.67,277){\rule{0.400pt}{1.445pt}}
\multiput(937.17,280.00)(1.000,-3.000){2}{\rule{0.400pt}{0.723pt}}
\put(938.0,283.0){\rule[-0.200pt]{0.400pt}{3.132pt}}
\put(938.67,302){\rule{0.400pt}{2.650pt}}
\multiput(938.17,307.50)(1.000,-5.500){2}{\rule{0.400pt}{1.325pt}}
\put(939.0,277.0){\rule[-0.200pt]{0.400pt}{8.672pt}}
\put(940.0,286.0){\rule[-0.200pt]{0.400pt}{3.854pt}}
\put(939.67,295){\rule{0.400pt}{3.614pt}}
\multiput(939.17,302.50)(1.000,-7.500){2}{\rule{0.400pt}{1.807pt}}
\put(940.0,286.0){\rule[-0.200pt]{0.400pt}{5.782pt}}
\put(940.67,280){\rule{0.400pt}{4.818pt}}
\multiput(940.17,290.00)(1.000,-10.000){2}{\rule{0.400pt}{2.409pt}}
\put(941.0,295.0){\rule[-0.200pt]{0.400pt}{1.204pt}}
\put(942.0,280.0){\rule[-0.200pt]{0.400pt}{6.986pt}}
\put(941.67,296){\rule{0.400pt}{1.686pt}}
\multiput(941.17,296.00)(1.000,3.500){2}{\rule{0.400pt}{0.843pt}}
\put(942.0,296.0){\rule[-0.200pt]{0.400pt}{3.132pt}}
\put(943.0,296.0){\rule[-0.200pt]{0.400pt}{1.686pt}}
\put(942.67,280){\rule{0.400pt}{5.541pt}}
\multiput(942.17,291.50)(1.000,-11.500){2}{\rule{0.400pt}{2.770pt}}
\put(943.0,296.0){\rule[-0.200pt]{0.400pt}{1.686pt}}
\put(944.0,280.0){\rule[-0.200pt]{0.400pt}{1.927pt}}
\put(943.67,286){\rule{0.400pt}{4.818pt}}
\multiput(943.17,286.00)(1.000,10.000){2}{\rule{0.400pt}{2.409pt}}
\put(944.0,286.0){\rule[-0.200pt]{0.400pt}{0.482pt}}
\put(944.67,278){\rule{0.400pt}{2.650pt}}
\multiput(944.17,283.50)(1.000,-5.500){2}{\rule{0.400pt}{1.325pt}}
\put(945.0,289.0){\rule[-0.200pt]{0.400pt}{4.095pt}}
\put(945.67,285){\rule{0.400pt}{4.336pt}}
\multiput(945.17,294.00)(1.000,-9.000){2}{\rule{0.400pt}{2.168pt}}
\put(946.0,278.0){\rule[-0.200pt]{0.400pt}{6.022pt}}
\put(947.0,285.0){\rule[-0.200pt]{0.400pt}{2.409pt}}
\put(946.67,290){\rule{0.400pt}{1.686pt}}
\multiput(946.17,290.00)(1.000,3.500){2}{\rule{0.400pt}{0.843pt}}
\put(947.0,290.0){\rule[-0.200pt]{0.400pt}{1.204pt}}
\put(948.0,297.0){\rule[-0.200pt]{0.400pt}{0.723pt}}
\put(947.67,281){\rule{0.400pt}{6.263pt}}
\multiput(947.17,281.00)(1.000,13.000){2}{\rule{0.400pt}{3.132pt}}
\put(948.0,281.0){\rule[-0.200pt]{0.400pt}{4.577pt}}
\put(948.67,296){\rule{0.400pt}{1.445pt}}
\multiput(948.17,299.00)(1.000,-3.000){2}{\rule{0.400pt}{0.723pt}}
\put(949.0,302.0){\rule[-0.200pt]{0.400pt}{1.204pt}}
\put(950.0,279.0){\rule[-0.200pt]{0.400pt}{4.095pt}}
\put(949.67,303){\rule{0.400pt}{0.723pt}}
\multiput(949.17,303.00)(1.000,1.500){2}{\rule{0.400pt}{0.361pt}}
\put(950.0,279.0){\rule[-0.200pt]{0.400pt}{5.782pt}}
\put(951.0,295.0){\rule[-0.200pt]{0.400pt}{2.650pt}}
\put(950.67,309){\rule{0.400pt}{0.482pt}}
\multiput(950.17,309.00)(1.000,1.000){2}{\rule{0.400pt}{0.241pt}}
\put(951.0,295.0){\rule[-0.200pt]{0.400pt}{3.373pt}}
\put(952,285.67){\rule{0.241pt}{0.400pt}}
\multiput(952.00,285.17)(0.500,1.000){2}{\rule{0.120pt}{0.400pt}}
\put(952.0,286.0){\rule[-0.200pt]{0.400pt}{6.022pt}}
\put(952.67,286){\rule{0.400pt}{3.854pt}}
\multiput(952.17,294.00)(1.000,-8.000){2}{\rule{0.400pt}{1.927pt}}
\put(953.0,287.0){\rule[-0.200pt]{0.400pt}{3.613pt}}
\put(954.0,281.0){\rule[-0.200pt]{0.400pt}{1.204pt}}
\put(953.67,288){\rule{0.400pt}{5.300pt}}
\multiput(953.17,288.00)(1.000,11.000){2}{\rule{0.400pt}{2.650pt}}
\put(954.0,281.0){\rule[-0.200pt]{0.400pt}{1.686pt}}
\put(954.67,290){\rule{0.400pt}{1.927pt}}
\multiput(954.17,294.00)(1.000,-4.000){2}{\rule{0.400pt}{0.964pt}}
\put(955.0,298.0){\rule[-0.200pt]{0.400pt}{2.891pt}}
\put(955.67,303){\rule{0.400pt}{2.168pt}}
\multiput(955.17,307.50)(1.000,-4.500){2}{\rule{0.400pt}{1.084pt}}
\put(956.0,290.0){\rule[-0.200pt]{0.400pt}{5.300pt}}
\put(957,309.67){\rule{0.241pt}{0.400pt}}
\multiput(957.00,310.17)(0.500,-1.000){2}{\rule{0.120pt}{0.400pt}}
\put(957.0,303.0){\rule[-0.200pt]{0.400pt}{1.927pt}}
\put(957.67,283){\rule{0.400pt}{5.541pt}}
\multiput(957.17,283.00)(1.000,11.500){2}{\rule{0.400pt}{2.770pt}}
\put(958.0,283.0){\rule[-0.200pt]{0.400pt}{6.504pt}}
\put(959.0,295.0){\rule[-0.200pt]{0.400pt}{2.650pt}}
\put(958.67,280){\rule{0.400pt}{3.854pt}}
\multiput(958.17,288.00)(1.000,-8.000){2}{\rule{0.400pt}{1.927pt}}
\put(959.0,295.0){\usebox{\plotpoint}}
\put(959.67,293){\rule{0.400pt}{1.204pt}}
\multiput(959.17,293.00)(1.000,2.500){2}{\rule{0.400pt}{0.602pt}}
\put(960.0,280.0){\rule[-0.200pt]{0.400pt}{3.132pt}}
\put(960.67,278){\rule{0.400pt}{7.709pt}}
\multiput(960.17,278.00)(1.000,16.000){2}{\rule{0.400pt}{3.854pt}}
\put(961.0,278.0){\rule[-0.200pt]{0.400pt}{4.818pt}}
\put(962.0,285.0){\rule[-0.200pt]{0.400pt}{6.022pt}}
\put(961.67,282){\rule{0.400pt}{5.059pt}}
\multiput(961.17,292.50)(1.000,-10.500){2}{\rule{0.400pt}{2.529pt}}
\put(962.0,285.0){\rule[-0.200pt]{0.400pt}{4.336pt}}
\put(963.0,282.0){\rule[-0.200pt]{0.400pt}{1.445pt}}
\put(962.67,281){\rule{0.400pt}{0.723pt}}
\multiput(962.17,281.00)(1.000,1.500){2}{\rule{0.400pt}{0.361pt}}
\put(963.0,281.0){\rule[-0.200pt]{0.400pt}{1.686pt}}
\put(964.0,284.0){\rule[-0.200pt]{0.400pt}{5.059pt}}
\put(964,279.67){\rule{0.241pt}{0.400pt}}
\multiput(964.00,280.17)(0.500,-1.000){2}{\rule{0.120pt}{0.400pt}}
\put(964.0,281.0){\rule[-0.200pt]{0.400pt}{5.782pt}}
\put(965.0,280.0){\rule[-0.200pt]{0.400pt}{7.709pt}}
\put(964.67,296){\rule{0.400pt}{2.650pt}}
\multiput(964.17,301.50)(1.000,-5.500){2}{\rule{0.400pt}{1.325pt}}
\put(965.0,307.0){\rule[-0.200pt]{0.400pt}{1.204pt}}
\put(965.67,296){\rule{0.400pt}{4.095pt}}
\multiput(965.17,304.50)(1.000,-8.500){2}{\rule{0.400pt}{2.048pt}}
\put(966.0,296.0){\rule[-0.200pt]{0.400pt}{4.095pt}}
\put(967.0,287.0){\rule[-0.200pt]{0.400pt}{2.168pt}}
\put(966.67,289){\rule{0.400pt}{5.059pt}}
\multiput(966.17,299.50)(1.000,-10.500){2}{\rule{0.400pt}{2.529pt}}
\put(967.0,287.0){\rule[-0.200pt]{0.400pt}{5.541pt}}
\put(968,289){\usebox{\plotpoint}}
\put(967.67,288){\rule{0.400pt}{0.964pt}}
\multiput(967.17,290.00)(1.000,-2.000){2}{\rule{0.400pt}{0.482pt}}
\put(968.0,289.0){\rule[-0.200pt]{0.400pt}{0.723pt}}
\put(969.0,288.0){\rule[-0.200pt]{0.400pt}{4.577pt}}
\put(968.67,280){\rule{0.400pt}{6.986pt}}
\multiput(968.17,280.00)(1.000,14.500){2}{\rule{0.400pt}{3.493pt}}
\put(969.0,280.0){\rule[-0.200pt]{0.400pt}{6.504pt}}
\put(969.67,293){\rule{0.400pt}{3.373pt}}
\multiput(969.17,300.00)(1.000,-7.000){2}{\rule{0.400pt}{1.686pt}}
\put(970.0,307.0){\rule[-0.200pt]{0.400pt}{0.482pt}}
\put(971.0,282.0){\rule[-0.200pt]{0.400pt}{2.650pt}}
\put(970.67,286){\rule{0.400pt}{2.891pt}}
\multiput(970.17,286.00)(1.000,6.000){2}{\rule{0.400pt}{1.445pt}}
\put(971.0,282.0){\rule[-0.200pt]{0.400pt}{0.964pt}}
\put(972.0,298.0){\rule[-0.200pt]{0.400pt}{1.686pt}}
\put(971.67,278){\rule{0.400pt}{1.445pt}}
\multiput(971.17,278.00)(1.000,3.000){2}{\rule{0.400pt}{0.723pt}}
\put(972.0,278.0){\rule[-0.200pt]{0.400pt}{6.504pt}}
\put(973.0,284.0){\usebox{\plotpoint}}
\put(972.67,277){\rule{0.400pt}{4.577pt}}
\multiput(972.17,277.00)(1.000,9.500){2}{\rule{0.400pt}{2.289pt}}
\put(973.0,277.0){\rule[-0.200pt]{0.400pt}{1.927pt}}
\put(973.67,286){\rule{0.400pt}{3.373pt}}
\multiput(973.17,286.00)(1.000,7.000){2}{\rule{0.400pt}{1.686pt}}
\put(974.0,286.0){\rule[-0.200pt]{0.400pt}{2.409pt}}
\put(975.0,300.0){\rule[-0.200pt]{0.400pt}{1.927pt}}
\put(974.67,286){\rule{0.400pt}{2.409pt}}
\multiput(974.17,286.00)(1.000,5.000){2}{\rule{0.400pt}{1.204pt}}
\put(975.0,286.0){\rule[-0.200pt]{0.400pt}{5.300pt}}
\put(976.0,296.0){\rule[-0.200pt]{0.400pt}{2.168pt}}
\put(975.67,281){\rule{0.400pt}{0.723pt}}
\multiput(975.17,282.50)(1.000,-1.500){2}{\rule{0.400pt}{0.361pt}}
\put(976.0,284.0){\rule[-0.200pt]{0.400pt}{5.059pt}}
\put(976.67,281){\rule{0.400pt}{2.650pt}}
\multiput(976.17,286.50)(1.000,-5.500){2}{\rule{0.400pt}{1.325pt}}
\put(977.0,281.0){\rule[-0.200pt]{0.400pt}{2.650pt}}
\put(978,281.67){\rule{0.241pt}{0.400pt}}
\multiput(978.00,281.17)(0.500,1.000){2}{\rule{0.120pt}{0.400pt}}
\put(978.0,281.0){\usebox{\plotpoint}}
\put(978.67,279){\rule{0.400pt}{4.577pt}}
\multiput(978.17,279.00)(1.000,9.500){2}{\rule{0.400pt}{2.289pt}}
\put(979.0,279.0){\rule[-0.200pt]{0.400pt}{0.964pt}}
\put(979.67,295){\rule{0.400pt}{2.409pt}}
\multiput(979.17,300.00)(1.000,-5.000){2}{\rule{0.400pt}{1.204pt}}
\put(980.0,298.0){\rule[-0.200pt]{0.400pt}{1.686pt}}
\put(981.0,295.0){\rule[-0.200pt]{0.400pt}{3.854pt}}
\put(980.67,281){\rule{0.400pt}{3.373pt}}
\multiput(980.17,288.00)(1.000,-7.000){2}{\rule{0.400pt}{1.686pt}}
\put(981.0,295.0){\rule[-0.200pt]{0.400pt}{3.854pt}}
\put(981.67,300){\rule{0.400pt}{1.927pt}}
\multiput(981.17,304.00)(1.000,-4.000){2}{\rule{0.400pt}{0.964pt}}
\put(982.0,281.0){\rule[-0.200pt]{0.400pt}{6.504pt}}
\put(983.0,286.0){\rule[-0.200pt]{0.400pt}{3.373pt}}
\put(982.67,278){\rule{0.400pt}{7.227pt}}
\multiput(982.17,293.00)(1.000,-15.000){2}{\rule{0.400pt}{3.613pt}}
\put(983.0,286.0){\rule[-0.200pt]{0.400pt}{5.300pt}}
\put(984.0,278.0){\rule[-0.200pt]{0.400pt}{8.191pt}}
\put(983.67,284){\rule{0.400pt}{3.854pt}}
\multiput(983.17,292.00)(1.000,-8.000){2}{\rule{0.400pt}{1.927pt}}
\put(984.0,300.0){\rule[-0.200pt]{0.400pt}{2.891pt}}
\put(984.67,279){\rule{0.400pt}{7.709pt}}
\multiput(984.17,295.00)(1.000,-16.000){2}{\rule{0.400pt}{3.854pt}}
\put(985.0,284.0){\rule[-0.200pt]{0.400pt}{6.504pt}}
\put(985.67,298){\rule{0.400pt}{0.964pt}}
\multiput(985.17,300.00)(1.000,-2.000){2}{\rule{0.400pt}{0.482pt}}
\put(986.0,279.0){\rule[-0.200pt]{0.400pt}{5.541pt}}
\put(987.0,298.0){\rule[-0.200pt]{0.400pt}{2.168pt}}
\put(987,294.67){\rule{0.241pt}{0.400pt}}
\multiput(987.00,295.17)(0.500,-1.000){2}{\rule{0.120pt}{0.400pt}}
\put(987.0,296.0){\rule[-0.200pt]{0.400pt}{2.650pt}}
\put(988.0,289.0){\rule[-0.200pt]{0.400pt}{1.445pt}}
\put(987.67,280){\rule{0.400pt}{3.132pt}}
\multiput(987.17,286.50)(1.000,-6.500){2}{\rule{0.400pt}{1.566pt}}
\put(988.0,289.0){\rule[-0.200pt]{0.400pt}{0.964pt}}
\put(988.67,282){\rule{0.400pt}{4.095pt}}
\multiput(988.17,290.50)(1.000,-8.500){2}{\rule{0.400pt}{2.048pt}}
\put(989.0,280.0){\rule[-0.200pt]{0.400pt}{4.577pt}}
\put(990.0,281.0){\usebox{\plotpoint}}
\put(989.67,282){\rule{0.400pt}{3.373pt}}
\multiput(989.17,282.00)(1.000,7.000){2}{\rule{0.400pt}{1.686pt}}
\put(990.0,281.0){\usebox{\plotpoint}}
\put(990.67,277){\rule{0.400pt}{3.373pt}}
\multiput(990.17,284.00)(1.000,-7.000){2}{\rule{0.400pt}{1.686pt}}
\put(991.0,291.0){\rule[-0.200pt]{0.400pt}{1.204pt}}
\put(991.67,292){\rule{0.400pt}{2.168pt}}
\multiput(991.17,296.50)(1.000,-4.500){2}{\rule{0.400pt}{1.084pt}}
\put(992.0,277.0){\rule[-0.200pt]{0.400pt}{5.782pt}}
\put(993.0,292.0){\rule[-0.200pt]{0.400pt}{1.686pt}}
\put(992.67,288){\rule{0.400pt}{1.927pt}}
\multiput(992.17,288.00)(1.000,4.000){2}{\rule{0.400pt}{0.964pt}}
\put(993.0,288.0){\rule[-0.200pt]{0.400pt}{2.650pt}}
\put(994.0,293.0){\rule[-0.200pt]{0.400pt}{0.723pt}}
\put(994,301.67){\rule{0.241pt}{0.400pt}}
\multiput(994.00,302.17)(0.500,-1.000){2}{\rule{0.120pt}{0.400pt}}
\put(994.0,293.0){\rule[-0.200pt]{0.400pt}{2.409pt}}
\put(995.0,302.0){\rule[-0.200pt]{0.400pt}{2.409pt}}
\put(995.0,312.0){\usebox{\plotpoint}}
\put(996.0,277.0){\rule[-0.200pt]{0.400pt}{8.431pt}}
\put(995.67,278){\rule{0.400pt}{4.095pt}}
\multiput(995.17,278.00)(1.000,8.500){2}{\rule{0.400pt}{2.048pt}}
\put(996.0,277.0){\usebox{\plotpoint}}
\put(997.0,295.0){\rule[-0.200pt]{0.400pt}{2.409pt}}
\put(997.0,289.0){\rule[-0.200pt]{0.400pt}{3.854pt}}
\put(997.0,289.0){\usebox{\plotpoint}}
\put(998.0,279.0){\rule[-0.200pt]{0.400pt}{2.409pt}}
\put(997.67,287){\rule{0.400pt}{4.336pt}}
\multiput(997.17,296.00)(1.000,-9.000){2}{\rule{0.400pt}{2.168pt}}
\put(998.0,279.0){\rule[-0.200pt]{0.400pt}{6.263pt}}
\put(998.67,278){\rule{0.400pt}{6.986pt}}
\multiput(998.17,278.00)(1.000,14.500){2}{\rule{0.400pt}{3.493pt}}
\put(999.0,278.0){\rule[-0.200pt]{0.400pt}{2.168pt}}
\put(1000.0,290.0){\rule[-0.200pt]{0.400pt}{4.095pt}}
\put(999.67,296){\rule{0.400pt}{3.614pt}}
\multiput(999.17,303.50)(1.000,-7.500){2}{\rule{0.400pt}{1.807pt}}
\put(1000.0,290.0){\rule[-0.200pt]{0.400pt}{5.059pt}}
\put(1000.67,286){\rule{0.400pt}{0.482pt}}
\multiput(1000.17,287.00)(1.000,-1.000){2}{\rule{0.400pt}{0.241pt}}
\put(1001.0,288.0){\rule[-0.200pt]{0.400pt}{1.927pt}}
\put(1002.0,286.0){\rule[-0.200pt]{0.400pt}{0.723pt}}
\put(1001.67,279){\rule{0.400pt}{5.059pt}}
\multiput(1001.17,279.00)(1.000,10.500){2}{\rule{0.400pt}{2.529pt}}
\put(1002.0,279.0){\rule[-0.200pt]{0.400pt}{2.409pt}}
\put(1002.67,292){\rule{0.400pt}{0.723pt}}
\multiput(1002.17,293.50)(1.000,-1.500){2}{\rule{0.400pt}{0.361pt}}
\put(1003.0,295.0){\rule[-0.200pt]{0.400pt}{1.204pt}}
\put(1004.0,282.0){\rule[-0.200pt]{0.400pt}{2.409pt}}
\put(1003.67,298){\rule{0.400pt}{0.964pt}}
\multiput(1003.17,298.00)(1.000,2.000){2}{\rule{0.400pt}{0.482pt}}
\put(1004.0,282.0){\rule[-0.200pt]{0.400pt}{3.854pt}}
\put(1005.0,301.0){\usebox{\plotpoint}}
\put(1004.67,286){\rule{0.400pt}{5.059pt}}
\multiput(1004.17,296.50)(1.000,-10.500){2}{\rule{0.400pt}{2.529pt}}
\put(1005.0,301.0){\rule[-0.200pt]{0.400pt}{1.445pt}}
\put(1006.0,286.0){\rule[-0.200pt]{0.400pt}{2.168pt}}
\put(1005.67,294){\rule{0.400pt}{1.204pt}}
\multiput(1005.17,294.00)(1.000,2.500){2}{\rule{0.400pt}{0.602pt}}
\put(1006.0,294.0){\usebox{\plotpoint}}
\put(1006.67,300){\rule{0.400pt}{2.409pt}}
\multiput(1006.17,300.00)(1.000,5.000){2}{\rule{0.400pt}{1.204pt}}
\put(1007.0,299.0){\usebox{\plotpoint}}
\put(1008.0,292.0){\rule[-0.200pt]{0.400pt}{4.336pt}}
\put(1007.67,278){\rule{0.400pt}{6.263pt}}
\multiput(1007.17,291.00)(1.000,-13.000){2}{\rule{0.400pt}{3.132pt}}
\put(1008.0,292.0){\rule[-0.200pt]{0.400pt}{2.891pt}}
\put(1008.67,282){\rule{0.400pt}{6.504pt}}
\multiput(1008.17,295.50)(1.000,-13.500){2}{\rule{0.400pt}{3.252pt}}
\put(1009.0,278.0){\rule[-0.200pt]{0.400pt}{7.468pt}}
\put(1010.0,282.0){\rule[-0.200pt]{0.400pt}{7.227pt}}
\put(1009.67,284){\rule{0.400pt}{5.300pt}}
\multiput(1009.17,284.00)(1.000,11.000){2}{\rule{0.400pt}{2.650pt}}
\put(1010.0,284.0){\rule[-0.200pt]{0.400pt}{6.745pt}}
\put(1010.67,286){\rule{0.400pt}{4.577pt}}
\multiput(1010.17,295.50)(1.000,-9.500){2}{\rule{0.400pt}{2.289pt}}
\put(1011.0,305.0){\usebox{\plotpoint}}
\put(1012.0,286.0){\rule[-0.200pt]{0.400pt}{5.059pt}}
\put(1011.67,279){\rule{0.400pt}{6.023pt}}
\multiput(1011.17,291.50)(1.000,-12.500){2}{\rule{0.400pt}{3.011pt}}
\put(1012.0,304.0){\rule[-0.200pt]{0.400pt}{0.723pt}}
\put(1013.0,279.0){\rule[-0.200pt]{0.400pt}{6.986pt}}
\put(1012.67,277){\rule{0.400pt}{6.745pt}}
\multiput(1012.17,291.00)(1.000,-14.000){2}{\rule{0.400pt}{3.373pt}}
\put(1013.0,305.0){\rule[-0.200pt]{0.400pt}{0.723pt}}
\put(1013.67,285){\rule{0.400pt}{4.818pt}}
\multiput(1013.17,295.00)(1.000,-10.000){2}{\rule{0.400pt}{2.409pt}}
\put(1014.0,277.0){\rule[-0.200pt]{0.400pt}{6.745pt}}
\put(1015.0,285.0){\rule[-0.200pt]{0.400pt}{1.686pt}}
\put(1014.67,285){\rule{0.400pt}{6.263pt}}
\multiput(1014.17,285.00)(1.000,13.000){2}{\rule{0.400pt}{3.132pt}}
\put(1015.0,285.0){\rule[-0.200pt]{0.400pt}{1.686pt}}
\put(1015.67,299){\rule{0.400pt}{1.686pt}}
\multiput(1015.17,302.50)(1.000,-3.500){2}{\rule{0.400pt}{0.843pt}}
\put(1016.0,306.0){\rule[-0.200pt]{0.400pt}{1.204pt}}
\put(1017.0,299.0){\rule[-0.200pt]{0.400pt}{2.650pt}}
\put(1016.67,278){\rule{0.400pt}{2.891pt}}
\multiput(1016.17,284.00)(1.000,-6.000){2}{\rule{0.400pt}{1.445pt}}
\put(1017.0,290.0){\rule[-0.200pt]{0.400pt}{4.818pt}}
\put(1018.0,278.0){\rule[-0.200pt]{0.400pt}{8.191pt}}
\put(1018,304.67){\rule{0.241pt}{0.400pt}}
\multiput(1018.00,304.17)(0.500,1.000){2}{\rule{0.120pt}{0.400pt}}
\put(1018.0,305.0){\rule[-0.200pt]{0.400pt}{1.686pt}}
\put(1019.0,288.0){\rule[-0.200pt]{0.400pt}{4.336pt}}
\put(1019.0,288.0){\rule[-0.200pt]{0.400pt}{0.964pt}}
\put(1019.0,292.0){\usebox{\plotpoint}}
\put(1019.67,280){\rule{0.400pt}{6.263pt}}
\multiput(1019.17,280.00)(1.000,13.000){2}{\rule{0.400pt}{3.132pt}}
\put(1020.0,280.0){\rule[-0.200pt]{0.400pt}{2.891pt}}
\put(1021.0,285.0){\rule[-0.200pt]{0.400pt}{5.059pt}}
\put(1020.67,289){\rule{0.400pt}{0.964pt}}
\multiput(1020.17,289.00)(1.000,2.000){2}{\rule{0.400pt}{0.482pt}}
\put(1021.0,285.0){\rule[-0.200pt]{0.400pt}{0.964pt}}
\put(1022.0,279.0){\rule[-0.200pt]{0.400pt}{3.373pt}}
\put(1021.67,289){\rule{0.400pt}{4.336pt}}
\multiput(1021.17,289.00)(1.000,9.000){2}{\rule{0.400pt}{2.168pt}}
\put(1022.0,279.0){\rule[-0.200pt]{0.400pt}{2.409pt}}
\put(1023.0,307.0){\rule[-0.200pt]{0.400pt}{0.964pt}}
\put(1023,277.67){\rule{0.241pt}{0.400pt}}
\multiput(1023.00,277.17)(0.500,1.000){2}{\rule{0.120pt}{0.400pt}}
\put(1023.0,278.0){\rule[-0.200pt]{0.400pt}{7.950pt}}
\put(1023.67,286){\rule{0.400pt}{5.541pt}}
\multiput(1023.17,286.00)(1.000,11.500){2}{\rule{0.400pt}{2.770pt}}
\put(1024.0,279.0){\rule[-0.200pt]{0.400pt}{1.686pt}}
\put(1024.67,281){\rule{0.400pt}{6.504pt}}
\multiput(1024.17,281.00)(1.000,13.500){2}{\rule{0.400pt}{3.252pt}}
\put(1025.0,281.0){\rule[-0.200pt]{0.400pt}{6.745pt}}
\put(1025.67,294){\rule{0.400pt}{4.336pt}}
\multiput(1025.17,294.00)(1.000,9.000){2}{\rule{0.400pt}{2.168pt}}
\put(1026.0,294.0){\rule[-0.200pt]{0.400pt}{3.373pt}}
\put(1026.67,291){\rule{0.400pt}{2.168pt}}
\multiput(1026.17,291.00)(1.000,4.500){2}{\rule{0.400pt}{1.084pt}}
\put(1027.0,291.0){\rule[-0.200pt]{0.400pt}{5.059pt}}
\put(1027.67,287){\rule{0.400pt}{4.577pt}}
\multiput(1027.17,296.50)(1.000,-9.500){2}{\rule{0.400pt}{2.289pt}}
\put(1028.0,300.0){\rule[-0.200pt]{0.400pt}{1.445pt}}
\put(1029.0,287.0){\rule[-0.200pt]{0.400pt}{3.373pt}}
\put(1028.67,297){\rule{0.400pt}{3.614pt}}
\multiput(1028.17,297.00)(1.000,7.500){2}{\rule{0.400pt}{1.807pt}}
\put(1029.0,297.0){\rule[-0.200pt]{0.400pt}{0.964pt}}
\put(1029.67,284){\rule{0.400pt}{1.686pt}}
\multiput(1029.17,284.00)(1.000,3.500){2}{\rule{0.400pt}{0.843pt}}
\put(1030.0,284.0){\rule[-0.200pt]{0.400pt}{6.745pt}}
\put(1031.0,280.0){\rule[-0.200pt]{0.400pt}{2.650pt}}
\put(1030.67,283){\rule{0.400pt}{1.927pt}}
\multiput(1030.17,283.00)(1.000,4.000){2}{\rule{0.400pt}{0.964pt}}
\put(1031.0,280.0){\rule[-0.200pt]{0.400pt}{0.723pt}}
\put(1032,290.67){\rule{0.241pt}{0.400pt}}
\multiput(1032.00,291.17)(0.500,-1.000){2}{\rule{0.120pt}{0.400pt}}
\put(1032.0,291.0){\usebox{\plotpoint}}
\put(1033.0,288.0){\rule[-0.200pt]{0.400pt}{0.723pt}}
\put(1032.67,291){\rule{0.400pt}{4.336pt}}
\multiput(1032.17,300.00)(1.000,-9.000){2}{\rule{0.400pt}{2.168pt}}
\put(1033.0,288.0){\rule[-0.200pt]{0.400pt}{5.059pt}}
\put(1034.0,286.0){\rule[-0.200pt]{0.400pt}{1.204pt}}
\put(1033.67,287){\rule{0.400pt}{2.891pt}}
\multiput(1033.17,287.00)(1.000,6.000){2}{\rule{0.400pt}{1.445pt}}
\put(1034.0,286.0){\usebox{\plotpoint}}
\put(1034.67,280){\rule{0.400pt}{2.409pt}}
\multiput(1034.17,280.00)(1.000,5.000){2}{\rule{0.400pt}{1.204pt}}
\put(1035.0,280.0){\rule[-0.200pt]{0.400pt}{4.577pt}}
\put(1035.67,289){\rule{0.400pt}{2.650pt}}
\multiput(1035.17,294.50)(1.000,-5.500){2}{\rule{0.400pt}{1.325pt}}
\put(1036.0,290.0){\rule[-0.200pt]{0.400pt}{2.409pt}}
\put(1037.0,289.0){\rule[-0.200pt]{0.400pt}{2.891pt}}
\put(1036.67,299){\rule{0.400pt}{0.964pt}}
\multiput(1036.17,299.00)(1.000,2.000){2}{\rule{0.400pt}{0.482pt}}
\put(1037.0,299.0){\rule[-0.200pt]{0.400pt}{0.482pt}}
\put(1038.0,303.0){\rule[-0.200pt]{0.400pt}{0.723pt}}
\put(1037.67,292){\rule{0.400pt}{2.409pt}}
\multiput(1037.17,297.00)(1.000,-5.000){2}{\rule{0.400pt}{1.204pt}}
\put(1038.0,302.0){\rule[-0.200pt]{0.400pt}{0.964pt}}
\put(1039.0,292.0){\rule[-0.200pt]{0.400pt}{3.613pt}}
\put(1038.67,298){\rule{0.400pt}{2.650pt}}
\multiput(1038.17,298.00)(1.000,5.500){2}{\rule{0.400pt}{1.325pt}}
\put(1039.0,298.0){\rule[-0.200pt]{0.400pt}{2.168pt}}
\put(1040.0,293.0){\rule[-0.200pt]{0.400pt}{3.854pt}}
\put(1039.67,301){\rule{0.400pt}{2.168pt}}
\multiput(1039.17,301.00)(1.000,4.500){2}{\rule{0.400pt}{1.084pt}}
\put(1040.0,293.0){\rule[-0.200pt]{0.400pt}{1.927pt}}
\put(1040.67,291){\rule{0.400pt}{0.723pt}}
\multiput(1040.17,292.50)(1.000,-1.500){2}{\rule{0.400pt}{0.361pt}}
\put(1041.0,294.0){\rule[-0.200pt]{0.400pt}{3.854pt}}
\put(1042.0,285.0){\rule[-0.200pt]{0.400pt}{1.445pt}}
\put(1041.67,288){\rule{0.400pt}{5.782pt}}
\multiput(1041.17,288.00)(1.000,12.000){2}{\rule{0.400pt}{2.891pt}}
\put(1042.0,285.0){\rule[-0.200pt]{0.400pt}{0.723pt}}
\put(1043.0,286.0){\rule[-0.200pt]{0.400pt}{6.263pt}}
\put(1043,300.67){\rule{0.241pt}{0.400pt}}
\multiput(1043.00,300.17)(0.500,1.000){2}{\rule{0.120pt}{0.400pt}}
\put(1043.0,286.0){\rule[-0.200pt]{0.400pt}{3.613pt}}
\put(1043.67,284){\rule{0.400pt}{6.263pt}}
\multiput(1043.17,297.00)(1.000,-13.000){2}{\rule{0.400pt}{3.132pt}}
\put(1044.0,302.0){\rule[-0.200pt]{0.400pt}{1.927pt}}
\put(1044.67,279){\rule{0.400pt}{1.445pt}}
\multiput(1044.17,279.00)(1.000,3.000){2}{\rule{0.400pt}{0.723pt}}
\put(1045.0,279.0){\rule[-0.200pt]{0.400pt}{1.204pt}}
\put(1046.0,285.0){\usebox{\plotpoint}}
\put(1045.67,282){\rule{0.400pt}{6.745pt}}
\multiput(1045.17,282.00)(1.000,14.000){2}{\rule{0.400pt}{3.373pt}}
\put(1046.0,282.0){\rule[-0.200pt]{0.400pt}{0.964pt}}
\put(1047.0,285.0){\rule[-0.200pt]{0.400pt}{6.022pt}}
\put(1046.67,280){\rule{0.400pt}{4.577pt}}
\multiput(1046.17,289.50)(1.000,-9.500){2}{\rule{0.400pt}{2.289pt}}
\put(1047.0,285.0){\rule[-0.200pt]{0.400pt}{3.373pt}}
\put(1047.67,284){\rule{0.400pt}{6.023pt}}
\multiput(1047.17,296.50)(1.000,-12.500){2}{\rule{0.400pt}{3.011pt}}
\put(1048.0,280.0){\rule[-0.200pt]{0.400pt}{6.986pt}}
\put(1048.67,285){\rule{0.400pt}{5.782pt}}
\multiput(1048.17,297.00)(1.000,-12.000){2}{\rule{0.400pt}{2.891pt}}
\put(1049.0,284.0){\rule[-0.200pt]{0.400pt}{6.022pt}}
\put(1050.0,285.0){\rule[-0.200pt]{0.400pt}{4.818pt}}
\put(1049.67,284){\rule{0.400pt}{0.964pt}}
\multiput(1049.17,284.00)(1.000,2.000){2}{\rule{0.400pt}{0.482pt}}
\put(1050.0,284.0){\rule[-0.200pt]{0.400pt}{5.059pt}}
\put(1051.0,288.0){\rule[-0.200pt]{0.400pt}{3.854pt}}
\put(1050.67,290){\rule{0.400pt}{3.854pt}}
\multiput(1050.17,290.00)(1.000,8.000){2}{\rule{0.400pt}{1.927pt}}
\put(1051.0,290.0){\rule[-0.200pt]{0.400pt}{3.373pt}}
\put(1052.0,306.0){\usebox{\plotpoint}}
\put(1051.67,298){\rule{0.400pt}{0.482pt}}
\multiput(1051.17,299.00)(1.000,-1.000){2}{\rule{0.400pt}{0.241pt}}
\put(1052.0,300.0){\rule[-0.200pt]{0.400pt}{1.686pt}}
\put(1052.67,286){\rule{0.400pt}{5.300pt}}
\multiput(1052.17,297.00)(1.000,-11.000){2}{\rule{0.400pt}{2.650pt}}
\put(1053.0,298.0){\rule[-0.200pt]{0.400pt}{2.409pt}}
\put(1054.0,282.0){\rule[-0.200pt]{0.400pt}{0.964pt}}
\put(1053.67,286){\rule{0.400pt}{6.023pt}}
\multiput(1053.17,298.50)(1.000,-12.500){2}{\rule{0.400pt}{3.011pt}}
\put(1054.0,282.0){\rule[-0.200pt]{0.400pt}{6.986pt}}
\put(1055.0,280.0){\rule[-0.200pt]{0.400pt}{1.445pt}}
\put(1054.67,289){\rule{0.400pt}{5.059pt}}
\multiput(1054.17,289.00)(1.000,10.500){2}{\rule{0.400pt}{2.529pt}}
\put(1055.0,280.0){\rule[-0.200pt]{0.400pt}{2.168pt}}
\put(1055.67,283){\rule{0.400pt}{6.263pt}}
\multiput(1055.17,283.00)(1.000,13.000){2}{\rule{0.400pt}{3.132pt}}
\put(1056.0,283.0){\rule[-0.200pt]{0.400pt}{6.504pt}}
\put(1057.0,291.0){\rule[-0.200pt]{0.400pt}{4.336pt}}
\put(1057.0,291.0){\usebox{\plotpoint}}
\put(1058.0,283.0){\rule[-0.200pt]{0.400pt}{1.927pt}}
\put(1057.67,295){\rule{0.400pt}{2.650pt}}
\multiput(1057.17,300.50)(1.000,-5.500){2}{\rule{0.400pt}{1.325pt}}
\put(1058.0,283.0){\rule[-0.200pt]{0.400pt}{5.541pt}}
\put(1059,295){\usebox{\plotpoint}}
\put(1058.67,294){\rule{0.400pt}{1.445pt}}
\multiput(1058.17,294.00)(1.000,3.000){2}{\rule{0.400pt}{0.723pt}}
\put(1059.0,294.0){\usebox{\plotpoint}}
\put(1060.0,300.0){\rule[-0.200pt]{0.400pt}{0.723pt}}
\put(1059.67,278){\rule{0.400pt}{2.891pt}}
\multiput(1059.17,284.00)(1.000,-6.000){2}{\rule{0.400pt}{1.445pt}}
\put(1060.0,290.0){\rule[-0.200pt]{0.400pt}{3.132pt}}
\put(1060.67,279){\rule{0.400pt}{1.686pt}}
\multiput(1060.17,282.50)(1.000,-3.500){2}{\rule{0.400pt}{0.843pt}}
\put(1061.0,278.0){\rule[-0.200pt]{0.400pt}{1.927pt}}
\put(1061.67,289){\rule{0.400pt}{2.891pt}}
\multiput(1061.17,295.00)(1.000,-6.000){2}{\rule{0.400pt}{1.445pt}}
\put(1062.0,279.0){\rule[-0.200pt]{0.400pt}{5.300pt}}
\put(1062.67,277){\rule{0.400pt}{5.300pt}}
\multiput(1062.17,288.00)(1.000,-11.000){2}{\rule{0.400pt}{2.650pt}}
\put(1063.0,289.0){\rule[-0.200pt]{0.400pt}{2.409pt}}
\put(1063.67,284){\rule{0.400pt}{6.023pt}}
\multiput(1063.17,296.50)(1.000,-12.500){2}{\rule{0.400pt}{3.011pt}}
\put(1064.0,277.0){\rule[-0.200pt]{0.400pt}{7.709pt}}
\put(1065.0,284.0){\rule[-0.200pt]{0.400pt}{6.022pt}}
\put(1064.67,283){\rule{0.400pt}{6.263pt}}
\multiput(1064.17,283.00)(1.000,13.000){2}{\rule{0.400pt}{3.132pt}}
\put(1065.0,283.0){\rule[-0.200pt]{0.400pt}{6.263pt}}
\put(1065.67,283){\rule{0.400pt}{3.373pt}}
\multiput(1065.17,290.00)(1.000,-7.000){2}{\rule{0.400pt}{1.686pt}}
\put(1066.0,297.0){\rule[-0.200pt]{0.400pt}{2.891pt}}
\put(1067.0,283.0){\rule[-0.200pt]{0.400pt}{6.745pt}}
\put(1066.67,278){\rule{0.400pt}{8.191pt}}
\multiput(1066.17,278.00)(1.000,17.000){2}{\rule{0.400pt}{4.095pt}}
\put(1067.0,278.0){\rule[-0.200pt]{0.400pt}{7.950pt}}
\put(1068.0,290.0){\rule[-0.200pt]{0.400pt}{5.300pt}}
\put(1067.67,294){\rule{0.400pt}{0.964pt}}
\multiput(1067.17,296.00)(1.000,-2.000){2}{\rule{0.400pt}{0.482pt}}
\put(1068.0,290.0){\rule[-0.200pt]{0.400pt}{1.927pt}}
\put(1069.0,294.0){\rule[-0.200pt]{0.400pt}{3.373pt}}
\put(1068.67,281){\rule{0.400pt}{1.204pt}}
\multiput(1068.17,281.00)(1.000,2.500){2}{\rule{0.400pt}{0.602pt}}
\put(1069.0,281.0){\rule[-0.200pt]{0.400pt}{6.504pt}}
\put(1069.67,287){\rule{0.400pt}{5.300pt}}
\multiput(1069.17,298.00)(1.000,-11.000){2}{\rule{0.400pt}{2.650pt}}
\put(1070.0,286.0){\rule[-0.200pt]{0.400pt}{5.541pt}}
\put(1071.0,287.0){\rule[-0.200pt]{0.400pt}{5.541pt}}
\put(1070.67,301){\rule{0.400pt}{2.409pt}}
\multiput(1070.17,301.00)(1.000,5.000){2}{\rule{0.400pt}{1.204pt}}
\put(1071.0,301.0){\rule[-0.200pt]{0.400pt}{2.168pt}}
\put(1071.67,282){\rule{0.400pt}{0.723pt}}
\multiput(1071.17,283.50)(1.000,-1.500){2}{\rule{0.400pt}{0.361pt}}
\put(1072.0,285.0){\rule[-0.200pt]{0.400pt}{6.263pt}}
\put(1073.0,282.0){\rule[-0.200pt]{0.400pt}{4.818pt}}
\put(1073,283.67){\rule{0.241pt}{0.400pt}}
\multiput(1073.00,284.17)(0.500,-1.000){2}{\rule{0.120pt}{0.400pt}}
\put(1073.0,285.0){\rule[-0.200pt]{0.400pt}{4.095pt}}
\put(1073.67,306){\rule{0.400pt}{0.723pt}}
\multiput(1073.17,306.00)(1.000,1.500){2}{\rule{0.400pt}{0.361pt}}
\put(1074.0,284.0){\rule[-0.200pt]{0.400pt}{5.300pt}}
\put(1075.0,290.0){\rule[-0.200pt]{0.400pt}{4.577pt}}
\put(1074.67,293){\rule{0.400pt}{2.409pt}}
\multiput(1074.17,293.00)(1.000,5.000){2}{\rule{0.400pt}{1.204pt}}
\put(1075.0,290.0){\rule[-0.200pt]{0.400pt}{0.723pt}}
\put(1076.0,301.0){\rule[-0.200pt]{0.400pt}{0.482pt}}
\put(1075.67,278){\rule{0.400pt}{7.227pt}}
\multiput(1075.17,293.00)(1.000,-15.000){2}{\rule{0.400pt}{3.613pt}}
\put(1076.0,301.0){\rule[-0.200pt]{0.400pt}{1.686pt}}
\put(1077.0,278.0){\rule[-0.200pt]{0.400pt}{6.986pt}}
\put(1076.67,279){\rule{0.400pt}{7.950pt}}
\multiput(1076.17,279.00)(1.000,16.500){2}{\rule{0.400pt}{3.975pt}}
\put(1077.0,279.0){\rule[-0.200pt]{0.400pt}{6.745pt}}
\put(1077.67,288){\rule{0.400pt}{2.650pt}}
\multiput(1077.17,288.00)(1.000,5.500){2}{\rule{0.400pt}{1.325pt}}
\put(1078.0,288.0){\rule[-0.200pt]{0.400pt}{5.782pt}}
\put(1079.0,299.0){\rule[-0.200pt]{0.400pt}{1.204pt}}
\put(1078.67,283){\rule{0.400pt}{0.723pt}}
\multiput(1078.17,284.50)(1.000,-1.500){2}{\rule{0.400pt}{0.361pt}}
\put(1079.0,286.0){\rule[-0.200pt]{0.400pt}{4.336pt}}
\put(1080.0,277.0){\rule[-0.200pt]{0.400pt}{1.445pt}}
\put(1079.67,302){\rule{0.400pt}{1.927pt}}
\multiput(1079.17,302.00)(1.000,4.000){2}{\rule{0.400pt}{0.964pt}}
\put(1080.0,277.0){\rule[-0.200pt]{0.400pt}{6.022pt}}
\put(1080.67,302){\rule{0.400pt}{2.650pt}}
\multiput(1080.17,302.00)(1.000,5.500){2}{\rule{0.400pt}{1.325pt}}
\put(1081.0,302.0){\rule[-0.200pt]{0.400pt}{1.927pt}}
\put(1081.67,297){\rule{0.400pt}{3.132pt}}
\multiput(1081.17,297.00)(1.000,6.500){2}{\rule{0.400pt}{1.566pt}}
\put(1082.0,297.0){\rule[-0.200pt]{0.400pt}{3.854pt}}
\put(1082.67,279){\rule{0.400pt}{1.204pt}}
\multiput(1082.17,279.00)(1.000,2.500){2}{\rule{0.400pt}{0.602pt}}
\put(1083.0,279.0){\rule[-0.200pt]{0.400pt}{7.468pt}}
\put(1084.0,280.0){\rule[-0.200pt]{0.400pt}{0.964pt}}
\put(1084.0,280.0){\rule[-0.200pt]{0.400pt}{3.613pt}}
\put(1084.0,295.0){\usebox{\plotpoint}}
\put(1085.0,295.0){\rule[-0.200pt]{0.400pt}{2.168pt}}
\put(1084.67,300){\rule{0.400pt}{1.686pt}}
\multiput(1084.17,300.00)(1.000,3.500){2}{\rule{0.400pt}{0.843pt}}
\put(1085.0,300.0){\rule[-0.200pt]{0.400pt}{0.964pt}}
\put(1085.67,296){\rule{0.400pt}{0.482pt}}
\multiput(1085.17,296.00)(1.000,1.000){2}{\rule{0.400pt}{0.241pt}}
\put(1086.0,296.0){\rule[-0.200pt]{0.400pt}{2.650pt}}
\put(1087,298.67){\rule{0.241pt}{0.400pt}}
\multiput(1087.00,298.17)(0.500,1.000){2}{\rule{0.120pt}{0.400pt}}
\put(1087.0,298.0){\usebox{\plotpoint}}
\put(1088.0,281.0){\rule[-0.200pt]{0.400pt}{4.577pt}}
\put(1087.67,280){\rule{0.400pt}{1.686pt}}
\multiput(1087.17,283.50)(1.000,-3.500){2}{\rule{0.400pt}{0.843pt}}
\put(1088.0,281.0){\rule[-0.200pt]{0.400pt}{1.445pt}}
\put(1089.0,279.0){\usebox{\plotpoint}}
\put(1088.67,289){\rule{0.400pt}{4.336pt}}
\multiput(1088.17,289.00)(1.000,9.000){2}{\rule{0.400pt}{2.168pt}}
\put(1089.0,279.0){\rule[-0.200pt]{0.400pt}{2.409pt}}
\put(1090.0,307.0){\usebox{\plotpoint}}
\put(1089.67,284){\rule{0.400pt}{2.891pt}}
\multiput(1089.17,284.00)(1.000,6.000){2}{\rule{0.400pt}{1.445pt}}
\put(1090.0,284.0){\rule[-0.200pt]{0.400pt}{5.782pt}}
\put(1090.67,281){\rule{0.400pt}{1.686pt}}
\multiput(1090.17,281.00)(1.000,3.500){2}{\rule{0.400pt}{0.843pt}}
\put(1091.0,281.0){\rule[-0.200pt]{0.400pt}{3.613pt}}
\put(1091.67,281){\rule{0.400pt}{7.468pt}}
\multiput(1091.17,296.50)(1.000,-15.500){2}{\rule{0.400pt}{3.734pt}}
\put(1092.0,288.0){\rule[-0.200pt]{0.400pt}{5.782pt}}
\put(1093.0,281.0){\rule[-0.200pt]{0.400pt}{3.373pt}}
\put(1092.67,278){\rule{0.400pt}{8.191pt}}
\multiput(1092.17,278.00)(1.000,17.000){2}{\rule{0.400pt}{4.095pt}}
\put(1093.0,278.0){\rule[-0.200pt]{0.400pt}{4.095pt}}
\put(1094,277.67){\rule{0.241pt}{0.400pt}}
\multiput(1094.00,278.17)(0.500,-1.000){2}{\rule{0.120pt}{0.400pt}}
\put(1094.0,279.0){\rule[-0.200pt]{0.400pt}{7.950pt}}
\put(1094.67,278){\rule{0.400pt}{4.095pt}}
\multiput(1094.17,286.50)(1.000,-8.500){2}{\rule{0.400pt}{2.048pt}}
\put(1095.0,278.0){\rule[-0.200pt]{0.400pt}{4.095pt}}
\put(1096.0,278.0){\rule[-0.200pt]{0.400pt}{3.132pt}}
\put(1095.67,279){\rule{0.400pt}{0.482pt}}
\multiput(1095.17,279.00)(1.000,1.000){2}{\rule{0.400pt}{0.241pt}}
\put(1096.0,279.0){\rule[-0.200pt]{0.400pt}{2.891pt}}
\put(1097.0,281.0){\rule[-0.200pt]{0.400pt}{3.854pt}}
\put(1096.67,283){\rule{0.400pt}{0.723pt}}
\multiput(1096.17,284.50)(1.000,-1.500){2}{\rule{0.400pt}{0.361pt}}
\put(1097.0,286.0){\rule[-0.200pt]{0.400pt}{2.650pt}}
\put(1097.67,299){\rule{0.400pt}{0.723pt}}
\multiput(1097.17,299.00)(1.000,1.500){2}{\rule{0.400pt}{0.361pt}}
\put(1098.0,283.0){\rule[-0.200pt]{0.400pt}{3.854pt}}
\put(1098.67,296){\rule{0.400pt}{3.854pt}}
\multiput(1098.17,296.00)(1.000,8.000){2}{\rule{0.400pt}{1.927pt}}
\put(1099.0,296.0){\rule[-0.200pt]{0.400pt}{1.445pt}}
\put(1099.67,293){\rule{0.400pt}{3.373pt}}
\multiput(1099.17,300.00)(1.000,-7.000){2}{\rule{0.400pt}{1.686pt}}
\put(1100.0,307.0){\rule[-0.200pt]{0.400pt}{1.204pt}}
\put(1101.0,293.0){\rule[-0.200pt]{0.400pt}{2.168pt}}
\put(1100.67,300){\rule{0.400pt}{0.482pt}}
\multiput(1100.17,300.00)(1.000,1.000){2}{\rule{0.400pt}{0.241pt}}
\put(1101.0,300.0){\rule[-0.200pt]{0.400pt}{0.482pt}}
\put(1102.0,302.0){\rule[-0.200pt]{0.400pt}{0.964pt}}
\put(1101.67,280){\rule{0.400pt}{1.445pt}}
\multiput(1101.17,280.00)(1.000,3.000){2}{\rule{0.400pt}{0.723pt}}
\put(1102.0,280.0){\rule[-0.200pt]{0.400pt}{6.263pt}}
\put(1102.67,280){\rule{0.400pt}{6.023pt}}
\multiput(1102.17,280.00)(1.000,12.500){2}{\rule{0.400pt}{3.011pt}}
\put(1103.0,280.0){\rule[-0.200pt]{0.400pt}{1.445pt}}
\put(1103.67,289){\rule{0.400pt}{2.168pt}}
\multiput(1103.17,289.00)(1.000,4.500){2}{\rule{0.400pt}{1.084pt}}
\put(1104.0,289.0){\rule[-0.200pt]{0.400pt}{3.854pt}}
\put(1105.0,298.0){\rule[-0.200pt]{0.400pt}{0.964pt}}
\put(1104.67,282){\rule{0.400pt}{1.927pt}}
\multiput(1104.17,282.00)(1.000,4.000){2}{\rule{0.400pt}{0.964pt}}
\put(1105.0,282.0){\rule[-0.200pt]{0.400pt}{4.818pt}}
\put(1106.0,286.0){\rule[-0.200pt]{0.400pt}{0.964pt}}
\put(1105.67,302){\rule{0.400pt}{1.445pt}}
\multiput(1105.17,305.00)(1.000,-3.000){2}{\rule{0.400pt}{0.723pt}}
\put(1106.0,286.0){\rule[-0.200pt]{0.400pt}{5.300pt}}
\put(1106.67,290){\rule{0.400pt}{3.132pt}}
\multiput(1106.17,290.00)(1.000,6.500){2}{\rule{0.400pt}{1.566pt}}
\put(1107.0,290.0){\rule[-0.200pt]{0.400pt}{2.891pt}}
\put(1108.0,299.0){\rule[-0.200pt]{0.400pt}{0.964pt}}
\put(1107.67,302){\rule{0.400pt}{0.723pt}}
\multiput(1107.17,303.50)(1.000,-1.500){2}{\rule{0.400pt}{0.361pt}}
\put(1108.0,299.0){\rule[-0.200pt]{0.400pt}{1.445pt}}
\put(1109.0,296.0){\rule[-0.200pt]{0.400pt}{1.445pt}}
\put(1108.67,291){\rule{0.400pt}{1.686pt}}
\multiput(1108.17,294.50)(1.000,-3.500){2}{\rule{0.400pt}{0.843pt}}
\put(1109.0,296.0){\rule[-0.200pt]{0.400pt}{0.482pt}}
\put(1109.67,286){\rule{0.400pt}{6.023pt}}
\multiput(1109.17,286.00)(1.000,12.500){2}{\rule{0.400pt}{3.011pt}}
\put(1110.0,286.0){\rule[-0.200pt]{0.400pt}{1.204pt}}
\put(1111.0,307.0){\rule[-0.200pt]{0.400pt}{0.964pt}}
\put(1110.67,280){\rule{0.400pt}{7.227pt}}
\multiput(1110.17,295.00)(1.000,-15.000){2}{\rule{0.400pt}{3.613pt}}
\put(1111.0,307.0){\rule[-0.200pt]{0.400pt}{0.723pt}}
\put(1111.67,291){\rule{0.400pt}{4.095pt}}
\multiput(1111.17,291.00)(1.000,8.500){2}{\rule{0.400pt}{2.048pt}}
\put(1112.0,280.0){\rule[-0.200pt]{0.400pt}{2.650pt}}
\put(1113.0,282.0){\rule[-0.200pt]{0.400pt}{6.263pt}}
\put(1112.67,306){\rule{0.400pt}{1.445pt}}
\multiput(1112.17,309.00)(1.000,-3.000){2}{\rule{0.400pt}{0.723pt}}
\put(1113.0,282.0){\rule[-0.200pt]{0.400pt}{7.227pt}}
\put(1113.67,282){\rule{0.400pt}{6.263pt}}
\multiput(1113.17,282.00)(1.000,13.000){2}{\rule{0.400pt}{3.132pt}}
\put(1114.0,282.0){\rule[-0.200pt]{0.400pt}{5.782pt}}
\put(1115.0,279.0){\rule[-0.200pt]{0.400pt}{6.986pt}}
\put(1114.67,297){\rule{0.400pt}{3.373pt}}
\multiput(1114.17,304.00)(1.000,-7.000){2}{\rule{0.400pt}{1.686pt}}
\put(1115.0,279.0){\rule[-0.200pt]{0.400pt}{7.709pt}}
\put(1115.67,279){\rule{0.400pt}{0.723pt}}
\multiput(1115.17,279.00)(1.000,1.500){2}{\rule{0.400pt}{0.361pt}}
\put(1116.0,279.0){\rule[-0.200pt]{0.400pt}{4.336pt}}
\put(1117.0,282.0){\rule[-0.200pt]{0.400pt}{6.745pt}}
\put(1117.0,310.0){\usebox{\plotpoint}}
\put(1117.67,283){\rule{0.400pt}{3.132pt}}
\multiput(1117.17,283.00)(1.000,6.500){2}{\rule{0.400pt}{1.566pt}}
\put(1118.0,283.0){\rule[-0.200pt]{0.400pt}{6.504pt}}
\put(1119,296){\usebox{\plotpoint}}
\put(1118.67,297){\rule{0.400pt}{2.409pt}}
\multiput(1118.17,302.00)(1.000,-5.000){2}{\rule{0.400pt}{1.204pt}}
\put(1119.0,296.0){\rule[-0.200pt]{0.400pt}{2.650pt}}
\put(1119.67,285){\rule{0.400pt}{1.686pt}}
\multiput(1119.17,285.00)(1.000,3.500){2}{\rule{0.400pt}{0.843pt}}
\put(1120.0,285.0){\rule[-0.200pt]{0.400pt}{2.891pt}}
\put(1121.0,278.0){\rule[-0.200pt]{0.400pt}{3.373pt}}
\put(1120.67,306){\rule{0.400pt}{1.445pt}}
\multiput(1120.17,306.00)(1.000,3.000){2}{\rule{0.400pt}{0.723pt}}
\put(1121.0,278.0){\rule[-0.200pt]{0.400pt}{6.745pt}}
\put(1121.67,283){\rule{0.400pt}{1.686pt}}
\multiput(1121.17,286.50)(1.000,-3.500){2}{\rule{0.400pt}{0.843pt}}
\put(1122.0,290.0){\rule[-0.200pt]{0.400pt}{5.300pt}}
\put(1123.0,283.0){\rule[-0.200pt]{0.400pt}{4.577pt}}
\put(1122.67,301){\rule{0.400pt}{1.927pt}}
\multiput(1122.17,301.00)(1.000,4.000){2}{\rule{0.400pt}{0.964pt}}
\put(1123.0,301.0){\usebox{\plotpoint}}
\put(1124,282.67){\rule{0.241pt}{0.400pt}}
\multiput(1124.00,282.17)(0.500,1.000){2}{\rule{0.120pt}{0.400pt}}
\put(1124.0,283.0){\rule[-0.200pt]{0.400pt}{6.263pt}}
\put(1125.0,284.0){\rule[-0.200pt]{0.400pt}{4.336pt}}
\put(1124.67,290){\rule{0.400pt}{4.577pt}}
\multiput(1124.17,290.00)(1.000,9.500){2}{\rule{0.400pt}{2.289pt}}
\put(1125.0,290.0){\rule[-0.200pt]{0.400pt}{2.891pt}}
\put(1126.0,291.0){\rule[-0.200pt]{0.400pt}{4.336pt}}
\put(1125.67,292){\rule{0.400pt}{2.891pt}}
\multiput(1125.17,292.00)(1.000,6.000){2}{\rule{0.400pt}{1.445pt}}
\put(1126.0,291.0){\usebox{\plotpoint}}
\put(1127.0,304.0){\rule[-0.200pt]{0.400pt}{1.927pt}}
\put(1126.67,289){\rule{0.400pt}{3.132pt}}
\multiput(1126.17,289.00)(1.000,6.500){2}{\rule{0.400pt}{1.566pt}}
\put(1127.0,289.0){\rule[-0.200pt]{0.400pt}{5.541pt}}
\put(1127.67,280){\rule{0.400pt}{3.132pt}}
\multiput(1127.17,286.50)(1.000,-6.500){2}{\rule{0.400pt}{1.566pt}}
\put(1128.0,293.0){\rule[-0.200pt]{0.400pt}{2.168pt}}
\put(1129.0,280.0){\rule[-0.200pt]{0.400pt}{2.409pt}}
\put(1128.67,283){\rule{0.400pt}{0.964pt}}
\multiput(1128.17,285.00)(1.000,-2.000){2}{\rule{0.400pt}{0.482pt}}
\put(1129.0,287.0){\rule[-0.200pt]{0.400pt}{0.723pt}}
\put(1130,283){\usebox{\plotpoint}}
\put(1129.67,279){\rule{0.400pt}{3.132pt}}
\multiput(1129.17,279.00)(1.000,6.500){2}{\rule{0.400pt}{1.566pt}}
\put(1130.0,279.0){\rule[-0.200pt]{0.400pt}{0.964pt}}
\put(1131.0,292.0){\rule[-0.200pt]{0.400pt}{3.132pt}}
\put(1130.67,281){\rule{0.400pt}{1.445pt}}
\multiput(1130.17,281.00)(1.000,3.000){2}{\rule{0.400pt}{0.723pt}}
\put(1131.0,281.0){\rule[-0.200pt]{0.400pt}{5.782pt}}
\put(1131.67,285){\rule{0.400pt}{1.927pt}}
\multiput(1131.17,285.00)(1.000,4.000){2}{\rule{0.400pt}{0.964pt}}
\put(1132.0,285.0){\rule[-0.200pt]{0.400pt}{0.482pt}}
\put(1133.0,293.0){\rule[-0.200pt]{0.400pt}{2.891pt}}
\put(1132.67,286){\rule{0.400pt}{1.927pt}}
\multiput(1132.17,286.00)(1.000,4.000){2}{\rule{0.400pt}{0.964pt}}
\put(1133.0,286.0){\rule[-0.200pt]{0.400pt}{4.577pt}}
\put(1134.0,294.0){\rule[-0.200pt]{0.400pt}{4.095pt}}
\put(1133.67,288){\rule{0.400pt}{4.336pt}}
\multiput(1133.17,297.00)(1.000,-9.000){2}{\rule{0.400pt}{2.168pt}}
\put(1134.0,306.0){\rule[-0.200pt]{0.400pt}{1.204pt}}
\put(1135,310.67){\rule{0.241pt}{0.400pt}}
\multiput(1135.00,311.17)(0.500,-1.000){2}{\rule{0.120pt}{0.400pt}}
\put(1135.0,288.0){\rule[-0.200pt]{0.400pt}{5.782pt}}
\put(1135.67,286){\rule{0.400pt}{2.891pt}}
\multiput(1135.17,292.00)(1.000,-6.000){2}{\rule{0.400pt}{1.445pt}}
\put(1136.0,298.0){\rule[-0.200pt]{0.400pt}{3.132pt}}
\put(1136.67,298){\rule{0.400pt}{3.132pt}}
\multiput(1136.17,298.00)(1.000,6.500){2}{\rule{0.400pt}{1.566pt}}
\put(1137.0,286.0){\rule[-0.200pt]{0.400pt}{2.891pt}}
\put(1137.67,296){\rule{0.400pt}{1.686pt}}
\multiput(1137.17,296.00)(1.000,3.500){2}{\rule{0.400pt}{0.843pt}}
\put(1138.0,296.0){\rule[-0.200pt]{0.400pt}{3.613pt}}
\put(1139.0,303.0){\rule[-0.200pt]{0.400pt}{0.482pt}}
\put(1138.67,278){\rule{0.400pt}{0.482pt}}
\multiput(1138.17,278.00)(1.000,1.000){2}{\rule{0.400pt}{0.241pt}}
\put(1139.0,278.0){\rule[-0.200pt]{0.400pt}{6.504pt}}
\put(1139.67,279){\rule{0.400pt}{7.468pt}}
\multiput(1139.17,294.50)(1.000,-15.500){2}{\rule{0.400pt}{3.734pt}}
\put(1140.0,280.0){\rule[-0.200pt]{0.400pt}{7.227pt}}
\put(1140.67,306){\rule{0.400pt}{0.482pt}}
\multiput(1140.17,306.00)(1.000,1.000){2}{\rule{0.400pt}{0.241pt}}
\put(1141.0,279.0){\rule[-0.200pt]{0.400pt}{6.504pt}}
\put(1142.0,286.0){\rule[-0.200pt]{0.400pt}{5.300pt}}
\put(1141.67,306){\rule{0.400pt}{0.964pt}}
\multiput(1141.17,308.00)(1.000,-2.000){2}{\rule{0.400pt}{0.482pt}}
\put(1142.0,286.0){\rule[-0.200pt]{0.400pt}{5.782pt}}
\put(1142.67,286){\rule{0.400pt}{5.300pt}}
\multiput(1142.17,286.00)(1.000,11.000){2}{\rule{0.400pt}{2.650pt}}
\put(1143.0,286.0){\rule[-0.200pt]{0.400pt}{4.818pt}}
\put(1144.0,277.0){\rule[-0.200pt]{0.400pt}{7.468pt}}
\put(1143.67,284){\rule{0.400pt}{2.891pt}}
\multiput(1143.17,290.00)(1.000,-6.000){2}{\rule{0.400pt}{1.445pt}}
\put(1144.0,277.0){\rule[-0.200pt]{0.400pt}{4.577pt}}
\put(1144.67,290){\rule{0.400pt}{2.891pt}}
\multiput(1144.17,290.00)(1.000,6.000){2}{\rule{0.400pt}{1.445pt}}
\put(1145.0,284.0){\rule[-0.200pt]{0.400pt}{1.445pt}}
\put(1146.0,296.0){\rule[-0.200pt]{0.400pt}{1.445pt}}
\put(1145.67,294){\rule{0.400pt}{3.132pt}}
\multiput(1145.17,300.50)(1.000,-6.500){2}{\rule{0.400pt}{1.566pt}}
\put(1146.0,296.0){\rule[-0.200pt]{0.400pt}{2.650pt}}
\put(1147.0,292.0){\rule[-0.200pt]{0.400pt}{0.482pt}}
\put(1146.67,306){\rule{0.400pt}{0.482pt}}
\multiput(1146.17,306.00)(1.000,1.000){2}{\rule{0.400pt}{0.241pt}}
\put(1147.0,292.0){\rule[-0.200pt]{0.400pt}{3.373pt}}
\put(1148.0,284.0){\rule[-0.200pt]{0.400pt}{5.782pt}}
\put(1147.67,297){\rule{0.400pt}{1.204pt}}
\multiput(1147.17,299.50)(1.000,-2.500){2}{\rule{0.400pt}{0.602pt}}
\put(1148.0,284.0){\rule[-0.200pt]{0.400pt}{4.336pt}}
\put(1148.67,277){\rule{0.400pt}{8.191pt}}
\multiput(1148.17,277.00)(1.000,17.000){2}{\rule{0.400pt}{4.095pt}}
\put(1149.0,277.0){\rule[-0.200pt]{0.400pt}{4.818pt}}
\put(1149.67,279){\rule{0.400pt}{2.650pt}}
\multiput(1149.17,279.00)(1.000,5.500){2}{\rule{0.400pt}{1.325pt}}
\put(1150.0,279.0){\rule[-0.200pt]{0.400pt}{7.709pt}}
\put(1151.0,290.0){\rule[-0.200pt]{0.400pt}{4.577pt}}
\put(1150.67,283){\rule{0.400pt}{3.132pt}}
\multiput(1150.17,283.00)(1.000,6.500){2}{\rule{0.400pt}{1.566pt}}
\put(1151.0,283.0){\rule[-0.200pt]{0.400pt}{6.263pt}}
\put(1152.0,296.0){\rule[-0.200pt]{0.400pt}{2.891pt}}
\put(1151.67,296){\rule{0.400pt}{1.204pt}}
\multiput(1151.17,298.50)(1.000,-2.500){2}{\rule{0.400pt}{0.602pt}}
\put(1152.0,301.0){\rule[-0.200pt]{0.400pt}{1.686pt}}
\put(1152.67,293){\rule{0.400pt}{2.891pt}}
\multiput(1152.17,299.00)(1.000,-6.000){2}{\rule{0.400pt}{1.445pt}}
\put(1153.0,296.0){\rule[-0.200pt]{0.400pt}{2.168pt}}
\put(1154.0,290.0){\rule[-0.200pt]{0.400pt}{0.723pt}}
\put(1153.67,288){\rule{0.400pt}{2.168pt}}
\multiput(1153.17,292.50)(1.000,-4.500){2}{\rule{0.400pt}{1.084pt}}
\put(1154.0,290.0){\rule[-0.200pt]{0.400pt}{1.686pt}}
\put(1155.0,283.0){\rule[-0.200pt]{0.400pt}{1.204pt}}
\put(1154.67,302){\rule{0.400pt}{1.927pt}}
\multiput(1154.17,306.00)(1.000,-4.000){2}{\rule{0.400pt}{0.964pt}}
\put(1155.0,283.0){\rule[-0.200pt]{0.400pt}{6.504pt}}
\put(1156.0,285.0){\rule[-0.200pt]{0.400pt}{4.095pt}}
\put(1156,302.67){\rule{0.241pt}{0.400pt}}
\multiput(1156.00,302.17)(0.500,1.000){2}{\rule{0.120pt}{0.400pt}}
\put(1156.0,285.0){\rule[-0.200pt]{0.400pt}{4.336pt}}
\put(1156.67,285){\rule{0.400pt}{3.132pt}}
\multiput(1156.17,285.00)(1.000,6.500){2}{\rule{0.400pt}{1.566pt}}
\put(1157.0,285.0){\rule[-0.200pt]{0.400pt}{4.577pt}}
\put(1158.0,290.0){\rule[-0.200pt]{0.400pt}{1.927pt}}
\put(1157.67,305){\rule{0.400pt}{0.723pt}}
\multiput(1157.17,305.00)(1.000,1.500){2}{\rule{0.400pt}{0.361pt}}
\put(1158.0,290.0){\rule[-0.200pt]{0.400pt}{3.613pt}}
\put(1159.0,296.0){\rule[-0.200pt]{0.400pt}{2.891pt}}
\put(1158.67,301){\rule{0.400pt}{1.686pt}}
\multiput(1158.17,304.50)(1.000,-3.500){2}{\rule{0.400pt}{0.843pt}}
\put(1159.0,296.0){\rule[-0.200pt]{0.400pt}{2.891pt}}
\put(1159.67,285){\rule{0.400pt}{3.614pt}}
\multiput(1159.17,285.00)(1.000,7.500){2}{\rule{0.400pt}{1.807pt}}
\put(1160.0,285.0){\rule[-0.200pt]{0.400pt}{3.854pt}}
\put(1161.0,300.0){\rule[-0.200pt]{0.400pt}{0.964pt}}
\put(1160.67,284){\rule{0.400pt}{1.204pt}}
\multiput(1160.17,284.00)(1.000,2.500){2}{\rule{0.400pt}{0.602pt}}
\put(1161.0,284.0){\rule[-0.200pt]{0.400pt}{4.818pt}}
\put(1161.67,296){\rule{0.400pt}{0.964pt}}
\multiput(1161.17,296.00)(1.000,2.000){2}{\rule{0.400pt}{0.482pt}}
\put(1162.0,289.0){\rule[-0.200pt]{0.400pt}{1.686pt}}
\put(1163,300){\usebox{\plotpoint}}
\put(1162.67,285){\rule{0.400pt}{5.782pt}}
\multiput(1162.17,285.00)(1.000,12.000){2}{\rule{0.400pt}{2.891pt}}
\put(1163.0,285.0){\rule[-0.200pt]{0.400pt}{3.613pt}}
\put(1163.67,292){\rule{0.400pt}{0.723pt}}
\multiput(1163.17,293.50)(1.000,-1.500){2}{\rule{0.400pt}{0.361pt}}
\put(1164.0,295.0){\rule[-0.200pt]{0.400pt}{3.373pt}}
\put(1165.0,290.0){\rule[-0.200pt]{0.400pt}{0.482pt}}
\put(1164.67,294){\rule{0.400pt}{1.927pt}}
\multiput(1164.17,298.00)(1.000,-4.000){2}{\rule{0.400pt}{0.964pt}}
\put(1165.0,290.0){\rule[-0.200pt]{0.400pt}{2.891pt}}
\put(1165.67,283){\rule{0.400pt}{5.782pt}}
\multiput(1165.17,295.00)(1.000,-12.000){2}{\rule{0.400pt}{2.891pt}}
\put(1166.0,294.0){\rule[-0.200pt]{0.400pt}{3.132pt}}
\put(1167.0,282.0){\usebox{\plotpoint}}
\put(1166.67,299){\rule{0.400pt}{0.723pt}}
\multiput(1166.17,300.50)(1.000,-1.500){2}{\rule{0.400pt}{0.361pt}}
\put(1167.0,282.0){\rule[-0.200pt]{0.400pt}{4.818pt}}
\put(1168.0,287.0){\rule[-0.200pt]{0.400pt}{2.891pt}}
\put(1167.67,288){\rule{0.400pt}{3.373pt}}
\multiput(1167.17,288.00)(1.000,7.000){2}{\rule{0.400pt}{1.686pt}}
\put(1168.0,287.0){\usebox{\plotpoint}}
\put(1169.0,302.0){\rule[-0.200pt]{0.400pt}{1.686pt}}
\put(1168.67,290){\rule{0.400pt}{3.854pt}}
\multiput(1168.17,290.00)(1.000,8.000){2}{\rule{0.400pt}{1.927pt}}
\put(1169.0,290.0){\rule[-0.200pt]{0.400pt}{4.577pt}}
\put(1169.67,287){\rule{0.400pt}{3.614pt}}
\multiput(1169.17,287.00)(1.000,7.500){2}{\rule{0.400pt}{1.807pt}}
\put(1170.0,287.0){\rule[-0.200pt]{0.400pt}{4.577pt}}
\put(1171.0,302.0){\rule[-0.200pt]{0.400pt}{1.204pt}}
\put(1170.67,280){\rule{0.400pt}{5.541pt}}
\multiput(1170.17,280.00)(1.000,11.500){2}{\rule{0.400pt}{2.770pt}}
\put(1171.0,280.0){\rule[-0.200pt]{0.400pt}{6.504pt}}
\put(1172.0,289.0){\rule[-0.200pt]{0.400pt}{3.373pt}}
\put(1172.0,289.0){\rule[-0.200pt]{0.400pt}{1.686pt}}
\put(1172.0,296.0){\usebox{\plotpoint}}
\put(1172.67,297){\rule{0.400pt}{1.204pt}}
\multiput(1172.17,299.50)(1.000,-2.500){2}{\rule{0.400pt}{0.602pt}}
\put(1173.0,296.0){\rule[-0.200pt]{0.400pt}{1.445pt}}
\put(1173.67,288){\rule{0.400pt}{1.686pt}}
\multiput(1173.17,291.50)(1.000,-3.500){2}{\rule{0.400pt}{0.843pt}}
\put(1174.0,295.0){\rule[-0.200pt]{0.400pt}{0.482pt}}
\put(1175.0,288.0){\rule[-0.200pt]{0.400pt}{0.482pt}}
\put(1174.67,285){\rule{0.400pt}{1.445pt}}
\multiput(1174.17,285.00)(1.000,3.000){2}{\rule{0.400pt}{0.723pt}}
\put(1175.0,285.0){\rule[-0.200pt]{0.400pt}{1.204pt}}
\put(1175.67,289){\rule{0.400pt}{3.854pt}}
\multiput(1175.17,297.00)(1.000,-8.000){2}{\rule{0.400pt}{1.927pt}}
\put(1176.0,291.0){\rule[-0.200pt]{0.400pt}{3.373pt}}
\put(1176.67,306){\rule{0.400pt}{0.964pt}}
\multiput(1176.17,306.00)(1.000,2.000){2}{\rule{0.400pt}{0.482pt}}
\put(1177.0,289.0){\rule[-0.200pt]{0.400pt}{4.095pt}}
\put(1177.67,288){\rule{0.400pt}{5.541pt}}
\multiput(1177.17,288.00)(1.000,11.500){2}{\rule{0.400pt}{2.770pt}}
\put(1178.0,288.0){\rule[-0.200pt]{0.400pt}{5.300pt}}
\put(1178.67,282){\rule{0.400pt}{1.445pt}}
\multiput(1178.17,282.00)(1.000,3.000){2}{\rule{0.400pt}{0.723pt}}
\put(1179.0,282.0){\rule[-0.200pt]{0.400pt}{6.986pt}}
\put(1180.0,281.0){\rule[-0.200pt]{0.400pt}{1.686pt}}
\put(1180.0,281.0){\rule[-0.200pt]{0.400pt}{6.745pt}}
\put(1180.0,309.0){\usebox{\plotpoint}}
\put(1181.0,285.0){\rule[-0.200pt]{0.400pt}{5.782pt}}
\put(1180.67,289){\rule{0.400pt}{1.927pt}}
\multiput(1180.17,289.00)(1.000,4.000){2}{\rule{0.400pt}{0.964pt}}
\put(1181.0,285.0){\rule[-0.200pt]{0.400pt}{0.964pt}}
\put(1181.67,288){\rule{0.400pt}{0.482pt}}
\multiput(1181.17,289.00)(1.000,-1.000){2}{\rule{0.400pt}{0.241pt}}
\put(1182.0,290.0){\rule[-0.200pt]{0.400pt}{1.686pt}}
\put(1182.67,299){\rule{0.400pt}{1.204pt}}
\multiput(1182.17,301.50)(1.000,-2.500){2}{\rule{0.400pt}{0.602pt}}
\put(1183.0,288.0){\rule[-0.200pt]{0.400pt}{3.854pt}}
\put(1184.0,292.0){\rule[-0.200pt]{0.400pt}{1.686pt}}
\put(1183.67,294){\rule{0.400pt}{1.927pt}}
\multiput(1183.17,298.00)(1.000,-4.000){2}{\rule{0.400pt}{0.964pt}}
\put(1184.0,292.0){\rule[-0.200pt]{0.400pt}{2.409pt}}
\put(1185.0,294.0){\rule[-0.200pt]{0.400pt}{2.650pt}}
\put(1184.67,294){\rule{0.400pt}{3.132pt}}
\multiput(1184.17,294.00)(1.000,6.500){2}{\rule{0.400pt}{1.566pt}}
\put(1185.0,294.0){\rule[-0.200pt]{0.400pt}{2.650pt}}
\put(1186.0,302.0){\rule[-0.200pt]{0.400pt}{1.204pt}}
\put(1185.67,290){\rule{0.400pt}{5.059pt}}
\multiput(1185.17,300.50)(1.000,-10.500){2}{\rule{0.400pt}{2.529pt}}
\put(1186.0,302.0){\rule[-0.200pt]{0.400pt}{2.168pt}}
\put(1186.67,299){\rule{0.400pt}{0.723pt}}
\multiput(1186.17,299.00)(1.000,1.500){2}{\rule{0.400pt}{0.361pt}}
\put(1187.0,290.0){\rule[-0.200pt]{0.400pt}{2.168pt}}
\put(1188.0,292.0){\rule[-0.200pt]{0.400pt}{2.409pt}}
\put(1187.67,290){\rule{0.400pt}{3.132pt}}
\multiput(1187.17,296.50)(1.000,-6.500){2}{\rule{0.400pt}{1.566pt}}
\put(1188.0,292.0){\rule[-0.200pt]{0.400pt}{2.650pt}}
\put(1188.67,281){\rule{0.400pt}{4.336pt}}
\multiput(1188.17,281.00)(1.000,9.000){2}{\rule{0.400pt}{2.168pt}}
\put(1189.0,281.0){\rule[-0.200pt]{0.400pt}{2.168pt}}
\put(1189.67,290){\rule{0.400pt}{0.964pt}}
\multiput(1189.17,290.00)(1.000,2.000){2}{\rule{0.400pt}{0.482pt}}
\put(1190.0,290.0){\rule[-0.200pt]{0.400pt}{2.168pt}}
\put(1190.67,297){\rule{0.400pt}{0.723pt}}
\multiput(1190.17,297.00)(1.000,1.500){2}{\rule{0.400pt}{0.361pt}}
\put(1191.0,294.0){\rule[-0.200pt]{0.400pt}{0.723pt}}
\put(1192.0,300.0){\rule[-0.200pt]{0.400pt}{2.650pt}}
\put(1191.67,291){\rule{0.400pt}{1.927pt}}
\multiput(1191.17,295.00)(1.000,-4.000){2}{\rule{0.400pt}{0.964pt}}
\put(1192.0,299.0){\rule[-0.200pt]{0.400pt}{2.891pt}}
\put(1193.0,289.0){\rule[-0.200pt]{0.400pt}{0.482pt}}
\put(1192.67,293){\rule{0.400pt}{4.336pt}}
\multiput(1192.17,293.00)(1.000,9.000){2}{\rule{0.400pt}{2.168pt}}
\put(1193.0,289.0){\rule[-0.200pt]{0.400pt}{0.964pt}}
\put(1193.67,282){\rule{0.400pt}{6.504pt}}
\multiput(1193.17,282.00)(1.000,13.500){2}{\rule{0.400pt}{3.252pt}}
\put(1194.0,282.0){\rule[-0.200pt]{0.400pt}{6.986pt}}
\put(1194.67,292){\rule{0.400pt}{4.336pt}}
\multiput(1194.17,301.00)(1.000,-9.000){2}{\rule{0.400pt}{2.168pt}}
\put(1195.0,309.0){\usebox{\plotpoint}}
\put(1196.0,292.0){\rule[-0.200pt]{0.400pt}{0.723pt}}
\put(1195.67,284){\rule{0.400pt}{4.577pt}}
\multiput(1195.17,284.00)(1.000,9.500){2}{\rule{0.400pt}{2.289pt}}
\put(1196.0,284.0){\rule[-0.200pt]{0.400pt}{2.650pt}}
\put(1197.0,303.0){\rule[-0.200pt]{0.400pt}{0.964pt}}
\put(1196.67,290){\rule{0.400pt}{1.204pt}}
\multiput(1196.17,290.00)(1.000,2.500){2}{\rule{0.400pt}{0.602pt}}
\put(1197.0,290.0){\rule[-0.200pt]{0.400pt}{4.095pt}}
\put(1198.0,295.0){\rule[-0.200pt]{0.400pt}{3.854pt}}
\put(1197.67,293){\rule{0.400pt}{0.482pt}}
\multiput(1197.17,294.00)(1.000,-1.000){2}{\rule{0.400pt}{0.241pt}}
\put(1198.0,295.0){\rule[-0.200pt]{0.400pt}{3.854pt}}
\put(1198.67,285){\rule{0.400pt}{6.504pt}}
\multiput(1198.17,298.50)(1.000,-13.500){2}{\rule{0.400pt}{3.252pt}}
\put(1199.0,293.0){\rule[-0.200pt]{0.400pt}{4.577pt}}
\put(1200.0,285.0){\rule[-0.200pt]{0.400pt}{5.782pt}}
\put(1199.67,279){\rule{0.400pt}{6.263pt}}
\multiput(1199.17,279.00)(1.000,13.000){2}{\rule{0.400pt}{3.132pt}}
\put(1200.0,279.0){\rule[-0.200pt]{0.400pt}{7.227pt}}
\put(1201.0,282.0){\rule[-0.200pt]{0.400pt}{5.541pt}}
\put(1200.67,303){\rule{0.400pt}{2.168pt}}
\multiput(1200.17,307.50)(1.000,-4.500){2}{\rule{0.400pt}{1.084pt}}
\put(1201.0,282.0){\rule[-0.200pt]{0.400pt}{7.227pt}}
\put(1202.0,303.0){\rule[-0.200pt]{0.400pt}{0.482pt}}
\put(1201.67,290){\rule{0.400pt}{0.723pt}}
\multiput(1201.17,291.50)(1.000,-1.500){2}{\rule{0.400pt}{0.361pt}}
\put(1202.0,293.0){\rule[-0.200pt]{0.400pt}{2.891pt}}
\put(1202.67,288){\rule{0.400pt}{3.373pt}}
\multiput(1202.17,295.00)(1.000,-7.000){2}{\rule{0.400pt}{1.686pt}}
\put(1203.0,290.0){\rule[-0.200pt]{0.400pt}{2.891pt}}
\put(1203.67,279){\rule{0.400pt}{5.541pt}}
\multiput(1203.17,279.00)(1.000,11.500){2}{\rule{0.400pt}{2.770pt}}
\put(1204.0,279.0){\rule[-0.200pt]{0.400pt}{2.168pt}}
\put(1205.0,302.0){\rule[-0.200pt]{0.400pt}{2.168pt}}
\put(1204.67,284){\rule{0.400pt}{3.614pt}}
\multiput(1204.17,291.50)(1.000,-7.500){2}{\rule{0.400pt}{1.807pt}}
\put(1205.0,299.0){\rule[-0.200pt]{0.400pt}{2.891pt}}
\put(1206.0,280.0){\rule[-0.200pt]{0.400pt}{0.964pt}}
\put(1205.67,280){\rule{0.400pt}{6.023pt}}
\multiput(1205.17,292.50)(1.000,-12.500){2}{\rule{0.400pt}{3.011pt}}
\put(1206.0,280.0){\rule[-0.200pt]{0.400pt}{6.022pt}}
\put(1207.0,280.0){\rule[-0.200pt]{0.400pt}{3.132pt}}
\put(1206.67,286){\rule{0.400pt}{0.482pt}}
\multiput(1206.17,287.00)(1.000,-1.000){2}{\rule{0.400pt}{0.241pt}}
\put(1207.0,288.0){\rule[-0.200pt]{0.400pt}{1.204pt}}
\put(1207.67,295){\rule{0.400pt}{2.650pt}}
\multiput(1207.17,300.50)(1.000,-5.500){2}{\rule{0.400pt}{1.325pt}}
\put(1208.0,286.0){\rule[-0.200pt]{0.400pt}{4.818pt}}
\put(1209.0,295.0){\rule[-0.200pt]{0.400pt}{0.482pt}}
\put(1208.67,279){\rule{0.400pt}{1.445pt}}
\multiput(1208.17,279.00)(1.000,3.000){2}{\rule{0.400pt}{0.723pt}}
\put(1209.0,279.0){\rule[-0.200pt]{0.400pt}{4.336pt}}
\put(1210.0,285.0){\rule[-0.200pt]{0.400pt}{6.504pt}}
\put(1209.67,278){\rule{0.400pt}{1.445pt}}
\multiput(1209.17,281.00)(1.000,-3.000){2}{\rule{0.400pt}{0.723pt}}
\put(1210.0,284.0){\rule[-0.200pt]{0.400pt}{6.745pt}}
\put(1211,278){\usebox{\plotpoint}}
\put(1211,296.67){\rule{0.241pt}{0.400pt}}
\multiput(1211.00,297.17)(0.500,-1.000){2}{\rule{0.120pt}{0.400pt}}
\put(1211.0,278.0){\rule[-0.200pt]{0.400pt}{4.818pt}}
\put(1211.67,302){\rule{0.400pt}{1.204pt}}
\multiput(1211.17,304.50)(1.000,-2.500){2}{\rule{0.400pt}{0.602pt}}
\put(1212.0,297.0){\rule[-0.200pt]{0.400pt}{2.409pt}}
\put(1213.0,289.0){\rule[-0.200pt]{0.400pt}{3.132pt}}
\put(1213,291.67){\rule{0.241pt}{0.400pt}}
\multiput(1213.00,291.17)(0.500,1.000){2}{\rule{0.120pt}{0.400pt}}
\put(1213.0,289.0){\rule[-0.200pt]{0.400pt}{0.723pt}}
\put(1214.0,293.0){\rule[-0.200pt]{0.400pt}{0.723pt}}
\put(1213.67,278){\rule{0.400pt}{2.409pt}}
\multiput(1213.17,283.00)(1.000,-5.000){2}{\rule{0.400pt}{1.204pt}}
\put(1214.0,288.0){\rule[-0.200pt]{0.400pt}{1.927pt}}
\put(1214.67,278){\rule{0.400pt}{8.191pt}}
\multiput(1214.17,295.00)(1.000,-17.000){2}{\rule{0.400pt}{4.095pt}}
\put(1215.0,278.0){\rule[-0.200pt]{0.400pt}{8.191pt}}
\put(1215.67,293){\rule{0.400pt}{4.336pt}}
\multiput(1215.17,302.00)(1.000,-9.000){2}{\rule{0.400pt}{2.168pt}}
\put(1216.0,278.0){\rule[-0.200pt]{0.400pt}{7.950pt}}
\put(1217.0,280.0){\rule[-0.200pt]{0.400pt}{3.132pt}}
\put(1216.67,298){\rule{0.400pt}{2.168pt}}
\multiput(1216.17,302.50)(1.000,-4.500){2}{\rule{0.400pt}{1.084pt}}
\put(1217.0,280.0){\rule[-0.200pt]{0.400pt}{6.504pt}}
\put(1217.67,284){\rule{0.400pt}{2.409pt}}
\multiput(1217.17,284.00)(1.000,5.000){2}{\rule{0.400pt}{1.204pt}}
\put(1218.0,284.0){\rule[-0.200pt]{0.400pt}{3.373pt}}
\put(1219.0,294.0){\rule[-0.200pt]{0.400pt}{2.650pt}}
\put(1218.67,299){\rule{0.400pt}{1.686pt}}
\multiput(1218.17,299.00)(1.000,3.500){2}{\rule{0.400pt}{0.843pt}}
\put(1219.0,299.0){\rule[-0.200pt]{0.400pt}{1.445pt}}
\put(1219.67,278){\rule{0.400pt}{0.964pt}}
\multiput(1219.17,278.00)(1.000,2.000){2}{\rule{0.400pt}{0.482pt}}
\put(1220.0,278.0){\rule[-0.200pt]{0.400pt}{6.745pt}}
\put(1220.67,307){\rule{0.400pt}{0.964pt}}
\multiput(1220.17,307.00)(1.000,2.000){2}{\rule{0.400pt}{0.482pt}}
\put(1221.0,282.0){\rule[-0.200pt]{0.400pt}{6.022pt}}
\put(1222.0,288.0){\rule[-0.200pt]{0.400pt}{5.541pt}}
\put(1221.67,293){\rule{0.400pt}{0.964pt}}
\multiput(1221.17,293.00)(1.000,2.000){2}{\rule{0.400pt}{0.482pt}}
\put(1222.0,288.0){\rule[-0.200pt]{0.400pt}{1.204pt}}
\put(1223.0,287.0){\rule[-0.200pt]{0.400pt}{2.409pt}}
\put(1222.67,287){\rule{0.400pt}{3.614pt}}
\multiput(1222.17,294.50)(1.000,-7.500){2}{\rule{0.400pt}{1.807pt}}
\put(1223.0,287.0){\rule[-0.200pt]{0.400pt}{3.613pt}}
\put(1223.67,277){\rule{0.400pt}{6.986pt}}
\multiput(1223.17,291.50)(1.000,-14.500){2}{\rule{0.400pt}{3.493pt}}
\put(1224.0,287.0){\rule[-0.200pt]{0.400pt}{4.577pt}}
\put(1225.0,277.0){\rule[-0.200pt]{0.400pt}{5.541pt}}
\put(1224.67,279){\rule{0.400pt}{3.854pt}}
\multiput(1224.17,287.00)(1.000,-8.000){2}{\rule{0.400pt}{1.927pt}}
\put(1225.0,295.0){\rule[-0.200pt]{0.400pt}{1.204pt}}
\put(1225.67,280){\rule{0.400pt}{6.986pt}}
\multiput(1225.17,294.50)(1.000,-14.500){2}{\rule{0.400pt}{3.493pt}}
\put(1226.0,279.0){\rule[-0.200pt]{0.400pt}{7.227pt}}
\put(1227.0,280.0){\rule[-0.200pt]{0.400pt}{6.986pt}}
\put(1227,295.67){\rule{0.241pt}{0.400pt}}
\multiput(1227.00,295.17)(0.500,1.000){2}{\rule{0.120pt}{0.400pt}}
\put(1227.0,296.0){\rule[-0.200pt]{0.400pt}{3.132pt}}
\put(1227.67,284){\rule{0.400pt}{4.818pt}}
\multiput(1227.17,284.00)(1.000,10.000){2}{\rule{0.400pt}{2.409pt}}
\put(1228.0,284.0){\rule[-0.200pt]{0.400pt}{3.132pt}}
\put(1229.0,291.0){\rule[-0.200pt]{0.400pt}{3.132pt}}
\put(1228.67,297){\rule{0.400pt}{2.409pt}}
\multiput(1228.17,302.00)(1.000,-5.000){2}{\rule{0.400pt}{1.204pt}}
\put(1229.0,291.0){\rule[-0.200pt]{0.400pt}{3.854pt}}
\put(1230.0,297.0){\rule[-0.200pt]{0.400pt}{1.686pt}}
\put(1230,294.67){\rule{0.241pt}{0.400pt}}
\multiput(1230.00,295.17)(0.500,-1.000){2}{\rule{0.120pt}{0.400pt}}
\put(1230.0,296.0){\rule[-0.200pt]{0.400pt}{1.927pt}}
\put(1231.0,295.0){\rule[-0.200pt]{0.400pt}{2.891pt}}
\put(1230.67,287){\rule{0.400pt}{5.541pt}}
\multiput(1230.17,287.00)(1.000,11.500){2}{\rule{0.400pt}{2.770pt}}
\put(1231.0,287.0){\rule[-0.200pt]{0.400pt}{4.818pt}}
\put(1232.0,286.0){\rule[-0.200pt]{0.400pt}{5.782pt}}
\put(1231.67,294){\rule{0.400pt}{1.204pt}}
\multiput(1231.17,296.50)(1.000,-2.500){2}{\rule{0.400pt}{0.602pt}}
\put(1232.0,286.0){\rule[-0.200pt]{0.400pt}{3.132pt}}
\put(1232.67,293){\rule{0.400pt}{3.373pt}}
\multiput(1232.17,293.00)(1.000,7.000){2}{\rule{0.400pt}{1.686pt}}
\put(1233.0,293.0){\usebox{\plotpoint}}
\put(1234,280.67){\rule{0.241pt}{0.400pt}}
\multiput(1234.00,280.17)(0.500,1.000){2}{\rule{0.120pt}{0.400pt}}
\put(1234.0,281.0){\rule[-0.200pt]{0.400pt}{6.263pt}}
\put(1235.0,282.0){\rule[-0.200pt]{0.400pt}{4.336pt}}
\put(1234.67,291){\rule{0.400pt}{4.095pt}}
\multiput(1234.17,291.00)(1.000,8.500){2}{\rule{0.400pt}{2.048pt}}
\put(1235.0,291.0){\rule[-0.200pt]{0.400pt}{2.168pt}}
\put(1236.0,286.0){\rule[-0.200pt]{0.400pt}{5.300pt}}
\put(1235.67,278){\rule{0.400pt}{2.409pt}}
\multiput(1235.17,283.00)(1.000,-5.000){2}{\rule{0.400pt}{1.204pt}}
\put(1236.0,286.0){\rule[-0.200pt]{0.400pt}{0.482pt}}
\put(1236.67,290){\rule{0.400pt}{2.168pt}}
\multiput(1236.17,290.00)(1.000,4.500){2}{\rule{0.400pt}{1.084pt}}
\put(1237.0,278.0){\rule[-0.200pt]{0.400pt}{2.891pt}}
\put(1238.0,278.0){\rule[-0.200pt]{0.400pt}{5.059pt}}
\put(1237.67,290){\rule{0.400pt}{3.373pt}}
\multiput(1237.17,297.00)(1.000,-7.000){2}{\rule{0.400pt}{1.686pt}}
\put(1238.0,278.0){\rule[-0.200pt]{0.400pt}{6.263pt}}
\put(1238.67,284){\rule{0.400pt}{5.782pt}}
\multiput(1238.17,296.00)(1.000,-12.000){2}{\rule{0.400pt}{2.891pt}}
\put(1239.0,290.0){\rule[-0.200pt]{0.400pt}{4.336pt}}
\put(1240.0,284.0){\rule[-0.200pt]{0.400pt}{6.263pt}}
\put(1239.67,287){\rule{0.400pt}{4.818pt}}
\multiput(1239.17,287.00)(1.000,10.000){2}{\rule{0.400pt}{2.409pt}}
\put(1240.0,287.0){\rule[-0.200pt]{0.400pt}{5.541pt}}
\put(1240.67,295){\rule{0.400pt}{1.927pt}}
\multiput(1240.17,299.00)(1.000,-4.000){2}{\rule{0.400pt}{0.964pt}}
\put(1241.0,303.0){\rule[-0.200pt]{0.400pt}{0.964pt}}
\put(1242.0,295.0){\rule[-0.200pt]{0.400pt}{1.204pt}}
\put(1242,281.67){\rule{0.241pt}{0.400pt}}
\multiput(1242.00,282.17)(0.500,-1.000){2}{\rule{0.120pt}{0.400pt}}
\put(1242.0,283.0){\rule[-0.200pt]{0.400pt}{4.095pt}}
\put(1242.67,299){\rule{0.400pt}{1.927pt}}
\multiput(1242.17,303.00)(1.000,-4.000){2}{\rule{0.400pt}{0.964pt}}
\put(1243.0,282.0){\rule[-0.200pt]{0.400pt}{6.022pt}}
\put(1243.67,280){\rule{0.400pt}{2.409pt}}
\multiput(1243.17,280.00)(1.000,5.000){2}{\rule{0.400pt}{1.204pt}}
\put(1244.0,280.0){\rule[-0.200pt]{0.400pt}{4.577pt}}
\put(1244.67,283){\rule{0.400pt}{6.263pt}}
\multiput(1244.17,283.00)(1.000,13.000){2}{\rule{0.400pt}{3.132pt}}
\put(1245.0,283.0){\rule[-0.200pt]{0.400pt}{1.686pt}}
\put(1246,281.67){\rule{0.241pt}{0.400pt}}
\multiput(1246.00,282.17)(0.500,-1.000){2}{\rule{0.120pt}{0.400pt}}
\put(1246.0,283.0){\rule[-0.200pt]{0.400pt}{6.263pt}}
\put(1246.67,289){\rule{0.400pt}{3.132pt}}
\multiput(1246.17,295.50)(1.000,-6.500){2}{\rule{0.400pt}{1.566pt}}
\put(1247.0,282.0){\rule[-0.200pt]{0.400pt}{4.818pt}}
\put(1247.67,293){\rule{0.400pt}{3.132pt}}
\multiput(1247.17,293.00)(1.000,6.500){2}{\rule{0.400pt}{1.566pt}}
\put(1248.0,289.0){\rule[-0.200pt]{0.400pt}{0.964pt}}
\put(1248.67,280){\rule{0.400pt}{4.577pt}}
\multiput(1248.17,289.50)(1.000,-9.500){2}{\rule{0.400pt}{2.289pt}}
\put(1249.0,299.0){\rule[-0.200pt]{0.400pt}{1.686pt}}
\put(1249.67,286){\rule{0.400pt}{5.059pt}}
\multiput(1249.17,286.00)(1.000,10.500){2}{\rule{0.400pt}{2.529pt}}
\put(1250.0,280.0){\rule[-0.200pt]{0.400pt}{1.445pt}}
\put(1251.0,289.0){\rule[-0.200pt]{0.400pt}{4.336pt}}
\put(1250.67,278){\rule{0.400pt}{2.891pt}}
\multiput(1250.17,284.00)(1.000,-6.000){2}{\rule{0.400pt}{1.445pt}}
\put(1251.0,289.0){\usebox{\plotpoint}}
\put(1251.67,294){\rule{0.400pt}{0.723pt}}
\multiput(1251.17,294.00)(1.000,1.500){2}{\rule{0.400pt}{0.361pt}}
\put(1252.0,278.0){\rule[-0.200pt]{0.400pt}{3.854pt}}
\put(1253.0,297.0){\rule[-0.200pt]{0.400pt}{1.927pt}}
\put(1253.0,305.0){\usebox{\plotpoint}}
\put(1254.0,293.0){\rule[-0.200pt]{0.400pt}{2.891pt}}
\put(1253.67,282){\rule{0.400pt}{2.891pt}}
\multiput(1253.17,288.00)(1.000,-6.000){2}{\rule{0.400pt}{1.445pt}}
\put(1254.0,293.0){\usebox{\plotpoint}}
\put(1255.0,282.0){\rule[-0.200pt]{0.400pt}{6.504pt}}
\put(1254.67,281){\rule{0.400pt}{0.723pt}}
\multiput(1254.17,282.50)(1.000,-1.500){2}{\rule{0.400pt}{0.361pt}}
\put(1255.0,284.0){\rule[-0.200pt]{0.400pt}{6.022pt}}
\put(1255.67,299){\rule{0.400pt}{1.686pt}}
\multiput(1255.17,302.50)(1.000,-3.500){2}{\rule{0.400pt}{0.843pt}}
\put(1256.0,281.0){\rule[-0.200pt]{0.400pt}{6.022pt}}
\put(1257.0,284.0){\rule[-0.200pt]{0.400pt}{3.613pt}}
\put(1256.67,291){\rule{0.400pt}{0.723pt}}
\multiput(1256.17,291.00)(1.000,1.500){2}{\rule{0.400pt}{0.361pt}}
\put(1257.0,284.0){\rule[-0.200pt]{0.400pt}{1.686pt}}
\put(1257.67,288){\rule{0.400pt}{5.300pt}}
\multiput(1257.17,299.00)(1.000,-11.000){2}{\rule{0.400pt}{2.650pt}}
\put(1258.0,294.0){\rule[-0.200pt]{0.400pt}{3.854pt}}
\put(1259.0,288.0){\rule[-0.200pt]{0.400pt}{3.373pt}}
\put(1258.67,290){\rule{0.400pt}{0.723pt}}
\multiput(1258.17,290.00)(1.000,1.500){2}{\rule{0.400pt}{0.361pt}}
\put(1259.0,290.0){\rule[-0.200pt]{0.400pt}{2.891pt}}
\put(1260.0,289.0){\rule[-0.200pt]{0.400pt}{0.964pt}}
\put(1259.67,300){\rule{0.400pt}{2.891pt}}
\multiput(1259.17,300.00)(1.000,6.000){2}{\rule{0.400pt}{1.445pt}}
\put(1260.0,289.0){\rule[-0.200pt]{0.400pt}{2.650pt}}
\put(1260.67,280){\rule{0.400pt}{6.023pt}}
\multiput(1260.17,292.50)(1.000,-12.500){2}{\rule{0.400pt}{3.011pt}}
\put(1261.0,305.0){\rule[-0.200pt]{0.400pt}{1.686pt}}
\put(1262,277.67){\rule{0.241pt}{0.400pt}}
\multiput(1262.00,278.17)(0.500,-1.000){2}{\rule{0.120pt}{0.400pt}}
\put(1262.0,279.0){\usebox{\plotpoint}}
\put(1263,309.67){\rule{0.241pt}{0.400pt}}
\multiput(1263.00,309.17)(0.500,1.000){2}{\rule{0.120pt}{0.400pt}}
\put(1263.0,278.0){\rule[-0.200pt]{0.400pt}{7.709pt}}
\put(1264.0,296.0){\rule[-0.200pt]{0.400pt}{3.613pt}}
\put(1263.67,291){\rule{0.400pt}{1.927pt}}
\multiput(1263.17,295.00)(1.000,-4.000){2}{\rule{0.400pt}{0.964pt}}
\put(1264.0,296.0){\rule[-0.200pt]{0.400pt}{0.723pt}}
\put(1264.67,300){\rule{0.400pt}{1.445pt}}
\multiput(1264.17,303.00)(1.000,-3.000){2}{\rule{0.400pt}{0.723pt}}
\put(1265.0,291.0){\rule[-0.200pt]{0.400pt}{3.613pt}}
\put(1265.67,282){\rule{0.400pt}{4.577pt}}
\multiput(1265.17,291.50)(1.000,-9.500){2}{\rule{0.400pt}{2.289pt}}
\put(1266.0,300.0){\usebox{\plotpoint}}
\put(1267.0,281.0){\usebox{\plotpoint}}
\put(1266.67,296){\rule{0.400pt}{0.723pt}}
\multiput(1266.17,296.00)(1.000,1.500){2}{\rule{0.400pt}{0.361pt}}
\put(1267.0,281.0){\rule[-0.200pt]{0.400pt}{3.613pt}}
\put(1268.0,288.0){\rule[-0.200pt]{0.400pt}{2.650pt}}
\put(1267.67,310){\rule{0.400pt}{0.723pt}}
\multiput(1267.17,310.00)(1.000,1.500){2}{\rule{0.400pt}{0.361pt}}
\put(1268.0,288.0){\rule[-0.200pt]{0.400pt}{5.300pt}}
\put(1269.0,306.0){\rule[-0.200pt]{0.400pt}{1.686pt}}
\put(1268.67,286){\rule{0.400pt}{6.023pt}}
\multiput(1268.17,298.50)(1.000,-12.500){2}{\rule{0.400pt}{3.011pt}}
\put(1269.0,306.0){\rule[-0.200pt]{0.400pt}{1.204pt}}
\put(1269.67,292){\rule{0.400pt}{3.132pt}}
\multiput(1269.17,292.00)(1.000,6.500){2}{\rule{0.400pt}{1.566pt}}
\put(1270.0,286.0){\rule[-0.200pt]{0.400pt}{1.445pt}}
\put(1270.67,279){\rule{0.400pt}{2.409pt}}
\multiput(1270.17,279.00)(1.000,5.000){2}{\rule{0.400pt}{1.204pt}}
\put(1271.0,279.0){\rule[-0.200pt]{0.400pt}{6.263pt}}
\put(1272.0,289.0){\rule[-0.200pt]{0.400pt}{1.445pt}}
\put(1272.0,295.0){\usebox{\plotpoint}}
\put(1273.0,287.0){\rule[-0.200pt]{0.400pt}{1.927pt}}
\put(1272.67,284){\rule{0.400pt}{3.614pt}}
\multiput(1272.17,291.50)(1.000,-7.500){2}{\rule{0.400pt}{1.807pt}}
\put(1273.0,287.0){\rule[-0.200pt]{0.400pt}{2.891pt}}
\put(1273.67,292){\rule{0.400pt}{3.614pt}}
\multiput(1273.17,299.50)(1.000,-7.500){2}{\rule{0.400pt}{1.807pt}}
\put(1274.0,284.0){\rule[-0.200pt]{0.400pt}{5.541pt}}
\put(1275.0,279.0){\rule[-0.200pt]{0.400pt}{3.132pt}}
\put(1274.67,280){\rule{0.400pt}{5.300pt}}
\multiput(1274.17,280.00)(1.000,11.000){2}{\rule{0.400pt}{2.650pt}}
\put(1275.0,279.0){\usebox{\plotpoint}}
\put(1276,301.67){\rule{0.241pt}{0.400pt}}
\multiput(1276.00,302.17)(0.500,-1.000){2}{\rule{0.120pt}{0.400pt}}
\put(1276.0,302.0){\usebox{\plotpoint}}
\put(1277,302){\usebox{\plotpoint}}
\put(1276.67,291){\rule{0.400pt}{3.132pt}}
\multiput(1276.17,291.00)(1.000,6.500){2}{\rule{0.400pt}{1.566pt}}
\put(1277.0,291.0){\rule[-0.200pt]{0.400pt}{2.650pt}}
\put(1277.67,288){\rule{0.400pt}{5.541pt}}
\multiput(1277.17,288.00)(1.000,11.500){2}{\rule{0.400pt}{2.770pt}}
\put(1278.0,288.0){\rule[-0.200pt]{0.400pt}{3.854pt}}
\put(1278.67,283){\rule{0.400pt}{6.023pt}}
\multiput(1278.17,283.00)(1.000,12.500){2}{\rule{0.400pt}{3.011pt}}
\put(1279.0,283.0){\rule[-0.200pt]{0.400pt}{6.745pt}}
\put(1280.0,287.0){\rule[-0.200pt]{0.400pt}{5.059pt}}
\put(1279.67,292){\rule{0.400pt}{4.336pt}}
\multiput(1279.17,301.00)(1.000,-9.000){2}{\rule{0.400pt}{2.168pt}}
\put(1280.0,287.0){\rule[-0.200pt]{0.400pt}{5.541pt}}
\put(1281.0,292.0){\rule[-0.200pt]{0.400pt}{3.373pt}}
\put(1281,292.67){\rule{0.241pt}{0.400pt}}
\multiput(1281.00,293.17)(0.500,-1.000){2}{\rule{0.120pt}{0.400pt}}
\put(1281.0,294.0){\rule[-0.200pt]{0.400pt}{2.891pt}}
\put(1282.0,293.0){\rule[-0.200pt]{0.400pt}{0.964pt}}
\put(1281.67,292){\rule{0.400pt}{1.686pt}}
\multiput(1281.17,292.00)(1.000,3.500){2}{\rule{0.400pt}{0.843pt}}
\put(1282.0,292.0){\rule[-0.200pt]{0.400pt}{1.204pt}}
\put(1282.67,278){\rule{0.400pt}{3.854pt}}
\multiput(1282.17,278.00)(1.000,8.000){2}{\rule{0.400pt}{1.927pt}}
\put(1283.0,278.0){\rule[-0.200pt]{0.400pt}{5.059pt}}
\put(1284.0,294.0){\rule[-0.200pt]{0.400pt}{1.204pt}}
\put(1283.67,285){\rule{0.400pt}{0.964pt}}
\multiput(1283.17,285.00)(1.000,2.000){2}{\rule{0.400pt}{0.482pt}}
\put(1284.0,285.0){\rule[-0.200pt]{0.400pt}{3.373pt}}
\put(1285.0,289.0){\rule[-0.200pt]{0.400pt}{4.095pt}}
\put(1284.67,301){\rule{0.400pt}{0.723pt}}
\multiput(1284.17,301.00)(1.000,1.500){2}{\rule{0.400pt}{0.361pt}}
\put(1285.0,301.0){\rule[-0.200pt]{0.400pt}{1.204pt}}
\put(1286.0,281.0){\rule[-0.200pt]{0.400pt}{5.541pt}}
\put(1286,290.67){\rule{0.241pt}{0.400pt}}
\multiput(1286.00,290.17)(0.500,1.000){2}{\rule{0.120pt}{0.400pt}}
\put(1286.0,281.0){\rule[-0.200pt]{0.400pt}{2.409pt}}
\put(1286.67,295){\rule{0.400pt}{3.373pt}}
\multiput(1286.17,295.00)(1.000,7.000){2}{\rule{0.400pt}{1.686pt}}
\put(1287.0,292.0){\rule[-0.200pt]{0.400pt}{0.723pt}}
\put(1287.67,289){\rule{0.400pt}{5.059pt}}
\multiput(1287.17,289.00)(1.000,10.500){2}{\rule{0.400pt}{2.529pt}}
\put(1288.0,289.0){\rule[-0.200pt]{0.400pt}{4.818pt}}
\put(1289.0,297.0){\rule[-0.200pt]{0.400pt}{3.132pt}}
\put(1288.67,283){\rule{0.400pt}{5.541pt}}
\multiput(1288.17,294.50)(1.000,-11.500){2}{\rule{0.400pt}{2.770pt}}
\put(1289.0,297.0){\rule[-0.200pt]{0.400pt}{2.168pt}}
\put(1289.67,288){\rule{0.400pt}{1.686pt}}
\multiput(1289.17,291.50)(1.000,-3.500){2}{\rule{0.400pt}{0.843pt}}
\put(1290.0,283.0){\rule[-0.200pt]{0.400pt}{2.891pt}}
\put(1290.67,282){\rule{0.400pt}{3.373pt}}
\multiput(1290.17,282.00)(1.000,7.000){2}{\rule{0.400pt}{1.686pt}}
\put(1291.0,282.0){\rule[-0.200pt]{0.400pt}{1.445pt}}
\put(1292.0,283.0){\rule[-0.200pt]{0.400pt}{3.132pt}}
\put(1291.67,287){\rule{0.400pt}{4.818pt}}
\multiput(1291.17,297.00)(1.000,-10.000){2}{\rule{0.400pt}{2.409pt}}
\put(1292.0,283.0){\rule[-0.200pt]{0.400pt}{5.782pt}}
\put(1292.67,296){\rule{0.400pt}{0.964pt}}
\multiput(1292.17,298.00)(1.000,-2.000){2}{\rule{0.400pt}{0.482pt}}
\put(1293.0,287.0){\rule[-0.200pt]{0.400pt}{3.132pt}}
\put(1293.67,296){\rule{0.400pt}{3.373pt}}
\multiput(1293.17,303.00)(1.000,-7.000){2}{\rule{0.400pt}{1.686pt}}
\put(1294.0,296.0){\rule[-0.200pt]{0.400pt}{3.373pt}}
\put(1294.67,283){\rule{0.400pt}{6.986pt}}
\multiput(1294.17,297.50)(1.000,-14.500){2}{\rule{0.400pt}{3.493pt}}
\put(1295.0,296.0){\rule[-0.200pt]{0.400pt}{3.854pt}}
\put(1296.0,283.0){\rule[-0.200pt]{0.400pt}{4.577pt}}
\put(1296,290.67){\rule{0.241pt}{0.400pt}}
\multiput(1296.00,290.17)(0.500,1.000){2}{\rule{0.120pt}{0.400pt}}
\put(1296.0,291.0){\rule[-0.200pt]{0.400pt}{2.650pt}}
\put(1296.67,285){\rule{0.400pt}{5.059pt}}
\multiput(1296.17,295.50)(1.000,-10.500){2}{\rule{0.400pt}{2.529pt}}
\put(1297.0,292.0){\rule[-0.200pt]{0.400pt}{3.373pt}}
\put(1298.0,281.0){\rule[-0.200pt]{0.400pt}{0.964pt}}
\put(1297.67,299){\rule{0.400pt}{0.482pt}}
\multiput(1297.17,300.00)(1.000,-1.000){2}{\rule{0.400pt}{0.241pt}}
\put(1298.0,281.0){\rule[-0.200pt]{0.400pt}{4.818pt}}
\put(1298.67,298){\rule{0.400pt}{2.650pt}}
\multiput(1298.17,303.50)(1.000,-5.500){2}{\rule{0.400pt}{1.325pt}}
\put(1299.0,299.0){\rule[-0.200pt]{0.400pt}{2.409pt}}
\put(1300.0,298.0){\rule[-0.200pt]{0.400pt}{1.686pt}}
\put(1299.67,284){\rule{0.400pt}{4.095pt}}
\multiput(1299.17,292.50)(1.000,-8.500){2}{\rule{0.400pt}{2.048pt}}
\put(1300.0,301.0){\rule[-0.200pt]{0.400pt}{0.964pt}}
\put(1301.0,284.0){\rule[-0.200pt]{0.400pt}{6.263pt}}
\put(1301,304.67){\rule{0.241pt}{0.400pt}}
\multiput(1301.00,305.17)(0.500,-1.000){2}{\rule{0.120pt}{0.400pt}}
\put(1301.0,306.0){\rule[-0.200pt]{0.400pt}{0.964pt}}
\put(1302.0,284.0){\rule[-0.200pt]{0.400pt}{5.059pt}}
\put(1301.67,284){\rule{0.400pt}{3.614pt}}
\multiput(1301.17,291.50)(1.000,-7.500){2}{\rule{0.400pt}{1.807pt}}
\put(1302.0,284.0){\rule[-0.200pt]{0.400pt}{3.613pt}}
\put(1302.67,287){\rule{0.400pt}{5.059pt}}
\multiput(1302.17,287.00)(1.000,10.500){2}{\rule{0.400pt}{2.529pt}}
\put(1303.0,284.0){\rule[-0.200pt]{0.400pt}{0.723pt}}
\put(1304.0,308.0){\rule[-0.200pt]{0.400pt}{0.723pt}}
\put(1303.67,285){\rule{0.400pt}{5.541pt}}
\multiput(1303.17,285.00)(1.000,11.500){2}{\rule{0.400pt}{2.770pt}}
\put(1304.0,285.0){\rule[-0.200pt]{0.400pt}{6.263pt}}
\put(1305.0,284.0){\rule[-0.200pt]{0.400pt}{5.782pt}}
\put(1304.67,306){\rule{0.400pt}{1.204pt}}
\multiput(1304.17,306.00)(1.000,2.500){2}{\rule{0.400pt}{0.602pt}}
\put(1305.0,284.0){\rule[-0.200pt]{0.400pt}{5.300pt}}
\put(1306.0,303.0){\rule[-0.200pt]{0.400pt}{1.927pt}}
\put(1305.67,298){\rule{0.400pt}{3.373pt}}
\multiput(1305.17,305.00)(1.000,-7.000){2}{\rule{0.400pt}{1.686pt}}
\put(1306.0,303.0){\rule[-0.200pt]{0.400pt}{2.168pt}}
\put(1307.0,298.0){\rule[-0.200pt]{0.400pt}{2.650pt}}
\put(1307,277.67){\rule{0.241pt}{0.400pt}}
\multiput(1307.00,278.17)(0.500,-1.000){2}{\rule{0.120pt}{0.400pt}}
\put(1307.0,279.0){\rule[-0.200pt]{0.400pt}{7.227pt}}
\put(1307.67,283){\rule{0.400pt}{6.745pt}}
\multiput(1307.17,297.00)(1.000,-14.000){2}{\rule{0.400pt}{3.373pt}}
\put(1308.0,278.0){\rule[-0.200pt]{0.400pt}{7.950pt}}
\put(1308.67,292){\rule{0.400pt}{4.577pt}}
\multiput(1308.17,301.50)(1.000,-9.500){2}{\rule{0.400pt}{2.289pt}}
\put(1309.0,283.0){\rule[-0.200pt]{0.400pt}{6.745pt}}
\put(1310.0,279.0){\rule[-0.200pt]{0.400pt}{3.132pt}}
\put(1309.67,293){\rule{0.400pt}{2.891pt}}
\multiput(1309.17,293.00)(1.000,6.000){2}{\rule{0.400pt}{1.445pt}}
\put(1310.0,279.0){\rule[-0.200pt]{0.400pt}{3.373pt}}
\put(1311.0,305.0){\rule[-0.200pt]{0.400pt}{1.927pt}}
\put(1310.67,297){\rule{0.400pt}{1.445pt}}
\multiput(1310.17,300.00)(1.000,-3.000){2}{\rule{0.400pt}{0.723pt}}
\put(1311.0,303.0){\rule[-0.200pt]{0.400pt}{2.409pt}}
\put(1311.67,282){\rule{0.400pt}{1.686pt}}
\multiput(1311.17,285.50)(1.000,-3.500){2}{\rule{0.400pt}{0.843pt}}
\put(1312.0,289.0){\rule[-0.200pt]{0.400pt}{1.927pt}}
\put(1312.67,289){\rule{0.400pt}{3.132pt}}
\multiput(1312.17,295.50)(1.000,-6.500){2}{\rule{0.400pt}{1.566pt}}
\put(1313.0,282.0){\rule[-0.200pt]{0.400pt}{4.818pt}}
\put(1313.67,292){\rule{0.400pt}{2.409pt}}
\multiput(1313.17,292.00)(1.000,5.000){2}{\rule{0.400pt}{1.204pt}}
\put(1314.0,289.0){\rule[-0.200pt]{0.400pt}{0.723pt}}
\put(1314.67,299){\rule{0.400pt}{2.409pt}}
\multiput(1314.17,299.00)(1.000,5.000){2}{\rule{0.400pt}{1.204pt}}
\put(1315.0,299.0){\rule[-0.200pt]{0.400pt}{0.723pt}}
\put(1315.67,279){\rule{0.400pt}{3.373pt}}
\multiput(1315.17,286.00)(1.000,-7.000){2}{\rule{0.400pt}{1.686pt}}
\put(1316.0,293.0){\rule[-0.200pt]{0.400pt}{3.854pt}}
\put(1316.67,302){\rule{0.400pt}{2.650pt}}
\multiput(1316.17,307.50)(1.000,-5.500){2}{\rule{0.400pt}{1.325pt}}
\put(1317.0,279.0){\rule[-0.200pt]{0.400pt}{8.191pt}}
\put(1318.0,302.0){\rule[-0.200pt]{0.400pt}{0.964pt}}
\put(1317.67,290){\rule{0.400pt}{4.818pt}}
\multiput(1317.17,290.00)(1.000,10.000){2}{\rule{0.400pt}{2.409pt}}
\put(1318.0,290.0){\rule[-0.200pt]{0.400pt}{3.854pt}}
\put(1318.67,285){\rule{0.400pt}{0.964pt}}
\multiput(1318.17,287.00)(1.000,-2.000){2}{\rule{0.400pt}{0.482pt}}
\put(1319.0,289.0){\rule[-0.200pt]{0.400pt}{5.059pt}}
\put(1319.67,279){\rule{0.400pt}{5.782pt}}
\multiput(1319.17,291.00)(1.000,-12.000){2}{\rule{0.400pt}{2.891pt}}
\put(1320.0,285.0){\rule[-0.200pt]{0.400pt}{4.336pt}}
\put(1321.0,279.0){\rule[-0.200pt]{0.400pt}{4.336pt}}
\put(1320.67,278){\rule{0.400pt}{5.059pt}}
\multiput(1320.17,278.00)(1.000,10.500){2}{\rule{0.400pt}{2.529pt}}
\put(1321.0,278.0){\rule[-0.200pt]{0.400pt}{4.577pt}}
\put(1321.67,309){\rule{0.400pt}{0.482pt}}
\multiput(1321.17,309.00)(1.000,1.000){2}{\rule{0.400pt}{0.241pt}}
\put(1322.0,299.0){\rule[-0.200pt]{0.400pt}{2.409pt}}
\put(1323.0,282.0){\rule[-0.200pt]{0.400pt}{6.986pt}}
\put(1322.67,288){\rule{0.400pt}{3.132pt}}
\multiput(1322.17,288.00)(1.000,6.500){2}{\rule{0.400pt}{1.566pt}}
\put(1323.0,282.0){\rule[-0.200pt]{0.400pt}{1.445pt}}
\put(1323.67,297){\rule{0.400pt}{2.168pt}}
\multiput(1323.17,297.00)(1.000,4.500){2}{\rule{0.400pt}{1.084pt}}
\put(1324.0,297.0){\rule[-0.200pt]{0.400pt}{0.964pt}}
\put(1325.0,306.0){\rule[-0.200pt]{0.400pt}{1.204pt}}
\put(1324.67,288){\rule{0.400pt}{3.614pt}}
\multiput(1324.17,288.00)(1.000,7.500){2}{\rule{0.400pt}{1.807pt}}
\put(1325.0,288.0){\rule[-0.200pt]{0.400pt}{5.541pt}}
\put(1326.0,303.0){\rule[-0.200pt]{0.400pt}{1.445pt}}
\put(1325.67,303){\rule{0.400pt}{2.168pt}}
\multiput(1325.17,303.00)(1.000,4.500){2}{\rule{0.400pt}{1.084pt}}
\put(1326.0,303.0){\rule[-0.200pt]{0.400pt}{1.445pt}}
\put(1327.0,278.0){\rule[-0.200pt]{0.400pt}{8.191pt}}
\put(1327,308.67){\rule{0.241pt}{0.400pt}}
\multiput(1327.00,309.17)(0.500,-1.000){2}{\rule{0.120pt}{0.400pt}}
\put(1327.0,278.0){\rule[-0.200pt]{0.400pt}{7.709pt}}
\put(1327.67,308){\rule{0.400pt}{0.482pt}}
\multiput(1327.17,308.00)(1.000,1.000){2}{\rule{0.400pt}{0.241pt}}
\put(1328.0,308.0){\usebox{\plotpoint}}
\put(1329.0,282.0){\rule[-0.200pt]{0.400pt}{6.745pt}}
\put(1328.67,297){\rule{0.400pt}{1.686pt}}
\multiput(1328.17,297.00)(1.000,3.500){2}{\rule{0.400pt}{0.843pt}}
\put(1329.0,282.0){\rule[-0.200pt]{0.400pt}{3.613pt}}
\put(1330.0,304.0){\rule[-0.200pt]{0.400pt}{0.723pt}}
\put(1329.67,279){\rule{0.400pt}{5.300pt}}
\multiput(1329.17,279.00)(1.000,11.000){2}{\rule{0.400pt}{2.650pt}}
\put(1330.0,279.0){\rule[-0.200pt]{0.400pt}{6.745pt}}
\put(1331.0,281.0){\rule[-0.200pt]{0.400pt}{4.818pt}}
\put(1330.67,305){\rule{0.400pt}{0.723pt}}
\multiput(1330.17,305.00)(1.000,1.500){2}{\rule{0.400pt}{0.361pt}}
\put(1331.0,281.0){\rule[-0.200pt]{0.400pt}{5.782pt}}
\put(1331.67,286){\rule{0.400pt}{1.445pt}}
\multiput(1331.17,289.00)(1.000,-3.000){2}{\rule{0.400pt}{0.723pt}}
\put(1332.0,292.0){\rule[-0.200pt]{0.400pt}{3.854pt}}
\put(1333,286){\usebox{\plotpoint}}
\put(1332.67,286){\rule{0.400pt}{2.650pt}}
\multiput(1332.17,286.00)(1.000,5.500){2}{\rule{0.400pt}{1.325pt}}
\put(1334,289.67){\rule{0.241pt}{0.400pt}}
\multiput(1334.00,290.17)(0.500,-1.000){2}{\rule{0.120pt}{0.400pt}}
\put(1334.0,291.0){\rule[-0.200pt]{0.400pt}{1.445pt}}
\put(1335.0,290.0){\usebox{\plotpoint}}
\put(1334.67,283){\rule{0.400pt}{0.482pt}}
\multiput(1334.17,283.00)(1.000,1.000){2}{\rule{0.400pt}{0.241pt}}
\put(1335.0,283.0){\rule[-0.200pt]{0.400pt}{1.927pt}}
\put(1336.0,285.0){\rule[-0.200pt]{0.400pt}{5.541pt}}
\put(1335.67,285){\rule{0.400pt}{3.373pt}}
\multiput(1335.17,292.00)(1.000,-7.000){2}{\rule{0.400pt}{1.686pt}}
\put(1336.0,299.0){\rule[-0.200pt]{0.400pt}{2.168pt}}
\put(1336.67,293){\rule{0.400pt}{1.686pt}}
\multiput(1336.17,293.00)(1.000,3.500){2}{\rule{0.400pt}{0.843pt}}
\put(1337.0,285.0){\rule[-0.200pt]{0.400pt}{1.927pt}}
\put(1338.0,300.0){\rule[-0.200pt]{0.400pt}{0.482pt}}
\put(1337.67,290){\rule{0.400pt}{5.541pt}}
\multiput(1337.17,290.00)(1.000,11.500){2}{\rule{0.400pt}{2.770pt}}
\put(1338.0,290.0){\rule[-0.200pt]{0.400pt}{2.891pt}}
\put(1339.0,287.0){\rule[-0.200pt]{0.400pt}{6.263pt}}
\put(1338.67,306){\rule{0.400pt}{0.723pt}}
\multiput(1338.17,307.50)(1.000,-1.500){2}{\rule{0.400pt}{0.361pt}}
\put(1339.0,287.0){\rule[-0.200pt]{0.400pt}{5.300pt}}
\put(1339.67,279){\rule{0.400pt}{7.950pt}}
\multiput(1339.17,295.50)(1.000,-16.500){2}{\rule{0.400pt}{3.975pt}}
\put(1340.0,306.0){\rule[-0.200pt]{0.400pt}{1.445pt}}
\put(1340.67,294){\rule{0.400pt}{4.336pt}}
\multiput(1340.17,303.00)(1.000,-9.000){2}{\rule{0.400pt}{2.168pt}}
\put(1341.0,279.0){\rule[-0.200pt]{0.400pt}{7.950pt}}
\put(1342.0,294.0){\rule[-0.200pt]{0.400pt}{1.445pt}}
\put(1341.67,290){\rule{0.400pt}{0.964pt}}
\multiput(1341.17,292.00)(1.000,-2.000){2}{\rule{0.400pt}{0.482pt}}
\put(1342.0,294.0){\rule[-0.200pt]{0.400pt}{1.445pt}}
\put(1343.0,281.0){\rule[-0.200pt]{0.400pt}{2.168pt}}
\put(1342.67,283){\rule{0.400pt}{1.686pt}}
\multiput(1342.17,286.50)(1.000,-3.500){2}{\rule{0.400pt}{0.843pt}}
\put(1343.0,281.0){\rule[-0.200pt]{0.400pt}{2.168pt}}
\put(1344.0,280.0){\rule[-0.200pt]{0.400pt}{0.723pt}}
\put(1343.67,283){\rule{0.400pt}{5.782pt}}
\multiput(1343.17,295.00)(1.000,-12.000){2}{\rule{0.400pt}{2.891pt}}
\put(1344.0,280.0){\rule[-0.200pt]{0.400pt}{6.504pt}}
\put(1344.67,278){\rule{0.400pt}{1.686pt}}
\multiput(1344.17,278.00)(1.000,3.500){2}{\rule{0.400pt}{0.843pt}}
\put(1345.0,278.0){\rule[-0.200pt]{0.400pt}{1.204pt}}
\put(1346.0,285.0){\rule[-0.200pt]{0.400pt}{4.818pt}}
\put(1345.67,279){\rule{0.400pt}{0.723pt}}
\multiput(1345.17,280.50)(1.000,-1.500){2}{\rule{0.400pt}{0.361pt}}
\put(1346.0,282.0){\rule[-0.200pt]{0.400pt}{5.541pt}}
\put(1347.0,279.0){\rule[-0.200pt]{0.400pt}{6.745pt}}
\put(1346.67,284){\rule{0.400pt}{5.782pt}}
\multiput(1346.17,284.00)(1.000,12.000){2}{\rule{0.400pt}{2.891pt}}
\put(1347.0,284.0){\rule[-0.200pt]{0.400pt}{5.541pt}}
\put(1348.0,308.0){\rule[-0.200pt]{0.400pt}{0.723pt}}
\put(1347.67,282){\rule{0.400pt}{3.132pt}}
\multiput(1347.17,282.00)(1.000,6.500){2}{\rule{0.400pt}{1.566pt}}
\put(1348.0,282.0){\rule[-0.200pt]{0.400pt}{6.986pt}}
\put(1348.67,285){\rule{0.400pt}{1.927pt}}
\multiput(1348.17,289.00)(1.000,-4.000){2}{\rule{0.400pt}{0.964pt}}
\put(1349.0,293.0){\rule[-0.200pt]{0.400pt}{0.482pt}}
\put(1350.0,280.0){\rule[-0.200pt]{0.400pt}{1.204pt}}
\put(1349.67,295){\rule{0.400pt}{1.445pt}}
\multiput(1349.17,298.00)(1.000,-3.000){2}{\rule{0.400pt}{0.723pt}}
\put(1350.0,280.0){\rule[-0.200pt]{0.400pt}{5.059pt}}
\put(1351.0,295.0){\rule[-0.200pt]{0.400pt}{0.964pt}}
\put(1350.67,284){\rule{0.400pt}{5.782pt}}
\multiput(1350.17,284.00)(1.000,12.000){2}{\rule{0.400pt}{2.891pt}}
\put(1351.0,284.0){\rule[-0.200pt]{0.400pt}{3.613pt}}
\put(1351.67,281){\rule{0.400pt}{3.132pt}}
\multiput(1351.17,281.00)(1.000,6.500){2}{\rule{0.400pt}{1.566pt}}
\put(1352.0,281.0){\rule[-0.200pt]{0.400pt}{6.504pt}}
\put(1352.67,280){\rule{0.400pt}{4.818pt}}
\multiput(1352.17,280.00)(1.000,10.000){2}{\rule{0.400pt}{2.409pt}}
\put(1353.0,280.0){\rule[-0.200pt]{0.400pt}{3.373pt}}
\put(1354.0,290.0){\rule[-0.200pt]{0.400pt}{2.409pt}}
\put(1353.67,291){\rule{0.400pt}{2.168pt}}
\multiput(1353.17,295.50)(1.000,-4.500){2}{\rule{0.400pt}{1.084pt}}
\put(1354.0,290.0){\rule[-0.200pt]{0.400pt}{2.409pt}}
\put(1355.0,279.0){\rule[-0.200pt]{0.400pt}{2.891pt}}
\put(1354.67,286){\rule{0.400pt}{3.132pt}}
\multiput(1354.17,292.50)(1.000,-6.500){2}{\rule{0.400pt}{1.566pt}}
\put(1355.0,279.0){\rule[-0.200pt]{0.400pt}{4.818pt}}
\put(1356.0,280.0){\rule[-0.200pt]{0.400pt}{1.445pt}}
\put(1355.67,289){\rule{0.400pt}{5.059pt}}
\multiput(1355.17,299.50)(1.000,-10.500){2}{\rule{0.400pt}{2.529pt}}
\put(1356.0,280.0){\rule[-0.200pt]{0.400pt}{7.227pt}}
\put(1357.0,277.0){\rule[-0.200pt]{0.400pt}{2.891pt}}
\put(1356.67,293){\rule{0.400pt}{3.373pt}}
\multiput(1356.17,300.00)(1.000,-7.000){2}{\rule{0.400pt}{1.686pt}}
\put(1357.0,277.0){\rule[-0.200pt]{0.400pt}{7.227pt}}
\put(1357.67,278){\rule{0.400pt}{2.168pt}}
\multiput(1357.17,278.00)(1.000,4.500){2}{\rule{0.400pt}{1.084pt}}
\put(1358.0,278.0){\rule[-0.200pt]{0.400pt}{3.613pt}}
\put(1359.0,287.0){\rule[-0.200pt]{0.400pt}{4.818pt}}
\put(1358.67,278){\rule{0.400pt}{1.927pt}}
\multiput(1358.17,282.00)(1.000,-4.000){2}{\rule{0.400pt}{0.964pt}}
\put(1359.0,286.0){\rule[-0.200pt]{0.400pt}{5.059pt}}
\put(1360.0,278.0){\rule[-0.200pt]{0.400pt}{4.336pt}}
\put(1359.67,283){\rule{0.400pt}{6.745pt}}
\multiput(1359.17,283.00)(1.000,14.000){2}{\rule{0.400pt}{3.373pt}}
\put(1360.0,283.0){\rule[-0.200pt]{0.400pt}{3.132pt}}
\put(1361.0,282.0){\rule[-0.200pt]{0.400pt}{6.986pt}}
\put(1360.67,293){\rule{0.400pt}{3.132pt}}
\multiput(1360.17,293.00)(1.000,6.500){2}{\rule{0.400pt}{1.566pt}}
\put(1361.0,282.0){\rule[-0.200pt]{0.400pt}{2.650pt}}
\put(1361.67,278){\rule{0.400pt}{0.723pt}}
\multiput(1361.17,278.00)(1.000,1.500){2}{\rule{0.400pt}{0.361pt}}
\put(1362.0,278.0){\rule[-0.200pt]{0.400pt}{6.745pt}}
\put(1363.0,281.0){\rule[-0.200pt]{0.400pt}{6.022pt}}
\put(1362.67,278){\rule{0.400pt}{4.818pt}}
\multiput(1362.17,288.00)(1.000,-10.000){2}{\rule{0.400pt}{2.409pt}}
\put(1363.0,298.0){\rule[-0.200pt]{0.400pt}{1.927pt}}
\put(1363.67,295){\rule{0.400pt}{4.336pt}}
\multiput(1363.17,295.00)(1.000,9.000){2}{\rule{0.400pt}{2.168pt}}
\put(1364.0,278.0){\rule[-0.200pt]{0.400pt}{4.095pt}}
\put(1365.0,278.0){\rule[-0.200pt]{0.400pt}{8.431pt}}
\put(1364.67,284){\rule{0.400pt}{0.723pt}}
\multiput(1364.17,284.00)(1.000,1.500){2}{\rule{0.400pt}{0.361pt}}
\put(1365.0,278.0){\rule[-0.200pt]{0.400pt}{1.445pt}}
\put(171,768){\rule{1pt}{1pt}}
\put(171,754){\rule{1pt}{1pt}}
\put(172,755){\rule{1pt}{1pt}}
\put(172,761){\rule{1pt}{1pt}}
\put(172,784){\rule{1pt}{1pt}}
\put(173,766){\rule{1pt}{1pt}}
\put(173,760){\rule{1pt}{1pt}}
\put(174,753){\rule{1pt}{1pt}}
\put(174,782){\rule{1pt}{1pt}}
\put(174,749){\rule{1pt}{1pt}}
\put(175,782){\rule{1pt}{1pt}}
\put(175,781){\rule{1pt}{1pt}}
\put(175,748){\rule{1pt}{1pt}}
\put(176,776){\rule{1pt}{1pt}}
\put(176,751){\rule{1pt}{1pt}}
\put(176,773){\rule{1pt}{1pt}}
\put(177,754){\rule{1pt}{1pt}}
\put(177,762){\rule{1pt}{1pt}}
\put(178,758){\rule{1pt}{1pt}}
\put(178,775){\rule{1pt}{1pt}}
\put(178,774){\rule{1pt}{1pt}}
\put(179,760){\rule{1pt}{1pt}}
\put(179,777){\rule{1pt}{1pt}}
\put(179,771){\rule{1pt}{1pt}}
\put(180,753){\rule{1pt}{1pt}}
\put(180,779){\rule{1pt}{1pt}}
\put(180,753){\rule{1pt}{1pt}}
\put(181,769){\rule{1pt}{1pt}}
\put(181,762){\rule{1pt}{1pt}}
\put(182,751){\rule{1pt}{1pt}}
\put(182,762){\rule{1pt}{1pt}}
\put(182,743){\rule{1pt}{1pt}}
\put(183,764){\rule{1pt}{1pt}}
\put(183,756){\rule{1pt}{1pt}}
\put(183,777){\rule{1pt}{1pt}}
\put(184,759){\rule{1pt}{1pt}}
\put(184,761){\rule{1pt}{1pt}}
\put(184,772){\rule{1pt}{1pt}}
\put(185,757){\rule{1pt}{1pt}}
\put(185,753){\rule{1pt}{1pt}}
\put(185,744){\rule{1pt}{1pt}}
\put(186,773){\rule{1pt}{1pt}}
\put(186,747){\rule{1pt}{1pt}}
\put(187,755){\rule{1pt}{1pt}}
\put(187,747){\rule{1pt}{1pt}}
\put(187,749){\rule{1pt}{1pt}}
\put(188,749){\rule{1pt}{1pt}}
\put(188,739){\rule{1pt}{1pt}}
\put(188,746){\rule{1pt}{1pt}}
\put(189,751){\rule{1pt}{1pt}}
\put(189,741){\rule{1pt}{1pt}}
\put(189,742){\rule{1pt}{1pt}}
\put(190,753){\rule{1pt}{1pt}}
\put(190,766){\rule{1pt}{1pt}}
\put(191,773){\rule{1pt}{1pt}}
\put(191,755){\rule{1pt}{1pt}}
\put(191,772){\rule{1pt}{1pt}}
\put(192,752){\rule{1pt}{1pt}}
\put(192,766){\rule{1pt}{1pt}}
\put(192,765){\rule{1pt}{1pt}}
\put(193,764){\rule{1pt}{1pt}}
\put(193,742){\rule{1pt}{1pt}}
\put(193,756){\rule{1pt}{1pt}}
\put(194,753){\rule{1pt}{1pt}}
\put(194,740){\rule{1pt}{1pt}}
\put(195,743){\rule{1pt}{1pt}}
\put(195,751){\rule{1pt}{1pt}}
\put(195,763){\rule{1pt}{1pt}}
\put(196,767){\rule{1pt}{1pt}}
\put(196,741){\rule{1pt}{1pt}}
\put(196,747){\rule{1pt}{1pt}}
\put(197,753){\rule{1pt}{1pt}}
\put(197,739){\rule{1pt}{1pt}}
\put(197,751){\rule{1pt}{1pt}}
\put(198,747){\rule{1pt}{1pt}}
\put(198,736){\rule{1pt}{1pt}}
\put(199,738){\rule{1pt}{1pt}}
\put(199,749){\rule{1pt}{1pt}}
\put(199,740){\rule{1pt}{1pt}}
\put(200,760){\rule{1pt}{1pt}}
\put(200,766){\rule{1pt}{1pt}}
\put(200,753){\rule{1pt}{1pt}}
\put(201,736){\rule{1pt}{1pt}}
\put(201,766){\rule{1pt}{1pt}}
\put(201,745){\rule{1pt}{1pt}}
\put(202,733){\rule{1pt}{1pt}}
\put(202,745){\rule{1pt}{1pt}}
\put(203,737){\rule{1pt}{1pt}}
\put(203,739){\rule{1pt}{1pt}}
\put(203,740){\rule{1pt}{1pt}}
\put(204,755){\rule{1pt}{1pt}}
\put(204,739){\rule{1pt}{1pt}}
\put(204,756){\rule{1pt}{1pt}}
\put(205,763){\rule{1pt}{1pt}}
\put(205,748){\rule{1pt}{1pt}}
\put(205,744){\rule{1pt}{1pt}}
\put(206,747){\rule{1pt}{1pt}}
\put(206,756){\rule{1pt}{1pt}}
\put(207,753){\rule{1pt}{1pt}}
\put(207,739){\rule{1pt}{1pt}}
\put(207,732){\rule{1pt}{1pt}}
\put(208,730){\rule{1pt}{1pt}}
\put(208,744){\rule{1pt}{1pt}}
\put(208,748){\rule{1pt}{1pt}}
\put(209,754){\rule{1pt}{1pt}}
\put(209,752){\rule{1pt}{1pt}}
\put(209,758){\rule{1pt}{1pt}}
\put(210,750){\rule{1pt}{1pt}}
\put(210,728){\rule{1pt}{1pt}}
\put(210,752){\rule{1pt}{1pt}}
\put(211,758){\rule{1pt}{1pt}}
\put(211,731){\rule{1pt}{1pt}}
\put(212,751){\rule{1pt}{1pt}}
\put(212,723){\rule{1pt}{1pt}}
\put(212,742){\rule{1pt}{1pt}}
\put(213,751){\rule{1pt}{1pt}}
\put(213,725){\rule{1pt}{1pt}}
\put(213,723){\rule{1pt}{1pt}}
\put(214,757){\rule{1pt}{1pt}}
\put(214,741){\rule{1pt}{1pt}}
\put(214,750){\rule{1pt}{1pt}}
\put(215,729){\rule{1pt}{1pt}}
\put(215,737){\rule{1pt}{1pt}}
\put(216,722){\rule{1pt}{1pt}}
\put(216,720){\rule{1pt}{1pt}}
\put(216,724){\rule{1pt}{1pt}}
\put(217,755){\rule{1pt}{1pt}}
\put(217,749){\rule{1pt}{1pt}}
\put(217,731){\rule{1pt}{1pt}}
\put(218,745){\rule{1pt}{1pt}}
\put(218,721){\rule{1pt}{1pt}}
\put(218,737){\rule{1pt}{1pt}}
\put(219,731){\rule{1pt}{1pt}}
\put(219,753){\rule{1pt}{1pt}}
\put(220,720){\rule{1pt}{1pt}}
\put(220,753){\rule{1pt}{1pt}}
\put(220,723){\rule{1pt}{1pt}}
\put(221,724){\rule{1pt}{1pt}}
\put(221,731){\rule{1pt}{1pt}}
\put(221,717){\rule{1pt}{1pt}}
\put(222,735){\rule{1pt}{1pt}}
\put(222,719){\rule{1pt}{1pt}}
\put(222,743){\rule{1pt}{1pt}}
\put(223,724){\rule{1pt}{1pt}}
\put(223,744){\rule{1pt}{1pt}}
\put(224,716){\rule{1pt}{1pt}}
\put(224,749){\rule{1pt}{1pt}}
\put(224,733){\rule{1pt}{1pt}}
\put(225,735){\rule{1pt}{1pt}}
\put(225,730){\rule{1pt}{1pt}}
\put(225,720){\rule{1pt}{1pt}}
\put(226,716){\rule{1pt}{1pt}}
\put(226,745){\rule{1pt}{1pt}}
\put(226,730){\rule{1pt}{1pt}}
\put(227,737){\rule{1pt}{1pt}}
\put(227,747){\rule{1pt}{1pt}}
\put(228,715){\rule{1pt}{1pt}}
\put(228,744){\rule{1pt}{1pt}}
\put(228,739){\rule{1pt}{1pt}}
\put(229,737){\rule{1pt}{1pt}}
\put(229,714){\rule{1pt}{1pt}}
\put(229,725){\rule{1pt}{1pt}}
\put(230,724){\rule{1pt}{1pt}}
\put(230,740){\rule{1pt}{1pt}}
\put(230,732){\rule{1pt}{1pt}}
\put(231,745){\rule{1pt}{1pt}}
\put(231,716){\rule{1pt}{1pt}}
\put(232,724){\rule{1pt}{1pt}}
\put(232,731){\rule{1pt}{1pt}}
\put(232,726){\rule{1pt}{1pt}}
\put(233,730){\rule{1pt}{1pt}}
\put(233,719){\rule{1pt}{1pt}}
\put(233,727){\rule{1pt}{1pt}}
\put(234,713){\rule{1pt}{1pt}}
\put(234,744){\rule{1pt}{1pt}}
\put(234,740){\rule{1pt}{1pt}}
\put(235,717){\rule{1pt}{1pt}}
\put(235,738){\rule{1pt}{1pt}}
\put(235,720){\rule{1pt}{1pt}}
\put(236,714){\rule{1pt}{1pt}}
\put(236,736){\rule{1pt}{1pt}}
\put(237,733){\rule{1pt}{1pt}}
\put(237,741){\rule{1pt}{1pt}}
\put(237,709){\rule{1pt}{1pt}}
\put(238,730){\rule{1pt}{1pt}}
\put(238,719){\rule{1pt}{1pt}}
\put(238,733){\rule{1pt}{1pt}}
\put(239,735){\rule{1pt}{1pt}}
\put(239,725){\rule{1pt}{1pt}}
\put(239,708){\rule{1pt}{1pt}}
\put(240,718){\rule{1pt}{1pt}}
\put(240,738){\rule{1pt}{1pt}}
\put(241,733){\rule{1pt}{1pt}}
\put(241,736){\rule{1pt}{1pt}}
\put(241,704){\rule{1pt}{1pt}}
\put(242,712){\rule{1pt}{1pt}}
\put(242,729){\rule{1pt}{1pt}}
\put(242,716){\rule{1pt}{1pt}}
\put(243,728){\rule{1pt}{1pt}}
\put(243,705){\rule{1pt}{1pt}}
\put(243,723){\rule{1pt}{1pt}}
\put(244,724){\rule{1pt}{1pt}}
\put(244,725){\rule{1pt}{1pt}}
\put(245,736){\rule{1pt}{1pt}}
\put(245,720){\rule{1pt}{1pt}}
\put(245,712){\rule{1pt}{1pt}}
\put(246,708){\rule{1pt}{1pt}}
\put(246,721){\rule{1pt}{1pt}}
\put(246,716){\rule{1pt}{1pt}}
\put(247,718){\rule{1pt}{1pt}}
\put(247,700){\rule{1pt}{1pt}}
\put(247,704){\rule{1pt}{1pt}}
\put(248,705){\rule{1pt}{1pt}}
\put(248,713){\rule{1pt}{1pt}}
\put(249,713){\rule{1pt}{1pt}}
\put(249,727){\rule{1pt}{1pt}}
\put(249,709){\rule{1pt}{1pt}}
\put(250,703){\rule{1pt}{1pt}}
\put(250,699){\rule{1pt}{1pt}}
\put(250,731){\rule{1pt}{1pt}}
\put(251,729){\rule{1pt}{1pt}}
\put(251,722){\rule{1pt}{1pt}}
\put(251,709){\rule{1pt}{1pt}}
\put(252,728){\rule{1pt}{1pt}}
\put(252,722){\rule{1pt}{1pt}}
\put(253,721){\rule{1pt}{1pt}}
\put(253,704){\rule{1pt}{1pt}}
\put(253,698){\rule{1pt}{1pt}}
\put(254,725){\rule{1pt}{1pt}}
\put(254,723){\rule{1pt}{1pt}}
\put(254,696){\rule{1pt}{1pt}}
\put(255,702){\rule{1pt}{1pt}}
\put(255,711){\rule{1pt}{1pt}}
\put(255,726){\rule{1pt}{1pt}}
\put(256,723){\rule{1pt}{1pt}}
\put(256,696){\rule{1pt}{1pt}}
\put(256,701){\rule{1pt}{1pt}}
\put(257,714){\rule{1pt}{1pt}}
\put(257,707){\rule{1pt}{1pt}}
\put(258,712){\rule{1pt}{1pt}}
\put(258,719){\rule{1pt}{1pt}}
\put(258,692){\rule{1pt}{1pt}}
\put(259,719){\rule{1pt}{1pt}}
\put(259,714){\rule{1pt}{1pt}}
\put(259,701){\rule{1pt}{1pt}}
\put(260,724){\rule{1pt}{1pt}}
\put(260,696){\rule{1pt}{1pt}}
\put(260,721){\rule{1pt}{1pt}}
\put(261,701){\rule{1pt}{1pt}}
\put(261,706){\rule{1pt}{1pt}}
\put(262,717){\rule{1pt}{1pt}}
\put(262,692){\rule{1pt}{1pt}}
\put(262,700){\rule{1pt}{1pt}}
\put(263,690){\rule{1pt}{1pt}}
\put(263,722){\rule{1pt}{1pt}}
\put(263,696){\rule{1pt}{1pt}}
\put(264,721){\rule{1pt}{1pt}}
\put(264,714){\rule{1pt}{1pt}}
\put(264,710){\rule{1pt}{1pt}}
\put(265,718){\rule{1pt}{1pt}}
\put(265,690){\rule{1pt}{1pt}}
\put(266,690){\rule{1pt}{1pt}}
\put(266,716){\rule{1pt}{1pt}}
\put(266,696){\rule{1pt}{1pt}}
\put(267,705){\rule{1pt}{1pt}}
\put(267,693){\rule{1pt}{1pt}}
\put(267,695){\rule{1pt}{1pt}}
\put(268,690){\rule{1pt}{1pt}}
\put(268,720){\rule{1pt}{1pt}}
\put(268,715){\rule{1pt}{1pt}}
\put(269,709){\rule{1pt}{1pt}}
\put(269,700){\rule{1pt}{1pt}}
\put(270,695){\rule{1pt}{1pt}}
\put(270,716){\rule{1pt}{1pt}}
\put(270,686){\rule{1pt}{1pt}}
\put(271,694){\rule{1pt}{1pt}}
\put(271,707){\rule{1pt}{1pt}}
\put(271,708){\rule{1pt}{1pt}}
\put(272,717){\rule{1pt}{1pt}}
\put(272,684){\rule{1pt}{1pt}}
\put(272,714){\rule{1pt}{1pt}}
\put(273,683){\rule{1pt}{1pt}}
\put(273,698){\rule{1pt}{1pt}}
\put(274,703){\rule{1pt}{1pt}}
\put(274,700){\rule{1pt}{1pt}}
\put(274,704){\rule{1pt}{1pt}}
\put(275,703){\rule{1pt}{1pt}}
\put(275,709){\rule{1pt}{1pt}}
\put(275,687){\rule{1pt}{1pt}}
\put(276,715){\rule{1pt}{1pt}}
\put(276,692){\rule{1pt}{1pt}}
\put(276,715){\rule{1pt}{1pt}}
\put(277,681){\rule{1pt}{1pt}}
\put(277,695){\rule{1pt}{1pt}}
\put(278,699){\rule{1pt}{1pt}}
\put(278,685){\rule{1pt}{1pt}}
\put(278,682){\rule{1pt}{1pt}}
\put(279,704){\rule{1pt}{1pt}}
\put(279,706){\rule{1pt}{1pt}}
\put(279,704){\rule{1pt}{1pt}}
\put(280,695){\rule{1pt}{1pt}}
\put(280,694){\rule{1pt}{1pt}}
\put(280,707){\rule{1pt}{1pt}}
\put(281,692){\rule{1pt}{1pt}}
\put(281,692){\rule{1pt}{1pt}}
\put(281,687){\rule{1pt}{1pt}}
\put(282,692){\rule{1pt}{1pt}}
\put(282,688){\rule{1pt}{1pt}}
\put(283,700){\rule{1pt}{1pt}}
\put(283,667){\rule{1pt}{1pt}}
\put(283,681){\rule{1pt}{1pt}}
\put(284,686){\rule{1pt}{1pt}}
\put(284,665){\rule{1pt}{1pt}}
\put(284,670){\rule{1pt}{1pt}}
\put(285,679){\rule{1pt}{1pt}}
\put(285,672){\rule{1pt}{1pt}}
\put(285,684){\rule{1pt}{1pt}}
\put(286,679){\rule{1pt}{1pt}}
\put(286,679){\rule{1pt}{1pt}}
\put(287,663){\rule{1pt}{1pt}}
\put(287,673){\rule{1pt}{1pt}}
\put(287,661){\rule{1pt}{1pt}}
\put(288,654){\rule{1pt}{1pt}}
\put(288,667){\rule{1pt}{1pt}}
\put(288,651){\rule{1pt}{1pt}}
\put(289,651){\rule{1pt}{1pt}}
\put(289,668){\rule{1pt}{1pt}}
\put(289,656){\rule{1pt}{1pt}}
\put(290,658){\rule{1pt}{1pt}}
\put(290,664){\rule{1pt}{1pt}}
\put(291,646){\rule{1pt}{1pt}}
\put(291,641){\rule{1pt}{1pt}}
\put(291,658){\rule{1pt}{1pt}}
\put(292,640){\rule{1pt}{1pt}}
\put(292,652){\rule{1pt}{1pt}}
\put(292,624){\rule{1pt}{1pt}}
\put(293,642){\rule{1pt}{1pt}}
\put(293,624){\rule{1pt}{1pt}}
\put(293,641){\rule{1pt}{1pt}}
\put(294,649){\rule{1pt}{1pt}}
\put(294,625){\rule{1pt}{1pt}}
\put(295,611){\rule{1pt}{1pt}}
\put(295,642){\rule{1pt}{1pt}}
\put(295,634){\rule{1pt}{1pt}}
\put(296,631){\rule{1pt}{1pt}}
\put(296,629){\rule{1pt}{1pt}}
\put(296,606){\rule{1pt}{1pt}}
\put(297,609){\rule{1pt}{1pt}}
\put(297,608){\rule{1pt}{1pt}}
\put(297,623){\rule{1pt}{1pt}}
\put(298,616){\rule{1pt}{1pt}}
\put(298,610){\rule{1pt}{1pt}}
\put(299,595){\rule{1pt}{1pt}}
\put(299,601){\rule{1pt}{1pt}}
\put(299,593){\rule{1pt}{1pt}}
\put(300,603){\rule{1pt}{1pt}}
\put(300,604){\rule{1pt}{1pt}}
\put(300,607){\rule{1pt}{1pt}}
\put(301,612){\rule{1pt}{1pt}}
\put(301,583){\rule{1pt}{1pt}}
\put(301,601){\rule{1pt}{1pt}}
\put(302,578){\rule{1pt}{1pt}}
\put(302,582){\rule{1pt}{1pt}}
\put(303,598){\rule{1pt}{1pt}}
\put(303,589){\rule{1pt}{1pt}}
\put(303,585){\rule{1pt}{1pt}}
\put(304,602){\rule{1pt}{1pt}}
\put(304,572){\rule{1pt}{1pt}}
\put(304,583){\rule{1pt}{1pt}}
\put(305,596){\rule{1pt}{1pt}}
\put(305,570){\rule{1pt}{1pt}}
\put(305,579){\rule{1pt}{1pt}}
\put(306,573){\rule{1pt}{1pt}}
\put(306,569){\rule{1pt}{1pt}}
\put(306,574){\rule{1pt}{1pt}}
\put(307,564){\rule{1pt}{1pt}}
\put(307,563){\rule{1pt}{1pt}}
\put(308,578){\rule{1pt}{1pt}}
\put(308,575){\rule{1pt}{1pt}}
\put(308,581){\rule{1pt}{1pt}}
\put(309,566){\rule{1pt}{1pt}}
\put(309,563){\rule{1pt}{1pt}}
\put(309,561){\rule{1pt}{1pt}}
\put(310,559){\rule{1pt}{1pt}}
\put(310,557){\rule{1pt}{1pt}}
\put(310,572){\rule{1pt}{1pt}}
\put(311,544){\rule{1pt}{1pt}}
\put(311,569){\rule{1pt}{1pt}}
\put(312,538){\rule{1pt}{1pt}}
\put(312,554){\rule{1pt}{1pt}}
\put(312,549){\rule{1pt}{1pt}}
\put(313,543){\rule{1pt}{1pt}}
\put(313,530){\rule{1pt}{1pt}}
\put(313,551){\rule{1pt}{1pt}}
\put(314,552){\rule{1pt}{1pt}}
\put(314,527){\rule{1pt}{1pt}}
\put(314,551){\rule{1pt}{1pt}}
\put(315,546){\rule{1pt}{1pt}}
\put(315,528){\rule{1pt}{1pt}}
\put(316,526){\rule{1pt}{1pt}}
\put(316,523){\rule{1pt}{1pt}}
\put(316,530){\rule{1pt}{1pt}}
\put(317,514){\rule{1pt}{1pt}}
\put(317,521){\rule{1pt}{1pt}}
\put(317,535){\rule{1pt}{1pt}}
\put(318,538){\rule{1pt}{1pt}}
\put(318,508){\rule{1pt}{1pt}}
\put(318,507){\rule{1pt}{1pt}}
\put(319,524){\rule{1pt}{1pt}}
\put(319,507){\rule{1pt}{1pt}}
\put(320,510){\rule{1pt}{1pt}}
\put(320,515){\rule{1pt}{1pt}}
\put(320,526){\rule{1pt}{1pt}}
\put(321,523){\rule{1pt}{1pt}}
\put(321,493){\rule{1pt}{1pt}}
\put(321,514){\rule{1pt}{1pt}}
\put(322,500){\rule{1pt}{1pt}}
\put(322,495){\rule{1pt}{1pt}}
\put(322,487){\rule{1pt}{1pt}}
\put(323,486){\rule{1pt}{1pt}}
\put(323,497){\rule{1pt}{1pt}}
\put(324,497){\rule{1pt}{1pt}}
\put(324,483){\rule{1pt}{1pt}}
\put(324,489){\rule{1pt}{1pt}}
\put(325,479){\rule{1pt}{1pt}}
\put(325,489){\rule{1pt}{1pt}}
\put(325,478){\rule{1pt}{1pt}}
\put(326,476){\rule{1pt}{1pt}}
\put(326,485){\rule{1pt}{1pt}}
\put(326,479){\rule{1pt}{1pt}}
\put(327,473){\rule{1pt}{1pt}}
\put(327,478){\rule{1pt}{1pt}}
\put(328,464){\rule{1pt}{1pt}}
\put(328,473){\rule{1pt}{1pt}}
\put(328,474){\rule{1pt}{1pt}}
\put(329,489){\rule{1pt}{1pt}}
\put(329,471){\rule{1pt}{1pt}}
\put(329,482){\rule{1pt}{1pt}}
\put(330,457){\rule{1pt}{1pt}}
\put(330,465){\rule{1pt}{1pt}}
\put(330,449){\rule{1pt}{1pt}}
\put(331,467){\rule{1pt}{1pt}}
\put(331,454){\rule{1pt}{1pt}}
\put(331,477){\rule{1pt}{1pt}}
\put(332,474){\rule{1pt}{1pt}}
\put(332,452){\rule{1pt}{1pt}}
\put(333,453){\rule{1pt}{1pt}}
\put(333,448){\rule{1pt}{1pt}}
\put(333,449){\rule{1pt}{1pt}}
\put(334,464){\rule{1pt}{1pt}}
\put(334,465){\rule{1pt}{1pt}}
\put(334,464){\rule{1pt}{1pt}}
\put(335,429){\rule{1pt}{1pt}}
\put(335,450){\rule{1pt}{1pt}}
\put(335,442){\rule{1pt}{1pt}}
\put(336,424){\rule{1pt}{1pt}}
\put(336,449){\rule{1pt}{1pt}}
\put(337,437){\rule{1pt}{1pt}}
\put(337,430){\rule{1pt}{1pt}}
\put(337,424){\rule{1pt}{1pt}}
\put(338,450){\rule{1pt}{1pt}}
\put(338,424){\rule{1pt}{1pt}}
\put(338,421){\rule{1pt}{1pt}}
\put(339,441){\rule{1pt}{1pt}}
\put(339,429){\rule{1pt}{1pt}}
\put(339,414){\rule{1pt}{1pt}}
\put(340,418){\rule{1pt}{1pt}}
\put(340,418){\rule{1pt}{1pt}}
\put(341,405){\rule{1pt}{1pt}}
\put(341,402){\rule{1pt}{1pt}}
\put(341,400){\rule{1pt}{1pt}}
\put(342,425){\rule{1pt}{1pt}}
\put(342,411){\rule{1pt}{1pt}}
\put(342,395){\rule{1pt}{1pt}}
\put(343,427){\rule{1pt}{1pt}}
\put(343,420){\rule{1pt}{1pt}}
\put(343,398){\rule{1pt}{1pt}}
\put(344,419){\rule{1pt}{1pt}}
\put(344,407){\rule{1pt}{1pt}}
\put(345,412){\rule{1pt}{1pt}}
\put(345,404){\rule{1pt}{1pt}}
\put(345,405){\rule{1pt}{1pt}}
\put(346,399){\rule{1pt}{1pt}}
\put(346,411){\rule{1pt}{1pt}}
\put(346,396){\rule{1pt}{1pt}}
\put(347,383){\rule{1pt}{1pt}}
\put(347,376){\rule{1pt}{1pt}}
\put(347,391){\rule{1pt}{1pt}}
\put(348,378){\rule{1pt}{1pt}}
\put(348,370){\rule{1pt}{1pt}}
\put(349,401){\rule{1pt}{1pt}}
\put(349,374){\rule{1pt}{1pt}}
\put(349,366){\rule{1pt}{1pt}}
\put(350,396){\rule{1pt}{1pt}}
\put(350,366){\rule{1pt}{1pt}}
\put(350,366){\rule{1pt}{1pt}}
\put(351,374){\rule{1pt}{1pt}}
\put(351,366){\rule{1pt}{1pt}}
\put(351,368){\rule{1pt}{1pt}}
\put(352,353){\rule{1pt}{1pt}}
\put(352,353){\rule{1pt}{1pt}}
\put(353,374){\rule{1pt}{1pt}}
\put(353,384){\rule{1pt}{1pt}}
\put(353,356){\rule{1pt}{1pt}}
\put(354,383){\rule{1pt}{1pt}}
\put(354,358){\rule{1pt}{1pt}}
\put(354,356){\rule{1pt}{1pt}}
\put(355,369){\rule{1pt}{1pt}}
\put(355,356){\rule{1pt}{1pt}}
\put(355,365){\rule{1pt}{1pt}}
\put(356,380){\rule{1pt}{1pt}}
\put(356,351){\rule{1pt}{1pt}}
\put(356,354){\rule{1pt}{1pt}}
\put(357,363){\rule{1pt}{1pt}}
\put(357,384){\rule{1pt}{1pt}}
\put(358,377){\rule{1pt}{1pt}}
\put(358,369){\rule{1pt}{1pt}}
\put(358,370){\rule{1pt}{1pt}}
\put(359,357){\rule{1pt}{1pt}}
\put(359,364){\rule{1pt}{1pt}}
\put(359,350){\rule{1pt}{1pt}}
\put(360,369){\rule{1pt}{1pt}}
\put(360,354){\rule{1pt}{1pt}}
\put(360,367){\rule{1pt}{1pt}}
\put(361,365){\rule{1pt}{1pt}}
\put(361,368){\rule{1pt}{1pt}}
\put(362,376){\rule{1pt}{1pt}}
\put(362,380){\rule{1pt}{1pt}}
\put(362,373){\rule{1pt}{1pt}}
\put(363,353){\rule{1pt}{1pt}}
\put(363,383){\rule{1pt}{1pt}}
\put(363,380){\rule{1pt}{1pt}}
\put(364,367){\rule{1pt}{1pt}}
\put(364,373){\rule{1pt}{1pt}}
\put(364,351){\rule{1pt}{1pt}}
\put(365,359){\rule{1pt}{1pt}}
\put(365,381){\rule{1pt}{1pt}}
\put(366,353){\rule{1pt}{1pt}}
\put(366,362){\rule{1pt}{1pt}}
\put(366,364){\rule{1pt}{1pt}}
\put(367,365){\rule{1pt}{1pt}}
\put(367,381){\rule{1pt}{1pt}}
\put(367,356){\rule{1pt}{1pt}}
\put(368,359){\rule{1pt}{1pt}}
\put(368,383){\rule{1pt}{1pt}}
\put(368,373){\rule{1pt}{1pt}}
\put(369,365){\rule{1pt}{1pt}}
\put(369,362){\rule{1pt}{1pt}}
\put(370,375){\rule{1pt}{1pt}}
\put(370,366){\rule{1pt}{1pt}}
\put(370,375){\rule{1pt}{1pt}}
\put(371,374){\rule{1pt}{1pt}}
\put(371,374){\rule{1pt}{1pt}}
\put(371,371){\rule{1pt}{1pt}}
\put(372,367){\rule{1pt}{1pt}}
\put(372,361){\rule{1pt}{1pt}}
\put(372,379){\rule{1pt}{1pt}}
\put(373,376){\rule{1pt}{1pt}}
\put(373,353){\rule{1pt}{1pt}}
\put(374,369){\rule{1pt}{1pt}}
\put(374,367){\rule{1pt}{1pt}}
\put(374,374){\rule{1pt}{1pt}}
\put(375,368){\rule{1pt}{1pt}}
\put(375,370){\rule{1pt}{1pt}}
\put(375,369){\rule{1pt}{1pt}}
\put(376,373){\rule{1pt}{1pt}}
\put(376,356){\rule{1pt}{1pt}}
\put(376,359){\rule{1pt}{1pt}}
\put(377,364){\rule{1pt}{1pt}}
\put(377,381){\rule{1pt}{1pt}}
\put(378,374){\rule{1pt}{1pt}}
\put(378,363){\rule{1pt}{1pt}}
\put(378,350){\rule{1pt}{1pt}}
\put(379,357){\rule{1pt}{1pt}}
\put(379,358){\rule{1pt}{1pt}}
\put(379,375){\rule{1pt}{1pt}}
\put(380,363){\rule{1pt}{1pt}}
\put(380,360){\rule{1pt}{1pt}}
\put(380,363){\rule{1pt}{1pt}}
\put(381,378){\rule{1pt}{1pt}}
\put(381,372){\rule{1pt}{1pt}}
\put(381,369){\rule{1pt}{1pt}}
\put(382,364){\rule{1pt}{1pt}}
\put(382,370){\rule{1pt}{1pt}}
\put(383,378){\rule{1pt}{1pt}}
\put(383,361){\rule{1pt}{1pt}}
\put(383,374){\rule{1pt}{1pt}}
\put(384,379){\rule{1pt}{1pt}}
\put(384,349){\rule{1pt}{1pt}}
\put(384,380){\rule{1pt}{1pt}}
\put(385,349){\rule{1pt}{1pt}}
\put(385,373){\rule{1pt}{1pt}}
\put(385,354){\rule{1pt}{1pt}}
\put(386,375){\rule{1pt}{1pt}}
\put(386,358){\rule{1pt}{1pt}}
\put(387,377){\rule{1pt}{1pt}}
\put(387,374){\rule{1pt}{1pt}}
\put(387,363){\rule{1pt}{1pt}}
\put(388,357){\rule{1pt}{1pt}}
\put(388,359){\rule{1pt}{1pt}}
\put(388,362){\rule{1pt}{1pt}}
\put(389,366){\rule{1pt}{1pt}}
\put(389,358){\rule{1pt}{1pt}}
\put(389,376){\rule{1pt}{1pt}}
\put(390,378){\rule{1pt}{1pt}}
\put(390,365){\rule{1pt}{1pt}}
\put(391,379){\rule{1pt}{1pt}}
\put(391,351){\rule{1pt}{1pt}}
\put(391,365){\rule{1pt}{1pt}}
\put(392,370){\rule{1pt}{1pt}}
\put(392,372){\rule{1pt}{1pt}}
\put(392,373){\rule{1pt}{1pt}}
\put(393,350){\rule{1pt}{1pt}}
\put(393,370){\rule{1pt}{1pt}}
\put(393,353){\rule{1pt}{1pt}}
\put(394,365){\rule{1pt}{1pt}}
\put(394,368){\rule{1pt}{1pt}}
\put(395,375){\rule{1pt}{1pt}}
\put(395,378){\rule{1pt}{1pt}}
\put(395,369){\rule{1pt}{1pt}}
\put(396,362){\rule{1pt}{1pt}}
\put(396,356){\rule{1pt}{1pt}}
\put(396,372){\rule{1pt}{1pt}}
\put(397,371){\rule{1pt}{1pt}}
\put(397,349){\rule{1pt}{1pt}}
\put(397,374){\rule{1pt}{1pt}}
\put(398,377){\rule{1pt}{1pt}}
\put(398,372){\rule{1pt}{1pt}}
\put(399,350){\rule{1pt}{1pt}}
\put(399,357){\rule{1pt}{1pt}}
\put(399,369){\rule{1pt}{1pt}}
\put(400,348){\rule{1pt}{1pt}}
\put(400,359){\rule{1pt}{1pt}}
\put(400,359){\rule{1pt}{1pt}}
\put(401,372){\rule{1pt}{1pt}}
\put(401,346){\rule{1pt}{1pt}}
\put(401,377){\rule{1pt}{1pt}}
\put(402,377){\rule{1pt}{1pt}}
\put(402,346){\rule{1pt}{1pt}}
\put(403,346){\rule{1pt}{1pt}}
\put(403,353){\rule{1pt}{1pt}}
\put(403,353){\rule{1pt}{1pt}}
\put(404,364){\rule{1pt}{1pt}}
\put(404,357){\rule{1pt}{1pt}}
\put(404,352){\rule{1pt}{1pt}}
\put(405,351){\rule{1pt}{1pt}}
\put(405,352){\rule{1pt}{1pt}}
\put(405,370){\rule{1pt}{1pt}}
\put(406,362){\rule{1pt}{1pt}}
\put(406,360){\rule{1pt}{1pt}}
\put(406,372){\rule{1pt}{1pt}}
\put(407,361){\rule{1pt}{1pt}}
\put(407,366){\rule{1pt}{1pt}}
\put(408,371){\rule{1pt}{1pt}}
\put(408,377){\rule{1pt}{1pt}}
\put(408,350){\rule{1pt}{1pt}}
\put(409,362){\rule{1pt}{1pt}}
\put(409,362){\rule{1pt}{1pt}}
\put(409,374){\rule{1pt}{1pt}}
\put(410,365){\rule{1pt}{1pt}}
\put(410,348){\rule{1pt}{1pt}}
\put(410,356){\rule{1pt}{1pt}}
\put(411,364){\rule{1pt}{1pt}}
\put(411,377){\rule{1pt}{1pt}}
\put(412,372){\rule{1pt}{1pt}}
\put(412,352){\rule{1pt}{1pt}}
\put(412,350){\rule{1pt}{1pt}}
\put(413,351){\rule{1pt}{1pt}}
\put(413,345){\rule{1pt}{1pt}}
\put(413,352){\rule{1pt}{1pt}}
\put(414,373){\rule{1pt}{1pt}}
\put(414,351){\rule{1pt}{1pt}}
\put(414,347){\rule{1pt}{1pt}}
\put(415,369){\rule{1pt}{1pt}}
\put(415,373){\rule{1pt}{1pt}}
\put(416,371){\rule{1pt}{1pt}}
\put(416,353){\rule{1pt}{1pt}}
\put(416,363){\rule{1pt}{1pt}}
\put(417,360){\rule{1pt}{1pt}}
\put(417,346){\rule{1pt}{1pt}}
\put(417,346){\rule{1pt}{1pt}}
\put(418,360){\rule{1pt}{1pt}}
\put(418,348){\rule{1pt}{1pt}}
\put(418,371){\rule{1pt}{1pt}}
\put(419,370){\rule{1pt}{1pt}}
\put(419,359){\rule{1pt}{1pt}}
\put(420,351){\rule{1pt}{1pt}}
\put(420,366){\rule{1pt}{1pt}}
\put(420,375){\rule{1pt}{1pt}}
\put(421,348){\rule{1pt}{1pt}}
\put(421,376){\rule{1pt}{1pt}}
\put(421,354){\rule{1pt}{1pt}}
\put(422,360){\rule{1pt}{1pt}}
\put(422,362){\rule{1pt}{1pt}}
\put(422,346){\rule{1pt}{1pt}}
\put(423,347){\rule{1pt}{1pt}}
\put(423,353){\rule{1pt}{1pt}}
\put(424,361){\rule{1pt}{1pt}}
\put(424,364){\rule{1pt}{1pt}}
\put(424,369){\rule{1pt}{1pt}}
\put(425,345){\rule{1pt}{1pt}}
\put(425,361){\rule{1pt}{1pt}}
\put(425,369){\rule{1pt}{1pt}}
\put(426,369){\rule{1pt}{1pt}}
\put(426,358){\rule{1pt}{1pt}}
\put(426,373){\rule{1pt}{1pt}}
\put(427,360){\rule{1pt}{1pt}}
\put(427,357){\rule{1pt}{1pt}}
\put(427,373){\rule{1pt}{1pt}}
\put(428,350){\rule{1pt}{1pt}}
\put(428,355){\rule{1pt}{1pt}}
\put(429,368){\rule{1pt}{1pt}}
\put(429,375){\rule{1pt}{1pt}}
\put(429,348){\rule{1pt}{1pt}}
\put(430,355){\rule{1pt}{1pt}}
\put(430,353){\rule{1pt}{1pt}}
\put(430,369){\rule{1pt}{1pt}}
\put(431,368){\rule{1pt}{1pt}}
\put(431,340){\rule{1pt}{1pt}}
\put(431,364){\rule{1pt}{1pt}}
\put(432,360){\rule{1pt}{1pt}}
\put(432,357){\rule{1pt}{1pt}}
\put(433,343){\rule{1pt}{1pt}}
\put(433,347){\rule{1pt}{1pt}}
\put(433,364){\rule{1pt}{1pt}}
\put(434,351){\rule{1pt}{1pt}}
\put(434,359){\rule{1pt}{1pt}}
\put(434,344){\rule{1pt}{1pt}}
\put(435,350){\rule{1pt}{1pt}}
\put(435,353){\rule{1pt}{1pt}}
\put(435,340){\rule{1pt}{1pt}}
\put(436,358){\rule{1pt}{1pt}}
\put(436,339){\rule{1pt}{1pt}}
\put(437,358){\rule{1pt}{1pt}}
\put(437,348){\rule{1pt}{1pt}}
\put(437,354){\rule{1pt}{1pt}}
\put(438,352){\rule{1pt}{1pt}}
\put(438,341){\rule{1pt}{1pt}}
\put(438,372){\rule{1pt}{1pt}}
\put(439,370){\rule{1pt}{1pt}}
\put(439,355){\rule{1pt}{1pt}}
\put(439,342){\rule{1pt}{1pt}}
\put(440,365){\rule{1pt}{1pt}}
\put(440,350){\rule{1pt}{1pt}}
\put(441,340){\rule{1pt}{1pt}}
\put(441,363){\rule{1pt}{1pt}}
\put(441,364){\rule{1pt}{1pt}}
\put(442,359){\rule{1pt}{1pt}}
\put(442,370){\rule{1pt}{1pt}}
\put(442,363){\rule{1pt}{1pt}}
\put(443,361){\rule{1pt}{1pt}}
\put(443,349){\rule{1pt}{1pt}}
\put(443,351){\rule{1pt}{1pt}}
\put(444,351){\rule{1pt}{1pt}}
\put(444,364){\rule{1pt}{1pt}}
\put(445,350){\rule{1pt}{1pt}}
\put(445,357){\rule{1pt}{1pt}}
\put(445,337){\rule{1pt}{1pt}}
\put(446,363){\rule{1pt}{1pt}}
\put(446,340){\rule{1pt}{1pt}}
\put(446,343){\rule{1pt}{1pt}}
\put(447,347){\rule{1pt}{1pt}}
\put(447,359){\rule{1pt}{1pt}}
\put(447,346){\rule{1pt}{1pt}}
\put(448,364){\rule{1pt}{1pt}}
\put(448,363){\rule{1pt}{1pt}}
\put(449,340){\rule{1pt}{1pt}}
\put(449,356){\rule{1pt}{1pt}}
\put(449,370){\rule{1pt}{1pt}}
\put(450,353){\rule{1pt}{1pt}}
\put(450,362){\rule{1pt}{1pt}}
\put(450,352){\rule{1pt}{1pt}}
\put(451,356){\rule{1pt}{1pt}}
\put(451,364){\rule{1pt}{1pt}}
\put(451,351){\rule{1pt}{1pt}}
\put(452,341){\rule{1pt}{1pt}}
\put(452,352){\rule{1pt}{1pt}}
\put(452,358){\rule{1pt}{1pt}}
\put(453,360){\rule{1pt}{1pt}}
\put(453,346){\rule{1pt}{1pt}}
\put(454,355){\rule{1pt}{1pt}}
\put(454,340){\rule{1pt}{1pt}}
\put(454,355){\rule{1pt}{1pt}}
\put(455,356){\rule{1pt}{1pt}}
\put(455,361){\rule{1pt}{1pt}}
\put(455,357){\rule{1pt}{1pt}}
\put(456,352){\rule{1pt}{1pt}}
\put(456,350){\rule{1pt}{1pt}}
\put(456,354){\rule{1pt}{1pt}}
\put(457,347){\rule{1pt}{1pt}}
\put(457,365){\rule{1pt}{1pt}}
\put(458,357){\rule{1pt}{1pt}}
\put(458,369){\rule{1pt}{1pt}}
\put(458,359){\rule{1pt}{1pt}}
\put(459,357){\rule{1pt}{1pt}}
\put(459,370){\rule{1pt}{1pt}}
\put(459,353){\rule{1pt}{1pt}}
\put(460,366){\rule{1pt}{1pt}}
\put(460,356){\rule{1pt}{1pt}}
\put(460,356){\rule{1pt}{1pt}}
\put(461,352){\rule{1pt}{1pt}}
\put(461,365){\rule{1pt}{1pt}}
\put(462,367){\rule{1pt}{1pt}}
\put(462,348){\rule{1pt}{1pt}}
\put(462,352){\rule{1pt}{1pt}}
\put(463,358){\rule{1pt}{1pt}}
\put(463,368){\rule{1pt}{1pt}}
\put(463,367){\rule{1pt}{1pt}}
\put(464,353){\rule{1pt}{1pt}}
\put(464,349){\rule{1pt}{1pt}}
\put(464,344){\rule{1pt}{1pt}}
\put(465,339){\rule{1pt}{1pt}}
\put(465,350){\rule{1pt}{1pt}}
\put(466,369){\rule{1pt}{1pt}}
\put(466,346){\rule{1pt}{1pt}}
\put(466,365){\rule{1pt}{1pt}}
\put(467,361){\rule{1pt}{1pt}}
\put(467,360){\rule{1pt}{1pt}}
\put(467,356){\rule{1pt}{1pt}}
\put(468,343){\rule{1pt}{1pt}}
\put(468,357){\rule{1pt}{1pt}}
\put(468,345){\rule{1pt}{1pt}}
\put(469,367){\rule{1pt}{1pt}}
\put(469,335){\rule{1pt}{1pt}}
\put(470,347){\rule{1pt}{1pt}}
\put(470,363){\rule{1pt}{1pt}}
\put(470,346){\rule{1pt}{1pt}}
\put(471,341){\rule{1pt}{1pt}}
\put(471,369){\rule{1pt}{1pt}}
\put(471,353){\rule{1pt}{1pt}}
\put(472,354){\rule{1pt}{1pt}}
\put(472,335){\rule{1pt}{1pt}}
\put(472,346){\rule{1pt}{1pt}}
\put(473,356){\rule{1pt}{1pt}}
\put(473,336){\rule{1pt}{1pt}}
\put(474,350){\rule{1pt}{1pt}}
\put(474,358){\rule{1pt}{1pt}}
\put(474,336){\rule{1pt}{1pt}}
\put(475,346){\rule{1pt}{1pt}}
\put(475,339){\rule{1pt}{1pt}}
\put(475,362){\rule{1pt}{1pt}}
\put(476,343){\rule{1pt}{1pt}}
\put(476,358){\rule{1pt}{1pt}}
\put(476,360){\rule{1pt}{1pt}}
\put(477,348){\rule{1pt}{1pt}}
\put(477,346){\rule{1pt}{1pt}}
\put(477,333){\rule{1pt}{1pt}}
\put(478,368){\rule{1pt}{1pt}}
\put(478,346){\rule{1pt}{1pt}}
\put(479,365){\rule{1pt}{1pt}}
\put(479,367){\rule{1pt}{1pt}}
\put(479,348){\rule{1pt}{1pt}}
\put(480,346){\rule{1pt}{1pt}}
\put(480,341){\rule{1pt}{1pt}}
\put(480,345){\rule{1pt}{1pt}}
\put(481,360){\rule{1pt}{1pt}}
\put(481,343){\rule{1pt}{1pt}}
\put(481,366){\rule{1pt}{1pt}}
\put(482,362){\rule{1pt}{1pt}}
\put(482,364){\rule{1pt}{1pt}}
\put(483,350){\rule{1pt}{1pt}}
\put(483,347){\rule{1pt}{1pt}}
\put(483,356){\rule{1pt}{1pt}}
\put(484,346){\rule{1pt}{1pt}}
\put(484,333){\rule{1pt}{1pt}}
\put(484,343){\rule{1pt}{1pt}}
\put(485,347){\rule{1pt}{1pt}}
\put(485,344){\rule{1pt}{1pt}}
\put(485,335){\rule{1pt}{1pt}}
\put(486,359){\rule{1pt}{1pt}}
\put(486,367){\rule{1pt}{1pt}}
\put(487,354){\rule{1pt}{1pt}}
\put(487,359){\rule{1pt}{1pt}}
\put(487,366){\rule{1pt}{1pt}}
\put(488,344){\rule{1pt}{1pt}}
\put(488,350){\rule{1pt}{1pt}}
\put(488,349){\rule{1pt}{1pt}}
\put(489,352){\rule{1pt}{1pt}}
\put(489,364){\rule{1pt}{1pt}}
\put(489,353){\rule{1pt}{1pt}}
\put(490,334){\rule{1pt}{1pt}}
\put(490,350){\rule{1pt}{1pt}}
\put(491,354){\rule{1pt}{1pt}}
\put(491,341){\rule{1pt}{1pt}}
\put(491,357){\rule{1pt}{1pt}}
\put(492,365){\rule{1pt}{1pt}}
\put(492,364){\rule{1pt}{1pt}}
\put(492,336){\rule{1pt}{1pt}}
\put(493,354){\rule{1pt}{1pt}}
\put(493,351){\rule{1pt}{1pt}}
\put(493,366){\rule{1pt}{1pt}}
\put(494,350){\rule{1pt}{1pt}}
\put(494,332){\rule{1pt}{1pt}}
\put(495,363){\rule{1pt}{1pt}}
\put(495,348){\rule{1pt}{1pt}}
\put(495,346){\rule{1pt}{1pt}}
\put(496,358){\rule{1pt}{1pt}}
\put(496,343){\rule{1pt}{1pt}}
\put(496,356){\rule{1pt}{1pt}}
\put(497,365){\rule{1pt}{1pt}}
\put(497,356){\rule{1pt}{1pt}}
\put(497,362){\rule{1pt}{1pt}}
\put(498,350){\rule{1pt}{1pt}}
\put(498,347){\rule{1pt}{1pt}}
\put(499,331){\rule{1pt}{1pt}}
\put(499,359){\rule{1pt}{1pt}}
\put(499,344){\rule{1pt}{1pt}}
\put(500,355){\rule{1pt}{1pt}}
\put(500,334){\rule{1pt}{1pt}}
\put(500,332){\rule{1pt}{1pt}}
\put(501,365){\rule{1pt}{1pt}}
\put(501,351){\rule{1pt}{1pt}}
\put(501,335){\rule{1pt}{1pt}}
\put(502,359){\rule{1pt}{1pt}}
\put(502,331){\rule{1pt}{1pt}}
\put(502,363){\rule{1pt}{1pt}}
\put(503,337){\rule{1pt}{1pt}}
\put(503,346){\rule{1pt}{1pt}}
\put(504,358){\rule{1pt}{1pt}}
\put(504,345){\rule{1pt}{1pt}}
\put(504,346){\rule{1pt}{1pt}}
\put(505,339){\rule{1pt}{1pt}}
\put(505,341){\rule{1pt}{1pt}}
\put(505,355){\rule{1pt}{1pt}}
\put(506,332){\rule{1pt}{1pt}}
\put(506,330){\rule{1pt}{1pt}}
\put(506,361){\rule{1pt}{1pt}}
\put(507,330){\rule{1pt}{1pt}}
\put(507,332){\rule{1pt}{1pt}}
\put(508,332){\rule{1pt}{1pt}}
\put(508,360){\rule{1pt}{1pt}}
\put(508,340){\rule{1pt}{1pt}}
\put(509,363){\rule{1pt}{1pt}}
\put(509,353){\rule{1pt}{1pt}}
\put(509,351){\rule{1pt}{1pt}}
\put(510,361){\rule{1pt}{1pt}}
\put(510,351){\rule{1pt}{1pt}}
\put(510,363){\rule{1pt}{1pt}}
\put(511,359){\rule{1pt}{1pt}}
\put(511,350){\rule{1pt}{1pt}}
\put(512,336){\rule{1pt}{1pt}}
\put(512,345){\rule{1pt}{1pt}}
\put(512,337){\rule{1pt}{1pt}}
\put(513,333){\rule{1pt}{1pt}}
\put(513,331){\rule{1pt}{1pt}}
\put(513,353){\rule{1pt}{1pt}}
\put(514,341){\rule{1pt}{1pt}}
\put(514,343){\rule{1pt}{1pt}}
\put(514,334){\rule{1pt}{1pt}}
\put(515,337){\rule{1pt}{1pt}}
\put(515,353){\rule{1pt}{1pt}}
\put(516,358){\rule{1pt}{1pt}}
\put(516,342){\rule{1pt}{1pt}}
\put(516,349){\rule{1pt}{1pt}}
\put(517,354){\rule{1pt}{1pt}}
\put(517,361){\rule{1pt}{1pt}}
\put(517,358){\rule{1pt}{1pt}}
\put(518,335){\rule{1pt}{1pt}}
\put(518,352){\rule{1pt}{1pt}}
\put(518,336){\rule{1pt}{1pt}}
\put(519,354){\rule{1pt}{1pt}}
\put(519,359){\rule{1pt}{1pt}}
\put(520,333){\rule{1pt}{1pt}}
\put(520,335){\rule{1pt}{1pt}}
\put(520,341){\rule{1pt}{1pt}}
\put(521,349){\rule{1pt}{1pt}}
\put(521,351){\rule{1pt}{1pt}}
\put(521,332){\rule{1pt}{1pt}}
\put(522,335){\rule{1pt}{1pt}}
\put(522,341){\rule{1pt}{1pt}}
\put(522,358){\rule{1pt}{1pt}}
\put(523,353){\rule{1pt}{1pt}}
\put(523,338){\rule{1pt}{1pt}}
\put(524,355){\rule{1pt}{1pt}}
\put(524,340){\rule{1pt}{1pt}}
\put(524,341){\rule{1pt}{1pt}}
\put(525,362){\rule{1pt}{1pt}}
\put(525,347){\rule{1pt}{1pt}}
\put(525,362){\rule{1pt}{1pt}}
\put(526,350){\rule{1pt}{1pt}}
\put(526,336){\rule{1pt}{1pt}}
\put(526,356){\rule{1pt}{1pt}}
\put(527,354){\rule{1pt}{1pt}}
\put(527,341){\rule{1pt}{1pt}}
\put(527,333){\rule{1pt}{1pt}}
\put(528,359){\rule{1pt}{1pt}}
\put(528,328){\rule{1pt}{1pt}}
\put(529,359){\rule{1pt}{1pt}}
\put(529,348){\rule{1pt}{1pt}}
\put(529,357){\rule{1pt}{1pt}}
\put(530,342){\rule{1pt}{1pt}}
\put(530,327){\rule{1pt}{1pt}}
\put(530,343){\rule{1pt}{1pt}}
\put(531,347){\rule{1pt}{1pt}}
\put(531,356){\rule{1pt}{1pt}}
\put(531,326){\rule{1pt}{1pt}}
\put(532,360){\rule{1pt}{1pt}}
\put(532,334){\rule{1pt}{1pt}}
\put(533,355){\rule{1pt}{1pt}}
\put(533,326){\rule{1pt}{1pt}}
\put(533,357){\rule{1pt}{1pt}}
\put(534,333){\rule{1pt}{1pt}}
\put(534,361){\rule{1pt}{1pt}}
\put(534,338){\rule{1pt}{1pt}}
\put(535,339){\rule{1pt}{1pt}}
\put(535,352){\rule{1pt}{1pt}}
\put(535,342){\rule{1pt}{1pt}}
\put(536,328){\rule{1pt}{1pt}}
\put(536,356){\rule{1pt}{1pt}}
\put(537,328){\rule{1pt}{1pt}}
\put(537,332){\rule{1pt}{1pt}}
\put(537,348){\rule{1pt}{1pt}}
\put(538,339){\rule{1pt}{1pt}}
\put(538,357){\rule{1pt}{1pt}}
\put(538,348){\rule{1pt}{1pt}}
\put(539,332){\rule{1pt}{1pt}}
\put(539,337){\rule{1pt}{1pt}}
\put(539,357){\rule{1pt}{1pt}}
\put(540,352){\rule{1pt}{1pt}}
\put(540,340){\rule{1pt}{1pt}}
\put(541,359){\rule{1pt}{1pt}}
\put(541,349){\rule{1pt}{1pt}}
\put(541,350){\rule{1pt}{1pt}}
\put(542,333){\rule{1pt}{1pt}}
\put(542,329){\rule{1pt}{1pt}}
\put(542,352){\rule{1pt}{1pt}}
\put(543,345){\rule{1pt}{1pt}}
\put(543,346){\rule{1pt}{1pt}}
\put(543,348){\rule{1pt}{1pt}}
\put(544,328){\rule{1pt}{1pt}}
\put(544,348){\rule{1pt}{1pt}}
\put(545,358){\rule{1pt}{1pt}}
\put(545,348){\rule{1pt}{1pt}}
\put(545,346){\rule{1pt}{1pt}}
\put(546,334){\rule{1pt}{1pt}}
\put(546,348){\rule{1pt}{1pt}}
\put(546,342){\rule{1pt}{1pt}}
\put(547,330){\rule{1pt}{1pt}}
\put(547,350){\rule{1pt}{1pt}}
\put(547,329){\rule{1pt}{1pt}}
\put(548,351){\rule{1pt}{1pt}}
\put(548,327){\rule{1pt}{1pt}}
\put(549,342){\rule{1pt}{1pt}}
\put(549,356){\rule{1pt}{1pt}}
\put(549,346){\rule{1pt}{1pt}}
\put(550,355){\rule{1pt}{1pt}}
\put(550,329){\rule{1pt}{1pt}}
\put(550,345){\rule{1pt}{1pt}}
\put(551,347){\rule{1pt}{1pt}}
\put(551,353){\rule{1pt}{1pt}}
\put(551,345){\rule{1pt}{1pt}}
\put(552,354){\rule{1pt}{1pt}}
\put(552,340){\rule{1pt}{1pt}}
\put(552,330){\rule{1pt}{1pt}}
\put(553,334){\rule{1pt}{1pt}}
\put(553,358){\rule{1pt}{1pt}}
\put(554,323){\rule{1pt}{1pt}}
\put(554,358){\rule{1pt}{1pt}}
\put(554,357){\rule{1pt}{1pt}}
\put(555,341){\rule{1pt}{1pt}}
\put(555,342){\rule{1pt}{1pt}}
\put(555,331){\rule{1pt}{1pt}}
\put(556,350){\rule{1pt}{1pt}}
\put(556,331){\rule{1pt}{1pt}}
\put(556,346){\rule{1pt}{1pt}}
\put(557,349){\rule{1pt}{1pt}}
\put(557,331){\rule{1pt}{1pt}}
\put(558,330){\rule{1pt}{1pt}}
\put(558,324){\rule{1pt}{1pt}}
\put(558,341){\rule{1pt}{1pt}}
\put(559,341){\rule{1pt}{1pt}}
\put(559,336){\rule{1pt}{1pt}}
\put(559,353){\rule{1pt}{1pt}}
\put(560,342){\rule{1pt}{1pt}}
\put(560,322){\rule{1pt}{1pt}}
\put(560,354){\rule{1pt}{1pt}}
\put(561,346){\rule{1pt}{1pt}}
\put(561,337){\rule{1pt}{1pt}}
\put(562,335){\rule{1pt}{1pt}}
\put(562,351){\rule{1pt}{1pt}}
\put(562,353){\rule{1pt}{1pt}}
\put(563,335){\rule{1pt}{1pt}}
\put(563,344){\rule{1pt}{1pt}}
\put(563,345){\rule{1pt}{1pt}}
\put(564,336){\rule{1pt}{1pt}}
\put(564,329){\rule{1pt}{1pt}}
\put(564,330){\rule{1pt}{1pt}}
\put(565,327){\rule{1pt}{1pt}}
\put(565,323){\rule{1pt}{1pt}}
\put(566,342){\rule{1pt}{1pt}}
\put(566,326){\rule{1pt}{1pt}}
\put(566,351){\rule{1pt}{1pt}}
\put(567,345){\rule{1pt}{1pt}}
\put(567,356){\rule{1pt}{1pt}}
\put(567,338){\rule{1pt}{1pt}}
\put(568,341){\rule{1pt}{1pt}}
\put(568,322){\rule{1pt}{1pt}}
\put(568,354){\rule{1pt}{1pt}}
\put(569,323){\rule{1pt}{1pt}}
\put(569,346){\rule{1pt}{1pt}}
\put(570,321){\rule{1pt}{1pt}}
\put(570,324){\rule{1pt}{1pt}}
\put(570,332){\rule{1pt}{1pt}}
\put(571,348){\rule{1pt}{1pt}}
\put(571,325){\rule{1pt}{1pt}}
\put(571,328){\rule{1pt}{1pt}}
\put(572,321){\rule{1pt}{1pt}}
\put(572,345){\rule{1pt}{1pt}}
\put(572,351){\rule{1pt}{1pt}}
\put(573,351){\rule{1pt}{1pt}}
\put(573,329){\rule{1pt}{1pt}}
\put(573,325){\rule{1pt}{1pt}}
\put(574,348){\rule{1pt}{1pt}}
\put(574,340){\rule{1pt}{1pt}}
\put(575,332){\rule{1pt}{1pt}}
\put(575,355){\rule{1pt}{1pt}}
\put(575,326){\rule{1pt}{1pt}}
\put(576,353){\rule{1pt}{1pt}}
\put(576,355){\rule{1pt}{1pt}}
\put(576,343){\rule{1pt}{1pt}}
\put(577,326){\rule{1pt}{1pt}}
\put(577,352){\rule{1pt}{1pt}}
\put(577,341){\rule{1pt}{1pt}}
\put(578,354){\rule{1pt}{1pt}}
\put(578,332){\rule{1pt}{1pt}}
\put(579,334){\rule{1pt}{1pt}}
\put(579,326){\rule{1pt}{1pt}}
\put(579,326){\rule{1pt}{1pt}}
\put(580,350){\rule{1pt}{1pt}}
\put(580,341){\rule{1pt}{1pt}}
\put(580,350){\rule{1pt}{1pt}}
\put(581,335){\rule{1pt}{1pt}}
\put(581,337){\rule{1pt}{1pt}}
\put(581,345){\rule{1pt}{1pt}}
\put(582,347){\rule{1pt}{1pt}}
\put(582,346){\rule{1pt}{1pt}}
\put(583,325){\rule{1pt}{1pt}}
\put(583,324){\rule{1pt}{1pt}}
\put(583,337){\rule{1pt}{1pt}}
\put(584,350){\rule{1pt}{1pt}}
\put(584,354){\rule{1pt}{1pt}}
\put(584,353){\rule{1pt}{1pt}}
\put(585,330){\rule{1pt}{1pt}}
\put(585,334){\rule{1pt}{1pt}}
\put(585,330){\rule{1pt}{1pt}}
\put(586,352){\rule{1pt}{1pt}}
\put(586,329){\rule{1pt}{1pt}}
\put(587,353){\rule{1pt}{1pt}}
\put(587,327){\rule{1pt}{1pt}}
\put(587,345){\rule{1pt}{1pt}}
\put(588,340){\rule{1pt}{1pt}}
\put(588,342){\rule{1pt}{1pt}}
\put(588,353){\rule{1pt}{1pt}}
\put(589,336){\rule{1pt}{1pt}}
\put(589,324){\rule{1pt}{1pt}}
\put(589,322){\rule{1pt}{1pt}}
\put(590,334){\rule{1pt}{1pt}}
\put(590,325){\rule{1pt}{1pt}}
\put(591,348){\rule{1pt}{1pt}}
\put(591,343){\rule{1pt}{1pt}}
\put(591,352){\rule{1pt}{1pt}}
\put(592,340){\rule{1pt}{1pt}}
\put(592,324){\rule{1pt}{1pt}}
\put(592,328){\rule{1pt}{1pt}}
\put(593,322){\rule{1pt}{1pt}}
\put(593,341){\rule{1pt}{1pt}}
\put(593,340){\rule{1pt}{1pt}}
\put(594,333){\rule{1pt}{1pt}}
\put(594,318){\rule{1pt}{1pt}}
\put(595,343){\rule{1pt}{1pt}}
\put(595,336){\rule{1pt}{1pt}}
\put(595,320){\rule{1pt}{1pt}}
\put(596,334){\rule{1pt}{1pt}}
\put(596,344){\rule{1pt}{1pt}}
\put(596,323){\rule{1pt}{1pt}}
\put(597,324){\rule{1pt}{1pt}}
\put(597,341){\rule{1pt}{1pt}}
\put(597,337){\rule{1pt}{1pt}}
\put(598,352){\rule{1pt}{1pt}}
\put(598,342){\rule{1pt}{1pt}}
\put(598,344){\rule{1pt}{1pt}}
\put(599,349){\rule{1pt}{1pt}}
\put(599,349){\rule{1pt}{1pt}}
\put(600,344){\rule{1pt}{1pt}}
\put(600,337){\rule{1pt}{1pt}}
\put(600,317){\rule{1pt}{1pt}}
\put(601,336){\rule{1pt}{1pt}}
\put(601,341){\rule{1pt}{1pt}}
\put(601,350){\rule{1pt}{1pt}}
\put(602,323){\rule{1pt}{1pt}}
\put(602,328){\rule{1pt}{1pt}}
\put(602,326){\rule{1pt}{1pt}}
\put(603,326){\rule{1pt}{1pt}}
\put(603,320){\rule{1pt}{1pt}}
\put(604,338){\rule{1pt}{1pt}}
\put(604,340){\rule{1pt}{1pt}}
\put(604,317){\rule{1pt}{1pt}}
\put(605,336){\rule{1pt}{1pt}}
\put(605,332){\rule{1pt}{1pt}}
\put(605,320){\rule{1pt}{1pt}}
\put(606,320){\rule{1pt}{1pt}}
\put(606,330){\rule{1pt}{1pt}}
\put(606,348){\rule{1pt}{1pt}}
\put(607,318){\rule{1pt}{1pt}}
\put(607,343){\rule{1pt}{1pt}}
\put(608,330){\rule{1pt}{1pt}}
\put(608,327){\rule{1pt}{1pt}}
\put(608,327){\rule{1pt}{1pt}}
\put(609,344){\rule{1pt}{1pt}}
\put(609,320){\rule{1pt}{1pt}}
\put(609,346){\rule{1pt}{1pt}}
\put(610,339){\rule{1pt}{1pt}}
\put(610,325){\rule{1pt}{1pt}}
\put(610,330){\rule{1pt}{1pt}}
\put(611,328){\rule{1pt}{1pt}}
\put(611,329){\rule{1pt}{1pt}}
\put(612,340){\rule{1pt}{1pt}}
\put(612,338){\rule{1pt}{1pt}}
\put(612,344){\rule{1pt}{1pt}}
\put(613,325){\rule{1pt}{1pt}}
\put(613,348){\rule{1pt}{1pt}}
\put(613,345){\rule{1pt}{1pt}}
\put(614,344){\rule{1pt}{1pt}}
\put(614,337){\rule{1pt}{1pt}}
\put(614,317){\rule{1pt}{1pt}}
\put(615,329){\rule{1pt}{1pt}}
\put(615,327){\rule{1pt}{1pt}}
\put(616,339){\rule{1pt}{1pt}}
\put(616,329){\rule{1pt}{1pt}}
\put(616,335){\rule{1pt}{1pt}}
\put(617,316){\rule{1pt}{1pt}}
\put(617,334){\rule{1pt}{1pt}}
\put(617,331){\rule{1pt}{1pt}}
\put(618,322){\rule{1pt}{1pt}}
\put(618,346){\rule{1pt}{1pt}}
\put(618,319){\rule{1pt}{1pt}}
\put(619,334){\rule{1pt}{1pt}}
\put(619,339){\rule{1pt}{1pt}}
\put(620,338){\rule{1pt}{1pt}}
\put(620,348){\rule{1pt}{1pt}}
\put(620,318){\rule{1pt}{1pt}}
\put(621,336){\rule{1pt}{1pt}}
\put(621,336){\rule{1pt}{1pt}}
\put(621,330){\rule{1pt}{1pt}}
\put(622,324){\rule{1pt}{1pt}}
\put(622,345){\rule{1pt}{1pt}}
\put(622,319){\rule{1pt}{1pt}}
\put(623,345){\rule{1pt}{1pt}}
\put(623,326){\rule{1pt}{1pt}}
\put(623,314){\rule{1pt}{1pt}}
\put(624,323){\rule{1pt}{1pt}}
\put(624,330){\rule{1pt}{1pt}}
\put(625,324){\rule{1pt}{1pt}}
\put(625,332){\rule{1pt}{1pt}}
\put(625,316){\rule{1pt}{1pt}}
\put(626,334){\rule{1pt}{1pt}}
\put(626,347){\rule{1pt}{1pt}}
\put(626,328){\rule{1pt}{1pt}}
\put(627,323){\rule{1pt}{1pt}}
\put(627,337){\rule{1pt}{1pt}}
\put(627,347){\rule{1pt}{1pt}}
\put(628,336){\rule{1pt}{1pt}}
\put(628,318){\rule{1pt}{1pt}}
\put(629,337){\rule{1pt}{1pt}}
\put(629,322){\rule{1pt}{1pt}}
\put(629,347){\rule{1pt}{1pt}}
\put(630,320){\rule{1pt}{1pt}}
\put(630,333){\rule{1pt}{1pt}}
\put(630,329){\rule{1pt}{1pt}}
\put(631,334){\rule{1pt}{1pt}}
\put(631,320){\rule{1pt}{1pt}}
\put(631,323){\rule{1pt}{1pt}}
\put(632,334){\rule{1pt}{1pt}}
\put(632,338){\rule{1pt}{1pt}}
\put(633,341){\rule{1pt}{1pt}}
\put(633,333){\rule{1pt}{1pt}}
\put(633,314){\rule{1pt}{1pt}}
\put(634,335){\rule{1pt}{1pt}}
\put(634,342){\rule{1pt}{1pt}}
\put(634,326){\rule{1pt}{1pt}}
\put(635,333){\rule{1pt}{1pt}}
\put(635,335){\rule{1pt}{1pt}}
\put(635,326){\rule{1pt}{1pt}}
\put(636,342){\rule{1pt}{1pt}}
\put(636,324){\rule{1pt}{1pt}}
\put(637,339){\rule{1pt}{1pt}}
\put(637,345){\rule{1pt}{1pt}}
\put(637,347){\rule{1pt}{1pt}}
\put(638,333){\rule{1pt}{1pt}}
\put(638,346){\rule{1pt}{1pt}}
\put(638,330){\rule{1pt}{1pt}}
\put(639,319){\rule{1pt}{1pt}}
\put(639,339){\rule{1pt}{1pt}}
\put(639,321){\rule{1pt}{1pt}}
\put(640,331){\rule{1pt}{1pt}}
\put(640,324){\rule{1pt}{1pt}}
\put(641,317){\rule{1pt}{1pt}}
\put(641,336){\rule{1pt}{1pt}}
\put(641,324){\rule{1pt}{1pt}}
\put(642,315){\rule{1pt}{1pt}}
\put(642,338){\rule{1pt}{1pt}}
\put(642,334){\rule{1pt}{1pt}}
\put(643,328){\rule{1pt}{1pt}}
\put(643,319){\rule{1pt}{1pt}}
\put(643,343){\rule{1pt}{1pt}}
\put(644,332){\rule{1pt}{1pt}}
\put(644,343){\rule{1pt}{1pt}}
\put(645,339){\rule{1pt}{1pt}}
\put(645,343){\rule{1pt}{1pt}}
\put(645,328){\rule{1pt}{1pt}}
\put(646,317){\rule{1pt}{1pt}}
\put(646,340){\rule{1pt}{1pt}}
\put(646,344){\rule{1pt}{1pt}}
\put(647,315){\rule{1pt}{1pt}}
\put(647,344){\rule{1pt}{1pt}}
\put(647,313){\rule{1pt}{1pt}}
\put(648,342){\rule{1pt}{1pt}}
\put(648,322){\rule{1pt}{1pt}}
\put(648,313){\rule{1pt}{1pt}}
\put(649,345){\rule{1pt}{1pt}}
\put(649,345){\rule{1pt}{1pt}}
\put(650,312){\rule{1pt}{1pt}}
\put(650,315){\rule{1pt}{1pt}}
\put(650,318){\rule{1pt}{1pt}}
\put(651,344){\rule{1pt}{1pt}}
\put(651,328){\rule{1pt}{1pt}}
\put(651,318){\rule{1pt}{1pt}}
\put(652,325){\rule{1pt}{1pt}}
\put(652,317){\rule{1pt}{1pt}}
\put(652,322){\rule{1pt}{1pt}}
\put(653,333){\rule{1pt}{1pt}}
\put(653,324){\rule{1pt}{1pt}}
\put(654,342){\rule{1pt}{1pt}}
\put(654,337){\rule{1pt}{1pt}}
\put(654,338){\rule{1pt}{1pt}}
\put(655,324){\rule{1pt}{1pt}}
\put(655,321){\rule{1pt}{1pt}}
\put(655,338){\rule{1pt}{1pt}}
\put(656,336){\rule{1pt}{1pt}}
\put(656,311){\rule{1pt}{1pt}}
\put(656,331){\rule{1pt}{1pt}}
\put(657,324){\rule{1pt}{1pt}}
\put(657,329){\rule{1pt}{1pt}}
\put(658,325){\rule{1pt}{1pt}}
\put(658,339){\rule{1pt}{1pt}}
\put(658,320){\rule{1pt}{1pt}}
\put(659,336){\rule{1pt}{1pt}}
\put(659,329){\rule{1pt}{1pt}}
\put(659,334){\rule{1pt}{1pt}}
\put(660,332){\rule{1pt}{1pt}}
\put(660,343){\rule{1pt}{1pt}}
\put(660,343){\rule{1pt}{1pt}}
\put(661,336){\rule{1pt}{1pt}}
\put(661,326){\rule{1pt}{1pt}}
\put(662,325){\rule{1pt}{1pt}}
\put(662,314){\rule{1pt}{1pt}}
\put(662,324){\rule{1pt}{1pt}}
\put(663,342){\rule{1pt}{1pt}}
\put(663,333){\rule{1pt}{1pt}}
\put(663,312){\rule{1pt}{1pt}}
\put(664,322){\rule{1pt}{1pt}}
\put(664,333){\rule{1pt}{1pt}}
\put(664,320){\rule{1pt}{1pt}}
\put(665,340){\rule{1pt}{1pt}}
\put(665,320){\rule{1pt}{1pt}}
\put(666,339){\rule{1pt}{1pt}}
\put(666,329){\rule{1pt}{1pt}}
\put(666,317){\rule{1pt}{1pt}}
\put(667,342){\rule{1pt}{1pt}}
\put(667,321){\rule{1pt}{1pt}}
\put(667,332){\rule{1pt}{1pt}}
\put(668,334){\rule{1pt}{1pt}}
\put(668,320){\rule{1pt}{1pt}}
\put(668,335){\rule{1pt}{1pt}}
\put(669,338){\rule{1pt}{1pt}}
\put(669,329){\rule{1pt}{1pt}}
\put(670,325){\rule{1pt}{1pt}}
\put(670,322){\rule{1pt}{1pt}}
\put(670,335){\rule{1pt}{1pt}}
\put(671,339){\rule{1pt}{1pt}}
\put(671,311){\rule{1pt}{1pt}}
\put(671,326){\rule{1pt}{1pt}}
\put(672,341){\rule{1pt}{1pt}}
\put(672,309){\rule{1pt}{1pt}}
\put(672,335){\rule{1pt}{1pt}}
\put(673,327){\rule{1pt}{1pt}}
\put(673,311){\rule{1pt}{1pt}}
\put(673,314){\rule{1pt}{1pt}}
\put(674,339){\rule{1pt}{1pt}}
\put(674,333){\rule{1pt}{1pt}}
\put(675,313){\rule{1pt}{1pt}}
\put(675,312){\rule{1pt}{1pt}}
\put(675,324){\rule{1pt}{1pt}}
\put(676,313){\rule{1pt}{1pt}}
\put(676,329){\rule{1pt}{1pt}}
\put(676,309){\rule{1pt}{1pt}}
\put(677,314){\rule{1pt}{1pt}}
\put(677,331){\rule{1pt}{1pt}}
\put(677,314){\rule{1pt}{1pt}}
\put(678,314){\rule{1pt}{1pt}}
\put(678,333){\rule{1pt}{1pt}}
\put(679,334){\rule{1pt}{1pt}}
\put(679,330){\rule{1pt}{1pt}}
\put(679,341){\rule{1pt}{1pt}}
\put(680,321){\rule{1pt}{1pt}}
\put(680,318){\rule{1pt}{1pt}}
\put(680,328){\rule{1pt}{1pt}}
\put(681,325){\rule{1pt}{1pt}}
\put(681,340){\rule{1pt}{1pt}}
\put(681,327){\rule{1pt}{1pt}}
\put(682,311){\rule{1pt}{1pt}}
\put(682,335){\rule{1pt}{1pt}}
\put(683,327){\rule{1pt}{1pt}}
\put(683,336){\rule{1pt}{1pt}}
\put(683,312){\rule{1pt}{1pt}}
\put(684,329){\rule{1pt}{1pt}}
\put(684,317){\rule{1pt}{1pt}}
\put(684,318){\rule{1pt}{1pt}}
\put(685,338){\rule{1pt}{1pt}}
\put(685,328){\rule{1pt}{1pt}}
\put(685,323){\rule{1pt}{1pt}}
\put(686,315){\rule{1pt}{1pt}}
\put(686,318){\rule{1pt}{1pt}}
\put(687,327){\rule{1pt}{1pt}}
\put(687,324){\rule{1pt}{1pt}}
\put(687,328){\rule{1pt}{1pt}}
\put(688,324){\rule{1pt}{1pt}}
\put(688,320){\rule{1pt}{1pt}}
\put(688,334){\rule{1pt}{1pt}}
\put(689,332){\rule{1pt}{1pt}}
\put(689,318){\rule{1pt}{1pt}}
\put(689,334){\rule{1pt}{1pt}}
\put(690,312){\rule{1pt}{1pt}}
\put(690,323){\rule{1pt}{1pt}}
\put(691,305){\rule{1pt}{1pt}}
\put(691,308){\rule{1pt}{1pt}}
\put(691,331){\rule{1pt}{1pt}}
\put(692,315){\rule{1pt}{1pt}}
\put(692,305){\rule{1pt}{1pt}}
\put(692,319){\rule{1pt}{1pt}}
\put(693,324){\rule{1pt}{1pt}}
\put(693,326){\rule{1pt}{1pt}}
\put(693,321){\rule{1pt}{1pt}}
\put(694,335){\rule{1pt}{1pt}}
\put(694,319){\rule{1pt}{1pt}}
\put(695,317){\rule{1pt}{1pt}}
\put(695,334){\rule{1pt}{1pt}}
\put(695,339){\rule{1pt}{1pt}}
\put(696,330){\rule{1pt}{1pt}}
\put(696,320){\rule{1pt}{1pt}}
\put(696,309){\rule{1pt}{1pt}}
\put(697,330){\rule{1pt}{1pt}}
\put(697,313){\rule{1pt}{1pt}}
\put(697,330){\rule{1pt}{1pt}}
\put(698,332){\rule{1pt}{1pt}}
\put(698,337){\rule{1pt}{1pt}}
\put(698,330){\rule{1pt}{1pt}}
\put(699,326){\rule{1pt}{1pt}}
\put(699,338){\rule{1pt}{1pt}}
\put(700,334){\rule{1pt}{1pt}}
\put(700,337){\rule{1pt}{1pt}}
\put(700,305){\rule{1pt}{1pt}}
\put(701,322){\rule{1pt}{1pt}}
\put(701,313){\rule{1pt}{1pt}}
\put(701,321){\rule{1pt}{1pt}}
\put(702,328){\rule{1pt}{1pt}}
\put(702,319){\rule{1pt}{1pt}}
\put(702,326){\rule{1pt}{1pt}}
\put(703,305){\rule{1pt}{1pt}}
\put(703,326){\rule{1pt}{1pt}}
\put(704,308){\rule{1pt}{1pt}}
\put(704,323){\rule{1pt}{1pt}}
\put(704,335){\rule{1pt}{1pt}}
\put(705,314){\rule{1pt}{1pt}}
\put(705,313){\rule{1pt}{1pt}}
\put(705,311){\rule{1pt}{1pt}}
\put(706,327){\rule{1pt}{1pt}}
\put(706,332){\rule{1pt}{1pt}}
\put(706,332){\rule{1pt}{1pt}}
\put(707,305){\rule{1pt}{1pt}}
\put(707,329){\rule{1pt}{1pt}}
\put(708,326){\rule{1pt}{1pt}}
\put(708,332){\rule{1pt}{1pt}}
\put(708,318){\rule{1pt}{1pt}}
\put(709,322){\rule{1pt}{1pt}}
\put(709,309){\rule{1pt}{1pt}}
\put(709,318){\rule{1pt}{1pt}}
\put(710,337){\rule{1pt}{1pt}}
\put(710,330){\rule{1pt}{1pt}}
\put(710,330){\rule{1pt}{1pt}}
\put(711,332){\rule{1pt}{1pt}}
\put(711,330){\rule{1pt}{1pt}}
\put(712,317){\rule{1pt}{1pt}}
\put(712,314){\rule{1pt}{1pt}}
\put(712,321){\rule{1pt}{1pt}}
\put(713,335){\rule{1pt}{1pt}}
\put(713,306){\rule{1pt}{1pt}}
\put(713,316){\rule{1pt}{1pt}}
\put(714,318){\rule{1pt}{1pt}}
\put(714,329){\rule{1pt}{1pt}}
\put(714,333){\rule{1pt}{1pt}}
\put(715,308){\rule{1pt}{1pt}}
\put(715,328){\rule{1pt}{1pt}}
\put(716,320){\rule{1pt}{1pt}}
\put(716,335){\rule{1pt}{1pt}}
\put(716,318){\rule{1pt}{1pt}}
\put(717,319){\rule{1pt}{1pt}}
\put(717,328){\rule{1pt}{1pt}}
\put(717,333){\rule{1pt}{1pt}}
\put(718,318){\rule{1pt}{1pt}}
\put(718,324){\rule{1pt}{1pt}}
\put(718,321){\rule{1pt}{1pt}}
\put(719,323){\rule{1pt}{1pt}}
\put(719,320){\rule{1pt}{1pt}}
\put(720,307){\rule{1pt}{1pt}}
\put(720,322){\rule{1pt}{1pt}}
\put(720,312){\rule{1pt}{1pt}}
\put(721,318){\rule{1pt}{1pt}}
\put(721,326){\rule{1pt}{1pt}}
\put(721,319){\rule{1pt}{1pt}}
\put(722,305){\rule{1pt}{1pt}}
\put(722,307){\rule{1pt}{1pt}}
\put(722,309){\rule{1pt}{1pt}}
\put(723,333){\rule{1pt}{1pt}}
\put(723,310){\rule{1pt}{1pt}}
\put(723,329){\rule{1pt}{1pt}}
\put(724,313){\rule{1pt}{1pt}}
\put(724,315){\rule{1pt}{1pt}}
\put(725,301){\rule{1pt}{1pt}}
\put(725,307){\rule{1pt}{1pt}}
\put(725,301){\rule{1pt}{1pt}}
\put(726,318){\rule{1pt}{1pt}}
\put(726,320){\rule{1pt}{1pt}}
\put(726,305){\rule{1pt}{1pt}}
\put(727,330){\rule{1pt}{1pt}}
\put(727,306){\rule{1pt}{1pt}}
\put(727,319){\rule{1pt}{1pt}}
\put(728,328){\rule{1pt}{1pt}}
\put(728,328){\rule{1pt}{1pt}}
\put(729,307){\rule{1pt}{1pt}}
\put(729,323){\rule{1pt}{1pt}}
\put(729,300){\rule{1pt}{1pt}}
\put(730,307){\rule{1pt}{1pt}}
\put(730,310){\rule{1pt}{1pt}}
\put(730,314){\rule{1pt}{1pt}}
\put(731,318){\rule{1pt}{1pt}}
\put(731,302){\rule{1pt}{1pt}}
\put(731,300){\rule{1pt}{1pt}}
\put(732,316){\rule{1pt}{1pt}}
\put(732,319){\rule{1pt}{1pt}}
\put(733,331){\rule{1pt}{1pt}}
\put(733,310){\rule{1pt}{1pt}}
\put(733,316){\rule{1pt}{1pt}}
\put(734,311){\rule{1pt}{1pt}}
\put(734,334){\rule{1pt}{1pt}}
\put(734,328){\rule{1pt}{1pt}}
\put(735,327){\rule{1pt}{1pt}}
\put(735,311){\rule{1pt}{1pt}}
\put(735,301){\rule{1pt}{1pt}}
\put(736,329){\rule{1pt}{1pt}}
\put(736,324){\rule{1pt}{1pt}}
\put(737,308){\rule{1pt}{1pt}}
\put(737,309){\rule{1pt}{1pt}}
\put(737,326){\rule{1pt}{1pt}}
\put(738,323){\rule{1pt}{1pt}}
\put(738,304){\rule{1pt}{1pt}}
\put(738,319){\rule{1pt}{1pt}}
\put(739,332){\rule{1pt}{1pt}}
\put(739,321){\rule{1pt}{1pt}}
\put(739,310){\rule{1pt}{1pt}}
\put(740,303){\rule{1pt}{1pt}}
\put(740,328){\rule{1pt}{1pt}}
\put(741,301){\rule{1pt}{1pt}}
\put(741,324){\rule{1pt}{1pt}}
\put(741,304){\rule{1pt}{1pt}}
\put(742,329){\rule{1pt}{1pt}}
\put(742,319){\rule{1pt}{1pt}}
\put(742,300){\rule{1pt}{1pt}}
\put(743,317){\rule{1pt}{1pt}}
\put(743,305){\rule{1pt}{1pt}}
\put(743,322){\rule{1pt}{1pt}}
\put(744,324){\rule{1pt}{1pt}}
\put(744,307){\rule{1pt}{1pt}}
\put(744,327){\rule{1pt}{1pt}}
\put(745,299){\rule{1pt}{1pt}}
\put(745,308){\rule{1pt}{1pt}}
\put(746,322){\rule{1pt}{1pt}}
\put(746,332){\rule{1pt}{1pt}}
\put(746,322){\rule{1pt}{1pt}}
\put(747,300){\rule{1pt}{1pt}}
\put(747,306){\rule{1pt}{1pt}}
\put(747,325){\rule{1pt}{1pt}}
\put(748,312){\rule{1pt}{1pt}}
\put(748,301){\rule{1pt}{1pt}}
\put(748,330){\rule{1pt}{1pt}}
\put(749,304){\rule{1pt}{1pt}}
\put(749,303){\rule{1pt}{1pt}}
\put(750,300){\rule{1pt}{1pt}}
\put(750,297){\rule{1pt}{1pt}}
\put(750,302){\rule{1pt}{1pt}}
\put(751,299){\rule{1pt}{1pt}}
\put(751,299){\rule{1pt}{1pt}}
\put(751,300){\rule{1pt}{1pt}}
\put(752,317){\rule{1pt}{1pt}}
\put(752,310){\rule{1pt}{1pt}}
\put(752,304){\rule{1pt}{1pt}}
\put(753,316){\rule{1pt}{1pt}}
\put(753,320){\rule{1pt}{1pt}}
\put(754,304){\rule{1pt}{1pt}}
\put(754,307){\rule{1pt}{1pt}}
\put(754,317){\rule{1pt}{1pt}}
\put(755,298){\rule{1pt}{1pt}}
\put(755,322){\rule{1pt}{1pt}}
\put(755,318){\rule{1pt}{1pt}}
\put(756,315){\rule{1pt}{1pt}}
\put(756,321){\rule{1pt}{1pt}}
\put(756,310){\rule{1pt}{1pt}}
\put(757,322){\rule{1pt}{1pt}}
\put(757,298){\rule{1pt}{1pt}}
\put(758,305){\rule{1pt}{1pt}}
\put(758,310){\rule{1pt}{1pt}}
\put(758,317){\rule{1pt}{1pt}}
\put(759,313){\rule{1pt}{1pt}}
\put(759,314){\rule{1pt}{1pt}}
\put(759,299){\rule{1pt}{1pt}}
\put(760,319){\rule{1pt}{1pt}}
\put(760,298){\rule{1pt}{1pt}}
\put(760,300){\rule{1pt}{1pt}}
\put(761,315){\rule{1pt}{1pt}}
\put(761,328){\rule{1pt}{1pt}}
\put(762,327){\rule{1pt}{1pt}}
\put(762,300){\rule{1pt}{1pt}}
\put(762,297){\rule{1pt}{1pt}}
\put(763,311){\rule{1pt}{1pt}}
\put(763,304){\rule{1pt}{1pt}}
\put(763,309){\rule{1pt}{1pt}}
\put(764,321){\rule{1pt}{1pt}}
\put(764,325){\rule{1pt}{1pt}}
\put(764,302){\rule{1pt}{1pt}}
\put(765,306){\rule{1pt}{1pt}}
\put(765,329){\rule{1pt}{1pt}}
\put(766,312){\rule{1pt}{1pt}}
\put(766,307){\rule{1pt}{1pt}}
\put(766,308){\rule{1pt}{1pt}}
\put(767,309){\rule{1pt}{1pt}}
\put(767,300){\rule{1pt}{1pt}}
\put(767,313){\rule{1pt}{1pt}}
\put(768,313){\rule{1pt}{1pt}}
\put(768,294){\rule{1pt}{1pt}}
\put(768,296){\rule{1pt}{1pt}}
\put(769,313){\rule{1pt}{1pt}}
\put(769,296){\rule{1pt}{1pt}}
\put(769,325){\rule{1pt}{1pt}}
\put(770,313){\rule{1pt}{1pt}}
\put(770,323){\rule{1pt}{1pt}}
\put(771,304){\rule{1pt}{1pt}}
\put(771,303){\rule{1pt}{1pt}}
\put(771,319){\rule{1pt}{1pt}}
\put(772,304){\rule{1pt}{1pt}}
\put(772,324){\rule{1pt}{1pt}}
\put(772,295){\rule{1pt}{1pt}}
\put(773,309){\rule{1pt}{1pt}}
\put(773,323){\rule{1pt}{1pt}}
\put(773,309){\rule{1pt}{1pt}}
\put(774,325){\rule{1pt}{1pt}}
\put(774,328){\rule{1pt}{1pt}}
\put(775,319){\rule{1pt}{1pt}}
\put(775,297){\rule{1pt}{1pt}}
\put(775,319){\rule{1pt}{1pt}}
\put(776,307){\rule{1pt}{1pt}}
\put(776,315){\rule{1pt}{1pt}}
\put(776,313){\rule{1pt}{1pt}}
\put(777,297){\rule{1pt}{1pt}}
\put(777,310){\rule{1pt}{1pt}}
\put(777,323){\rule{1pt}{1pt}}
\put(778,319){\rule{1pt}{1pt}}
\put(778,323){\rule{1pt}{1pt}}
\put(779,318){\rule{1pt}{1pt}}
\put(779,318){\rule{1pt}{1pt}}
\put(779,322){\rule{1pt}{1pt}}
\put(780,313){\rule{1pt}{1pt}}
\put(780,322){\rule{1pt}{1pt}}
\put(780,320){\rule{1pt}{1pt}}
\put(781,322){\rule{1pt}{1pt}}
\put(781,318){\rule{1pt}{1pt}}
\put(781,305){\rule{1pt}{1pt}}
\put(782,296){\rule{1pt}{1pt}}
\put(782,307){\rule{1pt}{1pt}}
\put(783,317){\rule{1pt}{1pt}}
\put(783,319){\rule{1pt}{1pt}}
\put(783,312){\rule{1pt}{1pt}}
\put(784,316){\rule{1pt}{1pt}}
\put(784,295){\rule{1pt}{1pt}}
\put(784,319){\rule{1pt}{1pt}}
\put(785,317){\rule{1pt}{1pt}}
\put(785,304){\rule{1pt}{1pt}}
\put(785,314){\rule{1pt}{1pt}}
\put(786,293){\rule{1pt}{1pt}}
\put(786,325){\rule{1pt}{1pt}}
\put(787,321){\rule{1pt}{1pt}}
\put(787,317){\rule{1pt}{1pt}}
\put(787,320){\rule{1pt}{1pt}}
\put(788,292){\rule{1pt}{1pt}}
\put(788,324){\rule{1pt}{1pt}}
\put(788,303){\rule{1pt}{1pt}}
\put(789,314){\rule{1pt}{1pt}}
\put(789,298){\rule{1pt}{1pt}}
\put(789,297){\rule{1pt}{1pt}}
\put(790,309){\rule{1pt}{1pt}}
\put(790,296){\rule{1pt}{1pt}}
\put(791,299){\rule{1pt}{1pt}}
\put(791,309){\rule{1pt}{1pt}}
\put(791,297){\rule{1pt}{1pt}}
\put(792,320){\rule{1pt}{1pt}}
\put(792,313){\rule{1pt}{1pt}}
\put(792,315){\rule{1pt}{1pt}}
\put(793,294){\rule{1pt}{1pt}}
\put(793,313){\rule{1pt}{1pt}}
\put(793,316){\rule{1pt}{1pt}}
\put(794,314){\rule{1pt}{1pt}}
\put(794,297){\rule{1pt}{1pt}}
\put(794,314){\rule{1pt}{1pt}}
\put(795,326){\rule{1pt}{1pt}}
\put(795,296){\rule{1pt}{1pt}}
\put(796,303){\rule{1pt}{1pt}}
\put(796,309){\rule{1pt}{1pt}}
\put(796,301){\rule{1pt}{1pt}}
\put(797,309){\rule{1pt}{1pt}}
\put(797,302){\rule{1pt}{1pt}}
\put(797,291){\rule{1pt}{1pt}}
\put(798,320){\rule{1pt}{1pt}}
\put(798,319){\rule{1pt}{1pt}}
\put(798,298){\rule{1pt}{1pt}}
\put(799,313){\rule{1pt}{1pt}}
\put(799,303){\rule{1pt}{1pt}}
\put(800,294){\rule{1pt}{1pt}}
\put(800,316){\rule{1pt}{1pt}}
\put(800,293){\rule{1pt}{1pt}}
\put(801,306){\rule{1pt}{1pt}}
\put(801,294){\rule{1pt}{1pt}}
\put(801,313){\rule{1pt}{1pt}}
\put(802,305){\rule{1pt}{1pt}}
\put(802,302){\rule{1pt}{1pt}}
\put(802,296){\rule{1pt}{1pt}}
\put(803,294){\rule{1pt}{1pt}}
\put(803,315){\rule{1pt}{1pt}}
\put(804,302){\rule{1pt}{1pt}}
\put(804,315){\rule{1pt}{1pt}}
\put(804,318){\rule{1pt}{1pt}}
\put(805,301){\rule{1pt}{1pt}}
\put(805,296){\rule{1pt}{1pt}}
\put(805,291){\rule{1pt}{1pt}}
\put(806,313){\rule{1pt}{1pt}}
\put(806,295){\rule{1pt}{1pt}}
\put(806,295){\rule{1pt}{1pt}}
\put(807,315){\rule{1pt}{1pt}}
\put(807,316){\rule{1pt}{1pt}}
\put(808,314){\rule{1pt}{1pt}}
\put(808,304){\rule{1pt}{1pt}}
\put(808,301){\rule{1pt}{1pt}}
\put(809,296){\rule{1pt}{1pt}}
\put(809,294){\rule{1pt}{1pt}}
\put(809,321){\rule{1pt}{1pt}}
\put(810,309){\rule{1pt}{1pt}}
\put(810,298){\rule{1pt}{1pt}}
\put(810,296){\rule{1pt}{1pt}}
\put(811,317){\rule{1pt}{1pt}}
\put(811,316){\rule{1pt}{1pt}}
\put(812,289){\rule{1pt}{1pt}}
\put(812,320){\rule{1pt}{1pt}}
\put(812,309){\rule{1pt}{1pt}}
\put(813,289){\rule{1pt}{1pt}}
\put(813,320){\rule{1pt}{1pt}}
\put(813,307){\rule{1pt}{1pt}}
\put(814,298){\rule{1pt}{1pt}}
\put(814,314){\rule{1pt}{1pt}}
\put(814,312){\rule{1pt}{1pt}}
\put(815,320){\rule{1pt}{1pt}}
\put(815,315){\rule{1pt}{1pt}}
\put(816,323){\rule{1pt}{1pt}}
\put(816,317){\rule{1pt}{1pt}}
\put(816,295){\rule{1pt}{1pt}}
\put(817,291){\rule{1pt}{1pt}}
\put(817,305){\rule{1pt}{1pt}}
\put(817,314){\rule{1pt}{1pt}}
\put(818,320){\rule{1pt}{1pt}}
\put(818,289){\rule{1pt}{1pt}}
\put(818,314){\rule{1pt}{1pt}}
\put(819,321){\rule{1pt}{1pt}}
\put(819,292){\rule{1pt}{1pt}}
\put(819,289){\rule{1pt}{1pt}}
\put(820,293){\rule{1pt}{1pt}}
\put(820,292){\rule{1pt}{1pt}}
\put(821,293){\rule{1pt}{1pt}}
\put(821,300){\rule{1pt}{1pt}}
\put(821,297){\rule{1pt}{1pt}}
\put(822,294){\rule{1pt}{1pt}}
\put(822,298){\rule{1pt}{1pt}}
\put(822,287){\rule{1pt}{1pt}}
\put(823,291){\rule{1pt}{1pt}}
\put(823,299){\rule{1pt}{1pt}}
\put(823,290){\rule{1pt}{1pt}}
\put(824,300){\rule{1pt}{1pt}}
\put(824,314){\rule{1pt}{1pt}}
\put(825,291){\rule{1pt}{1pt}}
\put(825,314){\rule{1pt}{1pt}}
\put(825,313){\rule{1pt}{1pt}}
\put(826,317){\rule{1pt}{1pt}}
\put(826,308){\rule{1pt}{1pt}}
\put(826,297){\rule{1pt}{1pt}}
\put(827,297){\rule{1pt}{1pt}}
\put(827,306){\rule{1pt}{1pt}}
\put(827,319){\rule{1pt}{1pt}}
\put(828,319){\rule{1pt}{1pt}}
\put(828,305){\rule{1pt}{1pt}}
\put(829,290){\rule{1pt}{1pt}}
\put(829,291){\rule{1pt}{1pt}}
\put(829,294){\rule{1pt}{1pt}}
\put(830,294){\rule{1pt}{1pt}}
\put(830,308){\rule{1pt}{1pt}}
\put(830,293){\rule{1pt}{1pt}}
\put(831,301){\rule{1pt}{1pt}}
\put(831,288){\rule{1pt}{1pt}}
\put(831,289){\rule{1pt}{1pt}}
\put(832,309){\rule{1pt}{1pt}}
\put(832,287){\rule{1pt}{1pt}}
\put(833,300){\rule{1pt}{1pt}}
\put(833,290){\rule{1pt}{1pt}}
\put(833,318){\rule{1pt}{1pt}}
\put(834,296){\rule{1pt}{1pt}}
\put(834,309){\rule{1pt}{1pt}}
\put(834,311){\rule{1pt}{1pt}}
\put(835,308){\rule{1pt}{1pt}}
\put(835,311){\rule{1pt}{1pt}}
\put(835,303){\rule{1pt}{1pt}}
\put(836,300){\rule{1pt}{1pt}}
\put(836,302){\rule{1pt}{1pt}}
\put(837,293){\rule{1pt}{1pt}}
\put(837,315){\rule{1pt}{1pt}}
\put(837,299){\rule{1pt}{1pt}}
\put(838,317){\rule{1pt}{1pt}}
\put(838,308){\rule{1pt}{1pt}}
\put(838,309){\rule{1pt}{1pt}}
\put(839,293){\rule{1pt}{1pt}}
\put(839,297){\rule{1pt}{1pt}}
\put(839,301){\rule{1pt}{1pt}}
\put(840,301){\rule{1pt}{1pt}}
\put(840,302){\rule{1pt}{1pt}}
\put(841,318){\rule{1pt}{1pt}}
\put(841,288){\rule{1pt}{1pt}}
\put(841,317){\rule{1pt}{1pt}}
\put(842,302){\rule{1pt}{1pt}}
\put(842,310){\rule{1pt}{1pt}}
\put(842,287){\rule{1pt}{1pt}}
\put(843,287){\rule{1pt}{1pt}}
\put(843,305){\rule{1pt}{1pt}}
\put(843,319){\rule{1pt}{1pt}}
\put(844,289){\rule{1pt}{1pt}}
\put(844,296){\rule{1pt}{1pt}}
\put(844,301){\rule{1pt}{1pt}}
\put(845,296){\rule{1pt}{1pt}}
\put(845,286){\rule{1pt}{1pt}}
\put(846,286){\rule{1pt}{1pt}}
\put(846,308){\rule{1pt}{1pt}}
\put(846,301){\rule{1pt}{1pt}}
\put(847,296){\rule{1pt}{1pt}}
\put(847,290){\rule{1pt}{1pt}}
\put(847,302){\rule{1pt}{1pt}}
\put(848,286){\rule{1pt}{1pt}}
\put(848,296){\rule{1pt}{1pt}}
\put(848,290){\rule{1pt}{1pt}}
\put(849,291){\rule{1pt}{1pt}}
\put(849,288){\rule{1pt}{1pt}}
\put(850,301){\rule{1pt}{1pt}}
\put(850,290){\rule{1pt}{1pt}}
\put(850,307){\rule{1pt}{1pt}}
\put(851,297){\rule{1pt}{1pt}}
\put(851,318){\rule{1pt}{1pt}}
\put(851,316){\rule{1pt}{1pt}}
\put(852,310){\rule{1pt}{1pt}}
\put(852,310){\rule{1pt}{1pt}}
\put(852,310){\rule{1pt}{1pt}}
\put(853,289){\rule{1pt}{1pt}}
\put(853,298){\rule{1pt}{1pt}}
\put(854,300){\rule{1pt}{1pt}}
\put(854,291){\rule{1pt}{1pt}}
\put(854,305){\rule{1pt}{1pt}}
\put(855,293){\rule{1pt}{1pt}}
\put(855,292){\rule{1pt}{1pt}}
\put(855,310){\rule{1pt}{1pt}}
\put(856,310){\rule{1pt}{1pt}}
\put(856,306){\rule{1pt}{1pt}}
\put(856,296){\rule{1pt}{1pt}}
\put(857,310){\rule{1pt}{1pt}}
\put(857,314){\rule{1pt}{1pt}}
\put(858,292){\rule{1pt}{1pt}}
\put(858,306){\rule{1pt}{1pt}}
\put(858,298){\rule{1pt}{1pt}}
\put(859,316){\rule{1pt}{1pt}}
\put(859,317){\rule{1pt}{1pt}}
\put(859,313){\rule{1pt}{1pt}}
\put(860,298){\rule{1pt}{1pt}}
\put(860,309){\rule{1pt}{1pt}}
\put(860,302){\rule{1pt}{1pt}}
\put(861,316){\rule{1pt}{1pt}}
\put(861,290){\rule{1pt}{1pt}}
\put(862,286){\rule{1pt}{1pt}}
\put(862,306){\rule{1pt}{1pt}}
\put(862,299){\rule{1pt}{1pt}}
\put(863,293){\rule{1pt}{1pt}}
\put(863,289){\rule{1pt}{1pt}}
\put(863,296){\rule{1pt}{1pt}}
\put(864,298){\rule{1pt}{1pt}}
\put(864,294){\rule{1pt}{1pt}}
\put(864,306){\rule{1pt}{1pt}}
\put(865,299){\rule{1pt}{1pt}}
\put(865,312){\rule{1pt}{1pt}}
\put(866,292){\rule{1pt}{1pt}}
\put(866,294){\rule{1pt}{1pt}}
\put(866,298){\rule{1pt}{1pt}}
\put(867,315){\rule{1pt}{1pt}}
\put(867,315){\rule{1pt}{1pt}}
\put(867,297){\rule{1pt}{1pt}}
\put(868,295){\rule{1pt}{1pt}}
\put(868,316){\rule{1pt}{1pt}}
\put(868,283){\rule{1pt}{1pt}}
\put(869,298){\rule{1pt}{1pt}}
\put(869,295){\rule{1pt}{1pt}}
\put(869,313){\rule{1pt}{1pt}}
\put(870,296){\rule{1pt}{1pt}}
\put(870,286){\rule{1pt}{1pt}}
\put(871,290){\rule{1pt}{1pt}}
\put(871,288){\rule{1pt}{1pt}}
\put(871,306){\rule{1pt}{1pt}}
\put(872,286){\rule{1pt}{1pt}}
\put(872,289){\rule{1pt}{1pt}}
\put(872,306){\rule{1pt}{1pt}}
\put(873,289){\rule{1pt}{1pt}}
\put(873,301){\rule{1pt}{1pt}}
\put(873,313){\rule{1pt}{1pt}}
\put(874,308){\rule{1pt}{1pt}}
\put(874,306){\rule{1pt}{1pt}}
\put(875,315){\rule{1pt}{1pt}}
\put(875,296){\rule{1pt}{1pt}}
\put(875,298){\rule{1pt}{1pt}}
\put(876,313){\rule{1pt}{1pt}}
\put(876,294){\rule{1pt}{1pt}}
\put(876,285){\rule{1pt}{1pt}}
\put(877,285){\rule{1pt}{1pt}}
\put(877,281){\rule{1pt}{1pt}}
\put(877,300){\rule{1pt}{1pt}}
\put(878,284){\rule{1pt}{1pt}}
\put(878,283){\rule{1pt}{1pt}}
\put(879,297){\rule{1pt}{1pt}}
\put(879,303){\rule{1pt}{1pt}}
\put(879,313){\rule{1pt}{1pt}}
\put(880,308){\rule{1pt}{1pt}}
\put(880,296){\rule{1pt}{1pt}}
\put(880,285){\rule{1pt}{1pt}}
\put(881,298){\rule{1pt}{1pt}}
\put(881,291){\rule{1pt}{1pt}}
\put(881,299){\rule{1pt}{1pt}}
\put(882,314){\rule{1pt}{1pt}}
\put(882,296){\rule{1pt}{1pt}}
\put(883,284){\rule{1pt}{1pt}}
\put(883,297){\rule{1pt}{1pt}}
\put(883,301){\rule{1pt}{1pt}}
\put(884,285){\rule{1pt}{1pt}}
\put(884,281){\rule{1pt}{1pt}}
\put(884,305){\rule{1pt}{1pt}}
\put(885,286){\rule{1pt}{1pt}}
\put(885,279){\rule{1pt}{1pt}}
\put(885,310){\rule{1pt}{1pt}}
\put(886,286){\rule{1pt}{1pt}}
\put(886,313){\rule{1pt}{1pt}}
\put(887,282){\rule{1pt}{1pt}}
\put(887,290){\rule{1pt}{1pt}}
\put(887,293){\rule{1pt}{1pt}}
\put(888,306){\rule{1pt}{1pt}}
\put(888,283){\rule{1pt}{1pt}}
\put(888,302){\rule{1pt}{1pt}}
\put(889,289){\rule{1pt}{1pt}}
\put(889,284){\rule{1pt}{1pt}}
\put(889,281){\rule{1pt}{1pt}}
\put(890,282){\rule{1pt}{1pt}}
\put(890,294){\rule{1pt}{1pt}}
\put(890,312){\rule{1pt}{1pt}}
\put(891,299){\rule{1pt}{1pt}}
\put(891,280){\rule{1pt}{1pt}}
\put(892,284){\rule{1pt}{1pt}}
\put(892,291){\rule{1pt}{1pt}}
\put(892,295){\rule{1pt}{1pt}}
\put(893,289){\rule{1pt}{1pt}}
\put(893,289){\rule{1pt}{1pt}}
\put(893,299){\rule{1pt}{1pt}}
\put(894,298){\rule{1pt}{1pt}}
\put(894,298){\rule{1pt}{1pt}}
\put(894,284){\rule{1pt}{1pt}}
\put(895,305){\rule{1pt}{1pt}}
\put(895,280){\rule{1pt}{1pt}}
\put(896,280){\rule{1pt}{1pt}}
\put(896,291){\rule{1pt}{1pt}}
\put(896,294){\rule{1pt}{1pt}}
\put(897,311){\rule{1pt}{1pt}}
\put(897,308){\rule{1pt}{1pt}}
\put(897,296){\rule{1pt}{1pt}}
\put(898,289){\rule{1pt}{1pt}}
\put(898,292){\rule{1pt}{1pt}}
\put(898,287){\rule{1pt}{1pt}}
\put(899,298){\rule{1pt}{1pt}}
\put(899,283){\rule{1pt}{1pt}}
\put(900,294){\rule{1pt}{1pt}}
\put(900,293){\rule{1pt}{1pt}}
\put(900,300){\rule{1pt}{1pt}}
\put(901,282){\rule{1pt}{1pt}}
\put(901,277){\rule{1pt}{1pt}}
\put(901,309){\rule{1pt}{1pt}}
\put(902,298){\rule{1pt}{1pt}}
\put(902,282){\rule{1pt}{1pt}}
\put(902,281){\rule{1pt}{1pt}}
\put(903,283){\rule{1pt}{1pt}}
\put(903,305){\rule{1pt}{1pt}}
\put(904,291){\rule{1pt}{1pt}}
\put(904,303){\rule{1pt}{1pt}}
\put(904,285){\rule{1pt}{1pt}}
\put(905,282){\rule{1pt}{1pt}}
\put(905,307){\rule{1pt}{1pt}}
\put(905,281){\rule{1pt}{1pt}}
\put(906,300){\rule{1pt}{1pt}}
\put(906,291){\rule{1pt}{1pt}}
\put(906,307){\rule{1pt}{1pt}}
\put(907,277){\rule{1pt}{1pt}}
\put(907,293){\rule{1pt}{1pt}}
\put(908,295){\rule{1pt}{1pt}}
\put(908,287){\rule{1pt}{1pt}}
\put(908,311){\rule{1pt}{1pt}}
\put(909,279){\rule{1pt}{1pt}}
\put(909,293){\rule{1pt}{1pt}}
\put(909,310){\rule{1pt}{1pt}}
\put(910,284){\rule{1pt}{1pt}}
\put(910,294){\rule{1pt}{1pt}}
\put(910,289){\rule{1pt}{1pt}}
\put(911,287){\rule{1pt}{1pt}}
\put(911,297){\rule{1pt}{1pt}}
\put(912,313){\rule{1pt}{1pt}}
\put(912,300){\rule{1pt}{1pt}}
\put(912,292){\rule{1pt}{1pt}}
\put(913,287){\rule{1pt}{1pt}}
\put(913,277){\rule{1pt}{1pt}}
\put(913,290){\rule{1pt}{1pt}}
\put(914,303){\rule{1pt}{1pt}}
\put(914,293){\rule{1pt}{1pt}}
\put(914,284){\rule{1pt}{1pt}}
\put(915,305){\rule{1pt}{1pt}}
\put(915,281){\rule{1pt}{1pt}}
\put(915,298){\rule{1pt}{1pt}}
\put(916,308){\rule{1pt}{1pt}}
\put(916,312){\rule{1pt}{1pt}}
\put(917,287){\rule{1pt}{1pt}}
\put(917,286){\rule{1pt}{1pt}}
\put(917,311){\rule{1pt}{1pt}}
\put(918,307){\rule{1pt}{1pt}}
\put(918,278){\rule{1pt}{1pt}}
\put(918,295){\rule{1pt}{1pt}}
\put(919,280){\rule{1pt}{1pt}}
\put(919,280){\rule{1pt}{1pt}}
\put(919,291){\rule{1pt}{1pt}}
\put(920,282){\rule{1pt}{1pt}}
\put(920,282){\rule{1pt}{1pt}}
\put(921,278){\rule{1pt}{1pt}}
\put(921,296){\rule{1pt}{1pt}}
\put(921,278){\rule{1pt}{1pt}}
\put(922,281){\rule{1pt}{1pt}}
\put(922,302){\rule{1pt}{1pt}}
\put(922,291){\rule{1pt}{1pt}}
\put(923,300){\rule{1pt}{1pt}}
\put(923,303){\rule{1pt}{1pt}}
\put(923,303){\rule{1pt}{1pt}}
\put(924,310){\rule{1pt}{1pt}}
\put(924,292){\rule{1pt}{1pt}}
\put(925,306){\rule{1pt}{1pt}}
\put(925,280){\rule{1pt}{1pt}}
\put(925,291){\rule{1pt}{1pt}}
\put(926,296){\rule{1pt}{1pt}}
\put(926,307){\rule{1pt}{1pt}}
\put(926,298){\rule{1pt}{1pt}}
\put(927,293){\rule{1pt}{1pt}}
\put(927,308){\rule{1pt}{1pt}}
\put(927,292){\rule{1pt}{1pt}}
\put(928,290){\rule{1pt}{1pt}}
\put(928,295){\rule{1pt}{1pt}}
\put(929,301){\rule{1pt}{1pt}}
\put(929,298){\rule{1pt}{1pt}}
\put(929,309){\rule{1pt}{1pt}}
\put(930,282){\rule{1pt}{1pt}}
\put(930,305){\rule{1pt}{1pt}}
\put(930,311){\rule{1pt}{1pt}}
\put(931,279){\rule{1pt}{1pt}}
\put(931,280){\rule{1pt}{1pt}}
\put(931,312){\rule{1pt}{1pt}}
\put(932,302){\rule{1pt}{1pt}}
\put(932,305){\rule{1pt}{1pt}}
\put(933,286){\rule{1pt}{1pt}}
\put(933,290){\rule{1pt}{1pt}}
\put(933,291){\rule{1pt}{1pt}}
\put(934,287){\rule{1pt}{1pt}}
\put(934,278){\rule{1pt}{1pt}}
\put(934,295){\rule{1pt}{1pt}}
\put(935,278){\rule{1pt}{1pt}}
\put(935,278){\rule{1pt}{1pt}}
\put(935,312){\rule{1pt}{1pt}}
\put(936,306){\rule{1pt}{1pt}}
\put(936,294){\rule{1pt}{1pt}}
\put(937,312){\rule{1pt}{1pt}}
\put(937,307){\rule{1pt}{1pt}}
\put(937,311){\rule{1pt}{1pt}}
\put(938,296){\rule{1pt}{1pt}}
\put(938,294){\rule{1pt}{1pt}}
\put(938,283){\rule{1pt}{1pt}}
\put(939,277){\rule{1pt}{1pt}}
\put(939,304){\rule{1pt}{1pt}}
\put(939,313){\rule{1pt}{1pt}}
\put(940,302){\rule{1pt}{1pt}}
\put(940,286){\rule{1pt}{1pt}}
\put(940,310){\rule{1pt}{1pt}}
\put(941,295){\rule{1pt}{1pt}}
\put(941,300){\rule{1pt}{1pt}}
\put(942,280){\rule{1pt}{1pt}}
\put(942,309){\rule{1pt}{1pt}}
\put(942,296){\rule{1pt}{1pt}}
\put(943,303){\rule{1pt}{1pt}}
\put(943,296){\rule{1pt}{1pt}}
\put(943,303){\rule{1pt}{1pt}}
\put(944,280){\rule{1pt}{1pt}}
\put(944,288){\rule{1pt}{1pt}}
\put(944,286){\rule{1pt}{1pt}}
\put(945,306){\rule{1pt}{1pt}}
\put(945,289){\rule{1pt}{1pt}}
\put(946,278){\rule{1pt}{1pt}}
\put(946,295){\rule{1pt}{1pt}}
\put(946,303){\rule{1pt}{1pt}}
\put(947,285){\rule{1pt}{1pt}}
\put(947,295){\rule{1pt}{1pt}}
\put(947,290){\rule{1pt}{1pt}}
\put(948,297){\rule{1pt}{1pt}}
\put(948,300){\rule{1pt}{1pt}}
\put(948,281){\rule{1pt}{1pt}}
\put(949,307){\rule{1pt}{1pt}}
\put(949,302){\rule{1pt}{1pt}}
\put(950,296){\rule{1pt}{1pt}}
\put(950,279){\rule{1pt}{1pt}}
\put(950,303){\rule{1pt}{1pt}}
\put(951,306){\rule{1pt}{1pt}}
\put(951,295){\rule{1pt}{1pt}}
\put(951,309){\rule{1pt}{1pt}}
\put(952,311){\rule{1pt}{1pt}}
\put(952,291){\rule{1pt}{1pt}}
\put(952,286){\rule{1pt}{1pt}}
\put(953,287){\rule{1pt}{1pt}}
\put(953,302){\rule{1pt}{1pt}}
\put(954,286){\rule{1pt}{1pt}}
\put(954,281){\rule{1pt}{1pt}}
\put(954,288){\rule{1pt}{1pt}}
\put(955,310){\rule{1pt}{1pt}}
\put(955,299){\rule{1pt}{1pt}}
\put(955,298){\rule{1pt}{1pt}}
\put(956,290){\rule{1pt}{1pt}}
\put(956,306){\rule{1pt}{1pt}}
\put(956,312){\rule{1pt}{1pt}}
\put(957,303){\rule{1pt}{1pt}}
\put(957,311){\rule{1pt}{1pt}}
\put(958,310){\rule{1pt}{1pt}}
\put(958,288){\rule{1pt}{1pt}}
\put(958,283){\rule{1pt}{1pt}}
\put(959,306){\rule{1pt}{1pt}}
\put(959,295){\rule{1pt}{1pt}}
\put(959,296){\rule{1pt}{1pt}}
\put(960,280){\rule{1pt}{1pt}}
\put(960,281){\rule{1pt}{1pt}}
\put(960,293){\rule{1pt}{1pt}}
\put(961,298){\rule{1pt}{1pt}}
\put(961,278){\rule{1pt}{1pt}}
\put(962,310){\rule{1pt}{1pt}}
\put(962,285){\rule{1pt}{1pt}}
\put(962,303){\rule{1pt}{1pt}}
\put(963,282){\rule{1pt}{1pt}}
\put(963,288){\rule{1pt}{1pt}}
\put(963,281){\rule{1pt}{1pt}}
\put(964,284){\rule{1pt}{1pt}}
\put(964,305){\rule{1pt}{1pt}}
\put(964,281){\rule{1pt}{1pt}}
\put(965,280){\rule{1pt}{1pt}}
\put(965,312){\rule{1pt}{1pt}}
\put(965,307){\rule{1pt}{1pt}}
\put(966,296){\rule{1pt}{1pt}}
\put(966,313){\rule{1pt}{1pt}}
\put(967,296){\rule{1pt}{1pt}}
\put(967,287){\rule{1pt}{1pt}}
\put(967,310){\rule{1pt}{1pt}}
\put(968,289){\rule{1pt}{1pt}}
\put(968,289){\rule{1pt}{1pt}}
\put(968,292){\rule{1pt}{1pt}}
\put(969,288){\rule{1pt}{1pt}}
\put(969,307){\rule{1pt}{1pt}}
\put(969,280){\rule{1pt}{1pt}}
\put(970,309){\rule{1pt}{1pt}}
\put(970,307){\rule{1pt}{1pt}}
\put(971,293){\rule{1pt}{1pt}}
\put(971,282){\rule{1pt}{1pt}}
\put(971,286){\rule{1pt}{1pt}}
\put(972,298){\rule{1pt}{1pt}}
\put(972,305){\rule{1pt}{1pt}}
\put(972,278){\rule{1pt}{1pt}}
\put(973,284){\rule{1pt}{1pt}}
\put(973,285){\rule{1pt}{1pt}}
\put(973,277){\rule{1pt}{1pt}}
\put(974,296){\rule{1pt}{1pt}}
\put(974,286){\rule{1pt}{1pt}}
\put(975,300){\rule{1pt}{1pt}}
\put(975,308){\rule{1pt}{1pt}}
\put(975,286){\rule{1pt}{1pt}}
\put(976,296){\rule{1pt}{1pt}}
\put(976,305){\rule{1pt}{1pt}}
\put(976,284){\rule{1pt}{1pt}}
\put(977,281){\rule{1pt}{1pt}}
\put(977,288){\rule{1pt}{1pt}}
\put(977,292){\rule{1pt}{1pt}}
\put(978,281){\rule{1pt}{1pt}}
\put(978,282){\rule{1pt}{1pt}}
\put(979,283){\rule{1pt}{1pt}}
\put(979,281){\rule{1pt}{1pt}}
\put(979,279){\rule{1pt}{1pt}}
\put(980,298){\rule{1pt}{1pt}}
\put(980,303){\rule{1pt}{1pt}}
\put(980,305){\rule{1pt}{1pt}}
\put(981,295){\rule{1pt}{1pt}}
\put(981,311){\rule{1pt}{1pt}}
\put(981,295){\rule{1pt}{1pt}}
\put(982,281){\rule{1pt}{1pt}}
\put(982,308){\rule{1pt}{1pt}}
\put(983,300){\rule{1pt}{1pt}}
\put(983,286){\rule{1pt}{1pt}}
\put(983,308){\rule{1pt}{1pt}}
\put(984,278){\rule{1pt}{1pt}}
\put(984,312){\rule{1pt}{1pt}}
\put(984,300){\rule{1pt}{1pt}}
\put(985,284){\rule{1pt}{1pt}}
\put(985,285){\rule{1pt}{1pt}}
\put(985,311){\rule{1pt}{1pt}}
\put(986,279){\rule{1pt}{1pt}}
\put(986,302){\rule{1pt}{1pt}}
\put(987,298){\rule{1pt}{1pt}}
\put(987,307){\rule{1pt}{1pt}}
\put(987,296){\rule{1pt}{1pt}}
\put(988,295){\rule{1pt}{1pt}}
\put(988,289){\rule{1pt}{1pt}}
\put(988,293){\rule{1pt}{1pt}}
\put(989,280){\rule{1pt}{1pt}}
\put(989,294){\rule{1pt}{1pt}}
\put(989,299){\rule{1pt}{1pt}}
\put(990,282){\rule{1pt}{1pt}}
\put(990,281){\rule{1pt}{1pt}}
\put(990,282){\rule{1pt}{1pt}}
\put(991,296){\rule{1pt}{1pt}}
\put(991,291){\rule{1pt}{1pt}}
\put(992,277){\rule{1pt}{1pt}}
\put(992,279){\rule{1pt}{1pt}}
\put(992,301){\rule{1pt}{1pt}}
\put(993,292){\rule{1pt}{1pt}}
\put(993,299){\rule{1pt}{1pt}}
\put(993,288){\rule{1pt}{1pt}}
\put(994,296){\rule{1pt}{1pt}}
\put(994,293){\rule{1pt}{1pt}}
\put(994,303){\rule{1pt}{1pt}}
\put(995,302){\rule{1pt}{1pt}}
\put(995,312){\rule{1pt}{1pt}}
\put(996,312){\rule{1pt}{1pt}}
\put(996,277){\rule{1pt}{1pt}}
\put(996,278){\rule{1pt}{1pt}}
\put(997,295){\rule{1pt}{1pt}}
\put(997,305){\rule{1pt}{1pt}}
\put(997,289){\rule{1pt}{1pt}}
\put(998,289){\rule{1pt}{1pt}}
\put(998,279){\rule{1pt}{1pt}}
\put(998,305){\rule{1pt}{1pt}}
\put(999,287){\rule{1pt}{1pt}}
\put(999,278){\rule{1pt}{1pt}}
\put(1000,307){\rule{1pt}{1pt}}
\put(1000,290){\rule{1pt}{1pt}}
\put(1000,311){\rule{1pt}{1pt}}
\put(1001,296){\rule{1pt}{1pt}}
\put(1001,293){\rule{1pt}{1pt}}
\put(1001,288){\rule{1pt}{1pt}}
\put(1002,286){\rule{1pt}{1pt}}
\put(1002,289){\rule{1pt}{1pt}}
\put(1002,279){\rule{1pt}{1pt}}
\put(1003,300){\rule{1pt}{1pt}}
\put(1003,295){\rule{1pt}{1pt}}
\put(1004,292){\rule{1pt}{1pt}}
\put(1004,282){\rule{1pt}{1pt}}
\put(1004,298){\rule{1pt}{1pt}}
\put(1005,302){\rule{1pt}{1pt}}
\put(1005,301){\rule{1pt}{1pt}}
\put(1005,307){\rule{1pt}{1pt}}
\put(1006,286){\rule{1pt}{1pt}}
\put(1006,295){\rule{1pt}{1pt}}
\put(1006,294){\rule{1pt}{1pt}}
\put(1007,299){\rule{1pt}{1pt}}
\put(1007,300){\rule{1pt}{1pt}}
\put(1008,310){\rule{1pt}{1pt}}
\put(1008,292){\rule{1pt}{1pt}}
\put(1008,304){\rule{1pt}{1pt}}
\put(1009,278){\rule{1pt}{1pt}}
\put(1009,284){\rule{1pt}{1pt}}
\put(1009,309){\rule{1pt}{1pt}}
\put(1010,282){\rule{1pt}{1pt}}
\put(1010,312){\rule{1pt}{1pt}}
\put(1010,284){\rule{1pt}{1pt}}
\put(1011,306){\rule{1pt}{1pt}}
\put(1011,305){\rule{1pt}{1pt}}
\put(1012,286){\rule{1pt}{1pt}}
\put(1012,307){\rule{1pt}{1pt}}
\put(1012,304){\rule{1pt}{1pt}}
\put(1013,279){\rule{1pt}{1pt}}
\put(1013,308){\rule{1pt}{1pt}}
\put(1013,305){\rule{1pt}{1pt}}
\put(1014,277){\rule{1pt}{1pt}}
\put(1014,299){\rule{1pt}{1pt}}
\put(1014,305){\rule{1pt}{1pt}}
\put(1015,285){\rule{1pt}{1pt}}
\put(1015,292){\rule{1pt}{1pt}}
\put(1015,285){\rule{1pt}{1pt}}
\put(1016,311){\rule{1pt}{1pt}}
\put(1016,306){\rule{1pt}{1pt}}
\put(1017,299){\rule{1pt}{1pt}}
\put(1017,310){\rule{1pt}{1pt}}
\put(1017,290){\rule{1pt}{1pt}}
\put(1018,278){\rule{1pt}{1pt}}
\put(1018,312){\rule{1pt}{1pt}}
\put(1018,305){\rule{1pt}{1pt}}
\put(1019,306){\rule{1pt}{1pt}}
\put(1019,288){\rule{1pt}{1pt}}
\put(1019,292){\rule{1pt}{1pt}}
\put(1020,292){\rule{1pt}{1pt}}
\put(1020,280){\rule{1pt}{1pt}}
\put(1021,306){\rule{1pt}{1pt}}
\put(1021,285){\rule{1pt}{1pt}}
\put(1021,289){\rule{1pt}{1pt}}
\put(1022,293){\rule{1pt}{1pt}}
\put(1022,279){\rule{1pt}{1pt}}
\put(1022,289){\rule{1pt}{1pt}}
\put(1023,307){\rule{1pt}{1pt}}
\put(1023,311){\rule{1pt}{1pt}}
\put(1023,278){\rule{1pt}{1pt}}
\put(1024,279){\rule{1pt}{1pt}}
\put(1024,286){\rule{1pt}{1pt}}
\put(1025,309){\rule{1pt}{1pt}}
\put(1025,298){\rule{1pt}{1pt}}
\put(1025,281){\rule{1pt}{1pt}}
\put(1026,308){\rule{1pt}{1pt}}
\put(1026,305){\rule{1pt}{1pt}}
\put(1026,294){\rule{1pt}{1pt}}
\put(1027,312){\rule{1pt}{1pt}}
\put(1027,308){\rule{1pt}{1pt}}
\put(1027,291){\rule{1pt}{1pt}}
\put(1028,300){\rule{1pt}{1pt}}
\put(1028,306){\rule{1pt}{1pt}}
\put(1029,287){\rule{1pt}{1pt}}
\put(1029,301){\rule{1pt}{1pt}}
\put(1029,297){\rule{1pt}{1pt}}
\put(1030,312){\rule{1pt}{1pt}}
\put(1030,287){\rule{1pt}{1pt}}
\put(1030,284){\rule{1pt}{1pt}}
\put(1031,291){\rule{1pt}{1pt}}
\put(1031,280){\rule{1pt}{1pt}}
\put(1031,283){\rule{1pt}{1pt}}
\put(1032,291){\rule{1pt}{1pt}}
\put(1032,292){\rule{1pt}{1pt}}
\put(1033,291){\rule{1pt}{1pt}}
\put(1033,288){\rule{1pt}{1pt}}
\put(1033,309){\rule{1pt}{1pt}}
\put(1034,291){\rule{1pt}{1pt}}
\put(1034,286){\rule{1pt}{1pt}}
\put(1034,287){\rule{1pt}{1pt}}
\put(1035,299){\rule{1pt}{1pt}}
\put(1035,286){\rule{1pt}{1pt}}
\put(1035,280){\rule{1pt}{1pt}}
\put(1036,290){\rule{1pt}{1pt}}
\put(1036,300){\rule{1pt}{1pt}}
\put(1037,289){\rule{1pt}{1pt}}
\put(1037,301){\rule{1pt}{1pt}}
\put(1037,299){\rule{1pt}{1pt}}
\put(1038,303){\rule{1pt}{1pt}}
\put(1038,306){\rule{1pt}{1pt}}
\put(1038,302){\rule{1pt}{1pt}}
\put(1039,292){\rule{1pt}{1pt}}
\put(1039,307){\rule{1pt}{1pt}}
\put(1039,298){\rule{1pt}{1pt}}
\put(1040,309){\rule{1pt}{1pt}}
\put(1040,293){\rule{1pt}{1pt}}
\put(1040,301){\rule{1pt}{1pt}}
\put(1041,310){\rule{1pt}{1pt}}
\put(1041,294){\rule{1pt}{1pt}}
\put(1042,291){\rule{1pt}{1pt}}
\put(1042,285){\rule{1pt}{1pt}}
\put(1042,288){\rule{1pt}{1pt}}
\put(1043,312){\rule{1pt}{1pt}}
\put(1043,286){\rule{1pt}{1pt}}
\put(1043,301){\rule{1pt}{1pt}}
\put(1044,302){\rule{1pt}{1pt}}
\put(1044,309){\rule{1pt}{1pt}}
\put(1044,310){\rule{1pt}{1pt}}
\put(1045,284){\rule{1pt}{1pt}}
\put(1045,279){\rule{1pt}{1pt}}
\put(1046,285){\rule{1pt}{1pt}}
\put(1046,286){\rule{1pt}{1pt}}
\put(1046,282){\rule{1pt}{1pt}}
\put(1047,310){\rule{1pt}{1pt}}
\put(1047,285){\rule{1pt}{1pt}}
\put(1047,299){\rule{1pt}{1pt}}
\put(1048,280){\rule{1pt}{1pt}}
\put(1048,301){\rule{1pt}{1pt}}
\put(1048,309){\rule{1pt}{1pt}}
\put(1049,284){\rule{1pt}{1pt}}
\put(1049,309){\rule{1pt}{1pt}}
\put(1050,285){\rule{1pt}{1pt}}
\put(1050,305){\rule{1pt}{1pt}}
\put(1050,284){\rule{1pt}{1pt}}
\put(1051,288){\rule{1pt}{1pt}}
\put(1051,304){\rule{1pt}{1pt}}
\put(1051,290){\rule{1pt}{1pt}}
\put(1052,306){\rule{1pt}{1pt}}
\put(1052,307){\rule{1pt}{1pt}}
\put(1052,300){\rule{1pt}{1pt}}
\put(1053,298){\rule{1pt}{1pt}}
\put(1053,308){\rule{1pt}{1pt}}
\put(1054,286){\rule{1pt}{1pt}}
\put(1054,282){\rule{1pt}{1pt}}
\put(1054,311){\rule{1pt}{1pt}}
\put(1055,286){\rule{1pt}{1pt}}
\put(1055,280){\rule{1pt}{1pt}}
\put(1055,289){\rule{1pt}{1pt}}
\put(1056,310){\rule{1pt}{1pt}}
\put(1056,290){\rule{1pt}{1pt}}
\put(1056,283){\rule{1pt}{1pt}}
\put(1057,309){\rule{1pt}{1pt}}
\put(1057,291){\rule{1pt}{1pt}}
\put(1058,291){\rule{1pt}{1pt}}
\put(1058,283){\rule{1pt}{1pt}}
\put(1058,306){\rule{1pt}{1pt}}
\put(1059,295){\rule{1pt}{1pt}}
\put(1059,295){\rule{1pt}{1pt}}
\put(1059,294){\rule{1pt}{1pt}}
\put(1060,300){\rule{1pt}{1pt}}
\put(1060,303){\rule{1pt}{1pt}}
\put(1060,290){\rule{1pt}{1pt}}
\put(1061,278){\rule{1pt}{1pt}}
\put(1061,280){\rule{1pt}{1pt}}
\put(1061,286){\rule{1pt}{1pt}}
\put(1062,279){\rule{1pt}{1pt}}
\put(1062,301){\rule{1pt}{1pt}}
\put(1063,289){\rule{1pt}{1pt}}
\put(1063,291){\rule{1pt}{1pt}}
\put(1063,299){\rule{1pt}{1pt}}
\put(1064,277){\rule{1pt}{1pt}}
\put(1064,301){\rule{1pt}{1pt}}
\put(1064,309){\rule{1pt}{1pt}}
\put(1065,284){\rule{1pt}{1pt}}
\put(1065,309){\rule{1pt}{1pt}}
\put(1065,283){\rule{1pt}{1pt}}
\put(1066,309){\rule{1pt}{1pt}}
\put(1066,297){\rule{1pt}{1pt}}
\put(1067,283){\rule{1pt}{1pt}}
\put(1067,311){\rule{1pt}{1pt}}
\put(1067,278){\rule{1pt}{1pt}}
\put(1068,312){\rule{1pt}{1pt}}
\put(1068,290){\rule{1pt}{1pt}}
\put(1068,298){\rule{1pt}{1pt}}
\put(1069,294){\rule{1pt}{1pt}}
\put(1069,308){\rule{1pt}{1pt}}
\put(1069,281){\rule{1pt}{1pt}}
\put(1070,286){\rule{1pt}{1pt}}
\put(1070,309){\rule{1pt}{1pt}}
\put(1071,287){\rule{1pt}{1pt}}
\put(1071,310){\rule{1pt}{1pt}}
\put(1071,301){\rule{1pt}{1pt}}
\put(1072,311){\rule{1pt}{1pt}}
\put(1072,305){\rule{1pt}{1pt}}
\put(1072,285){\rule{1pt}{1pt}}
\put(1073,282){\rule{1pt}{1pt}}
\put(1073,302){\rule{1pt}{1pt}}
\put(1073,285){\rule{1pt}{1pt}}
\put(1074,284){\rule{1pt}{1pt}}
\put(1074,306){\rule{1pt}{1pt}}
\put(1075,309){\rule{1pt}{1pt}}
\put(1075,290){\rule{1pt}{1pt}}
\put(1075,293){\rule{1pt}{1pt}}
\put(1076,303){\rule{1pt}{1pt}}
\put(1076,301){\rule{1pt}{1pt}}
\put(1076,308){\rule{1pt}{1pt}}
\put(1077,278){\rule{1pt}{1pt}}
\put(1077,307){\rule{1pt}{1pt}}
\put(1077,279){\rule{1pt}{1pt}}
\put(1078,312){\rule{1pt}{1pt}}
\put(1078,288){\rule{1pt}{1pt}}
\put(1079,299){\rule{1pt}{1pt}}
\put(1079,304){\rule{1pt}{1pt}}
\put(1079,286){\rule{1pt}{1pt}}
\put(1080,283){\rule{1pt}{1pt}}
\put(1080,277){\rule{1pt}{1pt}}
\put(1080,302){\rule{1pt}{1pt}}
\put(1081,310){\rule{1pt}{1pt}}
\put(1081,306){\rule{1pt}{1pt}}
\put(1081,302){\rule{1pt}{1pt}}
\put(1082,313){\rule{1pt}{1pt}}
\put(1082,297){\rule{1pt}{1pt}}
\put(1083,310){\rule{1pt}{1pt}}
\put(1083,289){\rule{1pt}{1pt}}
\put(1083,279){\rule{1pt}{1pt}}
\put(1084,284){\rule{1pt}{1pt}}
\put(1084,280){\rule{1pt}{1pt}}
\put(1084,295){\rule{1pt}{1pt}}
\put(1085,295){\rule{1pt}{1pt}}
\put(1085,304){\rule{1pt}{1pt}}
\put(1085,300){\rule{1pt}{1pt}}
\put(1086,307){\rule{1pt}{1pt}}
\put(1086,298){\rule{1pt}{1pt}}
\put(1086,296){\rule{1pt}{1pt}}
\put(1087,298){\rule{1pt}{1pt}}
\put(1087,299){\rule{1pt}{1pt}}
\put(1088,300){\rule{1pt}{1pt}}
\put(1088,281){\rule{1pt}{1pt}}
\put(1088,287){\rule{1pt}{1pt}}
\put(1089,280){\rule{1pt}{1pt}}
\put(1089,279){\rule{1pt}{1pt}}
\put(1089,289){\rule{1pt}{1pt}}
\put(1090,307){\rule{1pt}{1pt}}
\put(1090,308){\rule{1pt}{1pt}}
\put(1090,284){\rule{1pt}{1pt}}
\put(1091,296){\rule{1pt}{1pt}}
\put(1091,281){\rule{1pt}{1pt}}
\put(1092,288){\rule{1pt}{1pt}}
\put(1092,297){\rule{1pt}{1pt}}
\put(1092,312){\rule{1pt}{1pt}}
\put(1093,281){\rule{1pt}{1pt}}
\put(1093,295){\rule{1pt}{1pt}}
\put(1093,278){\rule{1pt}{1pt}}
\put(1094,312){\rule{1pt}{1pt}}
\put(1094,280){\rule{1pt}{1pt}}
\put(1094,279){\rule{1pt}{1pt}}
\put(1095,278){\rule{1pt}{1pt}}
\put(1095,295){\rule{1pt}{1pt}}
\put(1096,278){\rule{1pt}{1pt}}
\put(1096,291){\rule{1pt}{1pt}}
\put(1096,279){\rule{1pt}{1pt}}
\put(1097,281){\rule{1pt}{1pt}}
\put(1097,297){\rule{1pt}{1pt}}
\put(1097,286){\rule{1pt}{1pt}}
\put(1098,283){\rule{1pt}{1pt}}
\put(1098,288){\rule{1pt}{1pt}}
\put(1098,299){\rule{1pt}{1pt}}
\put(1099,302){\rule{1pt}{1pt}}
\put(1099,296){\rule{1pt}{1pt}}
\put(1100,312){\rule{1pt}{1pt}}
\put(1100,309){\rule{1pt}{1pt}}
\put(1100,307){\rule{1pt}{1pt}}
\put(1101,293){\rule{1pt}{1pt}}
\put(1101,302){\rule{1pt}{1pt}}
\put(1101,300){\rule{1pt}{1pt}}
\put(1102,302){\rule{1pt}{1pt}}
\put(1102,306){\rule{1pt}{1pt}}
\put(1102,280){\rule{1pt}{1pt}}
\put(1103,286){\rule{1pt}{1pt}}
\put(1103,280){\rule{1pt}{1pt}}
\put(1104,305){\rule{1pt}{1pt}}
\put(1104,299){\rule{1pt}{1pt}}
\put(1104,289){\rule{1pt}{1pt}}
\put(1105,298){\rule{1pt}{1pt}}
\put(1105,302){\rule{1pt}{1pt}}
\put(1105,282){\rule{1pt}{1pt}}
\put(1106,290){\rule{1pt}{1pt}}
\put(1106,286){\rule{1pt}{1pt}}
\put(1106,308){\rule{1pt}{1pt}}
\put(1107,302){\rule{1pt}{1pt}}
\put(1107,290){\rule{1pt}{1pt}}
\put(1108,303){\rule{1pt}{1pt}}
\put(1108,299){\rule{1pt}{1pt}}
\put(1108,305){\rule{1pt}{1pt}}
\put(1109,302){\rule{1pt}{1pt}}
\put(1109,296){\rule{1pt}{1pt}}
\put(1109,298){\rule{1pt}{1pt}}
\put(1110,291){\rule{1pt}{1pt}}
\put(1110,288){\rule{1pt}{1pt}}
\put(1110,286){\rule{1pt}{1pt}}
\put(1111,311){\rule{1pt}{1pt}}
\put(1111,307){\rule{1pt}{1pt}}
\put(1111,310){\rule{1pt}{1pt}}
\put(1112,280){\rule{1pt}{1pt}}
\put(1112,291){\rule{1pt}{1pt}}
\put(1113,308){\rule{1pt}{1pt}}
\put(1113,282){\rule{1pt}{1pt}}
\put(1113,312){\rule{1pt}{1pt}}
\put(1114,306){\rule{1pt}{1pt}}
\put(1114,286){\rule{1pt}{1pt}}
\put(1114,282){\rule{1pt}{1pt}}
\put(1115,308){\rule{1pt}{1pt}}
\put(1115,279){\rule{1pt}{1pt}}
\put(1115,311){\rule{1pt}{1pt}}
\put(1116,297){\rule{1pt}{1pt}}
\put(1116,279){\rule{1pt}{1pt}}
\put(1117,282){\rule{1pt}{1pt}}
\put(1117,309){\rule{1pt}{1pt}}
\put(1117,310){\rule{1pt}{1pt}}
\put(1118,310){\rule{1pt}{1pt}}
\put(1118,308){\rule{1pt}{1pt}}
\put(1118,283){\rule{1pt}{1pt}}
\put(1119,296){\rule{1pt}{1pt}}
\put(1119,296){\rule{1pt}{1pt}}
\put(1119,307){\rule{1pt}{1pt}}
\put(1120,297){\rule{1pt}{1pt}}
\put(1120,285){\rule{1pt}{1pt}}
\put(1121,292){\rule{1pt}{1pt}}
\put(1121,278){\rule{1pt}{1pt}}
\put(1121,306){\rule{1pt}{1pt}}
\put(1122,312){\rule{1pt}{1pt}}
\put(1122,309){\rule{1pt}{1pt}}
\put(1122,290){\rule{1pt}{1pt}}
\put(1123,283){\rule{1pt}{1pt}}
\put(1123,302){\rule{1pt}{1pt}}
\put(1123,301){\rule{1pt}{1pt}}
\put(1124,309){\rule{1pt}{1pt}}
\put(1124,283){\rule{1pt}{1pt}}
\put(1125,284){\rule{1pt}{1pt}}
\put(1125,302){\rule{1pt}{1pt}}
\put(1125,290){\rule{1pt}{1pt}}
\put(1126,309){\rule{1pt}{1pt}}
\put(1126,291){\rule{1pt}{1pt}}
\put(1126,292){\rule{1pt}{1pt}}
\put(1127,304){\rule{1pt}{1pt}}
\put(1127,312){\rule{1pt}{1pt}}
\put(1127,289){\rule{1pt}{1pt}}
\put(1128,302){\rule{1pt}{1pt}}
\put(1128,293){\rule{1pt}{1pt}}
\put(1129,280){\rule{1pt}{1pt}}
\put(1129,290){\rule{1pt}{1pt}}
\put(1129,287){\rule{1pt}{1pt}}
\put(1130,283){\rule{1pt}{1pt}}
\put(1130,283){\rule{1pt}{1pt}}
\put(1130,279){\rule{1pt}{1pt}}
\put(1131,292){\rule{1pt}{1pt}}
\put(1131,305){\rule{1pt}{1pt}}
\put(1131,281){\rule{1pt}{1pt}}
\put(1132,287){\rule{1pt}{1pt}}
\put(1132,285){\rule{1pt}{1pt}}
\put(1133,293){\rule{1pt}{1pt}}
\put(1133,305){\rule{1pt}{1pt}}
\put(1133,286){\rule{1pt}{1pt}}
\put(1134,294){\rule{1pt}{1pt}}
\put(1134,311){\rule{1pt}{1pt}}
\put(1134,306){\rule{1pt}{1pt}}
\put(1135,288){\rule{1pt}{1pt}}
\put(1135,307){\rule{1pt}{1pt}}
\put(1135,312){\rule{1pt}{1pt}}
\put(1136,311){\rule{1pt}{1pt}}
\put(1136,301){\rule{1pt}{1pt}}
\put(1136,298){\rule{1pt}{1pt}}
\put(1137,286){\rule{1pt}{1pt}}
\put(1137,298){\rule{1pt}{1pt}}
\put(1138,311){\rule{1pt}{1pt}}
\put(1138,299){\rule{1pt}{1pt}}
\put(1138,296){\rule{1pt}{1pt}}
\put(1139,303){\rule{1pt}{1pt}}
\put(1139,305){\rule{1pt}{1pt}}
\put(1139,278){\rule{1pt}{1pt}}
\put(1140,280){\rule{1pt}{1pt}}
\put(1140,286){\rule{1pt}{1pt}}
\put(1140,310){\rule{1pt}{1pt}}
\put(1141,279){\rule{1pt}{1pt}}
\put(1141,306){\rule{1pt}{1pt}}
\put(1142,308){\rule{1pt}{1pt}}
\put(1142,286){\rule{1pt}{1pt}}
\put(1142,310){\rule{1pt}{1pt}}
\put(1143,306){\rule{1pt}{1pt}}
\put(1143,287){\rule{1pt}{1pt}}
\put(1143,286){\rule{1pt}{1pt}}
\put(1144,308){\rule{1pt}{1pt}}
\put(1144,277){\rule{1pt}{1pt}}
\put(1144,296){\rule{1pt}{1pt}}
\put(1145,284){\rule{1pt}{1pt}}
\put(1145,290){\rule{1pt}{1pt}}
\put(1146,302){\rule{1pt}{1pt}}
\put(1146,296){\rule{1pt}{1pt}}
\put(1146,307){\rule{1pt}{1pt}}
\put(1147,294){\rule{1pt}{1pt}}
\put(1147,292){\rule{1pt}{1pt}}
\put(1147,306){\rule{1pt}{1pt}}
\put(1148,308){\rule{1pt}{1pt}}
\put(1148,284){\rule{1pt}{1pt}}
\put(1148,302){\rule{1pt}{1pt}}
\put(1149,297){\rule{1pt}{1pt}}
\put(1149,277){\rule{1pt}{1pt}}
\put(1150,311){\rule{1pt}{1pt}}
\put(1150,310){\rule{1pt}{1pt}}
\put(1150,279){\rule{1pt}{1pt}}
\put(1151,290){\rule{1pt}{1pt}}
\put(1151,309){\rule{1pt}{1pt}}
\put(1151,283){\rule{1pt}{1pt}}
\put(1152,296){\rule{1pt}{1pt}}
\put(1152,308){\rule{1pt}{1pt}}
\put(1152,301){\rule{1pt}{1pt}}
\put(1153,296){\rule{1pt}{1pt}}
\put(1153,305){\rule{1pt}{1pt}}
\put(1154,293){\rule{1pt}{1pt}}
\put(1154,290){\rule{1pt}{1pt}}
\put(1154,297){\rule{1pt}{1pt}}
\put(1155,288){\rule{1pt}{1pt}}
\put(1155,283){\rule{1pt}{1pt}}
\put(1155,310){\rule{1pt}{1pt}}
\put(1156,302){\rule{1pt}{1pt}}
\put(1156,285){\rule{1pt}{1pt}}
\put(1156,303){\rule{1pt}{1pt}}
\put(1157,304){\rule{1pt}{1pt}}
\put(1157,285){\rule{1pt}{1pt}}
\put(1158,298){\rule{1pt}{1pt}}
\put(1158,290){\rule{1pt}{1pt}}
\put(1158,305){\rule{1pt}{1pt}}
\put(1159,308){\rule{1pt}{1pt}}
\put(1159,296){\rule{1pt}{1pt}}
\put(1159,308){\rule{1pt}{1pt}}
\put(1160,301){\rule{1pt}{1pt}}
\put(1160,298){\rule{1pt}{1pt}}
\put(1160,285){\rule{1pt}{1pt}}
\put(1161,300){\rule{1pt}{1pt}}
\put(1161,304){\rule{1pt}{1pt}}
\put(1161,284){\rule{1pt}{1pt}}
\put(1162,289){\rule{1pt}{1pt}}
\put(1162,296){\rule{1pt}{1pt}}
\put(1163,300){\rule{1pt}{1pt}}
\put(1163,300){\rule{1pt}{1pt}}
\put(1163,285){\rule{1pt}{1pt}}
\put(1164,309){\rule{1pt}{1pt}}
\put(1164,307){\rule{1pt}{1pt}}
\put(1164,295){\rule{1pt}{1pt}}
\put(1165,292){\rule{1pt}{1pt}}
\put(1165,290){\rule{1pt}{1pt}}
\put(1165,302){\rule{1pt}{1pt}}
\put(1166,294){\rule{1pt}{1pt}}
\put(1166,307){\rule{1pt}{1pt}}
\put(1167,283){\rule{1pt}{1pt}}
\put(1167,282){\rule{1pt}{1pt}}
\put(1167,302){\rule{1pt}{1pt}}
\put(1168,299){\rule{1pt}{1pt}}
\put(1168,287){\rule{1pt}{1pt}}
\put(1168,288){\rule{1pt}{1pt}}
\put(1169,302){\rule{1pt}{1pt}}
\put(1169,309){\rule{1pt}{1pt}}
\put(1169,290){\rule{1pt}{1pt}}
\put(1170,306){\rule{1pt}{1pt}}
\put(1170,287){\rule{1pt}{1pt}}
\put(1171,302){\rule{1pt}{1pt}}
\put(1171,307){\rule{1pt}{1pt}}
\put(1171,280){\rule{1pt}{1pt}}
\put(1172,303){\rule{1pt}{1pt}}
\put(1172,289){\rule{1pt}{1pt}}
\put(1172,296){\rule{1pt}{1pt}}
\put(1173,296){\rule{1pt}{1pt}}
\put(1173,297){\rule{1pt}{1pt}}
\put(1173,302){\rule{1pt}{1pt}}
\put(1174,297){\rule{1pt}{1pt}}
\put(1174,295){\rule{1pt}{1pt}}
\put(1175,288){\rule{1pt}{1pt}}
\put(1175,290){\rule{1pt}{1pt}}
\put(1175,285){\rule{1pt}{1pt}}
\put(1176,291){\rule{1pt}{1pt}}
\put(1176,301){\rule{1pt}{1pt}}
\put(1176,305){\rule{1pt}{1pt}}
\put(1177,289){\rule{1pt}{1pt}}
\put(1177,299){\rule{1pt}{1pt}}
\put(1177,306){\rule{1pt}{1pt}}
\put(1178,310){\rule{1pt}{1pt}}
\put(1178,288){\rule{1pt}{1pt}}
\put(1179,311){\rule{1pt}{1pt}}
\put(1179,300){\rule{1pt}{1pt}}
\put(1179,282){\rule{1pt}{1pt}}
\put(1180,288){\rule{1pt}{1pt}}
\put(1180,281){\rule{1pt}{1pt}}
\put(1180,309){\rule{1pt}{1pt}}
\put(1181,309){\rule{1pt}{1pt}}
\put(1181,285){\rule{1pt}{1pt}}
\put(1181,289){\rule{1pt}{1pt}}
\put(1182,297){\rule{1pt}{1pt}}
\put(1182,290){\rule{1pt}{1pt}}
\put(1183,288){\rule{1pt}{1pt}}
\put(1183,297){\rule{1pt}{1pt}}
\put(1183,304){\rule{1pt}{1pt}}
\put(1184,299){\rule{1pt}{1pt}}
\put(1184,292){\rule{1pt}{1pt}}
\put(1184,302){\rule{1pt}{1pt}}
\put(1185,294){\rule{1pt}{1pt}}
\put(1185,305){\rule{1pt}{1pt}}
\put(1185,294){\rule{1pt}{1pt}}
\put(1186,307){\rule{1pt}{1pt}}
\put(1186,302){\rule{1pt}{1pt}}
\put(1186,311){\rule{1pt}{1pt}}
\put(1187,290){\rule{1pt}{1pt}}
\put(1187,299){\rule{1pt}{1pt}}
\put(1188,302){\rule{1pt}{1pt}}
\put(1188,292){\rule{1pt}{1pt}}
\put(1188,303){\rule{1pt}{1pt}}
\put(1189,290){\rule{1pt}{1pt}}
\put(1189,281){\rule{1pt}{1pt}}
\put(1189,281){\rule{1pt}{1pt}}
\put(1190,299){\rule{1pt}{1pt}}
\put(1190,294){\rule{1pt}{1pt}}
\put(1190,290){\rule{1pt}{1pt}}
\put(1191,294){\rule{1pt}{1pt}}
\put(1191,297){\rule{1pt}{1pt}}
\put(1192,300){\rule{1pt}{1pt}}
\put(1192,311){\rule{1pt}{1pt}}
\put(1192,299){\rule{1pt}{1pt}}
\put(1193,291){\rule{1pt}{1pt}}
\put(1193,289){\rule{1pt}{1pt}}
\put(1193,293){\rule{1pt}{1pt}}
\put(1194,311){\rule{1pt}{1pt}}
\put(1194,292){\rule{1pt}{1pt}}
\put(1194,282){\rule{1pt}{1pt}}
\put(1195,309){\rule{1pt}{1pt}}
\put(1195,310){\rule{1pt}{1pt}}
\put(1196,292){\rule{1pt}{1pt}}
\put(1196,295){\rule{1pt}{1pt}}
\put(1196,284){\rule{1pt}{1pt}}
\put(1197,303){\rule{1pt}{1pt}}
\put(1197,307){\rule{1pt}{1pt}}
\put(1197,290){\rule{1pt}{1pt}}
\put(1198,295){\rule{1pt}{1pt}}
\put(1198,311){\rule{1pt}{1pt}}
\put(1198,295){\rule{1pt}{1pt}}
\put(1199,293){\rule{1pt}{1pt}}
\put(1199,312){\rule{1pt}{1pt}}
\put(1200,285){\rule{1pt}{1pt}}
\put(1200,309){\rule{1pt}{1pt}}
\put(1200,279){\rule{1pt}{1pt}}
\put(1201,305){\rule{1pt}{1pt}}
\put(1201,282){\rule{1pt}{1pt}}
\put(1201,312){\rule{1pt}{1pt}}
\put(1202,303){\rule{1pt}{1pt}}
\put(1202,305){\rule{1pt}{1pt}}
\put(1202,293){\rule{1pt}{1pt}}
\put(1203,290){\rule{1pt}{1pt}}
\put(1203,302){\rule{1pt}{1pt}}
\put(1204,288){\rule{1pt}{1pt}}
\put(1204,286){\rule{1pt}{1pt}}
\put(1204,279){\rule{1pt}{1pt}}
\put(1205,302){\rule{1pt}{1pt}}
\put(1205,311){\rule{1pt}{1pt}}
\put(1205,299){\rule{1pt}{1pt}}
\put(1206,284){\rule{1pt}{1pt}}
\put(1206,280){\rule{1pt}{1pt}}
\put(1206,305){\rule{1pt}{1pt}}
\put(1207,280){\rule{1pt}{1pt}}
\put(1207,293){\rule{1pt}{1pt}}
\put(1207,288){\rule{1pt}{1pt}}
\put(1208,286){\rule{1pt}{1pt}}
\put(1208,306){\rule{1pt}{1pt}}
\put(1209,295){\rule{1pt}{1pt}}
\put(1209,297){\rule{1pt}{1pt}}
\put(1209,279){\rule{1pt}{1pt}}
\put(1210,285){\rule{1pt}{1pt}}
\put(1210,312){\rule{1pt}{1pt}}
\put(1210,284){\rule{1pt}{1pt}}
\put(1211,278){\rule{1pt}{1pt}}
\put(1211,278){\rule{1pt}{1pt}}
\put(1211,298){\rule{1pt}{1pt}}
\put(1212,297){\rule{1pt}{1pt}}
\put(1212,307){\rule{1pt}{1pt}}
\put(1213,302){\rule{1pt}{1pt}}
\put(1213,289){\rule{1pt}{1pt}}
\put(1213,292){\rule{1pt}{1pt}}
\put(1214,293){\rule{1pt}{1pt}}
\put(1214,296){\rule{1pt}{1pt}}
\put(1214,288){\rule{1pt}{1pt}}
\put(1215,278){\rule{1pt}{1pt}}
\put(1215,311){\rule{1pt}{1pt}}
\put(1215,312){\rule{1pt}{1pt}}
\put(1216,278){\rule{1pt}{1pt}}
\put(1216,311){\rule{1pt}{1pt}}
\put(1217,293){\rule{1pt}{1pt}}
\put(1217,280){\rule{1pt}{1pt}}
\put(1217,307){\rule{1pt}{1pt}}
\put(1218,298){\rule{1pt}{1pt}}
\put(1218,287){\rule{1pt}{1pt}}
\put(1218,284){\rule{1pt}{1pt}}
\put(1219,294){\rule{1pt}{1pt}}
\put(1219,305){\rule{1pt}{1pt}}
\put(1219,299){\rule{1pt}{1pt}}
\put(1220,306){\rule{1pt}{1pt}}
\put(1220,278){\rule{1pt}{1pt}}
\put(1221,282){\rule{1pt}{1pt}}
\put(1221,302){\rule{1pt}{1pt}}
\put(1221,307){\rule{1pt}{1pt}}
\put(1222,311){\rule{1pt}{1pt}}
\put(1222,288){\rule{1pt}{1pt}}
\put(1222,293){\rule{1pt}{1pt}}
\put(1223,297){\rule{1pt}{1pt}}
\put(1223,287){\rule{1pt}{1pt}}
\put(1223,302){\rule{1pt}{1pt}}
\put(1224,287){\rule{1pt}{1pt}}
\put(1224,306){\rule{1pt}{1pt}}
\put(1225,277){\rule{1pt}{1pt}}
\put(1225,300){\rule{1pt}{1pt}}
\put(1225,295){\rule{1pt}{1pt}}
\put(1226,279){\rule{1pt}{1pt}}
\put(1226,295){\rule{1pt}{1pt}}
\put(1226,309){\rule{1pt}{1pt}}
\put(1227,280){\rule{1pt}{1pt}}
\put(1227,309){\rule{1pt}{1pt}}
\put(1227,296){\rule{1pt}{1pt}}
\put(1228,297){\rule{1pt}{1pt}}
\put(1228,284){\rule{1pt}{1pt}}
\put(1229,304){\rule{1pt}{1pt}}
\put(1229,291){\rule{1pt}{1pt}}
\put(1229,307){\rule{1pt}{1pt}}
\put(1230,297){\rule{1pt}{1pt}}
\put(1230,304){\rule{1pt}{1pt}}
\put(1230,296){\rule{1pt}{1pt}}
\put(1231,295){\rule{1pt}{1pt}}
\put(1231,307){\rule{1pt}{1pt}}
\put(1231,287){\rule{1pt}{1pt}}
\put(1232,310){\rule{1pt}{1pt}}
\put(1232,286){\rule{1pt}{1pt}}
\put(1232,299){\rule{1pt}{1pt}}
\put(1233,294){\rule{1pt}{1pt}}
\put(1233,293){\rule{1pt}{1pt}}
\put(1234,307){\rule{1pt}{1pt}}
\put(1234,295){\rule{1pt}{1pt}}
\put(1234,281){\rule{1pt}{1pt}}
\put(1235,282){\rule{1pt}{1pt}}
\put(1235,300){\rule{1pt}{1pt}}
\put(1235,291){\rule{1pt}{1pt}}
\put(1236,308){\rule{1pt}{1pt}}
\put(1236,286){\rule{1pt}{1pt}}
\put(1236,288){\rule{1pt}{1pt}}
\put(1237,278){\rule{1pt}{1pt}}
\put(1237,290){\rule{1pt}{1pt}}
\put(1238,299){\rule{1pt}{1pt}}
\put(1238,278){\rule{1pt}{1pt}}
\put(1238,304){\rule{1pt}{1pt}}
\put(1239,290){\rule{1pt}{1pt}}
\put(1239,307){\rule{1pt}{1pt}}
\put(1239,308){\rule{1pt}{1pt}}
\put(1240,284){\rule{1pt}{1pt}}
\put(1240,310){\rule{1pt}{1pt}}
\put(1240,287){\rule{1pt}{1pt}}
\put(1241,307){\rule{1pt}{1pt}}
\put(1241,303){\rule{1pt}{1pt}}
\put(1242,295){\rule{1pt}{1pt}}
\put(1242,300){\rule{1pt}{1pt}}
\put(1242,283){\rule{1pt}{1pt}}
\put(1243,282){\rule{1pt}{1pt}}
\put(1243,291){\rule{1pt}{1pt}}
\put(1243,307){\rule{1pt}{1pt}}
\put(1244,299){\rule{1pt}{1pt}}
\put(1244,284){\rule{1pt}{1pt}}
\put(1244,280){\rule{1pt}{1pt}}
\put(1245,290){\rule{1pt}{1pt}}
\put(1245,283){\rule{1pt}{1pt}}
\put(1246,309){\rule{1pt}{1pt}}
\put(1246,293){\rule{1pt}{1pt}}
\put(1246,283){\rule{1pt}{1pt}}
\put(1247,282){\rule{1pt}{1pt}}
\put(1247,301){\rule{1pt}{1pt}}
\put(1247,302){\rule{1pt}{1pt}}
\put(1248,289){\rule{1pt}{1pt}}
\put(1248,293){\rule{1pt}{1pt}}
\put(1248,293){\rule{1pt}{1pt}}
\put(1249,306){\rule{1pt}{1pt}}
\put(1249,299){\rule{1pt}{1pt}}
\put(1250,280){\rule{1pt}{1pt}}
\put(1250,285){\rule{1pt}{1pt}}
\put(1250,286){\rule{1pt}{1pt}}
\put(1251,307){\rule{1pt}{1pt}}
\put(1251,289){\rule{1pt}{1pt}}
\put(1251,290){\rule{1pt}{1pt}}
\put(1252,278){\rule{1pt}{1pt}}
\put(1252,280){\rule{1pt}{1pt}}
\put(1252,294){\rule{1pt}{1pt}}
\put(1253,297){\rule{1pt}{1pt}}
\put(1253,305){\rule{1pt}{1pt}}
\put(1254,305){\rule{1pt}{1pt}}
\put(1254,293){\rule{1pt}{1pt}}
\put(1254,294){\rule{1pt}{1pt}}
\put(1255,282){\rule{1pt}{1pt}}
\put(1255,309){\rule{1pt}{1pt}}
\put(1255,284){\rule{1pt}{1pt}}
\put(1256,281){\rule{1pt}{1pt}}
\put(1256,295){\rule{1pt}{1pt}}
\put(1256,306){\rule{1pt}{1pt}}
\put(1257,299){\rule{1pt}{1pt}}
\put(1257,284){\rule{1pt}{1pt}}
\put(1257,291){\rule{1pt}{1pt}}
\put(1258,294){\rule{1pt}{1pt}}
\put(1258,310){\rule{1pt}{1pt}}
\put(1259,288){\rule{1pt}{1pt}}
\put(1259,302){\rule{1pt}{1pt}}
\put(1259,290){\rule{1pt}{1pt}}
\put(1260,293){\rule{1pt}{1pt}}
\put(1260,289){\rule{1pt}{1pt}}
\put(1260,300){\rule{1pt}{1pt}}
\put(1261,312){\rule{1pt}{1pt}}
\put(1261,307){\rule{1pt}{1pt}}
\put(1261,305){\rule{1pt}{1pt}}
\put(1262,280){\rule{1pt}{1pt}}
\put(1262,279){\rule{1pt}{1pt}}
\put(1263,278){\rule{1pt}{1pt}}
\put(1263,279){\rule{1pt}{1pt}}
\put(1263,310){\rule{1pt}{1pt}}
\put(1264,311){\rule{1pt}{1pt}}
\put(1264,296){\rule{1pt}{1pt}}
\put(1264,299){\rule{1pt}{1pt}}
\put(1265,291){\rule{1pt}{1pt}}
\put(1265,302){\rule{1pt}{1pt}}
\put(1265,306){\rule{1pt}{1pt}}
\put(1266,300){\rule{1pt}{1pt}}
\put(1266,301){\rule{1pt}{1pt}}
\put(1267,282){\rule{1pt}{1pt}}
\put(1267,281){\rule{1pt}{1pt}}
\put(1267,296){\rule{1pt}{1pt}}
\put(1268,299){\rule{1pt}{1pt}}
\put(1268,288){\rule{1pt}{1pt}}
\put(1268,310){\rule{1pt}{1pt}}
\put(1269,313){\rule{1pt}{1pt}}
\put(1269,306){\rule{1pt}{1pt}}
\put(1269,311){\rule{1pt}{1pt}}
\put(1270,286){\rule{1pt}{1pt}}
\put(1270,292){\rule{1pt}{1pt}}
\put(1271,305){\rule{1pt}{1pt}}
\put(1271,302){\rule{1pt}{1pt}}
\put(1271,279){\rule{1pt}{1pt}}
\put(1272,289){\rule{1pt}{1pt}}
\put(1272,294){\rule{1pt}{1pt}}
\put(1272,295){\rule{1pt}{1pt}}
\put(1273,295){\rule{1pt}{1pt}}
\put(1273,287){\rule{1pt}{1pt}}
\put(1273,299){\rule{1pt}{1pt}}
\put(1274,284){\rule{1pt}{1pt}}
\put(1274,307){\rule{1pt}{1pt}}
\put(1275,292){\rule{1pt}{1pt}}
\put(1275,279){\rule{1pt}{1pt}}
\put(1275,280){\rule{1pt}{1pt}}
\put(1276,302){\rule{1pt}{1pt}}
\put(1276,303){\rule{1pt}{1pt}}
\put(1276,303){\rule{1pt}{1pt}}
\put(1277,302){\rule{1pt}{1pt}}
\put(1277,302){\rule{1pt}{1pt}}
\put(1277,291){\rule{1pt}{1pt}}
\put(1278,304){\rule{1pt}{1pt}}
\put(1278,288){\rule{1pt}{1pt}}
\put(1279,311){\rule{1pt}{1pt}}
\put(1279,303){\rule{1pt}{1pt}}
\put(1279,283){\rule{1pt}{1pt}}
\put(1280,308){\rule{1pt}{1pt}}
\put(1280,287){\rule{1pt}{1pt}}
\put(1280,310){\rule{1pt}{1pt}}
\put(1281,292){\rule{1pt}{1pt}}
\put(1281,306){\rule{1pt}{1pt}}
\put(1281,294){\rule{1pt}{1pt}}
\put(1282,293){\rule{1pt}{1pt}}
\put(1282,297){\rule{1pt}{1pt}}
\put(1282,292){\rule{1pt}{1pt}}
\put(1283,299){\rule{1pt}{1pt}}
\put(1283,278){\rule{1pt}{1pt}}
\put(1284,294){\rule{1pt}{1pt}}
\put(1284,299){\rule{1pt}{1pt}}
\put(1284,285){\rule{1pt}{1pt}}
\put(1285,289){\rule{1pt}{1pt}}
\put(1285,306){\rule{1pt}{1pt}}
\put(1285,301){\rule{1pt}{1pt}}
\put(1286,304){\rule{1pt}{1pt}}
\put(1286,281){\rule{1pt}{1pt}}
\put(1286,291){\rule{1pt}{1pt}}
\put(1287,292){\rule{1pt}{1pt}}
\put(1287,295){\rule{1pt}{1pt}}
\put(1288,309){\rule{1pt}{1pt}}
\put(1288,297){\rule{1pt}{1pt}}
\put(1288,289){\rule{1pt}{1pt}}
\put(1289,310){\rule{1pt}{1pt}}
\put(1289,297){\rule{1pt}{1pt}}
\put(1289,306){\rule{1pt}{1pt}}
\put(1290,283){\rule{1pt}{1pt}}
\put(1290,295){\rule{1pt}{1pt}}
\put(1290,295){\rule{1pt}{1pt}}
\put(1291,288){\rule{1pt}{1pt}}
\put(1291,282){\rule{1pt}{1pt}}
\put(1292,296){\rule{1pt}{1pt}}
\put(1292,283){\rule{1pt}{1pt}}
\put(1292,307){\rule{1pt}{1pt}}
\put(1293,287){\rule{1pt}{1pt}}
\put(1293,288){\rule{1pt}{1pt}}
\put(1293,300){\rule{1pt}{1pt}}
\put(1294,296){\rule{1pt}{1pt}}
\put(1294,300){\rule{1pt}{1pt}}
\put(1294,310){\rule{1pt}{1pt}}
\put(1295,296){\rule{1pt}{1pt}}
\put(1295,312){\rule{1pt}{1pt}}
\put(1296,283){\rule{1pt}{1pt}}
\put(1296,302){\rule{1pt}{1pt}}
\put(1296,291){\rule{1pt}{1pt}}
\put(1297,292){\rule{1pt}{1pt}}
\put(1297,302){\rule{1pt}{1pt}}
\put(1297,306){\rule{1pt}{1pt}}
\put(1298,285){\rule{1pt}{1pt}}
\put(1298,281){\rule{1pt}{1pt}}
\put(1298,301){\rule{1pt}{1pt}}
\put(1299,299){\rule{1pt}{1pt}}
\put(1299,309){\rule{1pt}{1pt}}
\put(1300,298){\rule{1pt}{1pt}}
\put(1300,305){\rule{1pt}{1pt}}
\put(1300,301){\rule{1pt}{1pt}}
\put(1301,284){\rule{1pt}{1pt}}
\put(1301,310){\rule{1pt}{1pt}}
\put(1301,306){\rule{1pt}{1pt}}
\put(1302,305){\rule{1pt}{1pt}}
\put(1302,284){\rule{1pt}{1pt}}
\put(1302,299){\rule{1pt}{1pt}}
\put(1303,284){\rule{1pt}{1pt}}
\put(1303,287){\rule{1pt}{1pt}}
\put(1304,308){\rule{1pt}{1pt}}
\put(1304,311){\rule{1pt}{1pt}}
\put(1304,285){\rule{1pt}{1pt}}
\put(1305,308){\rule{1pt}{1pt}}
\put(1305,284){\rule{1pt}{1pt}}
\put(1305,306){\rule{1pt}{1pt}}
\put(1306,311){\rule{1pt}{1pt}}
\put(1306,303){\rule{1pt}{1pt}}
\put(1306,312){\rule{1pt}{1pt}}
\put(1307,298){\rule{1pt}{1pt}}
\put(1307,309){\rule{1pt}{1pt}}
\put(1307,279){\rule{1pt}{1pt}}
\put(1308,278){\rule{1pt}{1pt}}
\put(1308,311){\rule{1pt}{1pt}}
\put(1309,283){\rule{1pt}{1pt}}
\put(1309,306){\rule{1pt}{1pt}}
\put(1309,311){\rule{1pt}{1pt}}
\put(1310,292){\rule{1pt}{1pt}}
\put(1310,279){\rule{1pt}{1pt}}
\put(1310,293){\rule{1pt}{1pt}}
\put(1311,305){\rule{1pt}{1pt}}
\put(1311,313){\rule{1pt}{1pt}}
\put(1311,303){\rule{1pt}{1pt}}
\put(1312,297){\rule{1pt}{1pt}}
\put(1312,289){\rule{1pt}{1pt}}
\put(1313,282){\rule{1pt}{1pt}}
\put(1313,287){\rule{1pt}{1pt}}
\put(1313,302){\rule{1pt}{1pt}}
\put(1314,289){\rule{1pt}{1pt}}
\put(1314,292){\rule{1pt}{1pt}}
\put(1314,292){\rule{1pt}{1pt}}
\put(1315,302){\rule{1pt}{1pt}}
\put(1315,299){\rule{1pt}{1pt}}
\put(1315,299){\rule{1pt}{1pt}}
\put(1316,309){\rule{1pt}{1pt}}
\put(1316,293){\rule{1pt}{1pt}}
\put(1317,279){\rule{1pt}{1pt}}
\put(1317,280){\rule{1pt}{1pt}}
\put(1317,313){\rule{1pt}{1pt}}
\put(1318,302){\rule{1pt}{1pt}}
\put(1318,306){\rule{1pt}{1pt}}
\put(1318,290){\rule{1pt}{1pt}}
\put(1319,310){\rule{1pt}{1pt}}
\put(1319,291){\rule{1pt}{1pt}}
\put(1319,289){\rule{1pt}{1pt}}
\put(1320,285){\rule{1pt}{1pt}}
\put(1320,303){\rule{1pt}{1pt}}
\put(1321,279){\rule{1pt}{1pt}}
\put(1321,297){\rule{1pt}{1pt}}
\put(1321,278){\rule{1pt}{1pt}}
\put(1322,299){\rule{1pt}{1pt}}
\put(1322,309){\rule{1pt}{1pt}}
\put(1322,309){\rule{1pt}{1pt}}
\put(1323,311){\rule{1pt}{1pt}}
\put(1323,282){\rule{1pt}{1pt}}
\put(1323,288){\rule{1pt}{1pt}}
\put(1324,301){\rule{1pt}{1pt}}
\put(1324,297){\rule{1pt}{1pt}}
\put(1325,306){\rule{1pt}{1pt}}
\put(1325,311){\rule{1pt}{1pt}}
\put(1325,288){\rule{1pt}{1pt}}
\put(1326,303){\rule{1pt}{1pt}}
\put(1326,309){\rule{1pt}{1pt}}
\put(1326,303){\rule{1pt}{1pt}}
\put(1327,312){\rule{1pt}{1pt}}
\put(1327,278){\rule{1pt}{1pt}}
\put(1327,310){\rule{1pt}{1pt}}
\put(1328,309){\rule{1pt}{1pt}}
\put(1328,308){\rule{1pt}{1pt}}
\put(1329,310){\rule{1pt}{1pt}}
\put(1329,282){\rule{1pt}{1pt}}
\put(1329,297){\rule{1pt}{1pt}}
\put(1330,304){\rule{1pt}{1pt}}
\put(1330,307){\rule{1pt}{1pt}}
\put(1330,279){\rule{1pt}{1pt}}
\put(1331,301){\rule{1pt}{1pt}}
\put(1331,281){\rule{1pt}{1pt}}
\put(1331,305){\rule{1pt}{1pt}}
\put(1332,308){\rule{1pt}{1pt}}
\put(1332,300){\rule{1pt}{1pt}}
\put(1332,292){\rule{1pt}{1pt}}
\put(1333,286){\rule{1pt}{1pt}}
\put(1333,286){\rule{1pt}{1pt}}
\put(1334,297){\rule{1pt}{1pt}}
\put(1334,295){\rule{1pt}{1pt}}
\put(1334,291){\rule{1pt}{1pt}}
\put(1335,290){\rule{1pt}{1pt}}
\put(1335,291){\rule{1pt}{1pt}}
\put(1335,283){\rule{1pt}{1pt}}
\put(1336,285){\rule{1pt}{1pt}}
\put(1336,308){\rule{1pt}{1pt}}
\put(1336,299){\rule{1pt}{1pt}}
\put(1337,285){\rule{1pt}{1pt}}
\put(1337,293){\rule{1pt}{1pt}}
\put(1338,300){\rule{1pt}{1pt}}
\put(1338,302){\rule{1pt}{1pt}}
\put(1338,290){\rule{1pt}{1pt}}
\put(1339,313){\rule{1pt}{1pt}}
\put(1339,287){\rule{1pt}{1pt}}
\put(1339,309){\rule{1pt}{1pt}}
\put(1340,306){\rule{1pt}{1pt}}
\put(1340,308){\rule{1pt}{1pt}}
\put(1340,312){\rule{1pt}{1pt}}
\put(1341,279){\rule{1pt}{1pt}}
\put(1341,312){\rule{1pt}{1pt}}
\put(1342,294){\rule{1pt}{1pt}}
\put(1342,300){\rule{1pt}{1pt}}
\put(1342,294){\rule{1pt}{1pt}}
\put(1343,290){\rule{1pt}{1pt}}
\put(1343,281){\rule{1pt}{1pt}}
\put(1343,290){\rule{1pt}{1pt}}
\put(1344,283){\rule{1pt}{1pt}}
\put(1344,280){\rule{1pt}{1pt}}
\put(1344,307){\rule{1pt}{1pt}}
\put(1345,283){\rule{1pt}{1pt}}
\put(1345,278){\rule{1pt}{1pt}}
\put(1346,285){\rule{1pt}{1pt}}
\put(1346,305){\rule{1pt}{1pt}}
\put(1346,282){\rule{1pt}{1pt}}
\put(1347,279){\rule{1pt}{1pt}}
\put(1347,307){\rule{1pt}{1pt}}
\put(1347,284){\rule{1pt}{1pt}}
\put(1348,308){\rule{1pt}{1pt}}
\put(1348,311){\rule{1pt}{1pt}}
\put(1348,282){\rule{1pt}{1pt}}
\put(1349,295){\rule{1pt}{1pt}}
\put(1349,293){\rule{1pt}{1pt}}
\put(1350,285){\rule{1pt}{1pt}}
\put(1350,280){\rule{1pt}{1pt}}
\put(1350,301){\rule{1pt}{1pt}}
\put(1351,295){\rule{1pt}{1pt}}
\put(1351,299){\rule{1pt}{1pt}}
\put(1351,284){\rule{1pt}{1pt}}
\put(1352,308){\rule{1pt}{1pt}}
\put(1352,295){\rule{1pt}{1pt}}
\put(1352,281){\rule{1pt}{1pt}}
\put(1353,294){\rule{1pt}{1pt}}
\put(1353,280){\rule{1pt}{1pt}}
\put(1354,300){\rule{1pt}{1pt}}
\put(1354,290){\rule{1pt}{1pt}}
\put(1354,300){\rule{1pt}{1pt}}
\put(1355,291){\rule{1pt}{1pt}}
\put(1355,279){\rule{1pt}{1pt}}
\put(1355,299){\rule{1pt}{1pt}}
\put(1356,286){\rule{1pt}{1pt}}
\put(1356,280){\rule{1pt}{1pt}}
\put(1356,310){\rule{1pt}{1pt}}
\put(1357,289){\rule{1pt}{1pt}}
\put(1357,277){\rule{1pt}{1pt}}
\put(1357,307){\rule{1pt}{1pt}}
\put(1358,293){\rule{1pt}{1pt}}
\put(1358,278){\rule{1pt}{1pt}}
\put(1359,287){\rule{1pt}{1pt}}
\put(1359,307){\rule{1pt}{1pt}}
\put(1359,286){\rule{1pt}{1pt}}
\put(1360,278){\rule{1pt}{1pt}}
\put(1360,296){\rule{1pt}{1pt}}
\put(1360,283){\rule{1pt}{1pt}}
\put(1361,311){\rule{1pt}{1pt}}
\put(1361,282){\rule{1pt}{1pt}}
\put(1361,293){\rule{1pt}{1pt}}
\put(1362,306){\rule{1pt}{1pt}}
\put(1362,278){\rule{1pt}{1pt}}
\put(1363,281){\rule{1pt}{1pt}}
\put(1363,306){\rule{1pt}{1pt}}
\put(1363,298){\rule{1pt}{1pt}}
\put(1364,278){\rule{1pt}{1pt}}
\put(1364,295){\rule{1pt}{1pt}}
\put(1364,295){\rule{1pt}{1pt}}
\put(1365,313){\rule{1pt}{1pt}}
\put(1365,278){\rule{1pt}{1pt}}
\put(1365,284){\rule{1pt}{1pt}}
\put(1366,287){\rule{1pt}{1pt}}
\put(1366,298){\rule{1pt}{1pt}}
\put(1366.0,287.0){\rule[-0.200pt]{0.400pt}{2.650pt}}
\put(171.0,131.0){\rule[-0.200pt]{0.400pt}{175.375pt}}
\put(171.0,131.0){\rule[-0.200pt]{305.461pt}{0.400pt}}
\put(1439.0,131.0){\rule[-0.200pt]{0.400pt}{175.375pt}}
\put(171.0,859.0){\rule[-0.200pt]{305.461pt}{0.400pt}}
\end{picture}

    \caption{Typical learning curve for an MLP}
    \label{fig:learning}
\end{figure}

\section{Assignment 25}

The differences in each of these graphs show us that speed and accuracy of
learning can be controlled by scaling the axes. In the figure \ref{fig:b} we
can observe that when the axe X is logarithimic scaled won't be easy to reach 
the negative gradient. 

\begin{figure}[p]

        \centering
        % GNUPLOT: LaTeX picture
\setlength{\unitlength}{0.240900pt}
\ifx\plotpoint\undefined\newsavebox{\plotpoint}\fi
\sbox{\plotpoint}{\rule[-0.200pt]{0.400pt}{0.400pt}}%
\begin{picture}(1500,900)(0,0)
\sbox{\plotpoint}{\rule[-0.200pt]{0.400pt}{0.400pt}}%
\put(151.0,131.0){\rule[-0.200pt]{4.818pt}{0.400pt}}
\put(131,131){\makebox(0,0)[r]{ 0}}
\put(1419.0,131.0){\rule[-0.200pt]{4.818pt}{0.400pt}}
\put(151.0,276.0){\rule[-0.200pt]{4.818pt}{0.400pt}}
\put(131,276){\makebox(0,0)[r]{ 5}}
\put(1419.0,276.0){\rule[-0.200pt]{4.818pt}{0.400pt}}
\put(151.0,422.0){\rule[-0.200pt]{4.818pt}{0.400pt}}
\put(131,422){\makebox(0,0)[r]{ 10}}
\put(1419.0,422.0){\rule[-0.200pt]{4.818pt}{0.400pt}}
\put(151.0,568.0){\rule[-0.200pt]{4.818pt}{0.400pt}}
\put(131,568){\makebox(0,0)[r]{ 15}}
\put(1419.0,568.0){\rule[-0.200pt]{4.818pt}{0.400pt}}
\put(151.0,713.0){\rule[-0.200pt]{4.818pt}{0.400pt}}
\put(131,713){\makebox(0,0)[r]{ 20}}
\put(1419.0,713.0){\rule[-0.200pt]{4.818pt}{0.400pt}}
\put(151.0,859.0){\rule[-0.200pt]{4.818pt}{0.400pt}}
\put(131,859){\makebox(0,0)[r]{ 25}}
\put(1419.0,859.0){\rule[-0.200pt]{4.818pt}{0.400pt}}
\put(150.0,131.0){\rule[-0.200pt]{0.400pt}{4.818pt}}
\put(150,90){\makebox(0,0){ 0}}
\put(150.0,839.0){\rule[-0.200pt]{0.400pt}{4.818pt}}
\put(279.0,131.0){\rule[-0.200pt]{0.400pt}{4.818pt}}
\put(279,90){\makebox(0,0){ 1}}
\put(279.0,839.0){\rule[-0.200pt]{0.400pt}{4.818pt}}
\put(408.0,131.0){\rule[-0.200pt]{0.400pt}{4.818pt}}
\put(408,90){\makebox(0,0){ 2}}
\put(408.0,839.0){\rule[-0.200pt]{0.400pt}{4.818pt}}
\put(536.0,131.0){\rule[-0.200pt]{0.400pt}{4.818pt}}
\put(536,90){\makebox(0,0){ 3}}
\put(536.0,839.0){\rule[-0.200pt]{0.400pt}{4.818pt}}
\put(665.0,131.0){\rule[-0.200pt]{0.400pt}{4.818pt}}
\put(665,90){\makebox(0,0){ 4}}
\put(665.0,839.0){\rule[-0.200pt]{0.400pt}{4.818pt}}
\put(794.0,131.0){\rule[-0.200pt]{0.400pt}{4.818pt}}
\put(794,90){\makebox(0,0){ 5}}
\put(794.0,839.0){\rule[-0.200pt]{0.400pt}{4.818pt}}
\put(923.0,131.0){\rule[-0.200pt]{0.400pt}{4.818pt}}
\put(923,90){\makebox(0,0){ 6}}
\put(923.0,839.0){\rule[-0.200pt]{0.400pt}{4.818pt}}
\put(1052.0,131.0){\rule[-0.200pt]{0.400pt}{4.818pt}}
\put(1052,90){\makebox(0,0){ 7}}
\put(1052.0,839.0){\rule[-0.200pt]{0.400pt}{4.818pt}}
\put(1181.0,131.0){\rule[-0.200pt]{0.400pt}{4.818pt}}
\put(1181,90){\makebox(0,0){ 8}}
\put(1181.0,839.0){\rule[-0.200pt]{0.400pt}{4.818pt}}
\put(1310.0,131.0){\rule[-0.200pt]{0.400pt}{4.818pt}}
\put(1310,90){\makebox(0,0){ 9}}
\put(1310.0,839.0){\rule[-0.200pt]{0.400pt}{4.818pt}}
\put(1439.0,131.0){\rule[-0.200pt]{0.400pt}{4.818pt}}
\put(1439,90){\makebox(0,0){ 10}}
\put(1439.0,839.0){\rule[-0.200pt]{0.400pt}{4.818pt}}
\put(151.0,131.0){\rule[-0.200pt]{0.400pt}{175.375pt}}
\put(151.0,131.0){\rule[-0.200pt]{310.279pt}{0.400pt}}
\put(1439.0,131.0){\rule[-0.200pt]{0.400pt}{175.375pt}}
\put(151.0,859.0){\rule[-0.200pt]{310.279pt}{0.400pt}}
\put(30,495){\makebox(0,0){y}}
\put(795,29){\makebox(0,0){x}}
\put(1279,819){\makebox(0,0)[r]{2**( 2 - x + 4 * sin(x) )}}
\put(1299.0,819.0){\rule[-0.200pt]{24.090pt}{0.400pt}}
\put(151,250){\usebox{\plotpoint}}
\multiput(151.58,250.00)(0.493,1.052){23}{\rule{0.119pt}{0.931pt}}
\multiput(150.17,250.00)(13.000,25.068){2}{\rule{0.400pt}{0.465pt}}
\multiput(164.58,277.00)(0.493,1.329){23}{\rule{0.119pt}{1.146pt}}
\multiput(163.17,277.00)(13.000,31.621){2}{\rule{0.400pt}{0.573pt}}
\multiput(177.58,311.00)(0.493,1.567){23}{\rule{0.119pt}{1.331pt}}
\multiput(176.17,311.00)(13.000,37.238){2}{\rule{0.400pt}{0.665pt}}
\multiput(190.58,351.00)(0.493,1.805){23}{\rule{0.119pt}{1.515pt}}
\multiput(189.17,351.00)(13.000,42.855){2}{\rule{0.400pt}{0.758pt}}
\multiput(203.58,397.00)(0.493,2.083){23}{\rule{0.119pt}{1.731pt}}
\multiput(202.17,397.00)(13.000,49.408){2}{\rule{0.400pt}{0.865pt}}
\multiput(216.58,450.00)(0.493,2.281){23}{\rule{0.119pt}{1.885pt}}
\multiput(215.17,450.00)(13.000,54.088){2}{\rule{0.400pt}{0.942pt}}
\multiput(229.58,508.00)(0.493,2.400){23}{\rule{0.119pt}{1.977pt}}
\multiput(228.17,508.00)(13.000,56.897){2}{\rule{0.400pt}{0.988pt}}
\multiput(242.58,569.00)(0.493,2.400){23}{\rule{0.119pt}{1.977pt}}
\multiput(241.17,569.00)(13.000,56.897){2}{\rule{0.400pt}{0.988pt}}
\multiput(255.58,630.00)(0.493,2.320){23}{\rule{0.119pt}{1.915pt}}
\multiput(254.17,630.00)(13.000,55.025){2}{\rule{0.400pt}{0.958pt}}
\multiput(268.58,689.00)(0.493,2.003){23}{\rule{0.119pt}{1.669pt}}
\multiput(267.17,689.00)(13.000,47.535){2}{\rule{0.400pt}{0.835pt}}
\multiput(281.58,740.00)(0.493,1.607){23}{\rule{0.119pt}{1.362pt}}
\multiput(280.17,740.00)(13.000,38.174){2}{\rule{0.400pt}{0.681pt}}
\multiput(294.58,781.00)(0.493,1.012){23}{\rule{0.119pt}{0.900pt}}
\multiput(293.17,781.00)(13.000,24.132){2}{\rule{0.400pt}{0.450pt}}
\multiput(307.00,807.59)(0.824,0.488){13}{\rule{0.750pt}{0.117pt}}
\multiput(307.00,806.17)(11.443,8.000){2}{\rule{0.375pt}{0.400pt}}
\multiput(320.00,813.92)(0.652,-0.491){17}{\rule{0.620pt}{0.118pt}}
\multiput(320.00,814.17)(11.713,-10.000){2}{\rule{0.310pt}{0.400pt}}
\multiput(333.58,801.01)(0.493,-1.091){23}{\rule{0.119pt}{0.962pt}}
\multiput(332.17,803.00)(13.000,-26.004){2}{\rule{0.400pt}{0.481pt}}
\multiput(346.58,770.96)(0.493,-1.726){23}{\rule{0.119pt}{1.454pt}}
\multiput(345.17,773.98)(13.000,-40.982){2}{\rule{0.400pt}{0.727pt}}
\multiput(359.58,725.30)(0.493,-2.241){23}{\rule{0.119pt}{1.854pt}}
\multiput(358.17,729.15)(13.000,-53.152){2}{\rule{0.400pt}{0.927pt}}
\multiput(372.58,667.28)(0.493,-2.558){23}{\rule{0.119pt}{2.100pt}}
\multiput(371.17,671.64)(13.000,-60.641){2}{\rule{0.400pt}{1.050pt}}
\multiput(385.58,601.90)(0.493,-2.677){23}{\rule{0.119pt}{2.192pt}}
\multiput(384.17,606.45)(13.000,-63.450){2}{\rule{0.400pt}{1.096pt}}
\multiput(398.58,533.77)(0.493,-2.717){23}{\rule{0.119pt}{2.223pt}}
\multiput(397.17,538.39)(13.000,-64.386){2}{\rule{0.400pt}{1.112pt}}
\multiput(411.58,465.41)(0.493,-2.519){23}{\rule{0.119pt}{2.069pt}}
\multiput(410.17,469.71)(13.000,-59.705){2}{\rule{0.400pt}{1.035pt}}
\multiput(424.58,402.30)(0.493,-2.241){23}{\rule{0.119pt}{1.854pt}}
\multiput(423.17,406.15)(13.000,-53.152){2}{\rule{0.400pt}{0.927pt}}
\multiput(437.58,346.20)(0.493,-1.964){23}{\rule{0.119pt}{1.638pt}}
\multiput(436.17,349.60)(13.000,-46.599){2}{\rule{0.400pt}{0.819pt}}
\multiput(450.58,297.35)(0.493,-1.607){23}{\rule{0.119pt}{1.362pt}}
\multiput(449.17,300.17)(13.000,-38.174){2}{\rule{0.400pt}{0.681pt}}
\multiput(463.58,257.37)(0.493,-1.290){23}{\rule{0.119pt}{1.115pt}}
\multiput(462.17,259.68)(13.000,-30.685){2}{\rule{0.400pt}{0.558pt}}
\multiput(476.58,225.26)(0.493,-1.012){23}{\rule{0.119pt}{0.900pt}}
\multiput(475.17,227.13)(13.000,-24.132){2}{\rule{0.400pt}{0.450pt}}
\multiput(489.58,200.03)(0.493,-0.774){23}{\rule{0.119pt}{0.715pt}}
\multiput(488.17,201.52)(13.000,-18.515){2}{\rule{0.400pt}{0.358pt}}
\multiput(502.58,180.67)(0.493,-0.576){23}{\rule{0.119pt}{0.562pt}}
\multiput(501.17,181.83)(13.000,-13.834){2}{\rule{0.400pt}{0.281pt}}
\multiput(515.00,166.92)(0.590,-0.492){19}{\rule{0.573pt}{0.118pt}}
\multiput(515.00,167.17)(11.811,-11.000){2}{\rule{0.286pt}{0.400pt}}
\multiput(528.00,155.93)(0.950,-0.485){11}{\rule{0.843pt}{0.117pt}}
\multiput(528.00,156.17)(11.251,-7.000){2}{\rule{0.421pt}{0.400pt}}
\multiput(541.00,148.93)(1.123,-0.482){9}{\rule{0.967pt}{0.116pt}}
\multiput(541.00,149.17)(10.994,-6.000){2}{\rule{0.483pt}{0.400pt}}
\multiput(554.00,142.94)(1.797,-0.468){5}{\rule{1.400pt}{0.113pt}}
\multiput(554.00,143.17)(10.094,-4.000){2}{\rule{0.700pt}{0.400pt}}
\multiput(567.00,138.95)(2.695,-0.447){3}{\rule{1.833pt}{0.108pt}}
\multiput(567.00,139.17)(9.195,-3.000){2}{\rule{0.917pt}{0.400pt}}
\put(580,135.17){\rule{2.700pt}{0.400pt}}
\multiput(580.00,136.17)(7.396,-2.000){2}{\rule{1.350pt}{0.400pt}}
\put(593,133.67){\rule{3.132pt}{0.400pt}}
\multiput(593.00,134.17)(6.500,-1.000){2}{\rule{1.566pt}{0.400pt}}
\put(606,132.67){\rule{3.132pt}{0.400pt}}
\multiput(606.00,133.17)(6.500,-1.000){2}{\rule{1.566pt}{0.400pt}}
\put(632,131.67){\rule{3.132pt}{0.400pt}}
\multiput(632.00,132.17)(6.500,-1.000){2}{\rule{1.566pt}{0.400pt}}
\put(619.0,133.0){\rule[-0.200pt]{3.132pt}{0.400pt}}
\put(671,130.67){\rule{3.132pt}{0.400pt}}
\multiput(671.00,131.17)(6.500,-1.000){2}{\rule{1.566pt}{0.400pt}}
\put(645.0,132.0){\rule[-0.200pt]{6.263pt}{0.400pt}}
\put(684.0,131.0){\rule[-0.200pt]{15.658pt}{0.400pt}}
\put(919,130.67){\rule{3.132pt}{0.400pt}}
\multiput(919.00,130.17)(6.500,1.000){2}{\rule{1.566pt}{0.400pt}}
\put(831.0,131.0){\rule[-0.200pt]{21.199pt}{0.400pt}}
\put(971,131.67){\rule{3.132pt}{0.400pt}}
\multiput(971.00,131.17)(6.500,1.000){2}{\rule{1.566pt}{0.400pt}}
\put(932.0,132.0){\rule[-0.200pt]{9.395pt}{0.400pt}}
\put(997,132.67){\rule{3.132pt}{0.400pt}}
\multiput(997.00,132.17)(6.500,1.000){2}{\rule{1.566pt}{0.400pt}}
\put(1010,133.67){\rule{3.132pt}{0.400pt}}
\multiput(1010.00,133.17)(6.500,1.000){2}{\rule{1.566pt}{0.400pt}}
\put(984.0,133.0){\rule[-0.200pt]{3.132pt}{0.400pt}}
\put(1036,134.67){\rule{3.132pt}{0.400pt}}
\multiput(1036.00,134.17)(6.500,1.000){2}{\rule{1.566pt}{0.400pt}}
\put(1049,135.67){\rule{3.132pt}{0.400pt}}
\multiput(1049.00,135.17)(6.500,1.000){2}{\rule{1.566pt}{0.400pt}}
\put(1062,136.67){\rule{3.132pt}{0.400pt}}
\multiput(1062.00,136.17)(6.500,1.000){2}{\rule{1.566pt}{0.400pt}}
\put(1023.0,135.0){\rule[-0.200pt]{3.132pt}{0.400pt}}
\put(1088,137.67){\rule{3.132pt}{0.400pt}}
\multiput(1088.00,137.17)(6.500,1.000){2}{\rule{1.566pt}{0.400pt}}
\put(1075.0,138.0){\rule[-0.200pt]{3.132pt}{0.400pt}}
\put(1166,137.67){\rule{3.132pt}{0.400pt}}
\multiput(1166.00,138.17)(6.500,-1.000){2}{\rule{1.566pt}{0.400pt}}
\put(1179,136.67){\rule{3.132pt}{0.400pt}}
\multiput(1179.00,137.17)(6.500,-1.000){2}{\rule{1.566pt}{0.400pt}}
\put(1192,135.67){\rule{3.132pt}{0.400pt}}
\multiput(1192.00,136.17)(6.500,-1.000){2}{\rule{1.566pt}{0.400pt}}
\put(1205,134.67){\rule{3.132pt}{0.400pt}}
\multiput(1205.00,135.17)(6.500,-1.000){2}{\rule{1.566pt}{0.400pt}}
\put(1101.0,139.0){\rule[-0.200pt]{15.658pt}{0.400pt}}
\put(1231,133.67){\rule{3.132pt}{0.400pt}}
\multiput(1231.00,134.17)(6.500,-1.000){2}{\rule{1.566pt}{0.400pt}}
\put(1244,132.67){\rule{3.132pt}{0.400pt}}
\multiput(1244.00,133.17)(6.500,-1.000){2}{\rule{1.566pt}{0.400pt}}
\put(1218.0,135.0){\rule[-0.200pt]{3.132pt}{0.400pt}}
\put(1270,131.67){\rule{3.132pt}{0.400pt}}
\multiput(1270.00,132.17)(6.500,-1.000){2}{\rule{1.566pt}{0.400pt}}
\put(1257.0,133.0){\rule[-0.200pt]{3.132pt}{0.400pt}}
\put(1296,130.67){\rule{3.132pt}{0.400pt}}
\multiput(1296.00,131.17)(6.500,-1.000){2}{\rule{1.566pt}{0.400pt}}
\put(1283.0,132.0){\rule[-0.200pt]{3.132pt}{0.400pt}}
\put(1309.0,131.0){\rule[-0.200pt]{8.672pt}{0.400pt}}
\put(151.0,131.0){\rule[-0.200pt]{0.400pt}{175.375pt}}
\put(151.0,131.0){\rule[-0.200pt]{310.279pt}{0.400pt}}
\put(1439.0,131.0){\rule[-0.200pt]{0.400pt}{175.375pt}}
\put(151.0,859.0){\rule[-0.200pt]{310.279pt}{0.400pt}}
\end{picture}

        \caption{X and Y axes scaled linear}
        \label{fig:a}

        % GNUPLOT: LaTeX picture
\setlength{\unitlength}{0.240900pt}
\ifx\plotpoint\undefined\newsavebox{\plotpoint}\fi
\sbox{\plotpoint}{\rule[-0.200pt]{0.400pt}{0.400pt}}%
\begin{picture}(1500,900)(0,0)
\sbox{\plotpoint}{\rule[-0.200pt]{0.400pt}{0.400pt}}%
\put(151.0,131.0){\rule[-0.200pt]{4.818pt}{0.400pt}}
\put(131,131){\makebox(0,0)[r]{ 0}}
\put(1419.0,131.0){\rule[-0.200pt]{4.818pt}{0.400pt}}
\put(151.0,276.0){\rule[-0.200pt]{4.818pt}{0.400pt}}
\put(131,276){\makebox(0,0)[r]{ 5}}
\put(1419.0,276.0){\rule[-0.200pt]{4.818pt}{0.400pt}}
\put(151.0,422.0){\rule[-0.200pt]{4.818pt}{0.400pt}}
\put(131,422){\makebox(0,0)[r]{ 10}}
\put(1419.0,422.0){\rule[-0.200pt]{4.818pt}{0.400pt}}
\put(151.0,568.0){\rule[-0.200pt]{4.818pt}{0.400pt}}
\put(131,568){\makebox(0,0)[r]{ 15}}
\put(1419.0,568.0){\rule[-0.200pt]{4.818pt}{0.400pt}}
\put(151.0,713.0){\rule[-0.200pt]{4.818pt}{0.400pt}}
\put(131,713){\makebox(0,0)[r]{ 20}}
\put(1419.0,713.0){\rule[-0.200pt]{4.818pt}{0.400pt}}
\put(151.0,859.0){\rule[-0.200pt]{4.818pt}{0.400pt}}
\put(131,859){\makebox(0,0)[r]{ 25}}
\put(1419.0,859.0){\rule[-0.200pt]{4.818pt}{0.400pt}}
\put(151.0,131.0){\rule[-0.200pt]{0.400pt}{4.818pt}}
\put(151,90){\makebox(0,0){ 0.01}}
\put(151.0,839.0){\rule[-0.200pt]{0.400pt}{4.818pt}}
\put(280.0,131.0){\rule[-0.200pt]{0.400pt}{2.409pt}}
\put(280.0,849.0){\rule[-0.200pt]{0.400pt}{2.409pt}}
\put(356.0,131.0){\rule[-0.200pt]{0.400pt}{2.409pt}}
\put(356.0,849.0){\rule[-0.200pt]{0.400pt}{2.409pt}}
\put(409.0,131.0){\rule[-0.200pt]{0.400pt}{2.409pt}}
\put(409.0,849.0){\rule[-0.200pt]{0.400pt}{2.409pt}}
\put(451.0,131.0){\rule[-0.200pt]{0.400pt}{2.409pt}}
\put(451.0,849.0){\rule[-0.200pt]{0.400pt}{2.409pt}}
\put(485.0,131.0){\rule[-0.200pt]{0.400pt}{2.409pt}}
\put(485.0,849.0){\rule[-0.200pt]{0.400pt}{2.409pt}}
\put(514.0,131.0){\rule[-0.200pt]{0.400pt}{2.409pt}}
\put(514.0,849.0){\rule[-0.200pt]{0.400pt}{2.409pt}}
\put(539.0,131.0){\rule[-0.200pt]{0.400pt}{2.409pt}}
\put(539.0,849.0){\rule[-0.200pt]{0.400pt}{2.409pt}}
\put(561.0,131.0){\rule[-0.200pt]{0.400pt}{2.409pt}}
\put(561.0,849.0){\rule[-0.200pt]{0.400pt}{2.409pt}}
\put(580.0,131.0){\rule[-0.200pt]{0.400pt}{4.818pt}}
\put(580,90){\makebox(0,0){ 0.1}}
\put(580.0,839.0){\rule[-0.200pt]{0.400pt}{4.818pt}}
\put(710.0,131.0){\rule[-0.200pt]{0.400pt}{2.409pt}}
\put(710.0,849.0){\rule[-0.200pt]{0.400pt}{2.409pt}}
\put(785.0,131.0){\rule[-0.200pt]{0.400pt}{2.409pt}}
\put(785.0,849.0){\rule[-0.200pt]{0.400pt}{2.409pt}}
\put(839.0,131.0){\rule[-0.200pt]{0.400pt}{2.409pt}}
\put(839.0,849.0){\rule[-0.200pt]{0.400pt}{2.409pt}}
\put(880.0,131.0){\rule[-0.200pt]{0.400pt}{2.409pt}}
\put(880.0,849.0){\rule[-0.200pt]{0.400pt}{2.409pt}}
\put(914.0,131.0){\rule[-0.200pt]{0.400pt}{2.409pt}}
\put(914.0,849.0){\rule[-0.200pt]{0.400pt}{2.409pt}}
\put(943.0,131.0){\rule[-0.200pt]{0.400pt}{2.409pt}}
\put(943.0,849.0){\rule[-0.200pt]{0.400pt}{2.409pt}}
\put(968.0,131.0){\rule[-0.200pt]{0.400pt}{2.409pt}}
\put(968.0,849.0){\rule[-0.200pt]{0.400pt}{2.409pt}}
\put(990.0,131.0){\rule[-0.200pt]{0.400pt}{2.409pt}}
\put(990.0,849.0){\rule[-0.200pt]{0.400pt}{2.409pt}}
\put(1010.0,131.0){\rule[-0.200pt]{0.400pt}{4.818pt}}
\put(1010,90){\makebox(0,0){ 1}}
\put(1010.0,839.0){\rule[-0.200pt]{0.400pt}{4.818pt}}
\put(1139.0,131.0){\rule[-0.200pt]{0.400pt}{2.409pt}}
\put(1139.0,849.0){\rule[-0.200pt]{0.400pt}{2.409pt}}
\put(1215.0,131.0){\rule[-0.200pt]{0.400pt}{2.409pt}}
\put(1215.0,849.0){\rule[-0.200pt]{0.400pt}{2.409pt}}
\put(1268.0,131.0){\rule[-0.200pt]{0.400pt}{2.409pt}}
\put(1268.0,849.0){\rule[-0.200pt]{0.400pt}{2.409pt}}
\put(1310.0,131.0){\rule[-0.200pt]{0.400pt}{2.409pt}}
\put(1310.0,849.0){\rule[-0.200pt]{0.400pt}{2.409pt}}
\put(1344.0,131.0){\rule[-0.200pt]{0.400pt}{2.409pt}}
\put(1344.0,849.0){\rule[-0.200pt]{0.400pt}{2.409pt}}
\put(1372.0,131.0){\rule[-0.200pt]{0.400pt}{2.409pt}}
\put(1372.0,849.0){\rule[-0.200pt]{0.400pt}{2.409pt}}
\put(1397.0,131.0){\rule[-0.200pt]{0.400pt}{2.409pt}}
\put(1397.0,849.0){\rule[-0.200pt]{0.400pt}{2.409pt}}
\put(1419.0,131.0){\rule[-0.200pt]{0.400pt}{2.409pt}}
\put(1419.0,849.0){\rule[-0.200pt]{0.400pt}{2.409pt}}
\put(1439.0,131.0){\rule[-0.200pt]{0.400pt}{4.818pt}}
\put(1439,90){\makebox(0,0){ 10}}
\put(1439.0,839.0){\rule[-0.200pt]{0.400pt}{4.818pt}}
\put(151.0,131.0){\rule[-0.200pt]{0.400pt}{175.375pt}}
\put(151.0,131.0){\rule[-0.200pt]{310.279pt}{0.400pt}}
\put(1439.0,131.0){\rule[-0.200pt]{0.400pt}{175.375pt}}
\put(151.0,859.0){\rule[-0.200pt]{310.279pt}{0.400pt}}
\put(30,495){\makebox(0,0){y}}
\put(795,29){\makebox(0,0){x}}
\put(1279,819){\makebox(0,0)[r]{2**( 2 - x + 4 * sin(x) )}}
\put(1299.0,819.0){\rule[-0.200pt]{24.090pt}{0.400pt}}
\put(151,250){\usebox{\plotpoint}}
\put(203,249.67){\rule{3.132pt}{0.400pt}}
\multiput(203.00,249.17)(6.500,1.000){2}{\rule{1.566pt}{0.400pt}}
\put(151.0,250.0){\rule[-0.200pt]{12.527pt}{0.400pt}}
\put(242,250.67){\rule{3.132pt}{0.400pt}}
\multiput(242.00,250.17)(6.500,1.000){2}{\rule{1.566pt}{0.400pt}}
\put(216.0,251.0){\rule[-0.200pt]{6.263pt}{0.400pt}}
\put(281,251.67){\rule{3.132pt}{0.400pt}}
\multiput(281.00,251.17)(6.500,1.000){2}{\rule{1.566pt}{0.400pt}}
\put(255.0,252.0){\rule[-0.200pt]{6.263pt}{0.400pt}}
\put(320,252.67){\rule{3.132pt}{0.400pt}}
\multiput(320.00,252.17)(6.500,1.000){2}{\rule{1.566pt}{0.400pt}}
\put(294.0,253.0){\rule[-0.200pt]{6.263pt}{0.400pt}}
\put(346,253.67){\rule{3.132pt}{0.400pt}}
\multiput(346.00,253.17)(6.500,1.000){2}{\rule{1.566pt}{0.400pt}}
\put(333.0,254.0){\rule[-0.200pt]{3.132pt}{0.400pt}}
\put(372,254.67){\rule{3.132pt}{0.400pt}}
\multiput(372.00,254.17)(6.500,1.000){2}{\rule{1.566pt}{0.400pt}}
\put(385,255.67){\rule{3.132pt}{0.400pt}}
\multiput(385.00,255.17)(6.500,1.000){2}{\rule{1.566pt}{0.400pt}}
\put(359.0,255.0){\rule[-0.200pt]{3.132pt}{0.400pt}}
\put(411,256.67){\rule{3.132pt}{0.400pt}}
\multiput(411.00,256.17)(6.500,1.000){2}{\rule{1.566pt}{0.400pt}}
\put(424,257.67){\rule{3.132pt}{0.400pt}}
\multiput(424.00,257.17)(6.500,1.000){2}{\rule{1.566pt}{0.400pt}}
\put(437,258.67){\rule{3.132pt}{0.400pt}}
\multiput(437.00,258.17)(6.500,1.000){2}{\rule{1.566pt}{0.400pt}}
\put(450,259.67){\rule{3.132pt}{0.400pt}}
\multiput(450.00,259.17)(6.500,1.000){2}{\rule{1.566pt}{0.400pt}}
\put(463,260.67){\rule{3.132pt}{0.400pt}}
\multiput(463.00,260.17)(6.500,1.000){2}{\rule{1.566pt}{0.400pt}}
\put(476,261.67){\rule{3.132pt}{0.400pt}}
\multiput(476.00,261.17)(6.500,1.000){2}{\rule{1.566pt}{0.400pt}}
\put(489,262.67){\rule{3.132pt}{0.400pt}}
\multiput(489.00,262.17)(6.500,1.000){2}{\rule{1.566pt}{0.400pt}}
\put(502,264.17){\rule{2.700pt}{0.400pt}}
\multiput(502.00,263.17)(7.396,2.000){2}{\rule{1.350pt}{0.400pt}}
\put(515,265.67){\rule{3.132pt}{0.400pt}}
\multiput(515.00,265.17)(6.500,1.000){2}{\rule{1.566pt}{0.400pt}}
\put(528,267.17){\rule{2.700pt}{0.400pt}}
\multiput(528.00,266.17)(7.396,2.000){2}{\rule{1.350pt}{0.400pt}}
\put(541,268.67){\rule{3.132pt}{0.400pt}}
\multiput(541.00,268.17)(6.500,1.000){2}{\rule{1.566pt}{0.400pt}}
\put(554,270.17){\rule{2.700pt}{0.400pt}}
\multiput(554.00,269.17)(7.396,2.000){2}{\rule{1.350pt}{0.400pt}}
\put(567,272.17){\rule{2.700pt}{0.400pt}}
\multiput(567.00,271.17)(7.396,2.000){2}{\rule{1.350pt}{0.400pt}}
\put(580,274.17){\rule{2.700pt}{0.400pt}}
\multiput(580.00,273.17)(7.396,2.000){2}{\rule{1.350pt}{0.400pt}}
\multiput(593.00,276.61)(2.695,0.447){3}{\rule{1.833pt}{0.108pt}}
\multiput(593.00,275.17)(9.195,3.000){2}{\rule{0.917pt}{0.400pt}}
\put(606,279.17){\rule{2.700pt}{0.400pt}}
\multiput(606.00,278.17)(7.396,2.000){2}{\rule{1.350pt}{0.400pt}}
\multiput(619.00,281.61)(2.695,0.447){3}{\rule{1.833pt}{0.108pt}}
\multiput(619.00,280.17)(9.195,3.000){2}{\rule{0.917pt}{0.400pt}}
\multiput(632.00,284.61)(2.695,0.447){3}{\rule{1.833pt}{0.108pt}}
\multiput(632.00,283.17)(9.195,3.000){2}{\rule{0.917pt}{0.400pt}}
\multiput(645.00,287.61)(2.695,0.447){3}{\rule{1.833pt}{0.108pt}}
\multiput(645.00,286.17)(9.195,3.000){2}{\rule{0.917pt}{0.400pt}}
\multiput(658.00,290.60)(1.797,0.468){5}{\rule{1.400pt}{0.113pt}}
\multiput(658.00,289.17)(10.094,4.000){2}{\rule{0.700pt}{0.400pt}}
\multiput(671.00,294.60)(1.797,0.468){5}{\rule{1.400pt}{0.113pt}}
\multiput(671.00,293.17)(10.094,4.000){2}{\rule{0.700pt}{0.400pt}}
\multiput(684.00,298.60)(1.797,0.468){5}{\rule{1.400pt}{0.113pt}}
\multiput(684.00,297.17)(10.094,4.000){2}{\rule{0.700pt}{0.400pt}}
\multiput(697.00,302.59)(1.378,0.477){7}{\rule{1.140pt}{0.115pt}}
\multiput(697.00,301.17)(10.634,5.000){2}{\rule{0.570pt}{0.400pt}}
\multiput(710.00,307.59)(1.378,0.477){7}{\rule{1.140pt}{0.115pt}}
\multiput(710.00,306.17)(10.634,5.000){2}{\rule{0.570pt}{0.400pt}}
\multiput(723.00,312.59)(1.123,0.482){9}{\rule{0.967pt}{0.116pt}}
\multiput(723.00,311.17)(10.994,6.000){2}{\rule{0.483pt}{0.400pt}}
\multiput(736.00,318.59)(1.123,0.482){9}{\rule{0.967pt}{0.116pt}}
\multiput(736.00,317.17)(10.994,6.000){2}{\rule{0.483pt}{0.400pt}}
\multiput(749.00,324.59)(0.950,0.485){11}{\rule{0.843pt}{0.117pt}}
\multiput(749.00,323.17)(11.251,7.000){2}{\rule{0.421pt}{0.400pt}}
\multiput(762.00,331.59)(0.824,0.488){13}{\rule{0.750pt}{0.117pt}}
\multiput(762.00,330.17)(11.443,8.000){2}{\rule{0.375pt}{0.400pt}}
\multiput(775.00,339.59)(0.728,0.489){15}{\rule{0.678pt}{0.118pt}}
\multiput(775.00,338.17)(11.593,9.000){2}{\rule{0.339pt}{0.400pt}}
\multiput(788.00,348.59)(0.786,0.489){15}{\rule{0.722pt}{0.118pt}}
\multiput(788.00,347.17)(12.501,9.000){2}{\rule{0.361pt}{0.400pt}}
\multiput(802.00,357.58)(0.590,0.492){19}{\rule{0.573pt}{0.118pt}}
\multiput(802.00,356.17)(11.811,11.000){2}{\rule{0.286pt}{0.400pt}}
\multiput(815.00,368.58)(0.590,0.492){19}{\rule{0.573pt}{0.118pt}}
\multiput(815.00,367.17)(11.811,11.000){2}{\rule{0.286pt}{0.400pt}}
\multiput(828.00,379.58)(0.497,0.493){23}{\rule{0.500pt}{0.119pt}}
\multiput(828.00,378.17)(11.962,13.000){2}{\rule{0.250pt}{0.400pt}}
\multiput(841.58,392.00)(0.493,0.576){23}{\rule{0.119pt}{0.562pt}}
\multiput(840.17,392.00)(13.000,13.834){2}{\rule{0.400pt}{0.281pt}}
\multiput(854.58,407.00)(0.493,0.616){23}{\rule{0.119pt}{0.592pt}}
\multiput(853.17,407.00)(13.000,14.771){2}{\rule{0.400pt}{0.296pt}}
\multiput(867.58,423.00)(0.493,0.695){23}{\rule{0.119pt}{0.654pt}}
\multiput(866.17,423.00)(13.000,16.643){2}{\rule{0.400pt}{0.327pt}}
\multiput(880.58,441.00)(0.493,0.774){23}{\rule{0.119pt}{0.715pt}}
\multiput(879.17,441.00)(13.000,18.515){2}{\rule{0.400pt}{0.358pt}}
\multiput(893.58,461.00)(0.493,0.814){23}{\rule{0.119pt}{0.746pt}}
\multiput(892.17,461.00)(13.000,19.451){2}{\rule{0.400pt}{0.373pt}}
\multiput(906.58,482.00)(0.493,0.972){23}{\rule{0.119pt}{0.869pt}}
\multiput(905.17,482.00)(13.000,23.196){2}{\rule{0.400pt}{0.435pt}}
\multiput(919.58,507.00)(0.493,1.012){23}{\rule{0.119pt}{0.900pt}}
\multiput(918.17,507.00)(13.000,24.132){2}{\rule{0.400pt}{0.450pt}}
\multiput(932.58,533.00)(0.493,1.131){23}{\rule{0.119pt}{0.992pt}}
\multiput(931.17,533.00)(13.000,26.940){2}{\rule{0.400pt}{0.496pt}}
\multiput(945.58,562.00)(0.493,1.210){23}{\rule{0.119pt}{1.054pt}}
\multiput(944.17,562.00)(13.000,28.813){2}{\rule{0.400pt}{0.527pt}}
\multiput(958.58,593.00)(0.493,1.329){23}{\rule{0.119pt}{1.146pt}}
\multiput(957.17,593.00)(13.000,31.621){2}{\rule{0.400pt}{0.573pt}}
\multiput(971.58,627.00)(0.493,1.329){23}{\rule{0.119pt}{1.146pt}}
\multiput(970.17,627.00)(13.000,31.621){2}{\rule{0.400pt}{0.573pt}}
\multiput(984.58,661.00)(0.493,1.408){23}{\rule{0.119pt}{1.208pt}}
\multiput(983.17,661.00)(13.000,33.493){2}{\rule{0.400pt}{0.604pt}}
\multiput(997.58,697.00)(0.493,1.329){23}{\rule{0.119pt}{1.146pt}}
\multiput(996.17,697.00)(13.000,31.621){2}{\rule{0.400pt}{0.573pt}}
\multiput(1010.58,731.00)(0.493,1.250){23}{\rule{0.119pt}{1.085pt}}
\multiput(1009.17,731.00)(13.000,29.749){2}{\rule{0.400pt}{0.542pt}}
\multiput(1023.58,763.00)(0.493,1.052){23}{\rule{0.119pt}{0.931pt}}
\multiput(1022.17,763.00)(13.000,25.068){2}{\rule{0.400pt}{0.465pt}}
\multiput(1036.58,790.00)(0.493,0.734){23}{\rule{0.119pt}{0.685pt}}
\multiput(1035.17,790.00)(13.000,17.579){2}{\rule{0.400pt}{0.342pt}}
\multiput(1049.00,809.59)(1.123,0.482){9}{\rule{0.967pt}{0.116pt}}
\multiput(1049.00,808.17)(10.994,6.000){2}{\rule{0.483pt}{0.400pt}}
\multiput(1062.00,813.93)(0.728,-0.489){15}{\rule{0.678pt}{0.118pt}}
\multiput(1062.00,814.17)(11.593,-9.000){2}{\rule{0.339pt}{0.400pt}}
\multiput(1075.58,802.14)(0.493,-1.052){23}{\rule{0.119pt}{0.931pt}}
\multiput(1074.17,804.07)(13.000,-25.068){2}{\rule{0.400pt}{0.465pt}}
\multiput(1088.58,772.33)(0.493,-1.924){23}{\rule{0.119pt}{1.608pt}}
\multiput(1087.17,775.66)(13.000,-45.663){2}{\rule{0.400pt}{0.804pt}}
\multiput(1101.58,720.90)(0.493,-2.677){23}{\rule{0.119pt}{2.192pt}}
\multiput(1100.17,725.45)(13.000,-63.450){2}{\rule{0.400pt}{1.096pt}}
\multiput(1114.58,650.98)(0.493,-3.272){23}{\rule{0.119pt}{2.654pt}}
\multiput(1113.17,656.49)(13.000,-77.492){2}{\rule{0.400pt}{1.327pt}}
\multiput(1127.58,566.83)(0.493,-3.629){23}{\rule{0.119pt}{2.931pt}}
\multiput(1126.17,572.92)(13.000,-85.917){2}{\rule{0.400pt}{1.465pt}}
\multiput(1140.58,474.83)(0.493,-3.629){23}{\rule{0.119pt}{2.931pt}}
\multiput(1139.17,480.92)(13.000,-85.917){2}{\rule{0.400pt}{1.465pt}}
\multiput(1153.58,384.11)(0.493,-3.232){23}{\rule{0.119pt}{2.623pt}}
\multiput(1152.17,389.56)(13.000,-76.556){2}{\rule{0.400pt}{1.312pt}}
\multiput(1166.58,304.03)(0.493,-2.638){23}{\rule{0.119pt}{2.162pt}}
\multiput(1165.17,308.51)(13.000,-62.514){2}{\rule{0.400pt}{1.081pt}}
\multiput(1179.58,239.45)(0.493,-1.884){23}{\rule{0.119pt}{1.577pt}}
\multiput(1178.17,242.73)(13.000,-44.727){2}{\rule{0.400pt}{0.788pt}}
\multiput(1192.58,193.63)(0.493,-1.210){23}{\rule{0.119pt}{1.054pt}}
\multiput(1191.17,195.81)(13.000,-28.813){2}{\rule{0.400pt}{0.527pt}}
\multiput(1205.58,164.29)(0.493,-0.695){23}{\rule{0.119pt}{0.654pt}}
\multiput(1204.17,165.64)(13.000,-16.643){2}{\rule{0.400pt}{0.327pt}}
\multiput(1218.00,147.92)(0.652,-0.491){17}{\rule{0.620pt}{0.118pt}}
\multiput(1218.00,148.17)(11.713,-10.000){2}{\rule{0.310pt}{0.400pt}}
\multiput(1231.00,137.93)(1.378,-0.477){7}{\rule{1.140pt}{0.115pt}}
\multiput(1231.00,138.17)(10.634,-5.000){2}{\rule{0.570pt}{0.400pt}}
\put(1244,132.17){\rule{2.700pt}{0.400pt}}
\multiput(1244.00,133.17)(7.396,-2.000){2}{\rule{1.350pt}{0.400pt}}
\put(398.0,257.0){\rule[-0.200pt]{3.132pt}{0.400pt}}
\put(1270,130.67){\rule{3.132pt}{0.400pt}}
\multiput(1270.00,131.17)(6.500,-1.000){2}{\rule{1.566pt}{0.400pt}}
\put(1257.0,132.0){\rule[-0.200pt]{3.132pt}{0.400pt}}
\put(1283.0,131.0){\rule[-0.200pt]{3.373pt}{0.400pt}}
\put(1335,130.67){\rule{3.132pt}{0.400pt}}
\multiput(1335.00,130.17)(6.500,1.000){2}{\rule{1.566pt}{0.400pt}}
\put(1348,131.67){\rule{3.132pt}{0.400pt}}
\multiput(1348.00,131.17)(6.500,1.000){2}{\rule{1.566pt}{0.400pt}}
\multiput(1361.00,133.60)(1.797,0.468){5}{\rule{1.400pt}{0.113pt}}
\multiput(1361.00,132.17)(10.094,4.000){2}{\rule{0.700pt}{0.400pt}}
\put(1374,137.17){\rule{2.700pt}{0.400pt}}
\multiput(1374.00,136.17)(7.396,2.000){2}{\rule{1.350pt}{0.400pt}}
\put(1387,137.17){\rule{2.700pt}{0.400pt}}
\multiput(1387.00,138.17)(7.396,-2.000){2}{\rule{1.350pt}{0.400pt}}
\multiput(1400.00,135.93)(1.378,-0.477){7}{\rule{1.140pt}{0.115pt}}
\multiput(1400.00,136.17)(10.634,-5.000){2}{\rule{0.570pt}{0.400pt}}
\put(1413,130.67){\rule{3.132pt}{0.400pt}}
\multiput(1413.00,131.17)(6.500,-1.000){2}{\rule{1.566pt}{0.400pt}}
\put(1319.0,131.0){\rule[-0.200pt]{3.854pt}{0.400pt}}
\put(151.0,131.0){\rule[-0.200pt]{0.400pt}{175.375pt}}
\put(151.0,131.0){\rule[-0.200pt]{310.279pt}{0.400pt}}
\put(1439.0,131.0){\rule[-0.200pt]{0.400pt}{175.375pt}}
\put(151.0,859.0){\rule[-0.200pt]{310.279pt}{0.400pt}}
\end{picture}

        \caption{X axe scaled logarithmic }
        \label{fig:b}

\end{figure}

\begin{figure}[p]

        \centering
        % GNUPLOT: LaTeX picture
\setlength{\unitlength}{0.240900pt}
\ifx\plotpoint\undefined\newsavebox{\plotpoint}\fi
\sbox{\plotpoint}{\rule[-0.200pt]{0.400pt}{0.400pt}}%
\begin{picture}(1500,900)(0,0)
\sbox{\plotpoint}{\rule[-0.200pt]{0.400pt}{0.400pt}}%
\put(191.0,131.0){\rule[-0.200pt]{4.818pt}{0.400pt}}
\put(171,131){\makebox(0,0)[r]{ 0.01}}
\put(1419.0,131.0){\rule[-0.200pt]{4.818pt}{0.400pt}}
\put(191.0,186.0){\rule[-0.200pt]{2.409pt}{0.400pt}}
\put(1429.0,186.0){\rule[-0.200pt]{2.409pt}{0.400pt}}
\put(191.0,218.0){\rule[-0.200pt]{2.409pt}{0.400pt}}
\put(1429.0,218.0){\rule[-0.200pt]{2.409pt}{0.400pt}}
\put(191.0,241.0){\rule[-0.200pt]{2.409pt}{0.400pt}}
\put(1429.0,241.0){\rule[-0.200pt]{2.409pt}{0.400pt}}
\put(191.0,258.0){\rule[-0.200pt]{2.409pt}{0.400pt}}
\put(1429.0,258.0){\rule[-0.200pt]{2.409pt}{0.400pt}}
\put(191.0,273.0){\rule[-0.200pt]{2.409pt}{0.400pt}}
\put(1429.0,273.0){\rule[-0.200pt]{2.409pt}{0.400pt}}
\put(191.0,285.0){\rule[-0.200pt]{2.409pt}{0.400pt}}
\put(1429.0,285.0){\rule[-0.200pt]{2.409pt}{0.400pt}}
\put(191.0,295.0){\rule[-0.200pt]{2.409pt}{0.400pt}}
\put(1429.0,295.0){\rule[-0.200pt]{2.409pt}{0.400pt}}
\put(191.0,305.0){\rule[-0.200pt]{2.409pt}{0.400pt}}
\put(1429.0,305.0){\rule[-0.200pt]{2.409pt}{0.400pt}}
\put(191.0,313.0){\rule[-0.200pt]{4.818pt}{0.400pt}}
\put(171,313){\makebox(0,0)[r]{ 0.1}}
\put(1419.0,313.0){\rule[-0.200pt]{4.818pt}{0.400pt}}
\put(191.0,368.0){\rule[-0.200pt]{2.409pt}{0.400pt}}
\put(1429.0,368.0){\rule[-0.200pt]{2.409pt}{0.400pt}}
\put(191.0,400.0){\rule[-0.200pt]{2.409pt}{0.400pt}}
\put(1429.0,400.0){\rule[-0.200pt]{2.409pt}{0.400pt}}
\put(191.0,423.0){\rule[-0.200pt]{2.409pt}{0.400pt}}
\put(1429.0,423.0){\rule[-0.200pt]{2.409pt}{0.400pt}}
\put(191.0,440.0){\rule[-0.200pt]{2.409pt}{0.400pt}}
\put(1429.0,440.0){\rule[-0.200pt]{2.409pt}{0.400pt}}
\put(191.0,455.0){\rule[-0.200pt]{2.409pt}{0.400pt}}
\put(1429.0,455.0){\rule[-0.200pt]{2.409pt}{0.400pt}}
\put(191.0,467.0){\rule[-0.200pt]{2.409pt}{0.400pt}}
\put(1429.0,467.0){\rule[-0.200pt]{2.409pt}{0.400pt}}
\put(191.0,477.0){\rule[-0.200pt]{2.409pt}{0.400pt}}
\put(1429.0,477.0){\rule[-0.200pt]{2.409pt}{0.400pt}}
\put(191.0,487.0){\rule[-0.200pt]{2.409pt}{0.400pt}}
\put(1429.0,487.0){\rule[-0.200pt]{2.409pt}{0.400pt}}
\put(191.0,495.0){\rule[-0.200pt]{4.818pt}{0.400pt}}
\put(171,495){\makebox(0,0)[r]{ 1}}
\put(1419.0,495.0){\rule[-0.200pt]{4.818pt}{0.400pt}}
\put(191.0,550.0){\rule[-0.200pt]{2.409pt}{0.400pt}}
\put(1429.0,550.0){\rule[-0.200pt]{2.409pt}{0.400pt}}
\put(191.0,582.0){\rule[-0.200pt]{2.409pt}{0.400pt}}
\put(1429.0,582.0){\rule[-0.200pt]{2.409pt}{0.400pt}}
\put(191.0,605.0){\rule[-0.200pt]{2.409pt}{0.400pt}}
\put(1429.0,605.0){\rule[-0.200pt]{2.409pt}{0.400pt}}
\put(191.0,622.0){\rule[-0.200pt]{2.409pt}{0.400pt}}
\put(1429.0,622.0){\rule[-0.200pt]{2.409pt}{0.400pt}}
\put(191.0,637.0){\rule[-0.200pt]{2.409pt}{0.400pt}}
\put(1429.0,637.0){\rule[-0.200pt]{2.409pt}{0.400pt}}
\put(191.0,649.0){\rule[-0.200pt]{2.409pt}{0.400pt}}
\put(1429.0,649.0){\rule[-0.200pt]{2.409pt}{0.400pt}}
\put(191.0,659.0){\rule[-0.200pt]{2.409pt}{0.400pt}}
\put(1429.0,659.0){\rule[-0.200pt]{2.409pt}{0.400pt}}
\put(191.0,669.0){\rule[-0.200pt]{2.409pt}{0.400pt}}
\put(1429.0,669.0){\rule[-0.200pt]{2.409pt}{0.400pt}}
\put(191.0,677.0){\rule[-0.200pt]{4.818pt}{0.400pt}}
\put(171,677){\makebox(0,0)[r]{ 10}}
\put(1419.0,677.0){\rule[-0.200pt]{4.818pt}{0.400pt}}
\put(191.0,732.0){\rule[-0.200pt]{2.409pt}{0.400pt}}
\put(1429.0,732.0){\rule[-0.200pt]{2.409pt}{0.400pt}}
\put(191.0,764.0){\rule[-0.200pt]{2.409pt}{0.400pt}}
\put(1429.0,764.0){\rule[-0.200pt]{2.409pt}{0.400pt}}
\put(191.0,787.0){\rule[-0.200pt]{2.409pt}{0.400pt}}
\put(1429.0,787.0){\rule[-0.200pt]{2.409pt}{0.400pt}}
\put(191.0,804.0){\rule[-0.200pt]{2.409pt}{0.400pt}}
\put(1429.0,804.0){\rule[-0.200pt]{2.409pt}{0.400pt}}
\put(191.0,819.0){\rule[-0.200pt]{2.409pt}{0.400pt}}
\put(1429.0,819.0){\rule[-0.200pt]{2.409pt}{0.400pt}}
\put(191.0,831.0){\rule[-0.200pt]{2.409pt}{0.400pt}}
\put(1429.0,831.0){\rule[-0.200pt]{2.409pt}{0.400pt}}
\put(191.0,841.0){\rule[-0.200pt]{2.409pt}{0.400pt}}
\put(1429.0,841.0){\rule[-0.200pt]{2.409pt}{0.400pt}}
\put(191.0,851.0){\rule[-0.200pt]{2.409pt}{0.400pt}}
\put(1429.0,851.0){\rule[-0.200pt]{2.409pt}{0.400pt}}
\put(191.0,859.0){\rule[-0.200pt]{4.818pt}{0.400pt}}
\put(171,859){\makebox(0,0)[r]{ 100}}
\put(1419.0,859.0){\rule[-0.200pt]{4.818pt}{0.400pt}}
\put(190.0,131.0){\rule[-0.200pt]{0.400pt}{4.818pt}}
\put(190,90){\makebox(0,0){ 0}}
\put(190.0,839.0){\rule[-0.200pt]{0.400pt}{4.818pt}}
\put(315.0,131.0){\rule[-0.200pt]{0.400pt}{4.818pt}}
\put(315,90){\makebox(0,0){ 1}}
\put(315.0,839.0){\rule[-0.200pt]{0.400pt}{4.818pt}}
\put(440.0,131.0){\rule[-0.200pt]{0.400pt}{4.818pt}}
\put(440,90){\makebox(0,0){ 2}}
\put(440.0,839.0){\rule[-0.200pt]{0.400pt}{4.818pt}}
\put(565.0,131.0){\rule[-0.200pt]{0.400pt}{4.818pt}}
\put(565,90){\makebox(0,0){ 3}}
\put(565.0,839.0){\rule[-0.200pt]{0.400pt}{4.818pt}}
\put(689.0,131.0){\rule[-0.200pt]{0.400pt}{4.818pt}}
\put(689,90){\makebox(0,0){ 4}}
\put(689.0,839.0){\rule[-0.200pt]{0.400pt}{4.818pt}}
\put(814.0,131.0){\rule[-0.200pt]{0.400pt}{4.818pt}}
\put(814,90){\makebox(0,0){ 5}}
\put(814.0,839.0){\rule[-0.200pt]{0.400pt}{4.818pt}}
\put(939.0,131.0){\rule[-0.200pt]{0.400pt}{4.818pt}}
\put(939,90){\makebox(0,0){ 6}}
\put(939.0,839.0){\rule[-0.200pt]{0.400pt}{4.818pt}}
\put(1064.0,131.0){\rule[-0.200pt]{0.400pt}{4.818pt}}
\put(1064,90){\makebox(0,0){ 7}}
\put(1064.0,839.0){\rule[-0.200pt]{0.400pt}{4.818pt}}
\put(1189.0,131.0){\rule[-0.200pt]{0.400pt}{4.818pt}}
\put(1189,90){\makebox(0,0){ 8}}
\put(1189.0,839.0){\rule[-0.200pt]{0.400pt}{4.818pt}}
\put(1314.0,131.0){\rule[-0.200pt]{0.400pt}{4.818pt}}
\put(1314,90){\makebox(0,0){ 9}}
\put(1314.0,839.0){\rule[-0.200pt]{0.400pt}{4.818pt}}
\put(1439.0,131.0){\rule[-0.200pt]{0.400pt}{4.818pt}}
\put(1439,90){\makebox(0,0){ 10}}
\put(1439.0,839.0){\rule[-0.200pt]{0.400pt}{4.818pt}}
\put(191.0,131.0){\rule[-0.200pt]{0.400pt}{175.375pt}}
\put(191.0,131.0){\rule[-0.200pt]{300.643pt}{0.400pt}}
\put(1439.0,131.0){\rule[-0.200pt]{0.400pt}{175.375pt}}
\put(191.0,859.0){\rule[-0.200pt]{300.643pt}{0.400pt}}
\put(30,495){\makebox(0,0){y}}
\put(815,29){\makebox(0,0){x}}
\put(1279,819){\makebox(0,0)[r]{2**( 2 - x + 4 * sin(x) )}}
\put(1299.0,819.0){\rule[-0.200pt]{24.090pt}{0.400pt}}
\put(191,606){\usebox{\plotpoint}}
\multiput(191.58,606.00)(0.493,0.655){23}{\rule{0.119pt}{0.623pt}}
\multiput(190.17,606.00)(13.000,15.707){2}{\rule{0.400pt}{0.312pt}}
\multiput(204.58,623.00)(0.492,0.669){21}{\rule{0.119pt}{0.633pt}}
\multiput(203.17,623.00)(12.000,14.685){2}{\rule{0.400pt}{0.317pt}}
\multiput(216.58,639.00)(0.493,0.616){23}{\rule{0.119pt}{0.592pt}}
\multiput(215.17,639.00)(13.000,14.771){2}{\rule{0.400pt}{0.296pt}}
\multiput(229.58,655.00)(0.492,0.625){21}{\rule{0.119pt}{0.600pt}}
\multiput(228.17,655.00)(12.000,13.755){2}{\rule{0.400pt}{0.300pt}}
\multiput(241.58,670.00)(0.493,0.536){23}{\rule{0.119pt}{0.531pt}}
\multiput(240.17,670.00)(13.000,12.898){2}{\rule{0.400pt}{0.265pt}}
\multiput(254.00,684.58)(0.497,0.493){23}{\rule{0.500pt}{0.119pt}}
\multiput(254.00,683.17)(11.962,13.000){2}{\rule{0.250pt}{0.400pt}}
\multiput(267.00,697.58)(0.496,0.492){21}{\rule{0.500pt}{0.119pt}}
\multiput(267.00,696.17)(10.962,12.000){2}{\rule{0.250pt}{0.400pt}}
\multiput(279.00,709.58)(0.590,0.492){19}{\rule{0.573pt}{0.118pt}}
\multiput(279.00,708.17)(11.811,11.000){2}{\rule{0.286pt}{0.400pt}}
\multiput(292.00,720.59)(0.758,0.488){13}{\rule{0.700pt}{0.117pt}}
\multiput(292.00,719.17)(10.547,8.000){2}{\rule{0.350pt}{0.400pt}}
\multiput(304.00,728.59)(0.950,0.485){11}{\rule{0.843pt}{0.117pt}}
\multiput(304.00,727.17)(11.251,7.000){2}{\rule{0.421pt}{0.400pt}}
\multiput(317.00,735.59)(1.378,0.477){7}{\rule{1.140pt}{0.115pt}}
\multiput(317.00,734.17)(10.634,5.000){2}{\rule{0.570pt}{0.400pt}}
\multiput(330.00,740.60)(1.651,0.468){5}{\rule{1.300pt}{0.113pt}}
\multiput(330.00,739.17)(9.302,4.000){2}{\rule{0.650pt}{0.400pt}}
\put(342,743.67){\rule{3.132pt}{0.400pt}}
\multiput(342.00,743.17)(6.500,1.000){2}{\rule{1.566pt}{0.400pt}}
\put(355,743.17){\rule{2.500pt}{0.400pt}}
\multiput(355.00,744.17)(6.811,-2.000){2}{\rule{1.250pt}{0.400pt}}
\multiput(367.00,741.95)(2.695,-0.447){3}{\rule{1.833pt}{0.108pt}}
\multiput(367.00,742.17)(9.195,-3.000){2}{\rule{0.917pt}{0.400pt}}
\multiput(380.00,738.93)(1.123,-0.482){9}{\rule{0.967pt}{0.116pt}}
\multiput(380.00,739.17)(10.994,-6.000){2}{\rule{0.483pt}{0.400pt}}
\multiput(393.00,732.93)(0.874,-0.485){11}{\rule{0.786pt}{0.117pt}}
\multiput(393.00,733.17)(10.369,-7.000){2}{\rule{0.393pt}{0.400pt}}
\multiput(405.00,725.92)(0.652,-0.491){17}{\rule{0.620pt}{0.118pt}}
\multiput(405.00,726.17)(11.713,-10.000){2}{\rule{0.310pt}{0.400pt}}
\multiput(418.00,715.92)(0.497,-0.493){23}{\rule{0.500pt}{0.119pt}}
\multiput(418.00,716.17)(11.962,-13.000){2}{\rule{0.250pt}{0.400pt}}
\multiput(431.58,701.65)(0.492,-0.582){21}{\rule{0.119pt}{0.567pt}}
\multiput(430.17,702.82)(12.000,-12.824){2}{\rule{0.400pt}{0.283pt}}
\multiput(443.58,687.54)(0.493,-0.616){23}{\rule{0.119pt}{0.592pt}}
\multiput(442.17,688.77)(13.000,-14.771){2}{\rule{0.400pt}{0.296pt}}
\multiput(456.58,671.09)(0.492,-0.755){21}{\rule{0.119pt}{0.700pt}}
\multiput(455.17,672.55)(12.000,-16.547){2}{\rule{0.400pt}{0.350pt}}
\multiput(468.58,653.03)(0.493,-0.774){23}{\rule{0.119pt}{0.715pt}}
\multiput(467.17,654.52)(13.000,-18.515){2}{\rule{0.400pt}{0.358pt}}
\multiput(481.58,632.77)(0.493,-0.853){23}{\rule{0.119pt}{0.777pt}}
\multiput(480.17,634.39)(13.000,-20.387){2}{\rule{0.400pt}{0.388pt}}
\multiput(494.58,610.40)(0.492,-0.970){21}{\rule{0.119pt}{0.867pt}}
\multiput(493.17,612.20)(12.000,-21.201){2}{\rule{0.400pt}{0.433pt}}
\multiput(506.58,587.52)(0.493,-0.933){23}{\rule{0.119pt}{0.838pt}}
\multiput(505.17,589.26)(13.000,-22.260){2}{\rule{0.400pt}{0.419pt}}
\multiput(519.58,563.13)(0.492,-1.056){21}{\rule{0.119pt}{0.933pt}}
\multiput(518.17,565.06)(12.000,-23.063){2}{\rule{0.400pt}{0.467pt}}
\multiput(531.58,538.14)(0.493,-1.052){23}{\rule{0.119pt}{0.931pt}}
\multiput(530.17,540.07)(13.000,-25.068){2}{\rule{0.400pt}{0.465pt}}
\multiput(544.58,511.14)(0.493,-1.052){23}{\rule{0.119pt}{0.931pt}}
\multiput(543.17,513.07)(13.000,-25.068){2}{\rule{0.400pt}{0.465pt}}
\multiput(557.58,483.85)(0.492,-1.142){21}{\rule{0.119pt}{1.000pt}}
\multiput(556.17,485.92)(12.000,-24.924){2}{\rule{0.400pt}{0.500pt}}
\multiput(569.58,457.01)(0.493,-1.091){23}{\rule{0.119pt}{0.962pt}}
\multiput(568.17,459.00)(13.000,-26.004){2}{\rule{0.400pt}{0.481pt}}
\multiput(582.58,428.85)(0.492,-1.142){21}{\rule{0.119pt}{1.000pt}}
\multiput(581.17,430.92)(12.000,-24.924){2}{\rule{0.400pt}{0.500pt}}
\multiput(594.58,402.01)(0.493,-1.091){23}{\rule{0.119pt}{0.962pt}}
\multiput(593.17,404.00)(13.000,-26.004){2}{\rule{0.400pt}{0.481pt}}
\multiput(607.58,374.14)(0.493,-1.052){23}{\rule{0.119pt}{0.931pt}}
\multiput(606.17,376.07)(13.000,-25.068){2}{\rule{0.400pt}{0.465pt}}
\multiput(620.58,346.99)(0.492,-1.099){21}{\rule{0.119pt}{0.967pt}}
\multiput(619.17,348.99)(12.000,-23.994){2}{\rule{0.400pt}{0.483pt}}
\multiput(632.58,321.39)(0.493,-0.972){23}{\rule{0.119pt}{0.869pt}}
\multiput(631.17,323.20)(13.000,-23.196){2}{\rule{0.400pt}{0.435pt}}
\multiput(645.58,296.13)(0.492,-1.056){21}{\rule{0.119pt}{0.933pt}}
\multiput(644.17,298.06)(12.000,-23.063){2}{\rule{0.400pt}{0.467pt}}
\multiput(657.58,271.65)(0.493,-0.893){23}{\rule{0.119pt}{0.808pt}}
\multiput(656.17,273.32)(13.000,-21.324){2}{\rule{0.400pt}{0.404pt}}
\multiput(670.58,248.90)(0.493,-0.814){23}{\rule{0.119pt}{0.746pt}}
\multiput(669.17,250.45)(13.000,-19.451){2}{\rule{0.400pt}{0.373pt}}
\multiput(683.58,227.82)(0.492,-0.841){21}{\rule{0.119pt}{0.767pt}}
\multiput(682.17,229.41)(12.000,-18.409){2}{\rule{0.400pt}{0.383pt}}
\multiput(695.58,208.16)(0.493,-0.734){23}{\rule{0.119pt}{0.685pt}}
\multiput(694.17,209.58)(13.000,-17.579){2}{\rule{0.400pt}{0.342pt}}
\multiput(708.58,189.37)(0.492,-0.669){21}{\rule{0.119pt}{0.633pt}}
\multiput(707.17,190.69)(12.000,-14.685){2}{\rule{0.400pt}{0.317pt}}
\multiput(720.58,173.67)(0.493,-0.576){23}{\rule{0.119pt}{0.562pt}}
\multiput(719.17,174.83)(13.000,-13.834){2}{\rule{0.400pt}{0.281pt}}
\multiput(733.00,159.92)(0.539,-0.492){21}{\rule{0.533pt}{0.119pt}}
\multiput(733.00,160.17)(11.893,-12.000){2}{\rule{0.267pt}{0.400pt}}
\multiput(746.00,147.92)(0.600,-0.491){17}{\rule{0.580pt}{0.118pt}}
\multiput(746.00,148.17)(10.796,-10.000){2}{\rule{0.290pt}{0.400pt}}
\multiput(758.00,137.93)(0.824,-0.488){13}{\rule{0.750pt}{0.117pt}}
\multiput(758.00,138.17)(11.443,-8.000){2}{\rule{0.375pt}{0.400pt}}
\multiput(850.00,131.59)(0.933,0.477){7}{\rule{0.820pt}{0.115pt}}
\multiput(850.00,130.17)(7.298,5.000){2}{\rule{0.410pt}{0.400pt}}
\multiput(859.00,136.59)(0.728,0.489){15}{\rule{0.678pt}{0.118pt}}
\multiput(859.00,135.17)(11.593,9.000){2}{\rule{0.339pt}{0.400pt}}
\multiput(872.00,145.58)(0.600,0.491){17}{\rule{0.580pt}{0.118pt}}
\multiput(872.00,144.17)(10.796,10.000){2}{\rule{0.290pt}{0.400pt}}
\multiput(884.00,155.58)(0.539,0.492){21}{\rule{0.533pt}{0.119pt}}
\multiput(884.00,154.17)(11.893,12.000){2}{\rule{0.267pt}{0.400pt}}
\multiput(897.00,167.58)(0.497,0.493){23}{\rule{0.500pt}{0.119pt}}
\multiput(897.00,166.17)(11.962,13.000){2}{\rule{0.250pt}{0.400pt}}
\multiput(910.58,180.00)(0.492,0.582){21}{\rule{0.119pt}{0.567pt}}
\multiput(909.17,180.00)(12.000,12.824){2}{\rule{0.400pt}{0.283pt}}
\multiput(922.58,194.00)(0.493,0.576){23}{\rule{0.119pt}{0.562pt}}
\multiput(921.17,194.00)(13.000,13.834){2}{\rule{0.400pt}{0.281pt}}
\multiput(935.58,209.00)(0.492,0.669){21}{\rule{0.119pt}{0.633pt}}
\multiput(934.17,209.00)(12.000,14.685){2}{\rule{0.400pt}{0.317pt}}
\multiput(947.58,225.00)(0.493,0.616){23}{\rule{0.119pt}{0.592pt}}
\multiput(946.17,225.00)(13.000,14.771){2}{\rule{0.400pt}{0.296pt}}
\multiput(960.58,241.00)(0.493,0.655){23}{\rule{0.119pt}{0.623pt}}
\multiput(959.17,241.00)(13.000,15.707){2}{\rule{0.400pt}{0.312pt}}
\multiput(973.58,258.00)(0.492,0.669){21}{\rule{0.119pt}{0.633pt}}
\multiput(972.17,258.00)(12.000,14.685){2}{\rule{0.400pt}{0.317pt}}
\multiput(985.58,274.00)(0.493,0.655){23}{\rule{0.119pt}{0.623pt}}
\multiput(984.17,274.00)(13.000,15.707){2}{\rule{0.400pt}{0.312pt}}
\multiput(998.58,291.00)(0.492,0.625){21}{\rule{0.119pt}{0.600pt}}
\multiput(997.17,291.00)(12.000,13.755){2}{\rule{0.400pt}{0.300pt}}
\multiput(1010.58,306.00)(0.493,0.616){23}{\rule{0.119pt}{0.592pt}}
\multiput(1009.17,306.00)(13.000,14.771){2}{\rule{0.400pt}{0.296pt}}
\multiput(1023.58,322.00)(0.493,0.536){23}{\rule{0.119pt}{0.531pt}}
\multiput(1022.17,322.00)(13.000,12.898){2}{\rule{0.400pt}{0.265pt}}
\multiput(1036.58,336.00)(0.492,0.582){21}{\rule{0.119pt}{0.567pt}}
\multiput(1035.17,336.00)(12.000,12.824){2}{\rule{0.400pt}{0.283pt}}
\multiput(1048.00,350.58)(0.539,0.492){21}{\rule{0.533pt}{0.119pt}}
\multiput(1048.00,349.17)(11.893,12.000){2}{\rule{0.267pt}{0.400pt}}
\multiput(1061.00,362.58)(0.543,0.492){19}{\rule{0.536pt}{0.118pt}}
\multiput(1061.00,361.17)(10.887,11.000){2}{\rule{0.268pt}{0.400pt}}
\multiput(1073.00,373.59)(0.728,0.489){15}{\rule{0.678pt}{0.118pt}}
\multiput(1073.00,372.17)(11.593,9.000){2}{\rule{0.339pt}{0.400pt}}
\multiput(1086.00,382.59)(0.950,0.485){11}{\rule{0.843pt}{0.117pt}}
\multiput(1086.00,381.17)(11.251,7.000){2}{\rule{0.421pt}{0.400pt}}
\multiput(1099.00,389.59)(1.033,0.482){9}{\rule{0.900pt}{0.116pt}}
\multiput(1099.00,388.17)(10.132,6.000){2}{\rule{0.450pt}{0.400pt}}
\multiput(1111.00,395.60)(1.797,0.468){5}{\rule{1.400pt}{0.113pt}}
\multiput(1111.00,394.17)(10.094,4.000){2}{\rule{0.700pt}{0.400pt}}
\put(1124,398.67){\rule{2.891pt}{0.400pt}}
\multiput(1124.00,398.17)(6.000,1.000){2}{\rule{1.445pt}{0.400pt}}
\multiput(1149.00,398.95)(2.695,-0.447){3}{\rule{1.833pt}{0.108pt}}
\multiput(1149.00,399.17)(9.195,-3.000){2}{\rule{0.917pt}{0.400pt}}
\multiput(1162.00,395.93)(1.267,-0.477){7}{\rule{1.060pt}{0.115pt}}
\multiput(1162.00,396.17)(9.800,-5.000){2}{\rule{0.530pt}{0.400pt}}
\multiput(1174.00,390.93)(0.950,-0.485){11}{\rule{0.843pt}{0.117pt}}
\multiput(1174.00,391.17)(11.251,-7.000){2}{\rule{0.421pt}{0.400pt}}
\multiput(1187.00,383.92)(0.600,-0.491){17}{\rule{0.580pt}{0.118pt}}
\multiput(1187.00,384.17)(10.796,-10.000){2}{\rule{0.290pt}{0.400pt}}
\multiput(1199.00,373.92)(0.590,-0.492){19}{\rule{0.573pt}{0.118pt}}
\multiput(1199.00,374.17)(11.811,-11.000){2}{\rule{0.286pt}{0.400pt}}
\multiput(1212.58,361.80)(0.493,-0.536){23}{\rule{0.119pt}{0.531pt}}
\multiput(1211.17,362.90)(13.000,-12.898){2}{\rule{0.400pt}{0.265pt}}
\multiput(1225.58,347.37)(0.492,-0.669){21}{\rule{0.119pt}{0.633pt}}
\multiput(1224.17,348.69)(12.000,-14.685){2}{\rule{0.400pt}{0.317pt}}
\multiput(1237.58,331.29)(0.493,-0.695){23}{\rule{0.119pt}{0.654pt}}
\multiput(1236.17,332.64)(13.000,-16.643){2}{\rule{0.400pt}{0.327pt}}
\multiput(1250.58,313.16)(0.493,-0.734){23}{\rule{0.119pt}{0.685pt}}
\multiput(1249.17,314.58)(13.000,-17.579){2}{\rule{0.400pt}{0.342pt}}
\multiput(1263.58,293.68)(0.492,-0.884){21}{\rule{0.119pt}{0.800pt}}
\multiput(1262.17,295.34)(12.000,-19.340){2}{\rule{0.400pt}{0.400pt}}
\multiput(1275.58,272.65)(0.493,-0.893){23}{\rule{0.119pt}{0.808pt}}
\multiput(1274.17,274.32)(13.000,-21.324){2}{\rule{0.400pt}{0.404pt}}
\multiput(1288.58,249.26)(0.492,-1.013){21}{\rule{0.119pt}{0.900pt}}
\multiput(1287.17,251.13)(12.000,-22.132){2}{\rule{0.400pt}{0.450pt}}
\multiput(1300.58,225.39)(0.493,-0.972){23}{\rule{0.119pt}{0.869pt}}
\multiput(1299.17,227.20)(13.000,-23.196){2}{\rule{0.400pt}{0.435pt}}
\multiput(1313.58,200.26)(0.493,-1.012){23}{\rule{0.119pt}{0.900pt}}
\multiput(1312.17,202.13)(13.000,-24.132){2}{\rule{0.400pt}{0.450pt}}
\multiput(1326.58,173.85)(0.492,-1.142){21}{\rule{0.119pt}{1.000pt}}
\multiput(1325.17,175.92)(12.000,-24.924){2}{\rule{0.400pt}{0.500pt}}
\multiput(1338.58,147.26)(0.491,-1.017){17}{\rule{0.118pt}{0.900pt}}
\multiput(1337.17,149.13)(10.000,-18.132){2}{\rule{0.400pt}{0.450pt}}
\put(1136.0,400.0){\rule[-0.200pt]{3.132pt}{0.400pt}}
\put(191.0,131.0){\rule[-0.200pt]{0.400pt}{175.375pt}}
\put(191.0,131.0){\rule[-0.200pt]{300.643pt}{0.400pt}}
\put(1439.0,131.0){\rule[-0.200pt]{0.400pt}{175.375pt}}
\put(191.0,859.0){\rule[-0.200pt]{300.643pt}{0.400pt}}
\end{picture}

        \caption{Y axe scaled logarithmic }
        \label{fig:c}

        % GNUPLOT: LaTeX picture
\setlength{\unitlength}{0.240900pt}
\ifx\plotpoint\undefined\newsavebox{\plotpoint}\fi
\sbox{\plotpoint}{\rule[-0.200pt]{0.400pt}{0.400pt}}%
\begin{picture}(1500,900)(0,0)
\sbox{\plotpoint}{\rule[-0.200pt]{0.400pt}{0.400pt}}%
\put(191.0,131.0){\rule[-0.200pt]{4.818pt}{0.400pt}}
\put(171,131){\makebox(0,0)[r]{ 0.01}}
\put(1419.0,131.0){\rule[-0.200pt]{4.818pt}{0.400pt}}
\put(191.0,186.0){\rule[-0.200pt]{2.409pt}{0.400pt}}
\put(1429.0,186.0){\rule[-0.200pt]{2.409pt}{0.400pt}}
\put(191.0,218.0){\rule[-0.200pt]{2.409pt}{0.400pt}}
\put(1429.0,218.0){\rule[-0.200pt]{2.409pt}{0.400pt}}
\put(191.0,241.0){\rule[-0.200pt]{2.409pt}{0.400pt}}
\put(1429.0,241.0){\rule[-0.200pt]{2.409pt}{0.400pt}}
\put(191.0,258.0){\rule[-0.200pt]{2.409pt}{0.400pt}}
\put(1429.0,258.0){\rule[-0.200pt]{2.409pt}{0.400pt}}
\put(191.0,273.0){\rule[-0.200pt]{2.409pt}{0.400pt}}
\put(1429.0,273.0){\rule[-0.200pt]{2.409pt}{0.400pt}}
\put(191.0,285.0){\rule[-0.200pt]{2.409pt}{0.400pt}}
\put(1429.0,285.0){\rule[-0.200pt]{2.409pt}{0.400pt}}
\put(191.0,295.0){\rule[-0.200pt]{2.409pt}{0.400pt}}
\put(1429.0,295.0){\rule[-0.200pt]{2.409pt}{0.400pt}}
\put(191.0,305.0){\rule[-0.200pt]{2.409pt}{0.400pt}}
\put(1429.0,305.0){\rule[-0.200pt]{2.409pt}{0.400pt}}
\put(191.0,313.0){\rule[-0.200pt]{4.818pt}{0.400pt}}
\put(171,313){\makebox(0,0)[r]{ 0.1}}
\put(1419.0,313.0){\rule[-0.200pt]{4.818pt}{0.400pt}}
\put(191.0,368.0){\rule[-0.200pt]{2.409pt}{0.400pt}}
\put(1429.0,368.0){\rule[-0.200pt]{2.409pt}{0.400pt}}
\put(191.0,400.0){\rule[-0.200pt]{2.409pt}{0.400pt}}
\put(1429.0,400.0){\rule[-0.200pt]{2.409pt}{0.400pt}}
\put(191.0,423.0){\rule[-0.200pt]{2.409pt}{0.400pt}}
\put(1429.0,423.0){\rule[-0.200pt]{2.409pt}{0.400pt}}
\put(191.0,440.0){\rule[-0.200pt]{2.409pt}{0.400pt}}
\put(1429.0,440.0){\rule[-0.200pt]{2.409pt}{0.400pt}}
\put(191.0,455.0){\rule[-0.200pt]{2.409pt}{0.400pt}}
\put(1429.0,455.0){\rule[-0.200pt]{2.409pt}{0.400pt}}
\put(191.0,467.0){\rule[-0.200pt]{2.409pt}{0.400pt}}
\put(1429.0,467.0){\rule[-0.200pt]{2.409pt}{0.400pt}}
\put(191.0,477.0){\rule[-0.200pt]{2.409pt}{0.400pt}}
\put(1429.0,477.0){\rule[-0.200pt]{2.409pt}{0.400pt}}
\put(191.0,487.0){\rule[-0.200pt]{2.409pt}{0.400pt}}
\put(1429.0,487.0){\rule[-0.200pt]{2.409pt}{0.400pt}}
\put(191.0,495.0){\rule[-0.200pt]{4.818pt}{0.400pt}}
\put(171,495){\makebox(0,0)[r]{ 1}}
\put(1419.0,495.0){\rule[-0.200pt]{4.818pt}{0.400pt}}
\put(191.0,550.0){\rule[-0.200pt]{2.409pt}{0.400pt}}
\put(1429.0,550.0){\rule[-0.200pt]{2.409pt}{0.400pt}}
\put(191.0,582.0){\rule[-0.200pt]{2.409pt}{0.400pt}}
\put(1429.0,582.0){\rule[-0.200pt]{2.409pt}{0.400pt}}
\put(191.0,605.0){\rule[-0.200pt]{2.409pt}{0.400pt}}
\put(1429.0,605.0){\rule[-0.200pt]{2.409pt}{0.400pt}}
\put(191.0,622.0){\rule[-0.200pt]{2.409pt}{0.400pt}}
\put(1429.0,622.0){\rule[-0.200pt]{2.409pt}{0.400pt}}
\put(191.0,637.0){\rule[-0.200pt]{2.409pt}{0.400pt}}
\put(1429.0,637.0){\rule[-0.200pt]{2.409pt}{0.400pt}}
\put(191.0,649.0){\rule[-0.200pt]{2.409pt}{0.400pt}}
\put(1429.0,649.0){\rule[-0.200pt]{2.409pt}{0.400pt}}
\put(191.0,659.0){\rule[-0.200pt]{2.409pt}{0.400pt}}
\put(1429.0,659.0){\rule[-0.200pt]{2.409pt}{0.400pt}}
\put(191.0,669.0){\rule[-0.200pt]{2.409pt}{0.400pt}}
\put(1429.0,669.0){\rule[-0.200pt]{2.409pt}{0.400pt}}
\put(191.0,677.0){\rule[-0.200pt]{4.818pt}{0.400pt}}
\put(171,677){\makebox(0,0)[r]{ 10}}
\put(1419.0,677.0){\rule[-0.200pt]{4.818pt}{0.400pt}}
\put(191.0,732.0){\rule[-0.200pt]{2.409pt}{0.400pt}}
\put(1429.0,732.0){\rule[-0.200pt]{2.409pt}{0.400pt}}
\put(191.0,764.0){\rule[-0.200pt]{2.409pt}{0.400pt}}
\put(1429.0,764.0){\rule[-0.200pt]{2.409pt}{0.400pt}}
\put(191.0,787.0){\rule[-0.200pt]{2.409pt}{0.400pt}}
\put(1429.0,787.0){\rule[-0.200pt]{2.409pt}{0.400pt}}
\put(191.0,804.0){\rule[-0.200pt]{2.409pt}{0.400pt}}
\put(1429.0,804.0){\rule[-0.200pt]{2.409pt}{0.400pt}}
\put(191.0,819.0){\rule[-0.200pt]{2.409pt}{0.400pt}}
\put(1429.0,819.0){\rule[-0.200pt]{2.409pt}{0.400pt}}
\put(191.0,831.0){\rule[-0.200pt]{2.409pt}{0.400pt}}
\put(1429.0,831.0){\rule[-0.200pt]{2.409pt}{0.400pt}}
\put(191.0,841.0){\rule[-0.200pt]{2.409pt}{0.400pt}}
\put(1429.0,841.0){\rule[-0.200pt]{2.409pt}{0.400pt}}
\put(191.0,851.0){\rule[-0.200pt]{2.409pt}{0.400pt}}
\put(1429.0,851.0){\rule[-0.200pt]{2.409pt}{0.400pt}}
\put(191.0,859.0){\rule[-0.200pt]{4.818pt}{0.400pt}}
\put(171,859){\makebox(0,0)[r]{ 100}}
\put(1419.0,859.0){\rule[-0.200pt]{4.818pt}{0.400pt}}
\put(191.0,131.0){\rule[-0.200pt]{0.400pt}{4.818pt}}
\put(191,90){\makebox(0,0){ 0.01}}
\put(191.0,839.0){\rule[-0.200pt]{0.400pt}{4.818pt}}
\put(316.0,131.0){\rule[-0.200pt]{0.400pt}{2.409pt}}
\put(316.0,849.0){\rule[-0.200pt]{0.400pt}{2.409pt}}
\put(389.0,131.0){\rule[-0.200pt]{0.400pt}{2.409pt}}
\put(389.0,849.0){\rule[-0.200pt]{0.400pt}{2.409pt}}
\put(441.0,131.0){\rule[-0.200pt]{0.400pt}{2.409pt}}
\put(441.0,849.0){\rule[-0.200pt]{0.400pt}{2.409pt}}
\put(482.0,131.0){\rule[-0.200pt]{0.400pt}{2.409pt}}
\put(482.0,849.0){\rule[-0.200pt]{0.400pt}{2.409pt}}
\put(515.0,131.0){\rule[-0.200pt]{0.400pt}{2.409pt}}
\put(515.0,849.0){\rule[-0.200pt]{0.400pt}{2.409pt}}
\put(543.0,131.0){\rule[-0.200pt]{0.400pt}{2.409pt}}
\put(543.0,849.0){\rule[-0.200pt]{0.400pt}{2.409pt}}
\put(567.0,131.0){\rule[-0.200pt]{0.400pt}{2.409pt}}
\put(567.0,849.0){\rule[-0.200pt]{0.400pt}{2.409pt}}
\put(588.0,131.0){\rule[-0.200pt]{0.400pt}{2.409pt}}
\put(588.0,849.0){\rule[-0.200pt]{0.400pt}{2.409pt}}
\put(607.0,131.0){\rule[-0.200pt]{0.400pt}{4.818pt}}
\put(607,90){\makebox(0,0){ 0.1}}
\put(607.0,839.0){\rule[-0.200pt]{0.400pt}{4.818pt}}
\put(732.0,131.0){\rule[-0.200pt]{0.400pt}{2.409pt}}
\put(732.0,849.0){\rule[-0.200pt]{0.400pt}{2.409pt}}
\put(805.0,131.0){\rule[-0.200pt]{0.400pt}{2.409pt}}
\put(805.0,849.0){\rule[-0.200pt]{0.400pt}{2.409pt}}
\put(857.0,131.0){\rule[-0.200pt]{0.400pt}{2.409pt}}
\put(857.0,849.0){\rule[-0.200pt]{0.400pt}{2.409pt}}
\put(898.0,131.0){\rule[-0.200pt]{0.400pt}{2.409pt}}
\put(898.0,849.0){\rule[-0.200pt]{0.400pt}{2.409pt}}
\put(931.0,131.0){\rule[-0.200pt]{0.400pt}{2.409pt}}
\put(931.0,849.0){\rule[-0.200pt]{0.400pt}{2.409pt}}
\put(959.0,131.0){\rule[-0.200pt]{0.400pt}{2.409pt}}
\put(959.0,849.0){\rule[-0.200pt]{0.400pt}{2.409pt}}
\put(983.0,131.0){\rule[-0.200pt]{0.400pt}{2.409pt}}
\put(983.0,849.0){\rule[-0.200pt]{0.400pt}{2.409pt}}
\put(1004.0,131.0){\rule[-0.200pt]{0.400pt}{2.409pt}}
\put(1004.0,849.0){\rule[-0.200pt]{0.400pt}{2.409pt}}
\put(1023.0,131.0){\rule[-0.200pt]{0.400pt}{4.818pt}}
\put(1023,90){\makebox(0,0){ 1}}
\put(1023.0,839.0){\rule[-0.200pt]{0.400pt}{4.818pt}}
\put(1148.0,131.0){\rule[-0.200pt]{0.400pt}{2.409pt}}
\put(1148.0,849.0){\rule[-0.200pt]{0.400pt}{2.409pt}}
\put(1221.0,131.0){\rule[-0.200pt]{0.400pt}{2.409pt}}
\put(1221.0,849.0){\rule[-0.200pt]{0.400pt}{2.409pt}}
\put(1273.0,131.0){\rule[-0.200pt]{0.400pt}{2.409pt}}
\put(1273.0,849.0){\rule[-0.200pt]{0.400pt}{2.409pt}}
\put(1314.0,131.0){\rule[-0.200pt]{0.400pt}{2.409pt}}
\put(1314.0,849.0){\rule[-0.200pt]{0.400pt}{2.409pt}}
\put(1347.0,131.0){\rule[-0.200pt]{0.400pt}{2.409pt}}
\put(1347.0,849.0){\rule[-0.200pt]{0.400pt}{2.409pt}}
\put(1375.0,131.0){\rule[-0.200pt]{0.400pt}{2.409pt}}
\put(1375.0,849.0){\rule[-0.200pt]{0.400pt}{2.409pt}}
\put(1399.0,131.0){\rule[-0.200pt]{0.400pt}{2.409pt}}
\put(1399.0,849.0){\rule[-0.200pt]{0.400pt}{2.409pt}}
\put(1420.0,131.0){\rule[-0.200pt]{0.400pt}{2.409pt}}
\put(1420.0,849.0){\rule[-0.200pt]{0.400pt}{2.409pt}}
\put(1439.0,131.0){\rule[-0.200pt]{0.400pt}{4.818pt}}
\put(1439,90){\makebox(0,0){ 10}}
\put(1439.0,839.0){\rule[-0.200pt]{0.400pt}{4.818pt}}
\put(191.0,131.0){\rule[-0.200pt]{0.400pt}{175.375pt}}
\put(191.0,131.0){\rule[-0.200pt]{300.643pt}{0.400pt}}
\put(1439.0,131.0){\rule[-0.200pt]{0.400pt}{175.375pt}}
\put(191.0,859.0){\rule[-0.200pt]{300.643pt}{0.400pt}}
\put(30,495){\makebox(0,0){y}}
\put(815,29){\makebox(0,0){x}}
\put(1279,819){\makebox(0,0)[r]{2**( 2 - x + 4 * sin(x) )}}
\put(1299.0,819.0){\rule[-0.200pt]{24.090pt}{0.400pt}}
\put(191,606){\usebox{\plotpoint}}
\put(216,605.67){\rule{3.132pt}{0.400pt}}
\multiput(216.00,605.17)(6.500,1.000){2}{\rule{1.566pt}{0.400pt}}
\put(191.0,606.0){\rule[-0.200pt]{6.022pt}{0.400pt}}
\put(292,606.67){\rule{2.891pt}{0.400pt}}
\multiput(292.00,606.17)(6.000,1.000){2}{\rule{1.445pt}{0.400pt}}
\put(229.0,607.0){\rule[-0.200pt]{15.177pt}{0.400pt}}
\put(342,607.67){\rule{3.132pt}{0.400pt}}
\multiput(342.00,607.17)(6.500,1.000){2}{\rule{1.566pt}{0.400pt}}
\put(304.0,608.0){\rule[-0.200pt]{9.154pt}{0.400pt}}
\put(380,608.67){\rule{3.132pt}{0.400pt}}
\multiput(380.00,608.17)(6.500,1.000){2}{\rule{1.566pt}{0.400pt}}
\put(355.0,609.0){\rule[-0.200pt]{6.022pt}{0.400pt}}
\put(418,609.67){\rule{3.132pt}{0.400pt}}
\multiput(418.00,609.17)(6.500,1.000){2}{\rule{1.566pt}{0.400pt}}
\put(393.0,610.0){\rule[-0.200pt]{6.022pt}{0.400pt}}
\put(443,610.67){\rule{3.132pt}{0.400pt}}
\multiput(443.00,610.17)(6.500,1.000){2}{\rule{1.566pt}{0.400pt}}
\put(431.0,611.0){\rule[-0.200pt]{2.891pt}{0.400pt}}
\put(468,611.67){\rule{3.132pt}{0.400pt}}
\multiput(468.00,611.17)(6.500,1.000){2}{\rule{1.566pt}{0.400pt}}
\put(456.0,612.0){\rule[-0.200pt]{2.891pt}{0.400pt}}
\put(494,612.67){\rule{2.891pt}{0.400pt}}
\multiput(494.00,612.17)(6.000,1.000){2}{\rule{1.445pt}{0.400pt}}
\put(506,613.67){\rule{3.132pt}{0.400pt}}
\multiput(506.00,613.17)(6.500,1.000){2}{\rule{1.566pt}{0.400pt}}
\put(481.0,613.0){\rule[-0.200pt]{3.132pt}{0.400pt}}
\put(531,614.67){\rule{3.132pt}{0.400pt}}
\multiput(531.00,614.17)(6.500,1.000){2}{\rule{1.566pt}{0.400pt}}
\put(544,615.67){\rule{3.132pt}{0.400pt}}
\multiput(544.00,615.17)(6.500,1.000){2}{\rule{1.566pt}{0.400pt}}
\put(557,616.67){\rule{2.891pt}{0.400pt}}
\multiput(557.00,616.17)(6.000,1.000){2}{\rule{1.445pt}{0.400pt}}
\put(569,617.67){\rule{3.132pt}{0.400pt}}
\multiput(569.00,617.17)(6.500,1.000){2}{\rule{1.566pt}{0.400pt}}
\put(582,618.67){\rule{2.891pt}{0.400pt}}
\multiput(582.00,618.17)(6.000,1.000){2}{\rule{1.445pt}{0.400pt}}
\put(594,619.67){\rule{3.132pt}{0.400pt}}
\multiput(594.00,619.17)(6.500,1.000){2}{\rule{1.566pt}{0.400pt}}
\put(607,620.67){\rule{3.132pt}{0.400pt}}
\multiput(607.00,620.17)(6.500,1.000){2}{\rule{1.566pt}{0.400pt}}
\put(620,621.67){\rule{2.891pt}{0.400pt}}
\multiput(620.00,621.17)(6.000,1.000){2}{\rule{1.445pt}{0.400pt}}
\put(632,623.17){\rule{2.700pt}{0.400pt}}
\multiput(632.00,622.17)(7.396,2.000){2}{\rule{1.350pt}{0.400pt}}
\put(645,624.67){\rule{2.891pt}{0.400pt}}
\multiput(645.00,624.17)(6.000,1.000){2}{\rule{1.445pt}{0.400pt}}
\put(657,626.17){\rule{2.700pt}{0.400pt}}
\multiput(657.00,625.17)(7.396,2.000){2}{\rule{1.350pt}{0.400pt}}
\put(670,627.67){\rule{3.132pt}{0.400pt}}
\multiput(670.00,627.17)(6.500,1.000){2}{\rule{1.566pt}{0.400pt}}
\put(683,629.17){\rule{2.500pt}{0.400pt}}
\multiput(683.00,628.17)(6.811,2.000){2}{\rule{1.250pt}{0.400pt}}
\put(695,631.17){\rule{2.700pt}{0.400pt}}
\multiput(695.00,630.17)(7.396,2.000){2}{\rule{1.350pt}{0.400pt}}
\put(708,633.17){\rule{2.500pt}{0.400pt}}
\multiput(708.00,632.17)(6.811,2.000){2}{\rule{1.250pt}{0.400pt}}
\put(720,635.17){\rule{2.700pt}{0.400pt}}
\multiput(720.00,634.17)(7.396,2.000){2}{\rule{1.350pt}{0.400pt}}
\multiput(733.00,637.61)(2.695,0.447){3}{\rule{1.833pt}{0.108pt}}
\multiput(733.00,636.17)(9.195,3.000){2}{\rule{0.917pt}{0.400pt}}
\put(746,640.17){\rule{2.500pt}{0.400pt}}
\multiput(746.00,639.17)(6.811,2.000){2}{\rule{1.250pt}{0.400pt}}
\multiput(758.00,642.61)(2.695,0.447){3}{\rule{1.833pt}{0.108pt}}
\multiput(758.00,641.17)(9.195,3.000){2}{\rule{0.917pt}{0.400pt}}
\multiput(771.00,645.61)(2.472,0.447){3}{\rule{1.700pt}{0.108pt}}
\multiput(771.00,644.17)(8.472,3.000){2}{\rule{0.850pt}{0.400pt}}
\multiput(783.00,648.61)(2.695,0.447){3}{\rule{1.833pt}{0.108pt}}
\multiput(783.00,647.17)(9.195,3.000){2}{\rule{0.917pt}{0.400pt}}
\multiput(796.00,651.61)(2.695,0.447){3}{\rule{1.833pt}{0.108pt}}
\multiput(796.00,650.17)(9.195,3.000){2}{\rule{0.917pt}{0.400pt}}
\multiput(809.00,654.61)(2.472,0.447){3}{\rule{1.700pt}{0.108pt}}
\multiput(809.00,653.17)(8.472,3.000){2}{\rule{0.850pt}{0.400pt}}
\multiput(821.00,657.60)(1.797,0.468){5}{\rule{1.400pt}{0.113pt}}
\multiput(821.00,656.17)(10.094,4.000){2}{\rule{0.700pt}{0.400pt}}
\multiput(834.00,661.60)(1.797,0.468){5}{\rule{1.400pt}{0.113pt}}
\multiput(834.00,660.17)(10.094,4.000){2}{\rule{0.700pt}{0.400pt}}
\multiput(847.00,665.60)(1.651,0.468){5}{\rule{1.300pt}{0.113pt}}
\multiput(847.00,664.17)(9.302,4.000){2}{\rule{0.650pt}{0.400pt}}
\multiput(859.00,669.60)(1.797,0.468){5}{\rule{1.400pt}{0.113pt}}
\multiput(859.00,668.17)(10.094,4.000){2}{\rule{0.700pt}{0.400pt}}
\multiput(872.00,673.60)(1.651,0.468){5}{\rule{1.300pt}{0.113pt}}
\multiput(872.00,672.17)(9.302,4.000){2}{\rule{0.650pt}{0.400pt}}
\multiput(884.00,677.59)(1.378,0.477){7}{\rule{1.140pt}{0.115pt}}
\multiput(884.00,676.17)(10.634,5.000){2}{\rule{0.570pt}{0.400pt}}
\multiput(897.00,682.59)(1.378,0.477){7}{\rule{1.140pt}{0.115pt}}
\multiput(897.00,681.17)(10.634,5.000){2}{\rule{0.570pt}{0.400pt}}
\multiput(910.00,687.59)(1.267,0.477){7}{\rule{1.060pt}{0.115pt}}
\multiput(910.00,686.17)(9.800,5.000){2}{\rule{0.530pt}{0.400pt}}
\multiput(922.00,692.59)(1.378,0.477){7}{\rule{1.140pt}{0.115pt}}
\multiput(922.00,691.17)(10.634,5.000){2}{\rule{0.570pt}{0.400pt}}
\multiput(935.00,697.59)(1.033,0.482){9}{\rule{0.900pt}{0.116pt}}
\multiput(935.00,696.17)(10.132,6.000){2}{\rule{0.450pt}{0.400pt}}
\multiput(947.00,703.59)(1.378,0.477){7}{\rule{1.140pt}{0.115pt}}
\multiput(947.00,702.17)(10.634,5.000){2}{\rule{0.570pt}{0.400pt}}
\multiput(960.00,708.59)(1.123,0.482){9}{\rule{0.967pt}{0.116pt}}
\multiput(960.00,707.17)(10.994,6.000){2}{\rule{0.483pt}{0.400pt}}
\multiput(973.00,714.59)(1.267,0.477){7}{\rule{1.060pt}{0.115pt}}
\multiput(973.00,713.17)(9.800,5.000){2}{\rule{0.530pt}{0.400pt}}
\multiput(985.00,719.59)(1.378,0.477){7}{\rule{1.140pt}{0.115pt}}
\multiput(985.00,718.17)(10.634,5.000){2}{\rule{0.570pt}{0.400pt}}
\multiput(998.00,724.59)(1.267,0.477){7}{\rule{1.060pt}{0.115pt}}
\multiput(998.00,723.17)(9.800,5.000){2}{\rule{0.530pt}{0.400pt}}
\multiput(1010.00,729.59)(1.378,0.477){7}{\rule{1.140pt}{0.115pt}}
\multiput(1010.00,728.17)(10.634,5.000){2}{\rule{0.570pt}{0.400pt}}
\multiput(1023.00,734.60)(1.797,0.468){5}{\rule{1.400pt}{0.113pt}}
\multiput(1023.00,733.17)(10.094,4.000){2}{\rule{0.700pt}{0.400pt}}
\multiput(1036.00,738.60)(1.651,0.468){5}{\rule{1.300pt}{0.113pt}}
\multiput(1036.00,737.17)(9.302,4.000){2}{\rule{0.650pt}{0.400pt}}
\put(1048,742.17){\rule{2.700pt}{0.400pt}}
\multiput(1048.00,741.17)(7.396,2.000){2}{\rule{1.350pt}{0.400pt}}
\put(1061,743.67){\rule{2.891pt}{0.400pt}}
\multiput(1061.00,743.17)(6.000,1.000){2}{\rule{1.445pt}{0.400pt}}
\put(1073,743.17){\rule{2.700pt}{0.400pt}}
\multiput(1073.00,744.17)(7.396,-2.000){2}{\rule{1.350pt}{0.400pt}}
\multiput(1086.00,741.95)(2.695,-0.447){3}{\rule{1.833pt}{0.108pt}}
\multiput(1086.00,742.17)(9.195,-3.000){2}{\rule{0.917pt}{0.400pt}}
\multiput(1099.00,738.93)(1.033,-0.482){9}{\rule{0.900pt}{0.116pt}}
\multiput(1099.00,739.17)(10.132,-6.000){2}{\rule{0.450pt}{0.400pt}}
\multiput(1111.00,732.93)(0.728,-0.489){15}{\rule{0.678pt}{0.118pt}}
\multiput(1111.00,733.17)(11.593,-9.000){2}{\rule{0.339pt}{0.400pt}}
\multiput(1124.58,722.65)(0.492,-0.582){21}{\rule{0.119pt}{0.567pt}}
\multiput(1123.17,723.82)(12.000,-12.824){2}{\rule{0.400pt}{0.283pt}}
\multiput(1136.58,708.29)(0.493,-0.695){23}{\rule{0.119pt}{0.654pt}}
\multiput(1135.17,709.64)(13.000,-16.643){2}{\rule{0.400pt}{0.327pt}}
\multiput(1149.58,689.52)(0.493,-0.933){23}{\rule{0.119pt}{0.838pt}}
\multiput(1148.17,691.26)(13.000,-22.260){2}{\rule{0.400pt}{0.419pt}}
\multiput(1162.58,664.57)(0.492,-1.229){21}{\rule{0.119pt}{1.067pt}}
\multiput(1161.17,666.79)(12.000,-26.786){2}{\rule{0.400pt}{0.533pt}}
\multiput(1174.58,634.99)(0.493,-1.408){23}{\rule{0.119pt}{1.208pt}}
\multiput(1173.17,637.49)(13.000,-33.493){2}{\rule{0.400pt}{0.604pt}}
\multiput(1187.58,597.64)(0.492,-1.832){21}{\rule{0.119pt}{1.533pt}}
\multiput(1186.17,600.82)(12.000,-39.817){2}{\rule{0.400pt}{0.767pt}}
\multiput(1199.58,554.33)(0.493,-1.924){23}{\rule{0.119pt}{1.608pt}}
\multiput(1198.17,557.66)(13.000,-45.663){2}{\rule{0.400pt}{0.804pt}}
\multiput(1212.58,504.43)(0.493,-2.201){23}{\rule{0.119pt}{1.823pt}}
\multiput(1211.17,508.22)(13.000,-52.216){2}{\rule{0.400pt}{0.912pt}}
\multiput(1225.58,447.28)(0.492,-2.564){21}{\rule{0.119pt}{2.100pt}}
\multiput(1224.17,451.64)(12.000,-55.641){2}{\rule{0.400pt}{1.050pt}}
\multiput(1237.58,387.54)(0.493,-2.479){23}{\rule{0.119pt}{2.038pt}}
\multiput(1236.17,391.77)(13.000,-58.769){2}{\rule{0.400pt}{1.019pt}}
\multiput(1250.58,324.54)(0.493,-2.479){23}{\rule{0.119pt}{2.038pt}}
\multiput(1249.17,328.77)(13.000,-58.769){2}{\rule{0.400pt}{1.019pt}}
\multiput(1263.58,261.56)(0.492,-2.478){21}{\rule{0.119pt}{2.033pt}}
\multiput(1262.17,265.78)(12.000,-53.780){2}{\rule{0.400pt}{1.017pt}}
\multiput(1275.58,205.45)(0.493,-1.884){23}{\rule{0.119pt}{1.577pt}}
\multiput(1274.17,208.73)(13.000,-44.727){2}{\rule{0.400pt}{0.788pt}}
\multiput(1288.58,159.16)(0.492,-1.358){21}{\rule{0.119pt}{1.167pt}}
\multiput(1287.17,161.58)(12.000,-29.579){2}{\rule{0.400pt}{0.583pt}}
\put(1300,130.67){\rule{0.241pt}{0.400pt}}
\multiput(1300.00,131.17)(0.500,-1.000){2}{\rule{0.120pt}{0.400pt}}
\multiput(1323.00,131.61)(0.462,0.447){3}{\rule{0.500pt}{0.108pt}}
\multiput(1323.00,130.17)(1.962,3.000){2}{\rule{0.250pt}{0.400pt}}
\multiput(1326.58,134.00)(0.492,1.703){21}{\rule{0.119pt}{1.433pt}}
\multiput(1325.17,134.00)(12.000,37.025){2}{\rule{0.400pt}{0.717pt}}
\multiput(1338.58,174.00)(0.493,2.439){23}{\rule{0.119pt}{2.008pt}}
\multiput(1337.17,174.00)(13.000,57.833){2}{\rule{0.400pt}{1.004pt}}
\multiput(1351.58,236.00)(0.492,3.081){21}{\rule{0.119pt}{2.500pt}}
\multiput(1350.17,236.00)(12.000,66.811){2}{\rule{0.400pt}{1.250pt}}
\multiput(1363.58,308.00)(0.493,2.479){23}{\rule{0.119pt}{2.038pt}}
\multiput(1362.17,308.00)(13.000,58.769){2}{\rule{0.400pt}{1.019pt}}
\multiput(1376.58,371.00)(0.493,1.131){23}{\rule{0.119pt}{0.992pt}}
\multiput(1375.17,371.00)(13.000,26.940){2}{\rule{0.400pt}{0.496pt}}
\multiput(1389.58,395.71)(0.492,-1.186){21}{\rule{0.119pt}{1.033pt}}
\multiput(1388.17,397.86)(12.000,-25.855){2}{\rule{0.400pt}{0.517pt}}
\multiput(1401.58,359.07)(0.493,-3.867){23}{\rule{0.119pt}{3.115pt}}
\multiput(1400.17,365.53)(13.000,-91.534){2}{\rule{0.400pt}{1.558pt}}
\multiput(1414.58,252.00)(0.492,-6.722){19}{\rule{0.118pt}{5.300pt}}
\multiput(1413.17,263.00)(11.000,-132.000){2}{\rule{0.400pt}{2.650pt}}
\put(519.0,615.0){\rule[-0.200pt]{2.891pt}{0.400pt}}
\put(191.0,131.0){\rule[-0.200pt]{0.400pt}{175.375pt}}
\put(191.0,131.0){\rule[-0.200pt]{300.643pt}{0.400pt}}
\put(1439.0,131.0){\rule[-0.200pt]{0.400pt}{175.375pt}}
\put(191.0,859.0){\rule[-0.200pt]{300.643pt}{0.400pt}}
\end{picture}

        \caption{X and Y axes scaled logarithmic }
        \label{fig:d}

\end{figure}


\section{Assignment 26: Resilient-PROP algorithm}

The idea of RPROP algorithm~\cite{riedmiller1993direct} is to change the size of the weight-update $\Delta w_{ij}$ 
without considering the size of the partial derivative. The algorithms 
which are proposed to deal with the problem of appropiate weight-update by doing
some sort of parameters adaptation during learning still have some failures since that
the size of the actually taken weight-step $\Delta w_{ij}$ is not only depending
on the learning rate, but also on the partial derivative $ \frac{\partial E}{\partial w_{ij}} $.
The above problem can lead to oscillation, preventing the error to fall below a certain value.

Detail,

\begin{equation}
    ^{p}\Delta_{ij} = \left\{ 
                        \begin{array}{ll}
                            \eta^{+} \cdot\: ^{p-1}\Delta_{ij}, & \:if\: ^{p-1}(\frac{\partial E}{\partial w_{ij}}) \:\cdot\: ^{p}(\frac{\partial E}{\partial w_{ij}}) > 0 \\
                            \eta^{-} \cdot\: ^{p-1}\Delta_{ij}, & \:if\: ^{p-1}(\frac{\partial E}{\partial w_{ij}}) \:\cdot\: ^{p}(\frac{\partial E}{\partial w_{ij}}) < 0 \\
                                             ^{p-1}\Delta_{ij}, & otherwise
                        \end{array}
                      \right.
\end{equation}

Where, $ 0 < \eta^{-} < 1 < \eta^{+} $, here $\eta^{-}$ is the decrease factor
and $\eta^{+}$ is the increase factor.

\begin{equation}
    ^{p}\Delta w_{ij} = \left\{
                            \begin{array}{ll}
                                -\:^{p}\Delta_{ij}, & \:if\: ^{p}(\frac{\partial E}{\partial w_{ij}}) > 0 \\
                                +\:^{p}\Delta_{ij}, & \:if\: ^{p}(\frac{\partial e}{\partial w_{ij}}) < 0 \\
                                0, & otherwise $$ \\
                                -\:^{p-1} \Delta w_{ij}, & \:if\: ^{p-1}(\frac{\partial E}{\partial w_{ij}}) \:\cdot\: ^{p}(\frac{\partial E}{\partial w_{ij}}) < 0 \\
                            \end{array}
                        \right. \\
\end{equation}

$$ \:if\: ^{p-1}(\frac{\partial E}{\partial w_{ij}}) \:\cdot\: ^{p}(\frac{\partial E}{\partial w_{ij}}) < 0, ^{p}(\frac{\partial E}{\partial w_{ij}}) = 0 $$
$$ ^{p-1}w_{ij} =\: ^{p}w_{ij} \:+\: ^{p}\Delta w_{ij} $$


If gradient $\frac{\partial E}{\partial w_{ij}}$ retains its sign, the update-value is
slightly increased in order to accelerate convergence in shallow regions.

\begin{equation}
    \Rightarrow\: ^{p}\Delta_{ij} =\: \eta^{+} \:\cdot\: ^{p-1}\Delta_{ij}, \:if\: ^{p-1}(\frac{\partial E}{\partial w_{ij}}) \:\cdot\: ^{p}(\frac{\partial E}{\partial w_{ij}}) > 0\
\end{equation}

\begin{equation}
    \Rightarrow\: ^{p}\Delta w_{ij} = \left\{
                                        \begin{array}{ll}
                                            -\:^{p}\Delta_{ij}, \:if\: derivative > 0\\
                                            +\:^{p}\Delta_{ij}, \:if\: derivative < 0
                                        \end{array}
                                    \right.
\end{equation}

If gradient changes its sign, that indicates that the last update was too big
and the algorithm has jumped over a local minimun, the update-value $ \Delta_{ij} $
is decreased by the factor $ \eta^{-} $.

$$ \Rightarrow\: ^{p}\Delta_{ij} =\: \eta^{-} \:\cdot\: ^{p-1}\Delta_{ij} $$

and the previous weight-update is reverted which is denoted "back-tracking" weight-step.

$$ ^{p}\Delta w_{ij} =\: -\:^{p-1}\Delta w_{ij} $$

$$ ^{p}(\frac{\partial E}{\partial w_{ij}}) =\: 0 $$

% reference to biblatex
\bibliographystyle{ieeetr}
\bibliography{ex05}

\section{Assignment 27}

\subsection{Advantages of momentum}

\begin{itemize}

    \item Avoid oscillation: Single step learning can lead to oscillation when in steep
valleys, because of the large gradient. In contrast, cumulative learning can
reduce the step size, by adding the weight change of previous step.

    \item Accelerate on plateaus: Single step learning keeps its speed of covergence even
if the step is on flat plateaus, but cumulative learning increases the step
size on flat plateaus by adding the weight change of previous step to the new
one.

\end{itemize}

\subsection{Disadvantages of momentum}

\begin{itemize}

    \item Setting the momentum parameter \textit{too high} can create a risk of overshooting the
minimum.

    \item Setting the momentum parameter \textit{too low} can not reliably avoid local minima, and
slow down the speed of covergence.

\end{itemize}
\newpage

\section{Assignment 28}

\subsection{MLP 8-3-8 task}

$$ X \in 1-out-of-8\:binary\:coding $$

$$ Y = X $$

A three hidden layer whose neurons have binary output can represent 8
combinations. See table \ref{comb}.

\begin{table}[H]
    \centering
    \begin{tabular}{l c r}
        0& 0 &0 \\
        0& 0 &1 \\
        0& 1 &0 \\
        0& 1 &1 \\
        1& 0 &0 \\
        1& 0 &1 \\
        1& 1 &0 \\
        1& 1 &1 \\
    \end{tabular}
    \caption{Different outputs for a layer with three neurons and binary output}
    \label{comb}
\end{table}

So, it is enough to have three neurons in the hidden layer to emulate a
encoder-decoder.

\subsection{MLP 8-2-8 task}

Given a hidden layer $h$ with two neurons:

$$ net_{h} = w_{0} + w_{1} \cdot h_{1} +  w_{2} \cdot h_{2} $$

We can draw a line to separate eight points using a function like this:

\begin{equation}
    h_{2} = \begin{array}{l}
                    - \frac{w_{0} + w_{1}}{w_{2}}\\
                    - \frac{w_{1}}{w_{2}} \cdot h_{1} - \frac{w_{0}}{w_{2}}
            \end{array}
\end{equation}

\bigskip
\bigskip

\subsection{MLP 8-1-8 task}

Eight different states are eight different positions in (out h). The net
between hidden layer and output layer can implement a linear separation.
However it does not assure that is linear separable from all others.

\section{Assignment 29}

\begin{equation}
    \begin{aligned}
        \frac{\partial E}{\partial x_{n}} \:=\: & \frac{\partial E}{\partial net_{n}} \cdot                                                            &                                                              & \mathbf{ \frac{\partial net_{h}}{\partial x_{n}} }               \\[0.68em]
                                                & \mathrel{\makebox[\widthof{=}]{\vdots}}                                                              &                                                              & \frac{\partial}{\partial x_{n}} \sum_{i=0}^{n}x_{i} \cdot w_{ih} \\[0.68em]
                                                & \mathrel{\makebox[\widthof{=}]{\vdots}}                                                              &                                                              & w_{nh} \: \textit{(when i = n)}                                  \\[0.68em]
                                          \:=\: & \frac{\partial E}{\partial net_{n}}                                                                  &                                                              & \cdot w_{nh}                                                     \\[0.68em]
                                          \:=\: & \frac{\partial E}{\partial out_{n}}                                                                  & \cdot \frac{\partial out_{h}}{\partial net_{h}}              & \cdot w_{nh}                                                     \\[0.68em]
                                                & \mathrel{\makebox[\widthof{=}]{\vdots}}                                                              & \frac{\partial f(net_{h})}{\partial net_{h}} = f'(net_{h})   &                                                                  \\[0.68em]
                                          \:=\: & \frac{\partial E}{\partial out_{n}}                                                                  & \cdot f'(net_{h})                                            & \cdot w_{nh}                                                     \\[0.68em]
                                          \:=\: & \sum_{k=1}^{K} \frac{\partial E}{\partial \underline{net}_{k}} \cdot \frac{\partial \underline{net}_{k}}{\partial out_{n}}   &                                                              &                                                                  \\[0.68em]
        \:=\: & \mathrel{\makebox[\widthof{=}]{\vdots}}                                                              & \frac{\partial}{\partial out_{h}} \cdot \sum_{j=0}^{H} out_{j} \cdot \underline{w}_{jk} = \underline{w}_{hk}                                                                              \\[0.68em]
                                          \:=\: & \frac{\partial E}{\partial y_{m}} \cdot \frac{y_{m} }{\partial net_{m} }                             &                                                              &                                                                  \\[0.68em]
                                          \:=\: & \frac{\partial}{\partial y_{m}} \cdot \frac{1}{2} \sum_{j=1}^{M} (\hat{y}_{j} - y_{j})^{2}             &                                                              &                                                                  \\[0.68em]
                                          \:=\: & \frac{1}{2} \frac{\partial}{\partial y_{m}} (\hat{y}_{m} - y_{m})^{2}             &                                                              &                                                                  \\[0.68em]
                                          \:=\: & \frac{1}{2} \cdot 2 \cdot (\hat{y}_{m} - y_{m}) \cdot  \frac{\partial}{\partial y_{m}} \cdot (-y_{m} )            &                                                              &                                                                  \\[0.68em]
                                          \:=\: & -(\hat{y}_{m} - y_{m})            &                                                              &                                                                  \\[0.68em]
    \end{aligned}
\end{equation}

Conclusion, $$ \sum_{h=1}^{M} - (\hat{y}_{m} - y_{m}) \cdot f'(net_{m}) \cdot \underline{w}_{hm} \cdot f'(net_{h}) \cdot w_{nh}. $$

\end{document}
